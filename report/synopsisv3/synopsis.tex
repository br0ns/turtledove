\documentclass[a4paper,oneside,final]{article}

\usepackage[T1]{fontenc}
\usepackage[utf8]{inputenc}
\usepackage[english]{babel}

\usepackage[format=hang]{caption,subfig}
\usepackage{graphicx}
\usepackage[draft]{fixme}     % Indsæt "fixme" noter i drafts.
\usepackage{hyperref}  % Indsæter links (interne og eksterne) i PDF
\usepackage{mdwtab}
\usepackage{mathenv}
\usepackage{listings}
\usepackage{amssymb}

\setcounter{secnumdepth}{1} % Sæt overskriftsnummereringsdybde. Disable = -1.

\lstset{ %
% language=Octave,                % choose the language of the code
basicstyle=\ttfamily,        % the size of the fonts that are used for the code
basewidth=0.5em,
% numbers=left,                   % where to put the line-numbers
% numberstyle=\footnotesize,      % the size of the fonts that are used for the line-numbers
% stepnumber=2,                   % the step between two line-numbers. If it's 1 each line will be numbered
% numbersep=5pt,                  % how far the line-numbers are from the code
% backgroundcolor=\color{white},  % choose the background color. You must add \usepackage{color}
% showspaces=false,               % show spaces adding particular underscores
% showstringspaces=false,         % underline spaces within strings
% showtabs=false,                 % show tabs within strings adding particular underscores
% frame=single	                % adds a frame around the code
% tabsize=2,	                % sets default tabsize to 2 spaces
% captionpos=b,                   % sets the caption-position to bottom
% breaklines=true,                % sets automatic line breaking
% breakatwhitespace=false,        % sets if automatic breaks should only happen at whitespace
escapeinside={(@}{@)}          % if you want to add a comment within your code
}

\author{Morten Brøns-Pedersen \and Jesper Reenberg}
\title{Project contract}

\begin{document}

\maketitle

% \fixme{Search and replace ``plugins'' with new word ``tools''}
% \fixme{Search and replace ``IDE'' -> ``editor''}
% \fixme{Search and replace ``front end'' -> ``user interface''}
% \fixme{skift alle itemize ud med enumerate så der kan henvises til specifikke punkter uden for denne}

\section{Title} ``Turtledove: Tool assisted programming in SML, with special
emphasis on semi-automatic rewriting.''

\section{Motivation}
Today functional programming languages are not very widespread outside academic
circles. A major reason for this may be that not as rich development
environments exists (if at all) for functional programming languages compared to
certain imperative languages (take languages in the .NET platform as an
example).

We wish to make functional programming, specifically in SML, easier to approach
for the beginner as well as more advantageous for the veteran.

Statically typed functional programming languages has a variety of advantages
that makes them specially suited for quick and correct static analysis. The
effect of this is that it will be possible to make more complex tools for
functional programming languages than for imperative ones.
% \fixme{Er det dumt at hævde disse ting uden et opfølgende argument?}

We implement semi-automatic rewriting of terms according to a set of predefined
rules. We focus on modularity such that it will be relatively simple to
implement other tools based on our software in the future.

% \fixme{Hvorfor dette værktøj? i) Vi behøver ikke regne typer, ii) vi får lov at
%   føre nogle fine beviser, iii) med omskrivning kan vi let implementere flere
%   forskellige værktøjer - især hvis vi implementerer den transitive afslutning
%   af (dele af) omskrivningerne (tænk reduktioner).}

% Experienced programmers will find that code complying to certain standard
% patterns is easier to read. Another advantage of standard patterns is that
% compilers are able to specially optimize these, possibly making such code
% faster.

\section{Elaboration}
We wish to be able to suggest possible rewritings to the programmer in real
time, based on some predefined rules.

This implies that our tool is initiated and responds through some user
interface. There are two good reasons why we wish to be able to develop our tool
and its user interface separately:

\begin{enumerate}
\item As different programmers use different editors and generally have strong
  feelings about their choice we do not see it as an option to implement a new
  editor. Instead a user interface would be implemented on top of an existing
  editor. We do not wish to bind our tool to a single one.
\item It is not the goal of this project to implement a user interface for tool
  assisted programming.
\end{enumerate}

Since the user interface and the tool are two distinct pieces of software we
have to decide a way for them to communicate.

\subsection{Semi-automatic rewriting}
We implement a set of rewriting rules. This implies the development of a
suitable representation of rules and a theory for arguing about them.

An example of a potential rule is given below.

\subsubsection{Example}\label{sec:example} 
Given a suitable notion of instantiation and alpha equivalence, consider the
following rewriting rule for a standard map form:
\begin{eqnarray}[TlcTl]
\begin{lstlisting} fun f nil = nil | f (x :: xs) = (@$\heartsuit$@) :: f xs
\end{lstlisting} & \ \rightarrow \ &
\begin{lstlisting} val f = map (fn x => (@$\heartsuit$@))
\end{lstlisting}\label{eq:rule-map}
\end{eqnarray}

Notice that the \texttt{x} on the right hand side is related to the \texttt{x}
on the left as illustrated below.

The term
\begin{eqnarray}[Tl]\label{eq:term-gt}
\begin{lstlisting} fun gt nil = nil | gt (n :: ns) = n > 0 :: gt ns
\end{lstlisting}
\end{eqnarray} is an instance of the left hand side of (\ref{eq:rule-map}) with
instantiation
\begin{eqnarray*}[TlcTl] $\heartsuit$ & \ =\ & \texttt{n > 0}
\end{eqnarray*} and alpha conversion
\begin{eqnarray*}[TlcTl] \texttt{f} & \ =\ & \texttt{gt}\\ \texttt{x} & \ =\ &
\texttt{n}\\ \texttt{xs} & \ =\ & \texttt{ns}
\end{eqnarray*}

Now using (\ref{eq:rule-map}) we get the result
\begin{eqnarray}[Tl]\label{eq:term-gt-result}
\begin{lstlisting} val gt = map (fn n => n > 0)
\end{lstlisting}
\end{eqnarray}

\subsection{Turtledove}
Our implementation of the described tool is named Turtledove.\\

Since we intent that Turtledove should grow to include more than one tool we
design it in such a way as to accommodate this. This means that functionality
unique to a specific tool should be separated from functionality common to most
tools.

Common functionality is split between core functionality needed by every tool
and a collection of libraries that make it easier to perform common tasks.

\subsubsection{Core functionality}
The core functionality of Turtledove includes:

\begin{enumerate}
\item Turtledove will take the path of an SML file or a project file either as a
  command line argument or through the user interface after it has been
  started. It then reads and parses the file or files contained in the project.

  The user interface should signal when a file is changed and possibly what has
  changed within that file. The one extreme is to signal a file change only when
  a file is saved. The other is to signal every key stroke. Testing will reveal
  the right balance.

  Tools shall be signalled when the source code has changed.

\item Data is guided between the user interface and its
  destination. Communication is done over a protocol flexible enough to allow
  each destination to implement its own protocol.
\end{enumerate}

\section{Primary goals}
To summarize; Turtledove should be able to

\begin{enumerate}
\item Read single SML files or projects.
\item Analyze those files and suggest rewritings as described above.
\item Be signalled when a file changes and suggest new rewritings if any exists.
\end{enumerate}

We will implement rules for at least two rewritings: Map forms (as demonstrated
in Section \ref{sec:example}) and fold forms.

Equivalence of terms and their transformed counterparts should ideally be
proven. We acknowledge the complexity of such a proof, and is thus satisfied
with a good argument.

\section{Secondary goals}
Here follows a non-exhaustive list of further work, should time allow it:

\begin{enumerate}

\item Implement project management. A project manager should at least make it
  easy for the user to add and remove files from a project. The project manager
  should write changes to the ML Basis file containing the project to disk on
  the programmers behalf.

\item Make a nice user interface. Probably for Emacs as we use it ourselves and
  thus have more experience with this editor than other ones.

\item Extend the various parsers to be incremental where applicable (e.g., the
  parser for SML source code). The benefit would be speed gains for reparsing of
  files with small changes, which is often the case.

\item Extend the various parsers to handle syntax errors, for example by
  skipping the source code until the next valid declaration is found. The
  location in the skipped line of code where the first error was found should
  then be reported back to the user interface so it can make a proper
  notification of the user (e.g., underlining of erroneous code).

% Husk : Måske muligt med SML-Yacc..?

\item Implement transitive closures. Each rewriting rule or group of rules
  define a relation on terms. We suspect that some interesting rewritings can be
  defined as the transitive closure of a rule or a group of rules. An annotation
  in the rule or rule group will tell Turtledove whether it should compute the
  transitive closure of a rule or not.

\item Add a simplification group of term rewriting rules. Consider the rule
  \texttt{fn x => g x $\rightarrow$ g} as an example.

  When defining a rule it shall be possible to try other rules on the resulting
  term. The map rule from Section \ref{sec:example} will benefit from applying
  the above rule to its output. Of course if the rule does not match it should
  do nothing.

  Another option is to always try to apply a set of simplifying rules to the
  output of any rule.

\item Extend the IO file/source code handling part of the program to have a
  ``patch'' system such that code written in the editor is reflected ``live'' if
  the SML parser decides to parse the source files, even if the source files
  haven't been saved recently.

\item Extend Turtledove such that the SML source code parser can report back
  ``DocString'' information \footnote{A short description of a value or function
    as known from for example Eclipse or Visual Studio.} to the user interface
  which in turn can inform the user in an appropriate way.

\item Various parsings of the syntax tree to for example gradually remove
  syntactic sugar.

  With this a notion of ``run levels'' could be introduced, so at the end of
  each parsing all tools that hook a given level will be notified and able to
  act upon this specific version of the parse tree. While the next parse of the
  syntax tree should continue if there is tools depending on one of the next run
  levels.
\end{enumerate}

\section{Limitations}
\begin{enumerate}
\item Turtledove need not be able to read CM project files.

\item The SML parser need not handle syntax errors.

\item Changes to unsaved files need not be reflected inside Turtledove.

\item No user documentation of software need be produced (e.g., manual, user
  guide).
\end{enumerate}

\section{Learning objectives}
\begin{enumerate}
\item To develop structured, flexible and easy-to-understand software for
  manipulation of source code.
\item To understand and describe source code transformations formally, and to
  implement such transformations.
\end{enumerate}

\section{Tasks}
The project starts December the 14th (week 51) and ends June the 18th (week
24). A lot of the individual task can potentially be done in parallel as the
project has both theoretical and practical aspects.

Weeks 52-1, 4, 13 and 15 are reserved for holidays and exams.

The two task lists below is split in mostly theoretical and practical
tasks. Each list is ordered by descending priority.

\begin{description}
\item[Theoretical]\
  \begin{description}
  \item[week 51] Research state-of-the-art literature on code rewriting.
    
  \item[week 2-3] Make a draft of the formal definition of ``rewriting
    rules''. Experiment with different approaches. Determine and define a
    framework (helper functions, predicates on syntactic entities,
    transformations, canonical forms, etc.).
    \begin{itemize}
    \item[-] \it A short paper (approximately 20 pages) that defines term
      rewriting for a subset of SML. Examples of use are given.
    \end{itemize}
    
  \item[week 4-5] Formally define ``rewriting rules'' and framework.
    \begin{itemize}
    \item[-] \it The given definitions and framework is extended to full SML.
    \end{itemize}
    
  \item[week 5-7] Define a set of rewriting rules. Experiment with real world
    code examples.
    
  \item[week 8-9] Argue for the correctness and usefulness of the implemented
    rules.

  \item[week 17-20] Finish the report.
  \end{description}

\item[Practical]\
  \begin{description}
    
  \item[week 2-3] Design an API for intercommunication between the user
    interface and Turtledove. Maybe implement a small library of utility
    functions.
    
  \item[week 2-3] Implement a parser that can parse ML Basis project files
    \footnote{As described at
      \url{http://www.itu.dk/research/mlkit/index.php/ML_Basis_Files} (as of
      September 24th, 2009).}.
    
  \item[week 4-5] Implement a project manager that given the path of a ML Basis
    file parses the file, and models the project. It should be possible to read
    the contents of the files through the model.
    
  \item[week 5-6] Implement a representation of concrete syntax trees of SML
    source code.
    
  \item[week 7-8] Implement an SML parser.

  \item[week 9-11] Implement the term rewriting framework.
    
  \item[week 10-12] Implement the term rewriting tool. Including the actual
    rules.
    
  \item[week 13] Implement the designed API for intercommunication between the
    user interface and Turtledove.
    
  \item[week 14-16] Analyse Turtledove and apply threading where needed (e.g.,
    the parser for SML code and Turtledove itself). Improve performance and tidy
    up the code if needed.

  \end{description}

\item[week 20-24] Buffer, secondary goals.

\end{description}

\end{document}
