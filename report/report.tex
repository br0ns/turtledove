\documentclass[a4paper, oneside, draft]{memoir}
% Fixes "No room for a new \xxx" error by extending the default 256 fixed size
% LaTeX arrays
\usepackage{etex}
\reserveinserts{28}


\usepackage[T1]{fontenc}
\usepackage[utf8]{inputenc}
\usepackage[british]{babel}

% bedre orddeling Gør at der som minimum skal blive to tegn på linien ved
% orddeling og minimum flyttes to tegn ned på næste linie. Desværre er værdien
% anvendt af babel »12«, hvilket kan give orddelingen »h-vor«.
\renewcommand{\britishhyphenmins}{22} 

% Fix of fancyref to work with memoir. Makes references look
% nice. Redefines memoir \fref and \Fref to \refer and \Refer.
% \usepackage{refer}             %
% As we dont really have any use for \fref and \Fref we just undefine what
% memoir defined them as, so fancyref can define what it wants.
\let\fref\undefined
\let\Fref\undefined
\usepackage{fancyref} % Better reference. 

\usepackage{pdflscape} % Gør landscape-environmentet tilgængeligt
\usepackage{fixme}     % Indsæt "fixme" noter i drafts.
\usepackage{hyperref}  % Indsæter links (interne og eksterne) i PDF

\usepackage[rounded]{syntax} % Part of the mdwtools package

\usepackage{mdwtab}
\usepackage{mathenv}
\usepackage{amsfonts}
\usepackage{amsmath}
\usepackage{amssymb}
\usepackage{amsthm}
\usepackage{semantic} % for the \mathlig function

\usepackage[format=hang]{caption,subfig}
\usepackage{graphicx}
\usepackage{stmaryrd}
\usepackage[final]{listings} % Make sure we show the listing even though we are
                             % making a final report.
\usepackage{ulem} % \sout - strike-through
\usepackage{tikz}

\lstset{ %
% language=Octave,                % choose the language of the code
basicstyle=\ttfamily,        % the size of the fonts that are used for the code
basewidth=0.5em,
% numbers=left,                   % where to put the line-numbers
% numberstyle=\footnotesize,      % the size of the fonts that are used for the line-numbers
% stepnumber=2,                   % the step between two line-numbers. If it's 1 each line will be numbered
% numbersep=5pt,                  % how far the line-numbers are from the code
% backgroundcolor=\color{white},  % choose the background color. You must add \usepackage{color}
% showspaces=false,               % show spaces adding particular underscores
% showstringspaces=false,         % underline spaces within strings
% showtabs=false,                 % show tabs within strings adding particular underscores
% frame=single	                % adds a frame around the code
% tabsize=2,	                % sets default tabsize to 2 spaces
% captionpos=b,                   % sets the caption-position to bottom
% breaklines=true,                % sets automatic line breaking
% breakatwhitespace=false,        % sets if automatic breaks should only happen at whitespace
escapeinside={(@}{@)}          % if you want to add a comment within your code
}

\renewcommand{\ttdefault}{txtt} % Bedre typewriter font
%\usepackage[sc]{mathpazo}     % Palatino font
\renewcommand{\rmdefault}{ugm} % Garamond
%\usepackage[garamond]{mathdesign}

%\overfullrule=5pt
%\setsecnumdepth{part}
\setcounter{secnumdepth}{1} % Sæt overskriftsnummereringsdybde. Disable = -1.
\chapterstyle{hangnum} % changes style of chapters, to look nice.

\makeatletter
\newenvironment{nonfloatingfigure}{
  \vskip\intextsep
  \def\@captype{figure}
  }{
  \vskip\intextsep
}

\newenvironment{nonfloatingtable}{
  \vskip\intextsep
  \def\@captype{table}
  }{
  \vskip\intextsep
}
\makeatother

\renewcommand{\ttdefault}{txtt} % Bedre typewriter font
%% \usepackage[sc]{mathpazo}     % Palatino font
%% \renewcommand{\rmdefault}{ugm} % Garamond
%% \usepackage[garamond]{mathdesign}

% \overfullrule=5pt
% \setsecnumdepth{part}
\setcounter{secnumdepth}{1} % Sæt overskriftsnummereringsdybde. Disable = -1.
\chapterstyle{hangnum} % changes style of chapters, to look nice.

\theoremstyle{definition}
\newtheorem{judgment}{Judgment}
\newtheorem{definition}{Definition}
\newtheorem{lemma}{Lemma}
\newtheorem{theorem}{Theorem}
\newtheorem{corollary}{Corollary}
\newtheorem{example}{Example}

\newcommand*{\fancyrefdeflabelprefix}{def}
\fancyrefaddcaptions{english}{
  \newcommand*{\Frefdefname}{Definition}
  \newcommand*{\frefdefname}{\MakeLowercase{\Frefdefname}}
}
\frefformat{vario}{\fancyrefdeflabelprefix}{%
  \frefdefname\fancyrefdefaultspacing#1#3%
}
\Frefformat{vario}{\fancyrefdeflabelprefix}{%
  \Frefdefname\fancyrefdefaultspacing#1#3%
}

\newcommand*{\fancyreflemlabelprefix}{lem}
\fancyrefaddcaptions{english}{
  \newcommand*{\Freflemname}{Lemma}
  \newcommand*{\freflemname}{\MakeLowercase{\Freflemname}}
}
\frefformat{vario}{\fancyreflemlabelprefix}{%
  \freflemname\fancyrefdefaultspacing#1#3%
}
\Frefformat{vario}{\fancyreflemlabelprefix}{%
  \Freflemname\fancyrefdefaultspacing#1#3%
}
\frefformat{plain}{\fancyreflemlabelprefix}{%
  \freflemname\fancyrefdefaultspacing#1%
}
\Frefformat{plain}{\fancyreflemlabelprefix}{%
  \Freflemname\fancyrefdefaultspacing#1%
}

\newcommand*{\fancyrefthmlabelprefix}{thm}
\fancyrefaddcaptions{english}{
  \newcommand*{\Frefthmname}{Theorem}
  \newcommand*{\frefthmname}{\MakeLowercase{\Frefthmname}}
}
\frefformat{vario}{\fancyrefthmlabelprefix}{%
  \frefthmname\fancyrefdefaultspacing#1#3%
}
\Frefformat{vario}{\fancyrefthmlabelprefix}{%
  \Frefthmname\fancyrefdefaultspacing#1#3%
}

\newcommand*{\fancyrefcorlabelprefix}{cor}
\fancyrefaddcaptions{english}{
  \newcommand*{\Frefcorname}{Corollary}
  \newcommand*{\frefcorname}{\MakeLowercase{\Frefcorname}}
}
\frefformat{vario}{\fancyrefcorlabelprefix}{%
  \frefcorname\fancyrefdefaultspacing#1#3%
}
\Frefformat{vario}{\fancyrefcorlabelprefix}{%
  \Frefcorname\fancyrefdefaultspacing#1#3%
}

\newcommand*{\fancyrefexlabelprefix}{ex}
\fancyrefaddcaptions{english}{
  \newcommand*{\Frefexname}{Example}
  \newcommand*{\frefexname}{\MakeLowercase{\Frefexname}}
}
\frefformat{vario}{\fancyrefexlabelprefix}{%
  \frefexname\fancyrefdefaultspacing#1#3%
}
\Frefformat{vario}{\fancyrefexlabelprefix}{%
  \Frefexname\fancyrefdefaultspacing#1#3%
}

\newcommand{\ttt}[1]{\texttt{#1}}
\newcommand{\tnm}[1]{\textnormal{#1}}
\newcommand{\mrm}[1]{\mathrm{#1}}

\newcommand{\Cov}{\mathrm{Cov}}
\providecommand{\FV}{\mathrm{FV}}
\providecommand{\Dom}{\mathrm{Dom}}


\mathlig{||}{\parallel}
\mathlig{<'}{\prec}
\mathlig{>'}{\succ}
\mathlig{<='}{\preccurlyeq}
\mathlig{>='}{\succcurlyeq}
\mathlig{<=}{\leqslant}
\mathlig{>=}{\geqslant}
\mathlig{<>}{\neq}
\mathlig{|=}{\sqsubset}
\mathlig{=|}{\sqsupset}
\mathlig{==}{\equiv}
\mathlig{==a}{=_{\alpha}}
\mathlig{<|}{\lhd}
\mathlig{|>}{\rhd}
\mathlig{++}{\mathrel{\mbox{+\!\!\!+}}}
\mathlig{~>e}{\stackrel{elim}{\leadsto}}
\mathlig{~>g}{\stackrel{gen}{\leadsto}}

%%%%%%%%%%%%%%%%%%%%%%%%%%%%%%%%%%%%%%%%%%%%%%%%%%%%%%%%
%	    	     Forside
%%%%%%%%%%%%%%%%%%%%%%%%%%%%%%%%%%%%%%%%%%%%%%%%%%%%%%%%
\makeatletter % open mode for reading @ signed variables 
\def\maketitle{%
  \null
  \thispagestyle{empty}%
  \vfill
  \begin{center}\leavevmode
    \normalfont
    \Huge{\raggedleft \@title\par}%
    \hrulefill\par
    \Large{\raggedright \subtitle\par}%
    \vskip 2cm
    {\@date\par}%
  \end{center}%
  \vfill
\begin{minipage}{80pt}
\includegraphics*[scale=0.75]{imgs/nat-logo}
\end{minipage}
\begin{minipage}{300pt}
  \begin{flushleft}
    {\large \@author } \\
    {\footnotesize \suplementInfo }

  \end{flushleft}
\end{minipage}
\cleardoublepage % lave 1 ekstre side blank efter
  \clearpage % Terminates the page here. Everything else vil be placed on next page.
}
\makeatother % closing mode for reading @ signed variables
%%%%%%%%%%%%%%%%%%%%%%%%%%%%%%%%%%%%%%%%%%%%%%%%%%%%%%%%
%		Data til forside
%%%%%%%%%%%%%%%%%%%%%%%%%%%%%%%%%%%%%%%%%%%%%%%%%%%%%%%%
\title{Turtledove}
\def\subtitle{\footnotesize{Tool assisted programming in SML, with special emphasis on semi-automatic rewriting to
predefined standard forms.}}
\author{Morten Brøns-Pedersen {\footnotesize{(mortenbp@gmail.com)}}\\  
Jesper Reenberg \footnotesize{(jesper.reenberg@gmail.com)}}
\def\suplementInfo{
  \kern 5pt \hrule width 11pc \kern 5pt % putter 5pt spacing oven over og neden under stregen
  Dept. of Computer Science \\
  University of Copenhagen}
% \date{} % used to set explicit dates

\pagestyle{plain}

\begin{document}

\frontmatter

\maketitle
\thispagestyle{empty}

\begin{abstract}
\end{abstract}

\clearpage 
\chapter*{Preface}
This report is a 22.5 ECTS masters project at the Department of Computer
Science, University of Copenhagen (DIKU). The authors are Morten Brøns-Pedersen
and Jesper Reenberg. The project is supervised by Jakob Grue Simonsen, assistant
professor at DIKU.

\clearpage

\tableofcontents*

\mainmatter

\chapter{Introduction}

\section{Problem statement}
\label{sec:problem-statement}

\section{Motivation}
\label{sec:motivation}

\section{Reader expectations}

\section{Structure outline}

\chapter{Term rewriting}


%%% Local Variables: 
%%% mode: latex
%%% TeX-master: "../report"
%%% End: 




\chapter{Design}
\label{cha:design}



\section{Turtledove}
\label{sec:design-turtledove}


%\subsection{MLB parser}
%\label{sec:design-mlb-parser}

%\subsection{SML Parser}
%\label{sec:design-sml-parser}


\subsection{Project manager}
\label{sec:design-project-manager}

The project manager is responsible for managing the project and all actions done
to the current project. 

There are essentially two types of projects, that the manager needs to handle
ordinary projects and ``single file'' projects (both described below). 

When the project manager tries to open a project, it always tries to open it as
a ordinary project. As project files are JSON encoded\footnote{As we also need a
  JSON parser for other purposes this has been chosen as the project encoding to
  ease development time. Also the encoding is ``human readable'' which make
  human editing of the files relatively easy, though not encouraged.} and are
defined to contain a specialised \synt{JSON-Object} (see
\fref{fig:protocol-json}), the first character, except any trailing whitespace,
must be a ``\{'' else this is not a valid project file and it then defaults to
open the file as a ``single file'' project.

No action will be taken from the project manager if the file, opened as a
``single file'' project, is not valid SML code. In this case the SML parser will
try to parse the file and fail with some appropriate error message.

The following are valid actions that can be done to a project

\begin{description}
\item[Files] can be added, removed or renamed. \\

  If a filename is relative, we enforce that it must be relative to where
  the project file is located, sinve all relative filenames will be made
  absolute, by appending the path of the project file, when creating the MLB
  description. This is due to the MLB parser \fixme{MLB eller SML parseren?}
  only accepts absolute paths.

  It don't make sense to have the same file defined multiple times in a project,
  so it is not allowed as this will make it impossible to know which file to
  rename or delete. Also the same functionality can be accomplished by using
  dependencies.

\item[Groupings] can be added, removed or renamed. \\

  To ease the overall structure of a project, it can be grouped. A grouping is a
  logical collection of either files or other groupings referenced as
  objects. It must have a name, reference list of names of objects to be exposed
  and list of object definitions. The name of a grouping is the counterpart to
  the filename of files so that it can be referenced in dependency and exposure
  informations. The exposed object names must be of objects defined top most in
  this group and not of objects defined in sub groupings of this group.

  As with files, it is not allowed to add multiple groupings with the same name.


\item[Dependencies] can be added or removed. \\

  Dependencies make sure that files are parsed in the correct order. This is
  explained in more detail below. The most common use of dependencies is that
  structures must depend upon the signatures that they implement such that the
  signature is parsed before the structure.


  One might ask why this is not auto generated? In some (if not most) cases it
  can't be determined, most obviously if there exists two modules with the same
  name but with different meaning. However in most cases a fair guess can be
  generated, though it leaves the few cases unsolved where it is not possible
  and it would be complicated and time consuming to implement. So the other
  solution has been chosen which also offer more flexibility to the user at the
  cost of being a bit more complicated to use.


  If a file doesn't have any dependencies, then there is no guarantee of which
  order the file will be parsed in. In all cases where a structure implements a
  signature, and the signature is defined in another file), then the file
  defining the structure must depend on the file defining the signature, as this
  will make sure that the signature is parsed before the structure.

\item[Exposure] can be added or removed. \\

  Exposing an object or not, can bee seen as a form of public or private
  modified respectively. For example a grouping containing two files
  \textit{FooUtils.sml} and \textit{Boo.sml}, here we don't necessarily want to
  expose the functions defined in the utility file as these may only have
  meaning to the functions defined in the foo file. In this case we would also
  have a dependency from \textit{Foo.sml} to \textit{FooUtils.sml} such that the
  utility file is parsed before the foo file.

  This is explained in detail in \fref{sec:translating-into-mlb}.

\end{description}


\subsubsection{Ordinary Projects}


An ordinary project is defined by a project file (see
\fref{sec:protocol-project-file}), normally with the extension \texttt{.turt}.

To keep the project file as simple as possible, the actual project definition is
a grouping (the ``ProjectNode'' field in the below description of the project
file). Having it this way simplifies implementation since there is no need for
separate logic to handle the project definition and separate logic for the
containing groupings. In other terms, we can use the same recursive logic.

\paragraph{The project file}

The project file is a \synt{Project-File-Object} which is defined by the
following three fields, see \fref[plain]{fig:protocol-project-file} for the
definition of a project file and more examples of project files.
  

\begin{description}
\item[Properties] a \synt{JSON-Object} of key-value pairs that the editor can
  read and write any valid \synt{JSON-Value} to. The editor could save state
  information, like which files were open when the project was last used,
  editor version information or anything else. Turtledove will never read or
  write to this field.
  
  \begin{description}
  \item[Dependencies] a \synt{Depends-Array} where each \synt{Depend-Object}
    has two fields
    
    \begin{description}
    \item[Name] a \synt{JSON-String} that defines the group or filename this
      depend constraint applies to.
      
    \item[Depends] a \synt{String-Array} of any group or filename that this
      depend constraint depends upon.
    \end{description}
    
    
  \item[ProjectNode] a \synt{Node-Group-Object} that defines the overall
    structure of a grouping. The ``ProjectNode'' (grouping) is the topmost
    grouping and can thus not be removed. It has the following two fields
    
    \begin{description}
    \item[Name] The name of this grouping as a \synt {JSON-String}.
      
    \item[Value] A \synt{Value-Object} that contains sub groupings or stand
      alone files contained within this group. It has the following two fields
      
      \begin{description}
      \item[Exposes] A \synt{String-Array} of group or filenames from the
        \synt{Nodes-Array} that gets exposed at compile time to groups or files
        that depend on this grouping.
        
      \item[Nodes] A \synt{Nodes-Array} that contains an array of either
        \synt{JSON-String} that is an stand alone file name or
        \synt{Node-Group-Object} which is a sub group of this grouping that
        creates a recursive data structure of the above explained.
      \end{description}
    \end{description}      
  \end{description}
\end{description}


\begin{example}[Sample project file for a project named ``Turtledove'']\ 
  \label{ex:Sample-project-file-turtledove}
  
  The \texttt{Turtledove} project contains the following groupings
  
  \begin{itemize}
  \item Group \texttt{A} which
    \begin{itemize}
    \item Contains the file \texttt{x.sml}.
    \item Exposes the file \texttt{x.sml}.
    \item Doesn't depend on anything.
    \end{itemize}
    
  \item Group \texttt{B} which
    \begin{itemize}
    \item Contains the files \texttt{y.sml} and \texttt{z.sml}.
    \item Exposes the file \texttt{z.sml}.
    \item Has dependencies
      \begin{itemize}
      \item \texttt{y.sml} depends on \texttt{x.sml}.
      \item \texttt{z.sml} depends on \texttt{y.sml}.
      \end{itemize}
    \end{itemize}
    
  \item Group \texttt{C} which
    \begin{itemize}
    \item Contains the files \texttt{n.sml} and \texttt{m.sml}.
    \item Exposes the files \texttt{n.sml} and \texttt{m.sml}.
    \item Doesn't depend on anything.
    \end{itemize}
  \end{itemize}
  
  and files \texttt{u.sml} and \texttt{j.sml}. The project exposes the group
  \texttt{C} and file \texttt{j.sml}, and it has dependencies: \texttt{C}
  depends on \texttt{B}, \texttt{j.sml} depends on \texttt{u.sml} and
  \texttt{j.sml} depends on \texttt{B}.
  
  
  This results in the following project file
  
\begin{lstlisting}
{
  "Properties" : { },
  "Dependencies" : 
  [
    { "Name" : "C", "Depends" : [ "B" ] },
    { "Name" : "j.sml" , "Depends" : [ "u.sml", "B" ] },
    { "Name" : "y.sml" , "Depends" : [ "x.sml" ] },
    { "Name" : "z.sml", "Depends" : [ "y.sml" ] }
  ],
  "ProjectNode" :
  {
    "Name"  : "Turtledove",
    "Value" : 
    {
       "Exposes" : [ "C", "j.sml" ],
       "Nodes" :
       [
         {
           "Name" : "A",
           "Value" :
           {
             "Exposes" : [ "x.sml" ],
             "Nodes" : [ "x.sml" ]
           }          
         },
         {
           "Name" : "B",
           "Value" :
           {
             "Exposes" : [ "z.sml" ],
             "Nodes" : [ "y.sml", "z.sml" ]
           }          
         },
         {
           "Name" : "C",
           "Value" :
           {
             "Exposes" : [ "n.sml", "m.sml" ],
             "Nodes" : [ "n.sml", "m.sml" ]
           }          
         },
         "u.sml", 
         "j.sml"
       ]
    }
  }
}    
\end{lstlisting}
\end{example}

\paragraph{Sorting files and groupings}

When the project needs to be parsed, a resulting MLB description of the project
is generated. As files are parsed in the order they are listed in the MLB
description it is important to sort out the order in which files and groupings
should be listed from the defined dependencies. First this needs to be done for
files and then for groupings.

For files this can be done in two steps

\begin{itemize}
\item Create a DAG (Directed Acyclic Graph) with each filename as a node and
  each dependency as an edge from the file to all the files it depends on.

  
  However before the edges are added to the DAG, any group dependencies needs to
  be expanded. This include the following two steps
  
  \begin{enumerate}
  \item \label{item:expand-group-dependency1} Expand all dependencies to a
    group.
  \item \label{item:expand-group-dependency2} Expand all dependencies from a
    group.
  \end{enumerate}
  
  Expanding the group dependency \texttt{\{"Name" : "C", "Depends" : [ "B" ]\}}
  from the project file in \fref[plain]{ex:Sample-project-file-turtledove},
  would result in the following list by applying the above step
  \ref{item:expand-group-dependency1}
  
\begin{lstlisting}
  C depends on y.sml
  C depends on z.sml
\end{lstlisting}
  
  and by applying step \ref{item:expand-group-dependency2} we get the following
  4 edges
  
\begin{lstlisting}
  n.sml depends on y.sml
  m.sml depends on y.sml
  n.sml depends on z.sml
  m.sml depends on z.sml
\end{lstlisting}
  
  which needs to be added to the DAG as representing that dependency constraint.
  
\item Topological sort the DAG to get an ordered list of filenames.
\end{itemize}

Afterwards the same needs to be done for the groupings. As with the files,
groupings are added as nodes to the DAG but only dependencies from and to groups
are added as edges, without expanding them to their containing files.


The two resulting ordered list of files and groupings represent the order of how
they should be listed in the final MLB description.


\paragraph{Translating into MLB description}
\label{sec:translating-into-mlb}

Each file and grouping needs to be translated into its equivalent MLB
description, which can be done in two ways for both files and groupings. 

Since both files and groupings have been sorted, we can safely reference the
generated basis names of their dependencies with the MLB syntax \texttt{open
  BASIS_NAME} without checking if it actually exists.

The translation examples below uses files and groupings from
\fref[plain]{ex:Sample-project-file-turtledove}.


\begin{description}
  
\item[Translation of files] \
  \begin{description}
   
  \item[No dependencies]
    
    The file \texttt{x.sml} from group \texttt{A} translates to
    
\begin{lstlisting}
basis x = bas x.sml end      
\end{lstlisting}
    
    where \texttt{x.sml} is the absolute path of the filename, and \texttt{x} is
    the basis name of \texttt{x.sml} which is defined as \textit{the filename
      without its extension and suffixed with an incrementing number}.
        
  \item[With dependencies]
    
    The file \texttt{y.sml} from group \texttt{B} that depends on file
    \texttt{x.sml} from group \texttt{A} translates to
    
\begin{lstlisting}
basis y = bas let open x in y.sml end end
\end{lstlisting}
    
    where \texttt{y} is the basis name of \texttt{y.sml} and \texttt{x} is the
    basis name of \texttt{x.sml} from the previous example.
    
    If there had been dependencies on multiple files their basis names would
    just have been listed after \texttt{x}, for example \texttt{... open x g h
      in ...}.
    
  \end{description}
  
\item[Translation of groupings]
  
  As files always belong to a grouping (the project itself is the top grouping and
  must always be defined), they are the ones that are responsible for exposing
  files or groups that are contained within itself.
  
  \begin{description}
  \item[No dependencies]
    
    The grouping \texttt{A} exposes file \texttt{x.sml} so it translates to
    
\begin{lstlisting}
basis A = bas open x end
\end{lstlisting}
    
    no matter how many files or groupings it contains. Here \texttt{x} is the
    basis name of \texttt{x.sml} defined above and \texttt{A} is the basis name
    for the group defined as \textit{the name of the group suffixed with an
      incrementing number}
    
  \item[With dependencies] The grouping \texttt{B} exposes multiple files, depends
    on another grouping and file so it translates to
    
\begin{lstlisting}
basis B = bas let open A u in open y z end end
\end{lstlisting}
    
    where \texttt{B}, \texttt{A}, \texttt{u}, \texttt{y} and \texttt{z} are
    basis names of their respective grouping or file.
  \end{description}
  
  
\item[Translation of the outermost grouping]
  
  The actual project definition, which is the outer most grouping, needs to be
  handled a bid special as every file or grouping translation is encapsulated in
  a MLB basis declaration (\texttt{basis x = bas ... end}) and is thus not
  accessible outside the basis declaration unless opened.
  
  So for the project, to be able to expose anything to the environment trying to
  use this project, we need to open it without encapsulating it in a basis
  declaration and it needs to be done as the last thing in the MLB description.
  
  The project group \texttt{Turtledove} exposes the group \texttt{B} and the
  file \texttt{j.sml} so it would translate to
  
\begin{lstlisting}
open B j
\end{lstlisting}
  
  where \texttt{B} and \texttt{j} are basis names of their respective grouping
  or file.
  
\end{description}



It is important that the basis library is included in the resulting MLB
description as nothing is included by default in any compiler/parser that reads
MLB descriptions and any stand alone file can expect this environment to be
available.

So the following empty MLB description template is used, where \texttt{...} is
replaced with the result of the above algorithm where all file paths that are
relative to the project file is expanded to their absolute paths.

\begin{example}[Empty MLB description template.]\
\label{ex:empty-mlb-description}
\begin{lstlisting}
local
  (@\$@)(SML_LIB)/basis/basis.mlb
in
  ...
end
\end{lstlisting}
\end{example}

The resulting MLB description for the project in
\fref[plain]{ex:Sample-project-file-turtledove} where ordering of files without
dependencies or files with the same dependencies are not important (and thus up
to the implementation) and where the path of the project file is \texttt{/tmp/turtledove.turt}

\begin{example}[Resulting MLB description for {\fref[plain]{ex:Sample-project-file-turtledove}}]\
\begin{lstlisting}
local
  (@\$@)(SML_LIB)/basis/basis.mlb
in
  basis u_0 = bas /tmp/u.sml end
  basis x_1 = bas /tmp/x.sml end

  basis y_2 = bas let open x_1 in /tmp/y.sml end end
  
  basis z_3 = bas let open y_2 in /tmp/z.sml end end

  basis n_4 = bas let open y_2 z_3 in /tmp/n.sml end end
  basis m_5 = bas let open y_2 z_3 in /tmp/m.sml end end
  
  basis j_6 = bas let open u_0 y_2 z_3 in /tmp/j.sml end end

  basis A_7 = bas open x_1 end
  basis B_8 = bas open y_2 end

  basis C_9 = bas let open B_9 in open n_4 m_5 end end

  open C_9 j_6
end
\end{lstlisting}  
\end{example}


\subsubsection{``Single file'' projects}

These types of projects doesn't have any project file, a stand alone SML code
file has been opened. However as Turtledove expects to get a MLB compliant
description of the project, one still needs to be supplied.

As ``single file'' projects only contain one file it is very easy to reference
this file in the empty MLB description template (see
\ref{ex:empty-mlb-description}) that includes the basis library as the only
thing.

\begin{example}[Resulting MLB description for ``single file'' projects]\
\begin{lstlisting}
local
  (@\$@)(SML_LIB)/basis/basis.mlb
in
  ABSOLUTE_PATH_TO_STAND_ALONE_FILE
end
\end{lstlisting}  
\end{example}


The ``single file'' project is an important feature of Turtledove for multiple
reasons

\begin{itemize}
\item Novice programmers of functional languages normally don't learn about or
  create their own signatures and structures, so a MLB description will not be
  necessary to compile the program.
  
\item Students following a course teaching functional programming normally newer
  create multiple files for one assignment, and thus the hassle of setting up an
  full fledge project definition for each coding session will get tedious
  (ofcause this will depend on the editor that is used).
  
  \begin{itemize}
  \item When signatures and structures are taught as part of a class they are
    normally created inside the same file and thus the above still holds
  \end{itemize}
\end{itemize}

And thus this will properly bee the most used of the two types of projects.

Regardless of what type of project that has been opened, the resulting MLB
description is never directly saved to a file. The MLB parser accepts its input
as a string and the description is transfered this way internally as the MLB
description is likely to change when any part of the project changes.

No actions (adding/removing of files, groups, etc.) are allowed to the project
when opened this way since there is no point in saving a project file for just a
single code file, it forces the user to create an ordinary project if more
than one file is needed which is the right thing to do and this kind of project
are more supposed to be a quick way of editing a code file.


%%% Local Variables: 
%%% mode: latex
%%% TeX-master: "../../report"
%%% End: 



%%% Local Variables: 
%%% mode: latex
%%% TeX-master: "../report"
%%% End: 


\chapter{Implementation}


\section{Parsers}

The SML and Rule parsers are both implemented in SML-Lex and SML-Yacc. We say
SML-Lex/Yacc as both the sml/nj version ml-lex/ml-yacc and mlton version
mllex/mlyacc works, but the mosml version mosmllex and mosmlyac are having
problems.

\fixme{fix the above gibberish}

\subsection{Standard ML}

A fine note about whete the grammar was stolen from and possibly some other
clever stuff.

\subsection{Rule}

The Rule parser is based on the SML parser with some added tokens to the lexer
and rules to the grammar to handle the rule syntax (see
\fref{tab:rule-grammar}). 


\subsubsection{Unicode}

The rule syntax uses the symbols \texttt{£} and \texttt{§} to denote
transformers and meta patterns respectively. These were a bit tricky to
implement as they are not ASCII characters. By default SML-Lex\cite{ml-lex-yacc}
only supports 8-bit characters and thus if the upper parts of UTF-8 characters
are needed they must be specially handled. One way to handle it would be to
convert the input file into some fixed length encoding (i.e., UTF-32), but as
SML only uses ASCII chars that would demand a total remake of the Rule lexer
definition. Instead we went with a UTF-8 solution which only need to be
specially fitted for the transformer and meta patterns as it seems that most
editors now a days use UTF-8 as default encoding (seems to be the case for emacs
on linux). Implementing the two UTF-8 values in the lexer was then just a matter
of defining a named expression containing the conjunction of the two decimal
values making up each of the UTF-8 values (see \fref{tab:utf8-rule-values}).

\begin{table}
  \centering
  \begin{tabular}{|l|c|c|c|}
    \hline
    \textbf{Letter} & \textbf{UTF-8} & \textbf{Hex} & \textbf{Decimal} \\ \hline
    Pound sign (£)   & U+00A3 & 0xC2 0xA3 &  $194$ $163$ \\ \hline
    Section sign (§) & U+00A7 & 0xC2 0xA7 & $194$ $167$ \\ \hline
  \end{tabular}

  \caption{Table of UTF-8 and hex values of \texttt{£} and \texttt{§}}
  \label{tab:utf8-rule-values}
\end{table}

As we are using the internal SML-Lex position feature \texttt{yypos} we have to
decrement it by one each time a transformer or meta pattern is encountered as it
is counted as two characters, where it actually is just one (composite)
character.

\subsection{Abstract syntax tree}


%%% Local Variables: 
%%% mode: latex
%%% TeX-master: "../rewriting-syntax"
%%% End: 


\subsection{Resolvers}


\subsubsection{Infix resolving}


\subsubsection{Environment resolving}




%%% Local Variables: 
%%% mode: latex
%%% TeX-master: "../rewriting-syntax"
%%% End: 


\subsection{Rewriting rule syntax}

When specifying rewriting rules we 

\begin{nonfloatingfigure}

  \setlength{\grammarindent}{7.3em}
  \begin{grammar} 
    
    <rule-program> ::= <rule>$^{*}$
    
    <rule> ::= "rule" <rule-header> <scheme> "becomes" <clause>$^{+}$ "end"
    
    <rule-header> ::= <rule-type> <rule-name>
    
    <rule-type> ::= "clauses" 
    \alt "expression" 
    
    <rule-name> ::= <longid>
    
    <rule-self> ::= "self" <sexp>
    
    <scheme> ::= <clause>$^{+}$ [<cstrns>]
    
    <clause> ::= "|" <spat> "=>" <sexp>
    
    <cstrns> ::= "where" <cstrn-body> 
    
    <cstrn-body> ::= <cstrn-rel> 
    \alt <cstrn-rel> "," <cstrn-body>
    
    <cstrn-rel> ::= <longid> "(" <cstrn-rel-body>$^{+}$ ")"
    
    <cstrn-rel-body> ::= <spat>   
    
    <transformer> ::= "£" <longid> <sexp>

    <meta-pattern-sexp> ::= "§" <longid>  <sexp>$^{*}$
    
    <meta-pattern-spat> ::= "§" <longid>  <spat>$^{*}$

  \end{grammar}
  
  \caption{Complete rule grammar. See \fref{fig:scheme-expressions} for the
    \synt{sexp} grammar and \fref{fig:scheme-patterns} for the \synt{spat}
    grammar which uses the transformers and meta patterns}
  \label{fig:rule-grammar}
\end{nonfloatingfigure}


%%% Local Variables:
%%% mode: latex
%%% TeX-master: "../rewriting-syntax"
%%% End: 





%%% Local Variables: 
%%% mode: latex
%%% TeX-master: "../report"
%%% End: 


\chapter{...}
...


\chapter{Evaluation}
\label{chap:evaluation}

\chapter{Conclusions}

\bibliographystyle{bibliography/theseurl}
\bibliography{bibliography/bibliography}

\appendix

\chapter{Figures}
\section{Protocols}
\label{sec:protocols}


\subsection{JSON}
\label{sec:protocol-json}


\begin{nonfloatingfigure}


  \begin{grammar}
    <JSON-value> ::= \[[
    \begin{stack}
      <JSON-string> \\
      <JSON-number> \\
      <JSON-object> \\
      <JSON-array> \\
      "true" \\
      "false" \\
      "null"
    \end{stack}
    \]]

    <JSON-string> ::= \[[
    "\""
    \begin{stack}
      \\
      \fbox {
        \begin{minipage}{1,6in}
          \begin{center}
            Any\ unicode\ char\ except\ '{\char 34}',\ '$\backslash$'\
            and\ any\ control\ chars
          \end{center}
        \end{minipage}
      }
      \\
      "\\"
      \begin{stack}
        "\"" \\
        "\\" \\
        "/" \\
        "b" \\
        "f" \\
        "n" \\
        "r" \\
        "t" \\
        "u"
      \end{stack}
    \end{stack}
    "\""
    \]]

    <JSON-object> ::= \[[
    "{"
      \begin{stack}
        \\

        \begin{rep}
          <JSON-string> ":" <JSON-value> \\
          ","
        \end{rep}
      \end{stack}
      "}"
    \]]

    <JSON-array> ::= \[[
    "["
    \begin{stack}
      \\
      \begin{rep}
        <JSON-value> \\
        ","
      \end{rep}
    \end{stack}
    "]"
    \]]

    <JSON-number> ::= \[[
    \begin{stack}
      \\
      "-"
    \end{stack}
    \begin{stack}
      "0" \\
      1--9
      \begin{rep}
        \\
        0--9
      \end{rep}
    \end{stack}
    \begin{stack}
      \\
      "."
      \begin{rep}
        \\
        0--9
      \end{rep}
    \end{stack}
    \begin{stack}
      \\
      \begin{stack}
        e \\
        E
      \end{stack}
      \begin{stack}
        \\
        "+" \\
        "-"
      \end{stack}
      \begin{rep}
        \\
        0--9
      \end{rep}
    \end{stack}
    \]]

  \end{grammar}

  \caption{Definition of JSON as described 10th of January 2010 at
    \url{www.json.org} \cite{json}}
  \label{fig:protocol-json}
\end{nonfloatingfigure}


%%% Local Variables: 
%%% mode: latex
%%% TeX-master: "../../../report"
%%% reftex-fref-is-default: t
%%% End: 


\chapter{...}



\end{document}

%%% Local Variables: 
%%% mode: latex
%%% TeX-master: t
%%% reftex-fref-is-default: t
%%% End: 
