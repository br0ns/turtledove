\documentclass[a4paper, oneside, final]{memoir}
% Fixes "No room for a new \xxx" error by extending the default 256 fixed size
% LaTeX arrays
\usepackage{etex}
\reserveinserts{28}


\usepackage[T1]{fontenc}
\usepackage[utf8]{inputenc}
\usepackage[british]{babel}

% bedre orddeling Gør at der som minimum skal blive to tegn på linien ved
% orddeling og minimum flyttes to tegn ned på næste linie. Desværre er værdien
% anvendt af babel »12«, hvilket kan give orddelingen »h-vor«.
\renewcommand{\britishhyphenmins}{22} 

% Fix of fancyref to work with memoir. Makes references look
% nice. Redefines memoir \fref and \Fref to \refer and \Refer.
% \usepackage{refer}             %
% As we dont really have any use for \fref and \Fref we just undefine what
% memoir defined them as, so fancyref can define what it wants.
\let\fref\undefined
\let\Fref\undefined
\usepackage{fancyref} % Better reference. 

\usepackage{pdflscape} % Gør landscape-environmentet tilgængeligt
\usepackage{fixme}     % Indsæt "fixme" noter i drafts.
\usepackage{hyperref}  % Indsæter links (interne og eksterne) i PDF

\usepackage[rounded]{syntax} % Part of the mdwtools package

\usepackage{mdwtab}
\usepackage{mathenv}
\usepackage{amsfonts}
\usepackage{amsmath}
\usepackage{amssymb}
\usepackage{amsthm}
\usepackage{semantic} % for the \mathlig function

\usepackage[format=hang]{caption,subfig}
\usepackage{graphicx}
\usepackage{stmaryrd}
\usepackage{listings}
\usepackage{ulem} % \sout - strike-through
\usepackage{tikz}

\lstset{ %
% language=Octave,                % choose the language of the code
basicstyle=\ttfamily,        % the size of the fonts that are used for the code
basewidth=0.5em,
% numbers=left,                   % where to put the line-numbers
% numberstyle=\footnotesize,      % the size of the fonts that are used for the line-numbers
% stepnumber=2,                   % the step between two line-numbers. If it's 1 each line will be numbered
% numbersep=5pt,                  % how far the line-numbers are from the code
% backgroundcolor=\color{white},  % choose the background color. You must add \usepackage{color}
% showspaces=false,               % show spaces adding particular underscores
% showstringspaces=false,         % underline spaces within strings
% showtabs=false,                 % show tabs within strings adding particular underscores
% frame=single	                % adds a frame around the code
% tabsize=2,	                % sets default tabsize to 2 spaces
% captionpos=b,                   % sets the caption-position to bottom
% breaklines=true,                % sets automatic line breaking
% breakatwhitespace=false,        % sets if automatic breaks should only happen at whitespace
escapeinside={(@}{@)}          % if you want to add a comment within your code
}

\renewcommand{\ttdefault}{txtt} % Bedre typewriter font
%\usepackage[sc]{mathpazo}     % Palatino font
\renewcommand{\rmdefault}{ugm} % Garamond
%\usepackage[garamond]{mathdesign}

%\overfullrule=5pt
%\setsecnumdepth{part}
\setcounter{secnumdepth}{1} % Sæt overskriftsnummereringsdybde. Disable = -1.
\chapterstyle{hangnum} % changes style of chapters, to look nice.

\makeatletter
\newenvironment{nonfloatingfigure}{
  \vskip\intextsep
  \def\@captype{figure}
  }{
  \vskip\intextsep
}

\newenvironment{nonfloatingtable}{
  \vskip\intextsep
  \def\@captype{table}
  }{
  \vskip\intextsep
}
\makeatother

\renewcommand{\ttdefault}{txtt} % Bedre typewriter font
%% \usepackage[sc]{mathpazo}     % Palatino font
%% \renewcommand{\rmdefault}{ugm} % Garamond
%% \usepackage[garamond]{mathdesign}

% \overfullrule=5pt
% \setsecnumdepth{part}
\setcounter{secnumdepth}{1} % Sæt overskriftsnummereringsdybde. Disable = -1.
\chapterstyle{hangnum} % changes style of chapters, to look nice.

\theoremstyle{definition}
\newtheorem{judgment}{Judgment}
\newtheorem{definition}{Definition}
\newtheorem{lemma}{Lemma}
\newtheorem{theorem}{Theorem}
\newtheorem{corollary}{Corollary}
\newtheorem{example}{Example}

\newcommand*{\fancyrefdeflabelprefix}{def}
\fancyrefaddcaptions{english}{
  \newcommand*{\Frefdefname}{Definition}
  \newcommand*{\frefdefname}{\MakeLowercase{\Frefdefname}}
}
\frefformat{vario}{\fancyrefdeflabelprefix}{%
  \frefdefname\fancyrefdefaultspacing#1#3%
}
\Frefformat{vario}{\fancyrefdeflabelprefix}{%
  \Frefdefname\fancyrefdefaultspacing#1#3%
}

\newcommand*{\fancyreflemlabelprefix}{lem}
\fancyrefaddcaptions{english}{
  \newcommand*{\Freflemname}{Lemma}
  \newcommand*{\freflemname}{\MakeLowercase{\Freflemname}}
}
\frefformat{vario}{\fancyreflemlabelprefix}{%
  \freflemname\fancyrefdefaultspacing#1#3%
}
\Frefformat{vario}{\fancyreflemlabelprefix}{%
  \Freflemname\fancyrefdefaultspacing#1#3%
}
\frefformat{plain}{\fancyreflemlabelprefix}{%
  \freflemname\fancyrefdefaultspacing#1%
}
\Frefformat{plain}{\fancyreflemlabelprefix}{%
  \Freflemname\fancyrefdefaultspacing#1%
}

\newcommand*{\fancyrefthmlabelprefix}{thm}
\fancyrefaddcaptions{english}{
  \newcommand*{\Frefthmname}{Theorem}
  \newcommand*{\frefthmname}{\MakeLowercase{\Frefthmname}}
}
\frefformat{vario}{\fancyrefthmlabelprefix}{%
  \frefthmname\fancyrefdefaultspacing#1#3%
}
\Frefformat{vario}{\fancyrefthmlabelprefix}{%
  \Frefthmname\fancyrefdefaultspacing#1#3%
}

\newcommand*{\fancyrefcorlabelprefix}{cor}
\fancyrefaddcaptions{english}{
  \newcommand*{\Frefcorname}{Corollary}
  \newcommand*{\frefcorname}{\MakeLowercase{\Frefcorname}}
}
\frefformat{vario}{\fancyrefcorlabelprefix}{%
  \frefcorname\fancyrefdefaultspacing#1#3%
}
\Frefformat{vario}{\fancyrefcorlabelprefix}{%
  \Frefcorname\fancyrefdefaultspacing#1#3%
}

\newcommand*{\fancyrefexlabelprefix}{ex}
\fancyrefaddcaptions{english}{
  \newcommand*{\Frefexname}{Example}
  \newcommand*{\frefexname}{\MakeLowercase{\Frefexname}}
}
\frefformat{vario}{\fancyrefexlabelprefix}{%
  \frefexname\fancyrefdefaultspacing#1#3%
}
\Frefformat{vario}{\fancyrefexlabelprefix}{%
  \Frefexname\fancyrefdefaultspacing#1#3%
}

\newcommand{\ttt}[1]{\texttt{#1}}
\newcommand{\tnm}[1]{\textnormal{#1}}
\newcommand{\mrm}[1]{\mathrm{#1}}

\newcommand{\Cov}{\mathrm{Cov}}
\providecommand{\FV}{\mathrm{FV}}
\providecommand{\Dom}{\mathrm{Dom}}


\mathlig{||}{\parallel}
\mathlig{<'}{\prec}
\mathlig{>'}{\succ}
\mathlig{<='}{\preccurlyeq}
\mathlig{>='}{\succcurlyeq}
\mathlig{<=}{\leqslant}
\mathlig{>=}{\geqslant}
\mathlig{<>}{\neq}
\mathlig{|=}{\sqsubset}
\mathlig{=|}{\sqsupset}
\mathlig{==}{\equiv}
\mathlig{==a}{=_{\alpha}}
\mathlig{<|}{\lhd}
\mathlig{|>}{\rhd}
\mathlig{++}{\mathrel{\mbox{+\!\!\!+}}}
\mathlig{~>e}{\stackrel{elim}{\leadsto}}
\mathlig{~>g}{\stackrel{gen}{\leadsto}}

%%%%%%%%%%%%%%%%%%%%%%%%%%%%%%%%%%%%%%%%%%%%%%%%%%%%%%%%
%	    	     Forside
%%%%%%%%%%%%%%%%%%%%%%%%%%%%%%%%%%%%%%%%%%%%%%%%%%%%%%%%
\makeatletter % open mode for reading @ signed variables 
\def\maketitle{%
  \null
  \thispagestyle{empty}%
  \vfill
  \begin{center}\leavevmode
    \normalfont
    \Huge{\raggedleft \@title\par}%
    \hrulefill\par
    \Large{\raggedright \subtitle\par}%
    \vskip 2cm
    {\@date\par}%
  \end{center}%
  \vfill
\begin{minipage}{80pt}
\includegraphics*[scale=0.75]{imgs/nat-logo}
\end{minipage}
\begin{minipage}{300pt}
  \begin{flushleft}
    {\large \@author } \\
    {\footnotesize \suplementInfo }

  \end{flushleft}
\end{minipage}
\cleardoublepage % lave 1 ekstre side blank efter
  \clearpage % Terminates the page here. Everything else vil be placed on next page.
}
\makeatother % closing mode for reading @ signed variables
%%%%%%%%%%%%%%%%%%%%%%%%%%%%%%%%%%%%%%%%%%%%%%%%%%%%%%%%
%		Data til forside
%%%%%%%%%%%%%%%%%%%%%%%%%%%%%%%%%%%%%%%%%%%%%%%%%%%%%%%%
\title{Turtledove: Tool assisted programming in SML, with special emphasis on semi-automatic rewriting to
predefined standard forms.}
\def\subtitle{\footnotesize{A joined Smartypants inc. and Morning Wood Productions venture.}}
\author{Morten Brøns-Pedersen {\footnotesize{(mortenbp@gmail.com)}}\\  
Jesper Reenberg \footnotesize{(jesper.reenberg@gmail.com)}}
\def\suplementInfo{
  \kern 5pt \hrule width 11pc \kern 5pt % putter 5pt spacing oven over og neden under stregen
  Dept. of Computer Science \\
  University of Copenhagen}
% \date{} % used to set explicit dates

\pagestyle{plain}

\begin{document}

\frontmatter

\maketitle
\thispagestyle{empty}

\begin{abstract}
\end{abstract}

\clearpage 
\chapter*{Preface}
This report is a 22.5 ECTS masters project at the Department of Computer Science
at the University of Copenhagen (DIKU). The authors are Morten Brøns-Pedersen
and Jesper Reenberg. The project is supervised by Jakob Grue Simonsen, assistant
professor at DIKU.

\clearpage

\tableofcontents*

\mainmatter

\chapter{Introduction}

\section{Problem statement}
\label{sec:problem-statement}

\section{Motivation}
\label{sec:motivation}

\section{Reader expectations}

\section{Structure outline}



\chapter{Design}
\label{cha:design}

\section*{Intercommunication}


\subsection*{Protocol}


The communication protocol between the development environment and Turtledove is
very simple and versatile with almost no limitations. The protocol utilises JSON
(JavaScript Object Notation) \footnote{\url{http://www.json.org/}} for the data
payload which makes it easy to serialise different types of values to and from
the desired tools inside Turtledove.

Besides the data payload the protocol consists of a ``CallerID'' string and a
``DestinationID'' string which is described below

\begin{description}
\item[CallerID] is a string specified by the development environment. This
  string is not used in any way by Turtledove but is returned in the response
  string. This string is intended for internal bookkeeping in the development
  environment for example to distinguish which file/buffer and/or at which line
  and column the response originated from.

  As there is a possibility of Turtledove getting multi threaded, this is a good
  way of handling multiple request to Turtledove that may have different
  response times (One request waiting for a new complete reparse of the project
  code and another requesting a static lookup of some data).

  If the development environment doesn't need the use of this string an empty
  string can be passed to Turtledove.

  Some of the tools may report back when they are done doing something, without
  the development environment has invoked it. In these cases an empty string
  will be used as ``CallerID''.

\item[DestinationID] is a string that needs to match a named destination in
  Turtledove. Examples of such a named destination, also including name of
  tools:

  \begin{itemize}
  \item ``Turtledove'': This is for communication with the main program. This
    includes enabling/disabling of individual tools, status information and
    closing down the server gracefully.

  \item ``ProjectManager'': This is for communication with the project
    manager. This include actions such as modifying text, creation/deletion of
    files and projects.
  \end{itemize}
\end{description}


\begin{nonfloatingfigure}
  \begin{grammar}
    <request> ::= \[[ "CallerID" "\\n" "DestinationID" "\\n" <JSON-value> "\\n\\n" \]]
    
    <response> ::= \[[ "DestinationID" "\\n" "CallerID" "\\n" <JSON-value> "\\n\\n" \]]
  \end{grammar}
  
  \caption{Definition of the communication protocol between Turtledove (and its
    tools) and the editor.}
  \label{fig:intercom-protocol-definition}

\end{nonfloatingfigure}


It is important to remember that the resulting string sent to and from
Turtledove must not contain any newlines except the ones that are mandatory by
the protocol. This is a requirement even though the definition of JSON allows
whitespace between a pair of tokens, as it would then render it impossible to
determine when the request/response is done.

We have chosen the newline character as a separator of the fields as it
makes it easy for debug pretty printing and since a newline character must be
escaped i a name/id (and besides it is unlikely to be used here anyway).

The same argument applies to the decision of using double newline chars to mark
the end of a request or response. 


\paragraph{Examples}



\begin{example} If a file needs to be added then the following string could be
  sent to the ``Projectmanager'':
\begin{verbatim}
"some_id \n ProjectManager \n {\"command\" : \"addfile\", 
                                  \"file\" : \"myfoo.sml\"} \n\n"
\end{verbatim}
\end{example}


\subsection*{Useful commands}

 Below is a draft of which commands might be useful to implement in the final system. 

\subsubsection*{Turtledove}

\begin{description}
\item[quit] This command closes down the server in a graceful way.
\end{description}

\subsubsection*{ProjectManager }

\begin{description}
\item[OpenProject] Makes the project manager open a project. If the
  project doesn't exists, then a new project with the name is created.

  If another project is currently opened, then this project is closed, if told
  to do so.
\item[CloseProject] makes the project manager close the currently opened project,
  saving any unsaved changes if told to do so. 
\item[DeleteProject] Makes the project manager delete the currently opened project
  with all its files.
\item[AddFile] Adds a new file to the project.
\item[DeleteFile] Deletes a file contained in the project
\item[ChangeFile] Add or delete content in a open file contained in the
  project. This command should be used when ever the user adds or deletes text
  from any file contained in the project, so the project manager can maintain an
  up to date version of the file, without having to reread the file again and
  thus also at some point do incremental parsing of the changes only.
\end{description}


\paragraph{Get names in scope}

Returns a list of identifiers visible from the current scope. This is particular
handy for doing auto completion and 


\paragraph{Goto definition}

Gets the filename and position of the definition of the specified function

\paragraph{Rename}

Rename a function or identifier together with its definition and all its uses.


%%% Local Variables: 
%%% mode: latex
%%% reftex-fref-is-default: t
%%% TeX-master: "../../report"
%%% End: 


\section*{Mylib}

The idea behind this base library is to implement the most basic data structures
such as dictionaries, sets, trees and etc. 

\subsection*{Dictionarie}

\subsection*{Set}

\subsection*{Map}

\subsection*{Tree}

% In the design we can talk about the below mentioned two types.
% But in the implementation we just say that we have implemented the later.

There is a need for two different kind of trees.

\begin{enumerate}
\item Ordinary trees, binary as n-ary

\item Trees with enhanced possibility of walking around the nodes by getting
  a list of the children or getting the parent node.

This will be a necessary feature when representing the syntax trees.
\end{enumerate}




%%% Local Variables: 
%%% mode: latex
%%% TeX-master: "../../report"
%%% End: 


\documentclass[a4paper, oneside, final]{memoir}
% Fixes "No room for a new \xxx" error by extending the default 256 fixed size
% LaTeX arrays
\usepackage{etex}
\reserveinserts{28}


\usepackage[T1]{fontenc}
\usepackage[utf8]{inputenc}
\usepackage[british]{babel}

% bedre orddeling Gør at der som minimum skal blive to tegn på linien ved
% orddeling og minimum flyttes to tegn ned på næste linie. Desværre er værdien
% anvendt af babel »12«, hvilket kan give orddelingen »h-vor«.
\renewcommand{\britishhyphenmins}{22} 

% Fix of fancyref to work with memoir. Makes references look
% nice. Redefines memoir \fref and \Fref to \refer and \Refer.
% \usepackage{refer}             %
% As we dont really have any use for \fref and \Fref we just undefine what
% memoir defined them as, so fancyref can define what it wants.
\let\fref\undefined
\let\Fref\undefined
\usepackage{fancyref} % Better reference. 

\usepackage{pdflscape} % Gør landscape-environmentet tilgængeligt
%\usepackage[draft]{fixme}     % Indsæt "fixme" noter i drafts.
\usepackage{hyperref}  % Indsæter links (interne og eksterne) i PDF

\usepackage[rounded]{syntax} % Part of the mdwtools package


\usepackage{mdwtab}
\usepackage{mathenv}
\usepackage{amsfonts}
\usepackage{amsmath}
\usepackage{amssymb}
\usepackage{amsthm}
\usepackage{semantic} % for the \mathlig function

\usepackage{multirow} % seems to work nicely with mdwtab

\usepackage[usenames,dvipsnames]{color}

\usepackage[format=hang]{caption,subfig}
\usepackage{graphicx}
\usepackage{stmaryrd}
\usepackage[final]{listings} % Make sure we show the listing even though we are
                             % making a final report.
\usepackage{ulem} % \sout - strike-through
\usepackage{tikz}

\usepackage{multicol}

% Extend figures into either left or right margin
% Ex: \begin{narrow}{-1in}{0in} .. \end{narrow} will place 1in into left margin
\newenvironment{narrow}[2]{%
  \begin{list}{}{%
  \setlength{\topsep}{0pt}%
  \setlength{\leftmargin}{#1}%
  \setlength{\rightmargin}{#2}%
  \setlength{\listparindent}{\parindent}%
  \setlength{\itemindent}{\parindent}%
  \setlength{\parsep}{\parskip}}%
\item[]}{\end{list}}

\lstset{ %
% language=Octave,                % choose the language of the code
basicstyle=\ttfamily,        % the size of the fonts that are used for the code
basewidth=0.5em,
% numbers=left,                   % where to put the line-numbers
% numberstyle=\footnotesize,      % the size of the fonts that are used for the line-numbers
% stepnumber=2,                   % the step between two line-numbers. If it's 1 each line will be numbered
% numbersep=5pt,                  % how far the line-numbers are from the code
% backgroundcolor=\color{white},  % choose the background color. You must add \usepackage{color}
% showspaces=false,               % show spaces adding particular underscores
% showstringspaces=false,         % underline spaces within strings
% showtabs=false,                 % show tabs within strings adding particular underscores
% frame=single	                % adds a frame around the code
% tabsize=2,	                % sets default tabsize to 2 spaces
% captionpos=b,                   % sets the caption-position to bottom
% breaklines=true,                % sets automatic line breaking
% breakatwhitespace=false,        % sets if automatic breaks should only happen at whitespace
escapeinside={(@}{@)}          % if you want to add a comment within your code
}

\lstnewenvironment{sml}
{\lstset{xleftmargin=3em}} % starting code, ex a lstset that is more specific;
{} % ending code

\newcommand{\smlinline}[1]{\lstinline{#1}}
\newcommand{\mathsml}[1]{\textnormal{\lstinline{#1}}}

\newcommand{\floor}[1]{\ensuremath{\lfloor #1 \rfloor}}
\newcommand{\codeinline}[1]{\texttt{#1}}

\renewcommand{\ttdefault}{txtt} % Bedre typewriter font
%\usepackage[sc]{mathpazo}     % Palatino font
\renewcommand{\rmdefault}{ugm} % Garamond
%\usepackage[garamond]{mathdesign}

%\overfullrule=5pt
\setsecnumdepth{subsection}
\chapterstyle{hangnum} % changes style of chapters, to look nice.

\makeatletter
\newenvironment{nonfloatingfigure}{
  \vskip\intextsep
  \def\@captype{figure}
  }{
  \vskip\intextsep
}

\newenvironment{nonfloatingtable}{
  \vskip\intextsep
  \def\@captype{table}
  }{
  \vskip\intextsep
}
\makeatother


% \overfullrule=5pt
% \setsecnumdepth{part}
\setcounter{secnumdepth}{1} % Sæt overskriftsnummereringsdybde. Disable = -1.
\chapterstyle{hangnum} % changes style of chapters, to look nice.

\theoremstyle{definition}
\newtheorem{judgment}{Judgment}
\newtheorem{definition}{Definition}
\newtheorem{lemma}{Lemma}
\newtheorem{theorem}{Theorem}
\newtheorem{corollary}{Corollary}
\newtheorem{example}{Example}
\newtheorem{trace}{Trace}

\frefformat{plain}{\fancyrefchaplabelprefix}{%
  \frefchapname\fancyrefdefaultspacing#1%
}
\Frefformat{plain}{\fancyrefchaplabelprefix}{%
  \Frefchapname\fancyrefdefaultspacing#1%
}

\newcommand*{\fancyrefdeflabelprefix}{def}
\fancyrefaddcaptions{english}{
  \newcommand*{\Frefdefname}{Definition}
  \newcommand*{\frefdefname}{\MakeLowercase{\Frefdefname}}
}
\frefformat{vario}{\fancyrefdeflabelprefix}{%
  \frefdefname\fancyrefdefaultspacing#1#3%
}
\Frefformat{vario}{\fancyrefdeflabelprefix}{%
  \Frefdefname\fancyrefdefaultspacing#1#3%
}
\frefformat{plain}{\fancyrefdeflabelprefix}{%
  \frefdefname\fancyrefdefaultspacing#1%
}
\Frefformat{plain}{\fancyrefdeflabelprefix}{%
  \Frefdefname\fancyrefdefaultspacing#1%
}

\newcommand*{\fancyreflemlabelprefix}{lem}
\fancyrefaddcaptions{english}{
  \newcommand*{\Freflemname}{Lemma}
  \newcommand*{\freflemname}{\MakeLowercase{\Freflemname}}
}
\frefformat{vario}{\fancyreflemlabelprefix}{%
  \freflemname\fancyrefdefaultspacing#1#3%
}
\Frefformat{vario}{\fancyreflemlabelprefix}{%
  \Freflemname\fancyrefdefaultspacing#1#3%
}
\frefformat{plain}{\fancyreflemlabelprefix}{%
  \freflemname\fancyrefdefaultspacing#1%
}
\Frefformat{plain}{\fancyreflemlabelprefix}{%
  \Freflemname\fancyrefdefaultspacing#1%
}

\newcommand*{\fancyrefthmlabelprefix}{thm}
\fancyrefaddcaptions{english}{
  \newcommand*{\Frefthmname}{Theorem}
  \newcommand*{\frefthmname}{\MakeLowercase{\Frefthmname}}
}
\frefformat{vario}{\fancyrefthmlabelprefix}{%
  \frefthmname\fancyrefdefaultspacing#1#3%
}
\Frefformat{vario}{\fancyrefthmlabelprefix}{%
  \Frefthmname\fancyrefdefaultspacing#1#3%
}

\newcommand*{\fancyrefcorlabelprefix}{cor}
\fancyrefaddcaptions{english}{
  \newcommand*{\Frefcorname}{Corollary}
  \newcommand*{\frefcorname}{\MakeLowercase{\Frefcorname}}
}
\frefformat{vario}{\fancyrefcorlabelprefix}{%
  \frefcorname\fancyrefdefaultspacing#1#3%
}
\Frefformat{vario}{\fancyrefcorlabelprefix}{%
  \Frefcorname\fancyrefdefaultspacing#1#3%
}

\newcommand*{\fancyrefexlabelprefix}{ex}
\fancyrefaddcaptions{english}{
  \newcommand*{\Frefexname}{Example}
  \newcommand*{\frefexname}{\MakeLowercase{\Frefexname}}
}
\frefformat{vario}{\fancyrefexlabelprefix}{%
  \frefexname\fancyrefdefaultspacing#1#3%
}
\Frefformat{vario}{\fancyrefexlabelprefix}{%
  \Frefexname\fancyrefdefaultspacing#1#3%
}
\frefformat{plain}{\fancyrefexlabelprefix}{%
  \frefexname\fancyrefdefaultspacing#1%
}
\Frefformat{plain}{\fancyrefexlabelprefix}{%
  \Frefexname\fancyrefdefaultspacing#1%
}

\newcommand*{\fancyreftrlabelprefix}{tr}
\fancyrefaddcaptions{english}{
  \newcommand*{\Freftrname}{Trace}
  \newcommand*{\freftrname}{\MakeLowercase{\Freftrname}}
}
\frefformat{vario}{\fancyreftrlabelprefix}{%
  \freftrname\fancyrefdefaultspacing#1#3%
}
\Frefformat{vario}{\fancyreftrlabelprefix}{%
  \Freftrname\fancyrefdefaultspacing#1#3%
}
\frefformat{plain}{\fancyreftrlabelprefix}{%
  \freftrname\fancyrefdefaultspacing#1%
}
\Frefformat{plain}{\fancyreftrlabelprefix}{%
  \Freftrname\fancyrefdefaultspacing#1%
}

\newcommand{\ttt}[1]{\texttt{#1}}
\newcommand{\tnm}[1]{\textnormal{#1}}
\newcommand{\tsf}[1]{\textsf{#1}}
\newcommand{\mrm}[1]{\mathrm{#1}}
\newcommand{\ol}[1]{\overline{#1}}

\newcommand{\Cov}{\mathrm{Cov}}
\providecommand{\FV}{\mathrm{FV}}
\providecommand{\Dom}{\mathrm{Dom}}


\mathlig{||}{\parallel}
\mathlig{<'}{\prec}
\mathlig{>'}{\succ}
\mathlig{<='}{\preccurlyeq}
\mathlig{>='}{\succcurlyeq}
\mathlig{<=}{\leqslant}
\mathlig{>=}{\geqslant}
\mathlig{<>}{\neq}
\mathlig{|=}{\sqsubset}
\mathlig{=|}{\sqsupset}
\mathlig{==}{\equiv}
\mathlig{==a}{=_{\alpha}}
\mathlig{<|}{\lhd}
\mathlig{|>}{\rhd}
\mathlig{++}{\mathrel{\mbox{+\!\!\!+}}}
% ~>e or ~>g conflicts with the \cite command for some reason.
\mathlig{->e}{\stackrel{elim}{\leadsto}}
\mathlig{->g}{\stackrel{gen}{\leadsto}}
\mathlig{++}{\mathrel{\mbox{+\!\!\!+}}}

%%%%%%%%%%%%%%%%%%%%%%%%%%%%%%%%%%%%%%%%%%%%%%%%%%%%%%%%
%	    	     Forside
%%%%%%%%%%%%%%%%%%%%%%%%%%%%%%%%%%%%%%%%%%%%%%%%%%%%%%%%
\makeatletter % open mode for reading @ signed variables 
\def\maketitle{%
  \null
  \thispagestyle{empty}%
  \vfill
  \begin{center}\leavevmode
    \normalfont
    \Huge{\raggedleft \@title\par}%
    \small{\raggedright \hspace{15pt}\raisebox{-0.7em}{\@date}\par}%
    \hrulefill\par
    \Large{\raggedright \subtitle\par}%
  \end{center}%
  \vskip 3em
  \begin{abstract}
    \textit{\descript}
  \end{abstract}
  \vfill
\begin{minipage}{80pt}
\includegraphics*[scale=0.75]{imgs/nat-logo}
\end{minipage}
\begin{minipage}{300pt}
  \begin{flushleft}
    {\large \@author } \\
    {\footnotesize \suplementInfo }

  \end{flushleft}
\end{minipage}
\cleardoublepage % lave 1 ekstre side blank efter
  \clearpage % Terminates the page here. Everything else vil be placed on next page.
}
\makeatother % closing mode for reading @ signed variables
%%%%%%%%%%%%%%%%%%%%%%%%%%%%%%%%%%%%%%%%%%%%%%%%%%%%%%%%
%		Data til forside
%%%%%%%%%%%%%%%%%%%%%%%%%%%%%%%%%%%%%%%%%%%%%%%%%%%%%%%%
\title{{\color{White}Lipstick on a Bulldog aka. }Turtledove}
\def\subtitle{\footnotesize{Tool assisted programming in SML, with special
    emphasis on semi-automatic rewriting to predefined standard
    forms. {\color{White}In some sense this is exactly ``lipstick on a bulldog''
      --- we might be able to make the code look pretty but it's still broken
      nonetheless.}}}
\author{Morten Brøns-Pedersen {\footnotesize{(mortenbp@gmail.com)}}\\
  Jesper Reenberg \footnotesize{(jesper.reenberg@gmail.com)}}
\def\suplementInfo{ \kern 5pt \hrule width 11pc \kern
  5pt % putter 5pt spacing oven over og neden under stregen
  Dept. of Computer Science \\
  University of Copenhagen} \date{21st January
  2011} % used to set explicit dates

\def\descript{%
  We define a normal form for SML functions and implement a program that can
  convert functions to their normal form. Building on the normal form we then
  give a DSL for describing rewriting rules among whole functions in SML, along
  with its semantics and examples of use.

  A small suite of rewriting rules is developed and tested on a body of code
  written by ourselves and freshman students during an introductory programming
  course.

  We implement a full parser for SML and MLB and we give a novel internal
  representation of syntax trees that we find to work very well in
  practice. Furthermore we have implemented a general purpose library for
  programming in SML.  }

\pagestyle{plain}

\begin{document}

\frontmatter

\maketitle
\thispagestyle{empty}

\clearpage
\chapter*{Preface}
This report and the work described within it is a 22.5 ECTS masters project at
the Department of Computer Science, University of Copenhagen (DIKU). The authors
are Morten Brøns-Pedersen and Jesper Reenberg.\\[1em]
The project was suggested by Jakob Grue Simonsen, associate professor at DIKU,
who also supervised it.
\clearpage

\setcounter{tocdepth}{2}
\tableofcontents*

\mainmatter


\section{Problem Statement}

Huge explanation about the actual problem

\section{Motivation}

Huge explanation about why we want to fix the problem

\section{Reader expectations}


We expect that readers of this report to be able to read and understand SML
(Standard Meta Language) and in general has functional programming experience,
mathematical maturity and notion of lambda calculus. 

%\section{Structure outline}




%%% Local Variables: 
%%% mode: latex
%%% TeX-master: "../miniml"
%%% End: 


%%%%%%%%%%%%%%%%%%%%%%%%%%%%%%%%%%%%%%%%%%%%%%%%
\chapter{SML normalform}

\input{sml-normalform/rewriting}

\input{sml-normalform/examples}


%%% Local Variables:
%%% mode: latex
%%% TeX-master: "../report"
%%% End:


%%%%%%%%%%%%%%%%%%%%%%%%%%%%%%%%%%%%%%%%%%%%%%%%
\chapter{Rewriting rules}
In this chapter we define a DSL for describing SML rewriting rules. We also give
a system of inference rules for carrying out the rewritings. We assume that
matches has already been rewritten to a normal form (see \fref{chap:normal-form})
before a rewriting rule is applied.

\paragraph{Note.} This chapter was written at a much later time than
\fref[plain]{chap:normal-form} which is why they differ so much in style.

\input{rewriting-rules/definitions}

\input{rewriting-rules/semantics}

\input{rewriting-rules/intuition}


%%% Local Variables: 
%%% mode: latex
%%% TeX-master: "../report"
%%% End: 


%%%%%%%%%%%%%%%%%%%%%%%%%%%%%%%%%%%%%%%%%%%%%%%%
\chapter{Concrete examples}
\label{chap:concrete-examples}
In this chapter we present rewriting rules for \ttt{map}, \ttt{foldl} and
\ttt{foldr} forms. We then show the rules in action on a body of example code.

We have chosen rules concerning list combinators since they turn up very
often. The \ttt{map} form was chosen for its simplicity whereas the
\ttt{fold$($l$|$r$)$} forms was chosen for their complexity; it is our
experience as teaching assistants that students often struggle with these forms.

Other forms concerning list combinators that turn up often are \ttt{exists} and
\ttt{all} forms. But we have not made rules for these.

\input{rewriting-examples/rules}

\section{Examples}
Assignments and exams in the following are taken from the course ``Introduction
to Programming'' taught at the Department of Computer Science, University of
Copenhagen (DIKU).

When determining $\sigma$ and the $\theta$'s we do not distinguish sharply
between the two as the typesetting of the variables makes it clear what belongs
where.

\input{rewriting-examples/map}

\input{rewriting-examples/fold}

%%% Local Variables: 
%%% mode: latex
%%% TeX-master: "../report"
%%% End: 


%%%%%%%%%%%%%%%%%%%%%%%%%%%%%%%%%%%%%%%%%%%%%%%%

\chapter{Design}
\label{cha:design}

\input{design-decisions/intercommunication/intercommunication}

\input{design-decisions/mylib/mylib}

\input{design-decisions/report/report}

%%% Local Variables: 
%%% mode: latex
%%% TeX-master: "../report"
%%% End: 



%%%%%%%%%%%%%%%%%%%%%%%%%%%%%%%%%%%%%%%%%%%%%%%%
\chapter{Implementation}

\input{implementation/parsers}


\input{implementation/resolvers}


%%% Local Variables: 
%%% mode: latex
%%% TeX-master: "../report"
%%% End: 



%%%%%%%%%%%%%%%%%%%%%%%%%%%%%%%%%%%%%%%%%%%%%%%%
\input{using-turtledove/using-turtledove}

%%%%%%%%%%%%%%%%%%%%%%%%%%%%%%%%%%%%%%%%%%%%%%%%
\chapter{Evaluation}
\label{chap:evaluation}

\fixme{What is presented in this chapter?}

\newcommand{\cmt}[1]{\textcolor{Red}{>>#1}}

The console traces presented in \fref[plain]{sec:eval-normal-form} and
\fref[plain]{sec:eval-map-rewriting} are typeset in columns for brevity and
contains explanatory comments, that are not part of the original output, written
as ``\cmt{This is a comment}''

\input{evaluation/normalform}

\input{evaluation/maprewrite}

\input{evaluation/furtherwork}


%%% Local Variables: 
%%% mode: latex
%%% TeX-master: "../report"
%%% End: 



%%%%%%%%%%%%%%%%%%%%%%%%%%%%%%%%%%%%%%%%%%%%%%%%
\chapter{Conclusion}


\fixme[inline=true,margin=false]{Was it a good idea with the AST, only having one data type instead of
  many different}

\fixme[inline=true,margin=false]{Something about the issues with the map rule,
  that we need two of them.}

%%% Local Variables: 
%%% mode: latex
%%% TeX-master: "../report"
%%% End: 



\bibliographystyle{bibliography/theseurl}
\bibliography{bibliography/bibliography}

\appendix


%%%%%%%%%%%%%%%%%%%%%%%%%%%%%%%%%%%%%%%%%%%%%%%%
\input{appendix/figures/figures}

\input{appendix/examples/examples}

\input{appendix/grammar/grammar}

\input{appendix/rewriting-rules}

\input{appendix/sml-normalform/sml-normalform}

\input{appendix/implementation/implementation}

\input{appendix/traces/trace}

%%% Local Variables: 
%%% mode: latex
%%% TeX-master: "../report"
%%% End: 



\end{document}

%%% Local Variables: 
%%% mode: latex
%%% TeX-master: t
%%% reftex-fref-is-default: t
%%% End: 


%%% Local Variables: 
%%% mode: latex
%%% TeX-master: "../report"
%%% End: 


\chapter{Implementation}


\section{Parsers}

The SML and Rule parsers are both implemented in SML-Lex and SML-Yacc. We say
SML-Lex/Yacc as both the sml/nj version ml-lex/ml-yacc and mlton version
mllex/mlyacc works, but the mosml version mosmllex and mosmlyac are having
problems.

\fixme{fix the above gibberish, add something about the MLB}

\subsection{ML Basis files}

The MLB parser parses the files in two steps, First is a regular parse and the
second parse pulls out information about which files were included, etc. so any
MLB files referenced may also be parsed and their AST inserted.  ... bla blah

\fixme{Write this section}

\subsection{Standard ML}

A fine note about where the grammar was stolen from and possibly some other
clever stuff.

\fixme {Write this section}

\subsection{Rule}

The Rule parser is based on the SML parser with some added tokens to the lexer
and rules to the grammar to handle the rule syntax (see
\fref{tab:rule-grammar}). 


\subsubsection{Unicode}

The rule syntax uses the symbols \texttt{£} and \texttt{§} to denote
transformers and meta patterns respectively. These were a bit tricky to
implement as they are not ASCII characters. By default SML-Lex\cite{ml-lex-yacc}
only supports 8-bit characters and thus if the upper parts of UTF-8 characters
are needed they must be specially handled. One way to handle it would be to
convert the input file into some fixed length encoding (i.e., UTF-32), but as
SML only uses ASCII chars that would demand a total remake of the Rule lexer
definition. Instead we went with a UTF-8 solution which only need to be
specially fitted for the transformer and meta patterns as it seems that most
editors now a days use UTF-8 as default encoding (seems to be the case for emacs
on linux). Implementing the two UTF-8 values in the lexer was then just a matter
of defining a named expression containing the conjunction of the two decimal
values making up each of the UTF-8 values (see \fref{tab:utf8-rule-values}).

\begin{table}
  \centering
  \begin{tabular}{|l|c|c|c|}
    \hline
    \textbf{Letter} & \textbf{UTF-8} & \textbf{Hex} & \textbf{Decimal} \\ \hline
    Pound sign (£)   & U+00A3 & 0xC2 0xA3 &  $194$ $163$ \\ \hline
    Section sign (§) & U+00A7 & 0xC2 0xA7 & $194$ $167$ \\ \hline
  \end{tabular}

  \caption{Table of UTF-8 and hex values of \texttt{£} and \texttt{§}}
  \label{tab:utf8-rule-values}
\end{table}

As we are using the internal SML-Lex position feature \texttt{yypos} we have to
decrement it by one each time a transformer or meta pattern is encountered as it
is counted as two characters, where it actually is just one (composite)
character.

\subsection{Abstract syntax tree}

It seems that there exists one common way of representing an AST; Group
different parts of the language together in different data types, for example
patterns and expressions. In general it seems that the decision of how to break
up the language into different data types are based heavily on the BNF. This way
of representing the AST gives integrity safety for free, as for example an
\synt{exp} would not be able to contain an \synt{atpat}, which is most likely
the main motivation of doing it this way. However various examples are given in
\cite{mbp08} of how this way of representing the AST blows up code
vise when working with it as lots of almost identical functions are needed to
work on all the different data types making up the AST. Because of this we have
chosen a dramatical new approach\footnote{Compared to how all the SML
  interpresers/compilers such as SML/NJ, MLKit and MLton} as to how the AST is
represented; We have chosen to represent it using only one data type. This
doesn't give any gives integrity but it gives a great deal of flexibility
(``code by convention''). Such an integrity check seems to be a small ``price to
pay'' compared to the extra flexibility and what seems to be less code to work
with the AST.


This way of thinking generalised even further. Both the SML parser and Rule
parser uses the same AST, as this will make it much easier when comparing actual
SML code to a scheme in the defined rules.


\subsubsection{wrap}

Actually the AST is not just a n-ary tree structure of the above mentioned
single data type. Each node in the AST is a \textit{wrap} of a token (the data
type) and a left and right field both of type $\alpha$. These two fields, left
and right, are polymorph such that any manipulations to the AST can transform it
from some $\alpha$ to some $\beta$. 

Initially, out of the parser, the left and right fields are integer character
positions of left- and rightmost part of the text representing the token
including any children. This is later changed in the infix resolving to also
contain infix status and environment information, where the environment
information is that just before, in the left part, and the information after
this node, in the right part.

\fixme{give a reference to where all this is explained in detail}




%%% Local Variables: 
%%% mode: latex
%%% TeX-master: "../rewriting-syntax"
%%% End: 





\subsection{Parsers}


%%% Local Variables: 
%%% mode: latex
%%% TeX-master: "../rewriting-syntax"
%%% End: 



%%% Local Variables: 
%%% mode: latex
%%% TeX-master: "../report"
%%% End: 


\chapter{...}
...


\chapter{Evaluation}
\label{chap:evaluation}

\chapter{Conclusions}

\bibliographystyle{bibliography/theseurl}
\bibliography{bibliography/bibliography}

\appendix

\chapter{Figures}
\section{Protocols}
\label{sec:protocols}


\subsection{JSON}
\label{sec:protocol-json}


\begin{nonfloatingfigure}


  \begin{grammar}
    <JSON-value> ::= \[[
    \begin{stack}
      <JSON-string> \\
      <JSON-number> \\
      <JSON-object> \\
      <JSON-array> \\
      "true" \\
      "false" \\
      "null"
    \end{stack}
    \]]

    <JSON-string> ::= \[[
    "\""
    \begin{stack}
      \\
      \fbox {
        \begin{minipage}{1,6in}
          \begin{center}
            Any\ unicode\ char\ except\ '{\char 34}',\ '$\backslash$'\
            and\ any\ control\ chars
          \end{center}
        \end{minipage}
      }
      \\
      "\\"
      \begin{stack}
        "\"" \\
        "\\" \\
        "/" \\
        "b" \\
        "f" \\
        "n" \\
        "r" \\
        "t" \\
        "u"
      \end{stack}
    \end{stack}
    "\""
    \]]

    <JSON-object> ::= \[[
    "{"
      \begin{stack}
        \\

        \begin{rep}
          <JSON-string> ":" <JSON-value> \\
          ","
        \end{rep}
      \end{stack}
      "}"
    \]]

    <JSON-array> ::= \[[
    "["
    \begin{stack}
      \\
      \begin{rep}
        <JSON-value> \\
        ","
      \end{rep}
    \end{stack}
    "]"
    \]]

    <JSON-number> ::= \[[
    \begin{stack}
      \\
      "-"
    \end{stack}
    \begin{stack}
      "0" \\
      1--9
      \begin{rep}
        \\
        0--9
      \end{rep}
    \end{stack}
    \begin{stack}
      \\
      "."
      \begin{rep}
        \\
        0--9
      \end{rep}
    \end{stack}
    \begin{stack}
      \\
      \begin{stack}
        e \\
        E
      \end{stack}
      \begin{stack}
        \\
        "+" \\
        "-"
      \end{stack}
      \begin{rep}
        \\
        0--9
      \end{rep}
    \end{stack}
    \]]

  \end{grammar}


  \caption{Definition of JSON as described 10th of January 2010 at \url{www.json.org}  }
\end{nonfloatingfigure}


%%% Local Variables: 
%%% mode: latex
%%% TeX-master: "../../../report"
%%% reftex-fref-is-default: t
%%% End: 


\chapter{...}



\end{document}

%%% Local Variables: 
%%% mode: latex
%%% TeX-master: t
%%% reftex-fref-is-default: t
%%% End: 
