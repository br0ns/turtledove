\documentclass[a4paper, oneside, final]{memoir}
% Fixes "No room for a new \xxx" error by extending the default 256 fixed size
% LaTeX arrays
\usepackage{etex}
\reserveinserts{28}


\usepackage[T1]{fontenc}
\usepackage[utf8]{inputenc}
\usepackage[british]{babel}

% bedre orddeling Gør at der som minimum skal blive to tegn på linien ved
% orddeling og minimum flyttes to tegn ned på næste linie. Desværre er værdien
% anvendt af babel »12«, hvilket kan give orddelingen »h-vor«.
\renewcommand{\britishhyphenmins}{22} 

% Fix of fancyref to work with memoir. Makes references look
% nice. Redefines memoir \fref and \Fref to \refer and \Refer.
% \usepackage{refer}             %
% As we dont really have any use for \fref and \Fref we just undefine what
% memoir defined them as, so fancyref can define what it wants.
\let\fref\undefined
\let\Fref\undefined
\usepackage{fancyref} % Better reference. 

\usepackage{pdflscape} % Gør landscape-environmentet tilgængeligt
\usepackage{fixme}     % Indsæt "fixme" noter i drafts.
\usepackage{hyperref}  % Indsæter links (interne og eksterne) i PDF

\usepackage[rounded]{syntax} % Part of the mdwtools package

\usepackage{mdwtab}
\usepackage{mathenv}
\usepackage{amsfonts}
\usepackage{amsmath}
\usepackage{amssymb}
\usepackage{amsthm}
\usepackage{semantic} % for the \mathlig function

\usepackage[format=hang]{caption,subfig}
\usepackage{graphicx}
\usepackage{stmaryrd}
\usepackage{listings}
\usepackage{ulem} % \sout - strike-through
\usepackage{tikz}

\lstset{ %
% language=Octave,                % choose the language of the code
basicstyle=\ttfamily,        % the size of the fonts that are used for the code
basewidth=0.5em,
% numbers=left,                   % where to put the line-numbers
% numberstyle=\footnotesize,      % the size of the fonts that are used for the line-numbers
% stepnumber=2,                   % the step between two line-numbers. If it's 1 each line will be numbered
% numbersep=5pt,                  % how far the line-numbers are from the code
% backgroundcolor=\color{white},  % choose the background color. You must add \usepackage{color}
% showspaces=false,               % show spaces adding particular underscores
% showstringspaces=false,         % underline spaces within strings
% showtabs=false,                 % show tabs within strings adding particular underscores
% frame=single	                % adds a frame around the code
% tabsize=2,	                % sets default tabsize to 2 spaces
% captionpos=b,                   % sets the caption-position to bottom
% breaklines=true,                % sets automatic line breaking
% breakatwhitespace=false,        % sets if automatic breaks should only happen at whitespace
escapeinside={(@}{@)}          % if you want to add a comment within your code
}

\renewcommand{\ttdefault}{txtt} % Bedre typewriter font
%\usepackage[sc]{mathpazo}     % Palatino font
\renewcommand{\rmdefault}{ugm} % Garamond
%\usepackage[garamond]{mathdesign}

%\overfullrule=5pt
%\setsecnumdepth{part}
\setcounter{secnumdepth}{1} % Sæt overskriftsnummereringsdybde. Disable = -1.
\chapterstyle{hangnum} % changes style of chapters, to look nice.

\makeatletter
\newenvironment{nonfloatingfigure}{
  \vskip\intextsep
  \def\@captype{figure}
  }{
  \vskip\intextsep
}

\newenvironment{nonfloatingtable}{
  \vskip\intextsep
  \def\@captype{table}
  }{
  \vskip\intextsep
}
\makeatother

\renewcommand{\ttdefault}{txtt} % Bedre typewriter font
%% \usepackage[sc]{mathpazo}     % Palatino font
%% \renewcommand{\rmdefault}{ugm} % Garamond
%% \usepackage[garamond]{mathdesign}

% \overfullrule=5pt
% \setsecnumdepth{part}
\setcounter{secnumdepth}{1} % Sæt overskriftsnummereringsdybde. Disable = -1.
\chapterstyle{hangnum} % changes style of chapters, to look nice.

\theoremstyle{definition}
\newtheorem{judgment}{Judgment}
\newtheorem{definition}{Definition}
\newtheorem{lemma}{Lemma}
\newtheorem{theorem}{Theorem}
\newtheorem{corollary}{Corollary}
\newtheorem{example}{Example}

\newcommand*{\fancyrefdeflabelprefix}{def}
\fancyrefaddcaptions{english}{
  \newcommand*{\Frefdefname}{Definition}
  \newcommand*{\frefdefname}{\MakeLowercase{\Frefdefname}}
}
\frefformat{vario}{\fancyrefdeflabelprefix}{%
  \frefdefname\fancyrefdefaultspacing#1#3%
}
\Frefformat{vario}{\fancyrefdeflabelprefix}{%
  \Frefdefname\fancyrefdefaultspacing#1#3%
}

\newcommand*{\fancyreflemlabelprefix}{lem}
\fancyrefaddcaptions{english}{
  \newcommand*{\Freflemname}{Lemma}
  \newcommand*{\freflemname}{\MakeLowercase{\Freflemname}}
}
\frefformat{vario}{\fancyreflemlabelprefix}{%
  \freflemname\fancyrefdefaultspacing#1#3%
}
\Frefformat{vario}{\fancyreflemlabelprefix}{%
  \Freflemname\fancyrefdefaultspacing#1#3%
}
\frefformat{plain}{\fancyreflemlabelprefix}{%
  \freflemname\fancyrefdefaultspacing#1%
}
\Frefformat{plain}{\fancyreflemlabelprefix}{%
  \Freflemname\fancyrefdefaultspacing#1%
}

\newcommand*{\fancyrefthmlabelprefix}{thm}
\fancyrefaddcaptions{english}{
  \newcommand*{\Frefthmname}{Theorem}
  \newcommand*{\frefthmname}{\MakeLowercase{\Frefthmname}}
}
\frefformat{vario}{\fancyrefthmlabelprefix}{%
  \frefthmname\fancyrefdefaultspacing#1#3%
}
\Frefformat{vario}{\fancyrefthmlabelprefix}{%
  \Frefthmname\fancyrefdefaultspacing#1#3%
}

\newcommand*{\fancyrefcorlabelprefix}{cor}
\fancyrefaddcaptions{english}{
  \newcommand*{\Frefcorname}{Corollary}
  \newcommand*{\frefcorname}{\MakeLowercase{\Frefcorname}}
}
\frefformat{vario}{\fancyrefcorlabelprefix}{%
  \frefcorname\fancyrefdefaultspacing#1#3%
}
\Frefformat{vario}{\fancyrefcorlabelprefix}{%
  \Frefcorname\fancyrefdefaultspacing#1#3%
}

\newcommand*{\fancyrefexlabelprefix}{ex}
\fancyrefaddcaptions{english}{
  \newcommand*{\Frefexname}{Example}
  \newcommand*{\frefexname}{\MakeLowercase{\Frefexname}}
}
\frefformat{vario}{\fancyrefexlabelprefix}{%
  \frefexname\fancyrefdefaultspacing#1#3%
}
\Frefformat{vario}{\fancyrefexlabelprefix}{%
  \Frefexname\fancyrefdefaultspacing#1#3%
}

\newcommand{\ttt}[1]{\texttt{#1}}
\newcommand{\tnm}[1]{\textnormal{#1}}
\newcommand{\mrm}[1]{\mathrm{#1}}

\newcommand{\Cov}{\mathrm{Cov}}
\providecommand{\FV}{\mathrm{FV}}
\providecommand{\Dom}{\mathrm{Dom}}


\mathlig{||}{\parallel}
\mathlig{<'}{\prec}
\mathlig{>'}{\succ}
\mathlig{<='}{\preccurlyeq}
\mathlig{>='}{\succcurlyeq}
\mathlig{<=}{\leqslant}
\mathlig{>=}{\geqslant}
\mathlig{<>}{\neq}
\mathlig{|=}{\sqsubset}
\mathlig{=|}{\sqsupset}
\mathlig{==}{\equiv}
\mathlig{==a}{=_{\alpha}}
\mathlig{<|}{\lhd}
\mathlig{|>}{\rhd}
\mathlig{++}{\mathrel{\mbox{+\!\!\!+}}}
\mathlig{~>e}{\stackrel{elim}{\leadsto}}
\mathlig{~>g}{\stackrel{gen}{\leadsto}}

%%%%%%%%%%%%%%%%%%%%%%%%%%%%%%%%%%%%%%%%%%%%%%%%%%%%%%%%
%	    	     Forside
%%%%%%%%%%%%%%%%%%%%%%%%%%%%%%%%%%%%%%%%%%%%%%%%%%%%%%%%
\makeatletter % open mode for reading @ signed variables 
\def\maketitle{%
  \null
  \thispagestyle{empty}%
  \vfill
  \begin{center}\leavevmode
    \normalfont
    \Huge{\raggedleft \@title\par}%
    \hrulefill\par
    \Large{\raggedright \subtitle\par}%
    \vskip 2cm
    {\@date\par}%
  \end{center}%
  \vfill
\begin{minipage}{80pt}
\includegraphics*[scale=0.75]{imgs/nat-logo}
\end{minipage}
\begin{minipage}{300pt}
  \begin{flushleft}
    {\large \@author } \\
    {\footnotesize \suplementInfo }

  \end{flushleft}
\end{minipage}
\cleardoublepage % lave 1 ekstre side blank efter
  \clearpage % Terminates the page here. Everything else vil be placed on next page.
}
\makeatother % closing mode for reading @ signed variables
%%%%%%%%%%%%%%%%%%%%%%%%%%%%%%%%%%%%%%%%%%%%%%%%%%%%%%%%
%		Data til forside
%%%%%%%%%%%%%%%%%%%%%%%%%%%%%%%%%%%%%%%%%%%%%%%%%%%%%%%%
\title{Turtledove: Tool assisted programming in SML, with special emphasis on semi-automatic rewriting to
predefined standard forms.}
\def\subtitle{\footnotesize{A joined Smartypants inc. and Morning Wood Productions venture.}}
\author{Morten Brøns-Pedersen {\footnotesize{(mortenbp@gmail.com)}}\\  
Jesper Reenberg \footnotesize{(jesper.reenberg@gmail.com)}}
\def\suplementInfo{
  \kern 5pt \hrule width 11pc \kern 5pt % putter 5pt spacing oven over og neden under stregen
  Dept. of Computer Science \\
  University of Copenhagen}
% \date{} % used to set explicit dates

\pagestyle{plain}

\begin{document}

\frontmatter

\maketitle
\thispagestyle{empty}

\begin{abstract}
\end{abstract}

\clearpage 
\chapter*{Preface}
This report is a 22.5 ECTS masters project at the Department of Computer Science
at the University of Copenhagen (DIKU). The authors are Morten Brøns-Pedersen
and Jesper Reenberg. The project is supervised by Jakob Grue Simonsen, assistant
professor at DIKU.

\clearpage

\tableofcontents*

\mainmatter

\chapter{Introduction}

\section{Problem statement}
\label{sec:problem-statement}

\section{Motivation}
\label{sec:motivation}

\section{Reader expectations}

\section{Structure outline}

\chapter{Design}
\label{cha:design}
\documentclass[a4paper, oneside, final]{memoir}
\usepackage[T1]{fontenc}
\usepackage[utf8]{inputenc}
\usepackage[british]{babel}

\usepackage{hyperref}  % Indsæter links (interne og eksterne) i PDF
\usepackage[rounded]{syntax} % Part of the mdwtools package

\let\fref\undefined
\let\Fref\undefined
\usepackage{fancyref
}
\usepackage{amsthm} % theorems

\renewcommand{\ttdefault}{txtt} % Bedre typewriter font
%\usepackage[sc]{mathpazo}     % Palatino font
\renewcommand{\rmdefault}{ugm} % Garamond
%\usepackage[garamond]{mathdesign}


\theoremstyle{definition}
\newtheorem{example}{Example}

\newcommand*{\fancyrefexlabelprefix}{ex}
\fancyrefaddcaptions{english}{
  \newcommand*{\Frefexname}{Example}
  \newcommand*{\frefexname}{\MakeLowercase{\Frefexname}}
}
\frefformat{vario}{\fancyrefexlabelprefix}{%
  \frefexname\fancyrefdefaultspacing#1#3%
}
\Frefformat{vario}{\fancyrefexlabelprefix}{%
  \Frefexname\fancyrefdefaultspacing#1#3%
}


\begin{document}


\chapter*{Intercommunication}


\section*{Protocol}


The communication protocol between the development environment and Turtledove is
vary simple and versatile with almost no limitations. The protocol utilises JSON
(JavaScript Object Notation) \footnote{\url{http://www.json.org/}} for the data
payload which makes it easy to serialise different types of values to and from
the desired tools inside Turtledove.

Besides the data payload the protocol consists of a ``CallerID'' string and a
``DestinationID'' string which is described below

\begin{description}
\item[CallerID] is a string specified by the development environment. This
  string is not used in any way by Turtledove but is returned in the response
  string. This string is intended for internal bookkeeping in the development
  environment for example to distinguish which file/buffer and/or at which line
  and column the response originated from.

  As there is a possibility of Turtledove getting multi threaded, this is a good
  way of handling multiple request to Turtledove that may have different
  response times (One request waiting for a new complete reparse of the project code
  and another requesting a static lookup of some data).

  If the development environment don't need the use of this string an empty
  string can be passed to Turtledove.

  Some of the tools may report back when they are done doing something, without
  the development environment has invoked it. In these cases an empty string
  will be used as ``CallerID''.

\item[DestinationID] is a string that needs to match a named destination in
  Turtledove. Examples of such a named destination, also including name of
  tools:

  \begin{itemize}
  \item ``Turtledove'': This is for communication with the main program. This
    includes enabling/disabling of individual tools, status information and
    closing down the server gracefully.

  \item ``ProjectManager'': This is for communication with the project
    manager. This include actions such as modifying text, creation/deletion of
    files and projects.
  \end{itemize}
\end{description}


\begin{grammar}
  <request> ::= \[[ "CallerID" "\\n" "DestinationID" "\\n" <JSON-value> "\\n\\n" \]]

  <response> ::= \[[ "DestinationID" "\\n" "CallerID" "\\n" <JSON-value> "\\n\\n" \]]

  <JSON-value> ::= \[[ 
  \begin{stack}
    <JSON-string> \\
    <JSON-number> \\
    <JSON-object> \\ 
    <JSON-array> \\
    "true" \\
    "false" \\
    "null"
  \end{stack}
  \]]   

  <JSON-string> ::= \[[
  "\""
  \begin{stack}
    \\
    \fbox {
      \begin{minipage}{1,6in}
        \begin{center}
          Any\ unicode\ char\ except\ '{\char 34}',\ '$\backslash$'\ 
          and\ any\ control\ chars 
        \end{center}
      \end{minipage}
    }
    \\
    "\\"
    \begin{stack}
      "\"" \\
      "\\" \\
      "/" \\
      "b" \\
      "f" \\
      "n" \\
      "r" \\
      "t" \\
      "u"
    \end{stack}
  \end{stack}
  "\""
  \]]
  
  <JSON-object> ::= \[[
  "{"
    \begin{stack}
      \\
      
      \begin{rep}
        <JSON-string> ":" <JSON-value> \\
        ","
      \end{rep}
    \end{stack}
    "}"
  \]]

  <JSON-array> ::= \[[
  "["
  \begin{stack}
    \\
    \begin{rep}
      <JSON-value> \\
      ","
    \end{rep}
  \end{stack}
  "]"
  \]]

  <JSON-number> ::= \[[
  \begin{stack}
    \\
    "-"
  \end{stack}
  \begin{stack}
    "0" \\
    1--9
    \begin{rep}
      \\
      0--9
    \end{rep}
  \end{stack}
  \begin{stack}
    \\
    "." 
    \begin{rep}
      \\
      0--9
    \end{rep}
  \end{stack}
  \begin{stack}
    \\
    \begin{stack}
      e \\
      E
    \end{stack}
    \begin{stack}
      \\
      "+" \\
      "-" 
    \end{stack}
    \begin{rep}
      \\
      0--9
    \end{rep}
  \end{stack}
  \]]
  
\end{grammar}

It is important to remember that the resulting string sent to and from
Turtledove must not contain any newlines except the ones that are mandatory by
the protocol. This is a requirement even though the definition of JSON allows
whitespace between a pair of tokens, as it would then render it impossible to
determine when the request/response is done.

We have chosen the newline character as a separator of the fields as it
makes it easy for debug pretty printing and since a newline character is the least
likely character to be in a name/id.

The same argument applies to the decision of using double newline chars to mark
the end of a request or response. 


\paragraph{Examples}



\begin{example} If a file needs to be added then the following string could be
  sent to the ``Projectmanager'':
\begin{verbatim}
"some_id \n ProjectManager \n {\"command\" : \"addfile\", 
                                  \"file\" : \"myfoo.sml\"} \n\n"
\end{verbatim}
\end{example}


\section*{Useful commands}

 Below is a draft of which commands might be useful to implement in the final system. 

\subsection*{Turtledove}

\begin{description}
\item[quit] This command closes down the server in a graceful way.
\end{description}

\subsection*{ProjectManager }

\begin{description}
\item[OpenProject] Makes the project manager open a project. If the
  project doesn't exists, then a new project with the name is created.

  If another project is currently opened, then this project is closed, if told
  to do so.
\item[CloseProject] makes the project manager close the currently opened project,
  saving any unsaved changes if told to do so. 
\item[DeleteProject] Makes the project manager delete the currently opened project
  with all its files.
\item[AddFile] Adds a new file to the project.
\item[DeleteFile] Deletes a file contained in the project
\item[ChangeFile] Add or delete content in a open file contained in the
  project. This command should be used when ever the user adds or deletes text
  from any file contained in the project, so the project manager can an
  up to date version of the file, without having to reread the file again and
  thus also at some point do incremental parsing of the changes only.
\end{description}


\paragraph{Get names in scope}

Returns a list of identifiers visible from the current scope. This is particular
handy for doing auto completion and 


\paragraph{Goto definition}

Gets the filename and position of the definition of the specified function

\paragraph{Rename}

Rename a function or identifier together with its definition and all its uses.


\end{document}



%%% Local Variables: 
%%% mode: latex
%%% TeX-master: t
%%% reftex-fref-is-default: t
%%% End: 


\chapter{Implementation}

\chapter{...}
...


\chapter{Evaluation}
\label{chap:evaluation}

\chapter{Conclusions}

\bibliographystyle{bibliography/theseurl}
\bibliography{bibliography/bibliography}

\appendix

\chapter{Figures}
\section{Protocols}
\label{sec:protocols}


\subsection{JSON}
\label{sec:protocol-json}


\begin{nonfloatingfigure}


  \begin{grammar}
    <JSON-value> ::= \[[
    \begin{stack}
      <JSON-string> \\
      <JSON-number> \\
      <JSON-object> \\
      <JSON-array> \\
      "true" \\
      "false" \\
      "null"
    \end{stack}
    \]]

    <JSON-string> ::= \[[
    "\""
    \begin{stack}
      \\
      \fbox {
        \begin{minipage}{1,6in}
          \begin{center}
            Any\ unicode\ char\ except\ '{\char 34}',\ '$\backslash$'\
            and\ any\ control\ chars
          \end{center}
        \end{minipage}
      }
      \\
      "\\"
      \begin{stack}
        "\"" \\
        "\\" \\
        "/" \\
        "b" \\
        "f" \\
        "n" \\
        "r" \\
        "t" \\
        "u"
      \end{stack}
    \end{stack}
    "\""
    \]]

    <JSON-object> ::= \[[
    "{"
      \begin{stack}
        \\

        \begin{rep}
          <JSON-string> ":" <JSON-value> \\
          ","
        \end{rep}
      \end{stack}
      "}"
    \]]

    <JSON-array> ::= \[[
    "["
    \begin{stack}
      \\
      \begin{rep}
        <JSON-value> \\
        ","
      \end{rep}
    \end{stack}
    "]"
    \]]

    <JSON-number> ::= \[[
    \begin{stack}
      \\
      "-"
    \end{stack}
    \begin{stack}
      "0" \\
      1--9
      \begin{rep}
        \\
        0--9
      \end{rep}
    \end{stack}
    \begin{stack}
      \\
      "."
      \begin{rep}
        \\
        0--9
      \end{rep}
    \end{stack}
    \begin{stack}
      \\
      \begin{stack}
        e \\
        E
      \end{stack}
      \begin{stack}
        \\
        "+" \\
        "-"
      \end{stack}
      \begin{rep}
        \\
        0--9
      \end{rep}
    \end{stack}
    \]]

  \end{grammar}

  \caption{Definition of JSON as described 10th of January 2010 at
    \url{www.json.org} \cite{json}}
  \label{fig:protocol-json}
\end{nonfloatingfigure}


%%% Local Variables: 
%%% mode: latex
%%% TeX-master: "../../../report"
%%% reftex-fref-is-default: t
%%% End: 


\chapter{...}



\end{document}

%%% Local Variables: 
%%% mode: latex
%%% TeX-master: t
%%% reftex-fref-is-default: t
%%% End: 
