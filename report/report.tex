
\documentclass[a4paper, oneside, draft]{memoir}
% Fixes "No room for a new \xxx" error by extending the default 256 fixed size
% LaTeX arrays
\usepackage{etex}
\reserveinserts{28}


\usepackage[T1]{fontenc}
\usepackage[utf8]{inputenc}
\usepackage[british]{babel}

% bedre orddeling Gør at der som minimum skal blive to tegn på linien ved
% orddeling og minimum flyttes to tegn ned på næste linie. Desværre er værdien
% anvendt af babel »12«, hvilket kan give orddelingen »h-vor«.
\renewcommand{\britishhyphenmins}{22} 

% Fix of fancyref to work with memoir. Makes references look
% nice. Redefines memoir \fref and \Fref to \refer and \Refer.
% \usepackage{refer}             %
% As we dont really have any use for \fref and \Fref we just undefine what
% memoir defined them as, so fancyref can define what it wants.
\let\fref\undefined
\let\Fref\undefined
\usepackage{fancyref} % Better reference. 

\usepackage{pdflscape} % Gør landscape-environmentet tilgængeligt
\usepackage{fixme}     % Indsæt "fixme" noter i drafts.
\usepackage{hyperref}  % Indsæter links (interne og eksterne) i PDF

\usepackage[rounded]{syntax} % Part of the mdwtools package

\usepackage{mdwtab}
\usepackage{mathenv}
\usepackage{amsfonts}
\usepackage{amsmath}
\usepackage{amssymb}
\usepackage{amsthm}
\usepackage{semantic} % for the \mathlig function

\usepackage[format=hang]{caption,subfig}
\usepackage{graphicx}
\usepackage{stmaryrd}
\usepackage[final]{listings} % Make sure we show the listing even though we are
                             % making a final report.
\usepackage{ulem} % \sout - strike-through
\usepackage{tikz}

\lstset{ %
% language=Octave,                % choose the language of the code
basicstyle=\ttfamily,        % the size of the fonts that are used for the code
basewidth=0.5em,
% numbers=left,                   % where to put the line-numbers
% numberstyle=\footnotesize,      % the size of the fonts that are used for the line-numbers
% stepnumber=2,                   % the step between two line-numbers. If it's 1 each line will be numbered
% numbersep=5pt,                  % how far the line-numbers are from the code
% backgroundcolor=\color{white},  % choose the background color. You must add \usepackage{color}
% showspaces=false,               % show spaces adding particular underscores
% showstringspaces=false,         % underline spaces within strings
% showtabs=false,                 % show tabs within strings adding particular underscores
% frame=single	                % adds a frame around the code
% tabsize=2,	                % sets default tabsize to 2 spaces
% captionpos=b,                   % sets the caption-position to bottom
% breaklines=true,                % sets automatic line breaking
% breakatwhitespace=false,        % sets if automatic breaks should only happen at whitespace
escapeinside={(@}{@)}          % if you want to add a comment within your code
}

\renewcommand{\ttdefault}{txtt} % Bedre typewriter font
%\usepackage[sc]{mathpazo}     % Palatino font
\renewcommand{\rmdefault}{ugm} % Garamond
%\usepackage[garamond]{mathdesign}

%\overfullrule=5pt
%\setsecnumdepth{part}
\setcounter{secnumdepth}{1} % Sæt overskriftsnummereringsdybde. Disable = -1.
\chapterstyle{hangnum} % changes style of chapters, to look nice.

\makeatletter
\newenvironment{nonfloatingfigure}{
  \vskip\intextsep
  \def\@captype{figure}
  }{
  \vskip\intextsep
}

\newenvironment{nonfloatingtable}{
  \vskip\intextsep
  \def\@captype{table}
  }{
  \vskip\intextsep
}
\makeatother

\renewcommand{\ttdefault}{txtt} % Bedre typewriter font
%% \usepackage[sc]{mathpazo}     % Palatino font
%% \renewcommand{\rmdefault}{ugm} % Garamond
%% \usepackage[garamond]{mathdesign}

% \overfullrule=5pt
% \setsecnumdepth{part}
\setcounter{secnumdepth}{1} % Sæt overskriftsnummereringsdybde. Disable = -1.
\chapterstyle{hangnum} % changes style of chapters, to look nice.

\theoremstyle{definition}
\newtheorem{judgment}{Judgment}
\newtheorem{definition}{Definition}
\newtheorem{lemma}{Lemma}
\newtheorem{theorem}{Theorem}
\newtheorem{corollary}{Corollary}
\newtheorem{example}{Example}

\newcommand*{\fancyrefdeflabelprefix}{def}
\fancyrefaddcaptions{english}{
  \newcommand*{\Frefdefname}{Definition}
  \newcommand*{\frefdefname}{\MakeLowercase{\Frefdefname}}
}
\frefformat{vario}{\fancyrefdeflabelprefix}{%
  \frefdefname\fancyrefdefaultspacing#1#3%
}
\Frefformat{vario}{\fancyrefdeflabelprefix}{%
  \Frefdefname\fancyrefdefaultspacing#1#3%
}

\newcommand*{\fancyreflemlabelprefix}{lem}
\fancyrefaddcaptions{english}{
  \newcommand*{\Freflemname}{Lemma}
  \newcommand*{\freflemname}{\MakeLowercase{\Freflemname}}
}
\frefformat{vario}{\fancyreflemlabelprefix}{%
  \freflemname\fancyrefdefaultspacing#1#3%
}
\Frefformat{vario}{\fancyreflemlabelprefix}{%
  \Freflemname\fancyrefdefaultspacing#1#3%
}
\frefformat{plain}{\fancyreflemlabelprefix}{%
  \freflemname\fancyrefdefaultspacing#1%
}
\Frefformat{plain}{\fancyreflemlabelprefix}{%
  \Freflemname\fancyrefdefaultspacing#1%
}

\newcommand*{\fancyrefthmlabelprefix}{thm}
\fancyrefaddcaptions{english}{
  \newcommand*{\Frefthmname}{Theorem}
  \newcommand*{\frefthmname}{\MakeLowercase{\Frefthmname}}
}
\frefformat{vario}{\fancyrefthmlabelprefix}{%
  \frefthmname\fancyrefdefaultspacing#1#3%
}
\Frefformat{vario}{\fancyrefthmlabelprefix}{%
  \Frefthmname\fancyrefdefaultspacing#1#3%
}

\newcommand*{\fancyrefcorlabelprefix}{cor}
\fancyrefaddcaptions{english}{
  \newcommand*{\Frefcorname}{Corollary}
  \newcommand*{\frefcorname}{\MakeLowercase{\Frefcorname}}
}
\frefformat{vario}{\fancyrefcorlabelprefix}{%
  \frefcorname\fancyrefdefaultspacing#1#3%
}
\Frefformat{vario}{\fancyrefcorlabelprefix}{%
  \Frefcorname\fancyrefdefaultspacing#1#3%
}

\newcommand*{\fancyrefexlabelprefix}{ex}
\fancyrefaddcaptions{english}{
  \newcommand*{\Frefexname}{Example}
  \newcommand*{\frefexname}{\MakeLowercase{\Frefexname}}
}
\frefformat{vario}{\fancyrefexlabelprefix}{%
  \frefexname\fancyrefdefaultspacing#1#3%
}
\Frefformat{vario}{\fancyrefexlabelprefix}{%
  \Frefexname\fancyrefdefaultspacing#1#3%
}

\newcommand{\ttt}[1]{\texttt{#1}}
\newcommand{\tnm}[1]{\textnormal{#1}}
\newcommand{\mrm}[1]{\mathrm{#1}}

\newcommand{\Cov}{\mathrm{Cov}}
\providecommand{\FV}{\mathrm{FV}}
\providecommand{\Dom}{\mathrm{Dom}}


\mathlig{||}{\parallel}
\mathlig{<'}{\prec}
\mathlig{>'}{\succ}
\mathlig{<='}{\preccurlyeq}
\mathlig{>='}{\succcurlyeq}
\mathlig{<=}{\leqslant}
\mathlig{>=}{\geqslant}
\mathlig{<>}{\neq}
\mathlig{|=}{\sqsubset}
\mathlig{=|}{\sqsupset}
\mathlig{==}{\equiv}
\mathlig{==a}{=_{\alpha}}
\mathlig{<|}{\lhd}
\mathlig{|>}{\rhd}
\mathlig{++}{\mathrel{\mbox{+\!\!\!+}}}
\mathlig{~>e}{\stackrel{elim}{\leadsto}}
\mathlig{~>g}{\stackrel{gen}{\leadsto}}

%%%%%%%%%%%%%%%%%%%%%%%%%%%%%%%%%%%%%%%%%%%%%%%%%%%%%%%%
%	    	     Forside
%%%%%%%%%%%%%%%%%%%%%%%%%%%%%%%%%%%%%%%%%%%%%%%%%%%%%%%%
\makeatletter % open mode for reading @ signed variables 
\def\maketitle{%
  \null
  \thispagestyle{empty}%
  \vfill
  \begin{center}\leavevmode
    \normalfont
    \Huge{\raggedleft \@title\par}%
    \hrulefill\par
    \Large{\raggedright \subtitle\par}%
    \vskip 2cm
    {\@date\par}%
  \end{center}%
  \vfill
\begin{minipage}{80pt}
\includegraphics*[scale=0.75]{imgs/nat-logo}
\end{minipage}
\begin{minipage}{300pt}
  \begin{flushleft}
    {\large \@author } \\
    {\footnotesize \suplementInfo }

  \end{flushleft}
\end{minipage}
\cleardoublepage % lave 1 ekstre side blank efter
  \clearpage % Terminates the page here. Everything else vil be placed on next page.
}
\makeatother % closing mode for reading @ signed variables
%%%%%%%%%%%%%%%%%%%%%%%%%%%%%%%%%%%%%%%%%%%%%%%%%%%%%%%%
%		Data til forside
%%%%%%%%%%%%%%%%%%%%%%%%%%%%%%%%%%%%%%%%%%%%%%%%%%%%%%%%
\title{Turtledove}
\def\subtitle{\footnotesize{Tool assisted programming in SML, with special emphasis on semi-automatic rewriting to
predefined standard forms.}}
\author{Morten Brøns-Pedersen {\footnotesize{(mortenbp@gmail.com)}}\\  
Jesper Reenberg \footnotesize{(jesper.reenberg@gmail.com)}}
\def\suplementInfo{
  \kern 5pt \hrule width 11pc \kern 5pt % putter 5pt spacing oven over og neden under stregen
  Dept. of Computer Science \\
  University of Copenhagen}
% \date{} % used to set explicit dates

\pagestyle{plain}

\begin{document}

\frontmatter

\maketitle
\thispagestyle{empty}

\begin{abstract}

\end{abstract}

\clearpage 
\chapter*{Preface}
This report is a 22.5 ECTS masters project at the Department of Computer
Science, University of Copenhagen (DIKU). The authors are Morten Brøns-Pedersen
and Jesper Reenberg. The project is supervised by Jakob Grue Simonsen, assistant
professor at DIKU.

\clearpage

\tableofcontents*

\mainmatter

\chapter{Introduction}

\chapter{Introduction}

\section{Problem Statement}

{\footnotesize [Huge explanation about the actual problem.]}

\section{Motivation}
\label{sec:motivation}
Suppose that we want to find code that could be written shorter using the
\texttt{map} function.
Here is an obvious example:
\begin{sml}
fun add (x :: xs) = x + 1 :: add xs
  | add nil       = nil
\end{sml}
can be rewritten to
\begin{sml}
val add = map (fn x => x + 1)
\end{sml}

But suppose that the first function was instead
\begin{sml}
fun add nil       = nil
  | add (x :: xs) = x + 1 :: add xs
\end{sml}
or even\footnote{We have seen novice SML programmers write functions similar to
  this.}
\begin{sml}
fun add (x :: xs) = x + 1 :: add xs
  | add [x]       = x + 1 :: add nil
  | add nil       = nil
\end{sml}
Of course all three examples can be rewritten to the same form. So should we
have three rewriting rules? Infinitely many? No.

Our problem here is that equivalent matches (a match is a list of pairs of
patterns and expressions) can take many forms.

In this paper we define a language similar to Core SML. We define a
normal form for matches. We then show how to obtain an equivalent normal form
from an arbitrary match.

\paragraph{Further work.}
The reader might find it odd that the second line in the last example above ends
in \smlinline{:: add nil}. Novice programmer or not, real code probably does not
look like this. The reason is that if \smlinline{:: add nil} is left out, the three
versions of the function \smlinline{add} does not have the same normal form.

We would like to determine that
\begin{sml}
fun add (x :: xs) = x + 1 :: add xs
  | add [x]       = [x + 1]
  | add nil       = nil
\end{sml}
is indeed equivalent to
\begin{sml}
fun add (x :: xs) = x + 1 :: add xs
  | add nil       = nil
\end{sml}
In the first function the patterns in line one and two is \smlinline{x :: xs}
and \smlinline{x :: nil}, so we can try to instantiate \smlinline{xs} to
\smlinline{nil} in the first function body, to try to eliminate the second line
of the function.

The first body becomes \smlinline{x + 1 :: add nil} which is not equal to the
second which is \smlinline{x + 1 :: nil}. But if we inline the definition of
\smlinline{add} in the first body we get \smlinline{x + 1 :: nil}. And so the
second line of the body can be eliminated.

Of course inlining function definitions to find a normal form, makes the
reduction to normal forms an undecidable problem.

For this reason we have decided to define a normal form that can always be
found, and then build on top of that.

Consider this other example:
\begin{sml}
fun add (x :: xs, b) = x + b :: add xs
  | add (nil, _)     = nil
\end{sml}
That too can be written using \smlinline{map}. But again we would need a new
rewriting rule for that. A solution could be to use meta patterns in rewriting
rules.

\fixme[inline,margin=false]{Ok, I'm just ranting here. Keeping the example for
  further brainstorm later on.}

\section{Readers prerequisites}


Readers of this text should, as a minimal prerequisite, be familiar with
Standard ML. Some knowledge of compiler design and programming language theory,
is also recommended. Also some algorithmic and mathematically maturity will be
of help. Any to-be computer scientist with a few years experience should be able
to either directly understand this text, or be able to easily acquire the needed
knowledge.

\section{Structure outline}


%%% Local Variables: 
%%% mode: latex
%%% TeX-master: "../report"
%%% End: 


\section{Structure outline}


%%%%%%%%%%%%%%%%%%%%%%%%%%%%%%%%%%%%%%%%%%%%%%%%
\chapter{Term rewriting}


%%% Local Variables: 
%%% mode: latex
%%% TeX-master: "../report"
%%% End: 



%%%%%%%%%%%%%%%%%%%%%%%%%%%%%%%%%%%%%%%%%%%%%%%%

\chapter{Design}
\label{cha:design}



\section{Turtledove}
\label{sec:design-turtledove}


%\subsection{MLB parser}
%\label{sec:design-mlb-parser}

%\subsection{SML Parser}
%\label{sec:design-sml-parser}


\subsection{Project manager}
\label{sec:design-project-manager}

The project manager is responsible for managing the project and all actions done
to the current project. 

There are essentially two types of projects, that the manager needs to handle
ordinary projects and ``single file'' projects (both described below). 

When the project manager tries to open a project, it always tries to open it as
a ordinary project. As project files are JSON encoded\footnote{As we also need a
  JSON parser for other purposes this has been chosen as the project encoding to
  ease development time. Also the encoding is ``human readable'' which make
  human editing of the files relatively easy, though not encouraged.} and are
defined to contain a specialised \synt{JSON-Object} (see
\fref{fig:protocol-json}), the first character, except any trailing whitespace,
must be a ``\{'' else this is not a valid project file and it then defaults to
open the file as a ``single file'' project.

No action will be taken from the project manager if the file, opened as a
``single file'' project, is not valid SML code. In this case the SML parser will
try to parse the file and fail with some appropriate error message.

The following are valid actions that can be done to a project

\begin{description}
\item[Files] can be added, removed or renamed. \\

  If a filename is relative, we enforce that it must be relative to where
  the project file is located, sinve all relative filenames will be made
  absolute, by appending the path of the project file, when creating the MLB
  description. This is due to the MLB parser \fixme{MLB eller SML parseren?}
  only accepts absolute paths.

  It don't make sense to have the same file defined multiple times in a project,
  so it is not allowed as this will make it impossible to know which file to
  rename or delete. Also the same functionality can be accomplished by using
  dependencies.

\item[Groupings] can be added, removed or renamed. \\

  To ease the overall structure of a project, it can be grouped. A grouping is a
  logical collection of either files or other groupings referenced as
  objects. It must have a name, reference list of names of objects to be exposed
  and list of object definitions. The name of a grouping is the counterpart to
  the filename of files so that it can be referenced in dependency and exposure
  informations. The exposed object names must be of objects defined top most in
  this group and not of objects defined in sub groupings of this group.

  As with files, it is not allowed to add multiple groupings with the same name.


\item[Dependencies] can be added or removed. \\

  Dependencies make sure that files are parsed in the correct order. This is
  explained in more detail below. The most common use of dependencies is that
  structures must depend upon the signatures that they implement such that the
  signature is parsed before the structure.


  One might ask why this is not auto generated? In some (if not most) cases it
  can't be determined, most obviously if there exists two modules with the same
  name but with different meaning. However in most cases a fair guess can be
  generated, though it leaves the few cases unsolved where it is not possible
  and it would be complicated and time consuming to implement. So the other
  solution has been chosen which also offer more flexibility to the user at the
  cost of being a bit more complicated to use.


  If a file doesn't have any dependencies, then there is no guarantee of which
  order the file will be parsed in. In all cases where a structure implements a
  signature, and the signature is defined in another file), then the file
  defining the structure must depend on the file defining the signature, as this
  will make sure that the signature is parsed before the structure.

\item[Exposure] can be added or removed. \\

  Exposing an object or not, can bee seen as a form of public or private
  modified respectively. For example a grouping containing two files
  \textit{FooUtils.sml} and \textit{Boo.sml}, here we don't necessarily want to
  expose the functions defined in the utility file as these may only have
  meaning to the functions defined in the foo file. In this case we would also
  have a dependency from \textit{Foo.sml} to \textit{FooUtils.sml} such that the
  utility file is parsed before the foo file.

  This is explained in detail in \fref{sec:translating-into-mlb}.

\end{description}


\subsubsection{Ordinary Projects}


An ordinary project is defined by a project file (see
\fref{sec:protocol-project-file}), normally with the extension \texttt{.turt}.

To keep the project file as simple as possible, the actual project definition is
a grouping (the ``ProjectNode'' field in the below description of the project
file). Having it this way simplifies implementation since there is no need for
separate logic to handle the project definition and separate logic for the
containing groupings. In other terms, we can use the same recursive logic.

\paragraph{The project file}

The project file is a \synt{Project-File-Object} which is defined by the
following three fields, see \fref[plain]{fig:protocol-project-file} for the
definition of a project file and more examples of project files.
  

\begin{description}
\item[Properties] a \synt{JSON-Object} of key-value pairs that the editor can
  read and write any valid \synt{JSON-Value} to. The editor could save state
  information, like which files were open when the project was last used,
  editor version information or anything else. Turtledove will never read or
  write to this field.
  
  \begin{description}
  \item[Dependencies] a \synt{Depends-Array} where each \synt{Depend-Object}
    has two fields
    
    \begin{description}
    \item[Name] a \synt{JSON-String} that defines the group or filename this
      depend constraint applies to.
      
    \item[Depends] a \synt{String-Array} of any group or filename that this
      depend constraint depends upon.
    \end{description}
    
    
  \item[ProjectNode] a \synt{Node-Group-Object} that defines the overall
    structure of a grouping. The ``ProjectNode'' (grouping) is the topmost
    grouping and can thus not be removed. It has the following two fields
    
    \begin{description}
    \item[Name] The name of this grouping as a \synt {JSON-String}.
      
    \item[Value] A \synt{Value-Object} that contains sub groupings or stand
      alone files contained within this group. It has the following two fields
      
      \begin{description}
      \item[Exposes] A \synt{String-Array} of group or filenames from the
        \synt{Nodes-Array} that gets exposed at compile time to groups or files
        that depend on this grouping.
        
      \item[Nodes] A \synt{Nodes-Array} that contains an array of either
        \synt{JSON-String} that is an stand alone file name or
        \synt{Node-Group-Object} which is a sub group of this grouping that
        creates a recursive data structure of the above explained.
      \end{description}
    \end{description}      
  \end{description}
\end{description}


\begin{example}[Sample project file for a project named ``Turtledove'']\ 
  \label{ex:Sample-project-file-turtledove}
  
  The \texttt{Turtledove} project contains the following groupings
  
  \begin{itemize}
  \item Group \texttt{A} which
    \begin{itemize}
    \item Contains the file \texttt{x.sml}.
    \item Exposes the file \texttt{x.sml}.
    \item Doesn't depend on anything.
    \end{itemize}
    
  \item Group \texttt{B} which
    \begin{itemize}
    \item Contains the files \texttt{y.sml} and \texttt{z.sml}.
    \item Exposes the file \texttt{z.sml}.
    \item Has dependencies
      \begin{itemize}
      \item \texttt{y.sml} depends on \texttt{x.sml}.
      \item \texttt{z.sml} depends on \texttt{y.sml}.
      \end{itemize}
    \end{itemize}
    
  \item Group \texttt{C} which
    \begin{itemize}
    \item Contains the files \texttt{n.sml} and \texttt{m.sml}.
    \item Exposes the files \texttt{n.sml} and \texttt{m.sml}.
    \item Doesn't depend on anything.
    \end{itemize}
  \end{itemize}
  
  and files \texttt{u.sml} and \texttt{j.sml}. The project exposes the group
  \texttt{C} and file \texttt{j.sml}, and it has dependencies: \texttt{C}
  depends on \texttt{B}, \texttt{j.sml} depends on \texttt{u.sml} and
  \texttt{j.sml} depends on \texttt{B}.
  
  
  This results in the following project file
  
\begin{lstlisting}
{
  "Properties" : { },
  "Dependencies" : 
  [
    { "Name" : "C", "Depends" : [ "B" ] },
    { "Name" : "j.sml" , "Depends" : [ "u.sml", "B" ] },
    { "Name" : "y.sml" , "Depends" : [ "x.sml" ] },
    { "Name" : "z.sml", "Depends" : [ "y.sml" ] }
  ],
  "ProjectNode" :
  {
    "Name"  : "Turtledove",
    "Value" : 
    {
       "Exposes" : [ "C", "j.sml" ],
       "Nodes" :
       [
         {
           "Name" : "A",
           "Value" :
           {
             "Exposes" : [ "x.sml" ],
             "Nodes" : [ "x.sml" ]
           }          
         },
         {
           "Name" : "B",
           "Value" :
           {
             "Exposes" : [ "z.sml" ],
             "Nodes" : [ "y.sml", "z.sml" ]
           }          
         },
         {
           "Name" : "C",
           "Value" :
           {
             "Exposes" : [ "n.sml", "m.sml" ],
             "Nodes" : [ "n.sml", "m.sml" ]
           }          
         },
         "u.sml", 
         "j.sml"
       ]
    }
  }
}    
\end{lstlisting}
\end{example}

\paragraph{Sorting files and groupings}

When the project needs to be parsed, a resulting MLB description of the project
is generated. As files are parsed in the order they are listed in the MLB
description it is important to sort out the order in which files and groupings
should be listed from the defined dependencies. First this needs to be done for
files and then for groupings.

For files this can be done in two steps

\begin{itemize}
\item Create a DAG (Directed Acyclic Graph) with each filename as a node and
  each dependency as an edge from the file to all the files it depends on.

  
  However before the edges are added to the DAG, any group dependencies needs to
  be expanded. This include the following two steps
  
  \begin{enumerate}
  \item \label{item:expand-group-dependency1} Expand all dependencies to a
    group.
  \item \label{item:expand-group-dependency2} Expand all dependencies from a
    group.
  \end{enumerate}
  
  Expanding the group dependency \texttt{\{"Name" : "C", "Depends" : [ "B" ]\}}
  from the project file in \fref[plain]{ex:Sample-project-file-turtledove},
  would result in the following list by applying the above step
  \ref{item:expand-group-dependency1}
  
\begin{lstlisting}
  C depends on y.sml
  C depends on z.sml
\end{lstlisting}
  
  and by applying step \ref{item:expand-group-dependency2} we get the following
  4 edges
  
\begin{lstlisting}
  n.sml depends on y.sml
  m.sml depends on y.sml
  n.sml depends on z.sml
  m.sml depends on z.sml
\end{lstlisting}
  
  which needs to be added to the DAG as representing that dependency constraint.
  
\item Topological sort the DAG to get an ordered list of filenames.
\end{itemize}

Afterwards the same needs to be done for the groupings. As with the files,
groupings are added as nodes to the DAG but only dependencies from and to groups
are added as edges, without expanding them to their containing files.


The two resulting ordered list of files and groupings represent the order of how
they should be listed in the final MLB description.


\paragraph{Translating into MLB description}
\label{sec:translating-into-mlb}

Each file and grouping needs to be translated into its equivalent MLB
description, which can be done in two ways for both files and groupings. 

Since both files and groupings have been sorted, we can safely reference the
generated basis names of their dependencies with the MLB syntax \texttt{open
  BASIS_NAME} without checking if it actually exists.

The translation examples below uses files and groupings from
\fref[plain]{ex:Sample-project-file-turtledove}.


\begin{description}
  
\item[Translation of files] \
  \begin{description}
   
  \item[No dependencies]
    
    The file \texttt{x.sml} from group \texttt{A} translates to
    
\begin{lstlisting}
basis x = bas x.sml end      
\end{lstlisting}
    
    where \texttt{x.sml} is the absolute path of the filename, and \texttt{x} is
    the basis name of \texttt{x.sml} which is defined as \textit{the filename
      without its extension and suffixed with an incrementing number}.
        
  \item[With dependencies]
    
    The file \texttt{y.sml} from group \texttt{B} that depends on file
    \texttt{x.sml} from group \texttt{A} translates to
    
\begin{lstlisting}
basis y = bas let open x in y.sml end end
\end{lstlisting}
    
    where \texttt{y} is the basis name of \texttt{y.sml} and \texttt{x} is the
    basis name of \texttt{x.sml} from the previous example.
    
    If there had been dependencies on multiple files their basis names would
    just have been listed after \texttt{x}, for example \texttt{... open x g h
      in ...}.
    
  \end{description}
  
\item[Translation of groupings]
  
  As files always belong to a grouping (the project itself is the top grouping and
  must always be defined), they are the ones that are responsible for exposing
  files or groups that are contained within itself.
  
  \begin{description}
  \item[No dependencies]
    
    The grouping \texttt{A} exposes file \texttt{x.sml} so it translates to
    
\begin{lstlisting}
basis A = bas open x end
\end{lstlisting}
    
    no matter how many files or groupings it contains. Here \texttt{x} is the
    basis name of \texttt{x.sml} defined above and \texttt{A} is the basis name
    for the group defined as \textit{the name of the group suffixed with an
      incrementing number}
    
  \item[With dependencies] The grouping \texttt{B} exposes multiple files, depends
    on another grouping and file so it translates to
    
\begin{lstlisting}
basis B = bas let open A u in open y z end end
\end{lstlisting}
    
    where \texttt{B}, \texttt{A}, \texttt{u}, \texttt{y} and \texttt{z} are
    basis names of their respective grouping or file.
  \end{description}
  
  
\item[Translation of the outermost grouping]
  
  The actual project definition, which is the outer most grouping, needs to be
  handled a bid special as every file or grouping translation is encapsulated in
  a MLB basis declaration (\texttt{basis x = bas ... end}) and is thus not
  accessible outside the basis declaration unless opened.
  
  So for the project, to be able to expose anything to the environment trying to
  use this project, we need to open it without encapsulating it in a basis
  declaration and it needs to be done as the last thing in the MLB description.
  
  The project group \texttt{Turtledove} exposes the group \texttt{B} and the
  file \texttt{j.sml} so it would translate to
  
\begin{lstlisting}
open B j
\end{lstlisting}
  
  where \texttt{B} and \texttt{j} are basis names of their respective grouping
  or file.
  
\end{description}



It is important that the basis library is included in the resulting MLB
description as nothing is included by default in any compiler/parser that reads
MLB descriptions and any stand alone file can expect this environment to be
available.

So the following empty MLB description template is used, where \texttt{...} is
replaced with the result of the above algorithm where all file paths that are
relative to the project file is expanded to their absolute paths.

\begin{example}[Empty MLB description template.]\
\label{ex:empty-mlb-description}
\begin{lstlisting}
local
  (@\$@)(SML_LIB)/basis/basis.mlb
in
  ...
end
\end{lstlisting}
\end{example}

The resulting MLB description for the project in
\fref[plain]{ex:Sample-project-file-turtledove} where ordering of files without
dependencies or files with the same dependencies are not important (and thus up
to the implementation) and where the path of the project file is \texttt{/tmp/turtledove.turt}

\begin{example}[Resulting MLB description for {\fref[plain]{ex:Sample-project-file-turtledove}}]\
\begin{lstlisting}
local
  (@\$@)(SML_LIB)/basis/basis.mlb
in
  basis u_0 = bas /tmp/u.sml end
  basis x_1 = bas /tmp/x.sml end

  basis y_2 = bas let open x_1 in /tmp/y.sml end end
  
  basis z_3 = bas let open y_2 in /tmp/z.sml end end

  basis n_4 = bas let open y_2 z_3 in /tmp/n.sml end end
  basis m_5 = bas let open y_2 z_3 in /tmp/m.sml end end
  
  basis j_6 = bas let open u_0 y_2 z_3 in /tmp/j.sml end end

  basis A_7 = bas open x_1 end
  basis B_8 = bas open y_2 end

  basis C_9 = bas let open B_9 in open n_4 m_5 end end

  open C_9 j_6
end
\end{lstlisting}  
\end{example}


\subsubsection{``Single file'' projects}

These types of projects doesn't have any project file, a stand alone SML code
file has been opened. However as Turtledove expects to get a MLB compliant
description of the project, one still needs to be supplied.

As ``single file'' projects only contain one file it is very easy to reference
this file in the empty MLB description template (see
\ref{ex:empty-mlb-description}) that includes the basis library as the only
thing.

\begin{example}[Resulting MLB description for ``single file'' projects]\
\begin{lstlisting}
local
  (@\$@)(SML_LIB)/basis/basis.mlb
in
  ABSOLUTE_PATH_TO_STAND_ALONE_FILE
end
\end{lstlisting}  
\end{example}


The ``single file'' project is an important feature of Turtledove for multiple
reasons

\begin{itemize}
\item Novice programmers of functional languages normally don't learn about or
  create their own signatures and structures, so a MLB description will not be
  necessary to compile the program.
  
\item Students following a course teaching functional programming normally newer
  create multiple files for one assignment, and thus the hassle of setting up an
  full fledge project definition for each coding session will get tedious
  (ofcause this will depend on the editor that is used).
  
  \begin{itemize}
  \item When signatures and structures are taught as part of a class they are
    normally created inside the same file and thus the above still holds
  \end{itemize}
\end{itemize}

And thus this will properly bee the most used of the two types of projects.

Regardless of what type of project that has been opened, the resulting MLB
description is never directly saved to a file. The MLB parser accepts its input
as a string and the description is transfered this way internally as the MLB
description is likely to change when any part of the project changes.

No actions (adding/removing of files, groups, etc.) are allowed to the project
when opened this way since there is no point in saving a project file for just a
single code file, it forces the user to create an ordinary project if more
than one file is needed which is the right thing to do and this kind of project
are more supposed to be a quick way of editing a code file.


%%% Local Variables: 
%%% mode: latex
%%% TeX-master: "../../report"
%%% End: 



%%% Local Variables: 
%%% mode: latex
%%% TeX-master: "../report"
%%% End: 



%%%%%%%%%%%%%%%%%%%%%%%%%%%%%%%%%%%%%%%%%%%%%%%%
\chapter{Implementation}


\section{Parsers}

The SML and Rule parsers are both implemented in SML-Lex and SML-Yacc. We say
SML-Lex/Yacc as both the sml/nj version ml-lex/ml-yacc and mlton version
mllex/mlyacc works, but the mosml version mosmllex and mosmlyac are having
problems.

\fixme{fix the above gibberish}

\subsection{Standard ML}

A fine note about whete the grammar was stolen from and possibly some other
clever stuff.

\subsection{Rule}

The Rule parser is based on the SML parser with some added tokens to the lexer
and rules to the grammar to handle the rule syntax (see
\fref{tab:rule-grammar}). 


\subsubsection{Unicode}

The rule syntax uses the symbols \texttt{£} and \texttt{§} to denote
transformers and meta patterns respectively. These were a bit tricky to
implement as they are not ASCII characters. By default SML-Lex\cite{ml-lex-yacc}
only supports 8-bit characters and thus if the upper parts of UTF-8 characters
are needed they must be specially handled. One way to handle it would be to
convert the input file into some fixed length encoding (i.e., UTF-32), but as
SML only uses ASCII chars that would demand a total remake of the Rule lexer
definition. Instead we went with a UTF-8 solution which only need to be
specially fitted for the transformer and meta patterns as it seems that most
editors now a days use UTF-8 as default encoding (seems to be the case for emacs
on linux). Implementing the two UTF-8 values in the lexer was then just a matter
of defining a named expression containing the conjunction of the two decimal
values making up each of the UTF-8 values (see \fref{tab:utf8-rule-values}).

\begin{table}
  \centering
  \begin{tabular}{|l|c|c|c|}
    \hline
    \textbf{Letter} & \textbf{UTF-8} & \textbf{Hex} & \textbf{Decimal} \\ \hline
    Pound sign (£)   & U+00A3 & 0xC2 0xA3 &  $194$ $163$ \\ \hline
    Section sign (§) & U+00A7 & 0xC2 0xA7 & $194$ $167$ \\ \hline
  \end{tabular}

  \caption{Table of UTF-8 and hex values of \texttt{£} and \texttt{§}}
  \label{tab:utf8-rule-values}
\end{table}

As we are using the internal SML-Lex position feature \texttt{yypos} we have to
decrement it by one each time a transformer or meta pattern is encountered as it
is counted as two characters, where it actually is just one (composite)
character.

\subsection{Abstract syntax tree}


%%% Local Variables: 
%%% mode: latex
%%% TeX-master: "../rewriting-syntax"
%%% End: 


\subsection{Resolvers}


\subsubsection{Infix resolving}


\subsubsection{Environment resolving}




%%% Local Variables: 
%%% mode: latex
%%% TeX-master: "../rewriting-syntax"
%%% End: 


\subsection{Rewriting rule syntax}

When specifying rewriting rules we 

\begin{nonfloatingfigure}

  \setlength{\grammarindent}{7.3em}
  \begin{grammar} 
    
    <rule-program> ::= <rule>$^{*}$
    
    <rule> ::= "rule" <rule-header> <scheme> "becomes" <clause>$^{+}$ "end"
    
    <rule-header> ::= <rule-type> <rule-name>
    
    <rule-type> ::= "clauses" 
    \alt "expression" 
    
    <rule-name> ::= <longid>
    
    <rule-self> ::= "self" <sexp>
    
    <scheme> ::= <clause>$^{+}$ [<cstrns>]
    
    <clause> ::= "|" <spat> "=>" <sexp>
    
    <cstrns> ::= "where" <cstrn-body> 
    
    <cstrn-body> ::= <cstrn-rel> 
    \alt <cstrn-rel> "," <cstrn-body>
    
    <cstrn-rel> ::= <longid> "(" <cstrn-rel-body>$^{+}$ ")"
    
    <cstrn-rel-body> ::= <spat>   
    
    <transformer> ::= "£" <longid> <sexp>

    <meta-pattern-sexp> ::= "§" <longid>  <sexp>$^{*}$
    
    <meta-pattern-spat> ::= "§" <longid>  <spat>$^{*}$

  \end{grammar}
  
  \caption{Complete rule grammar. See \fref{fig:scheme-expressions} for the
    \synt{sexp} grammar and \fref{fig:scheme-patterns} for the \synt{spat}
    grammar which uses the transformers and meta patterns}
  \label{fig:rule-grammar}
\end{nonfloatingfigure}


%%% Local Variables:
%%% mode: latex
%%% TeX-master: "../rewriting-syntax"
%%% End: 





%%% Local Variables: 
%%% mode: latex
%%% TeX-master: "../report"
%%% End: 


\chapter{...}
...


%%%%%%%%%%%%%%%%%%%%%%%%%%%%%%%%%%%%%%%%%%%%%%%%
\chapter{Evaluation}
\label{chap:evaluation}

\fixme{What is presented in this chapter?}

\newcommand{\cmt}[1]{\textcolor{Red}{>>#1}}

The console traces presented in \fref[plain]{sec:eval-normal-form} and
\fref[plain]{sec:eval-map-rewriting} are typeset in columns for brevity and
contains explanatory comments, that are not part of the original output, written
as ``\cmt{This is a comment}''

\section{Grammar}

\newcommand{\fn}{\ttt{fn}\ }
\newcommand{\rec}{\ttt{rec}\ }

An SML like grammar, without types, is presented below which will be the basis
of defining normal forms for SML functions. Well formed code in the below
grammar also includes that all constructors used in the patterns of a match must
be of the same data type. For example the unary constructors made from
\ttt{[0-9]}$^{+}$ could be defined to belong to the data type \textit{number}

\begin{eqnarray*}[rqcql:Tl]
  var & = & \ttt{[a-z]$^{+}$}                          & Identifiers\\
  con & = & \ttt{[A-Z][a-z]$^{*}$ | [0-9]$^{+}$}        & Constructors\\
% Matches
  match & ::= & \epsilon                              & Empty match\\
  & & pat\texttt{.}exp\ \texttt{|}\ match             & Pattern \ttt{=>} expression\\
% Patterns
  pat & ::= & var                                     & Variable\\
  & & con                                             & Constructor of arity $0$\\
  & & con\texttt{(}pat_1\texttt{,} \ldots\texttt{,} pat_n\texttt{)} & Constructor of arity $n$\\
% Expressions
  exp & ::= & var                                     & Variable\\
  & & exp_1\ exp_2                                     & Application\\
  & & \fn match                                       & Function\\
  & & con                                             & Constructor of arity $0$\\
  & & con\texttt{(}exp_1\texttt{,} \ldots\texttt{,} exp_n\texttt{)} & Constructor of arity $n$\\
% Declarations
  dec & ::= & var \mapsto exp                         & value binding\\
  & & \rec var \mapsto exp                            & Recursive value binding\\
  & & dec_1 \ttt{;} \cdots \ttt{;} dec_n              & Sequence, $n \geq 2$\\
  & & \epsilon                                        & Empty program\\
\end{eqnarray*}

We use (with superfixes, subfixes and primes) $v$, $c$, $m$, $p$, $e$ and $d$ to
range over $var$, $con$, $match$, $pat$, $exp$ and $dec$ respectively.

No pattern may contain a given variable more than once.

\paragraph{Environment.} For simplicity it is not possible to define data
types/constructors in the grammar, but we will assume two fixed constructor
environments, $\rho$ and $\psi$, containing this information as this is solely
used when determining coverage (see \fref[plain]{sec:cover}). $\rho$ is a mapping
of constructors to their data type and $\psi$ is a mapping of data types to a
list of all constructors of that data type.


\section{A note about evaluation}
We expect programs to be run in an environment containing predefined functions
(that is variables bound to predefined functions) and constructors. Thus the
program
\begin{quote}
% \begin{verbatim}
\ttt{x $\mapsto$ plus (pair (1, 8))}
% \end{verbatim}
\end{quote}
might make perfect sense (if in particular \ttt{plus} is a variable bound to a
suitable function (perhaps addition), and \texttt{pair}, \texttt{1} and
\texttt{8} are constructors of arity 2, 0 and 0, respectively).
\section{Auxiliary definitions}
\label{sec:auxil-defin}

In the following we define what we mean by equivalence of patterns (with a
permutation of variables), free variables (for expressions, matches and
patterns), substitution (in expressions) and alpha equivalence (of expressions).

\paragraph{Note.}
\begin{enumerate}
\item
\label{item:note-plusplus}
If $f : A -> B$ and $g : A -> B$ are arbitrary mappings then
\begin{eqnarray*}[rlqTl]
  (g ++ f)(x) &= f(x) & if $x \in \Dom(f)$\\
  (g ++ f)(x) &= g(x) & otherwise
\end{eqnarray*}
and
\[
  \Dom (g ++ f) = \Dom (g) \cup \Dom (f).
\]


\item
We write $p \subseteq p'$ to mean that $p$ is a subpattern of $p'$. More
precisely this is the case if $p = p'$ or if $p' = c \ttt{(} p_1 \ttt{,} \ldots
\ttt{,} p_n \ttt{)}$ and $p \subseteq p_i$ for some $i \in \{1, \ldots, n\}$.

In particular we have\footnote{See \fref{sec:free-variables}
  for the definition of $\FV_{pat}$.} $x \sqsubseteq p$ exactly when $x \in
\FV_{pat}(p)$. The relation is obviously reflexive.

\begin{example}\ \\
  \label{ex:suppattern1}
  Recall that lowercase identifiers are variables, and uppercase ones are
  constructors. Variables as subpatterns:
  \begin{eqnarray*}
    \ttt{x} \sqsubseteq \ttt{A(x,y)} \qquad
    \ttt{y} \sqsubseteq \ttt{A(x,y)} \qquad
    \ttt{z} \not \sqsubseteq \ttt{A(x,y)}
  \end{eqnarray*}
  Patterns as subpatterns:
  \begin{eqnarray*}
    \ttt{A(x,y)} \sqsubseteq \ttt{B(A(x,y),z)} \qquad
    \ttt{A(x,y)} \not \sqsubseteq \ttt{A(A(a,b),c)} \qquad
  \end{eqnarray*}
\end{example}

\item
The syntactic category $pat$ is a proper subset of $exp$. Let $\kappa : pat ->
exp$ be the canonical mapping from $pat$ to $exp$. It is injective so it has a
left inverse $\kappa^{-1} : exp -> pat$. $\kappa^{-1}$ is clearly not total.
\end{enumerate}

\subsection{Equivalence of patterns}
\label{sec:equivalence-patterns}
We say that two patterns are equivalent if they can be transformed into
each other by a suitable renaming of the variables.

If $p_1$ and $p_2$ are equivalent we write $==_\pi$ where $\pi$ is a permutation
of variables, such that for each variable $x$ in $p_1$ its counterpart in $p_2$
is $\pi(x)$.

For example we have $A(x,y) ==_\pi A(z,x)$ where $\pi = [x \mapsto z, y \mapsto x]$.

\fixme{maybe: show that sigma is a permutation of variables}

\begin{definition}[Equivalence of patterns, $==_\pi$]
\label{def:equivalence-patterns}
  \begin{eqnarray}[rlqTl]
    v_1 &==_{\pi} v_2  & where $\pi = [v_1 \mapsto v_2]$ \label{eq:struct-eq-var} \\
    c\ttt{(}p^1_1 \ttt{,} \ldots \ttt{,} p^1_n \ttt{)} & ==_{\pi}
    c\ttt{(}p^2_1 \ttt{,} \ldots \ttt{,} p^2_n \ttt{)} & \label{eq:struct-eq-con}
  \end{eqnarray}
where \fref{eq:struct-eq-con} holds if
\begin{eqnarray*}[c]
  p^1_1 ==_{\pi_1} p^2_1 \\
  \vdots \\
  p^1_n ==_{\pi_n} p^2_n
\end{eqnarray*}
and $\pi = \pi_1 ++ \ldots ++ \pi_n$.

Note that the domains of each of the $\sigma$s are disjoint because no variable
can occur more than once in a pattern (by the definition of the syntax).

We write $==$ to mean $==_\pi$ (with a suitable non-fixed $\pi$) where $\pi$ has no
interest. Equivalence of patterns is defined by $==_\pi$, \emph{only} when $\pi$
is not fixed.
\end{definition}

\begin{example}[Equivalence of patterns, $==_{\pi}$]
  \label{ex:pattern-equiv1}
  \begin{eqnarray*}
    \ttt{A(x,y)} &==_{\pi}& \ttt{A(f,g)}\\
    \ttt{A(f,g)} &==_{\pi'}& \ttt{A(x,y)}
  \end{eqnarray*}
with
  \begin{eqnarray*}
    \pi &=& [x\mapsto f, y \mapsto g] \\
    \pi' &=& [f \mapsto x, g \mapsto y]
\end{eqnarray*}

Whereas these patterns are not equivalent

  \begin{eqnarray*}
    \ttt{A(x,y)} &\not ==& \ttt{B(h,j)} \\
    \ttt{1} &\not ==& \ttt{2}
  \end{eqnarray*}
\end{example}

\fixme{Show equivalence relation: reflexive, symmetric and transitive.}
\fixme{maybe: show $p_1 ==_\pi p_2$ iff $p_2 ==_{\pi^{-1}} p_1$.}


\subsection{Free variables}\label{sec:free-variables}

We denote the free variables of expressions, matches and patterns with the tree
functions $\FV_{exp}$, $\FV_{match}$ and $\FV_{pat}$, respectively.

\begin{definition}[Free variables of expressions, $\FV_{exp}$]\ \\
  Inductively defined:
  \begin{eqnarray}
    \FV_{exp} (v) &=& \{v\} \\
    \FV_{exp} (\fn m) &=& \FV_{match} (m) \\
    \FV_{exp} (e_1e_2) &=& \FV_{exp} (e_1) \cup \FV_{exp} (e_2) \\
    \FV_{exp} (c\ttt{(}e_1\ttt{,} \ldots \ttt{,} e_n \ttt{)}) &=& \FV_{exp}
    (e_1) \cup \ldots \cup \FV_{exp} (e_n)
  \end{eqnarray}
\end{definition}

\begin{definition}[Free variables of matches, $\FV_{match}$]\ \\
  Inductively defined:
  \begin{eqnarray}
    \FV_{match} (\epsilon) &=& \emptyset \\
    \FV_{match} (p\ttt{.}e\ \ttt{|}\ m) &=& \left( \FV_{exp}(e) \setminus
      \FV_{pat}(p) \right) \cup \FV_{match} (m)
  \end{eqnarray}
\end{definition}

\begin{definition}[Free variables of patterns, $\FV_{pat}$] \ \\
  Inductively defined:
  \begin{eqnarray}
    \FV_{pat} (v) &=& \{v\} \\
    \FV_{pat} (c\ttt{(}p_1\ttt{,} \ldots \ttt{,} p_n\ttt{)}) &=& \FV_{pat} (p_1)
    \cup \ldots \cup \FV_{pat} (p_n)
  \end{eqnarray}
\end{definition}

\begin{example}[Free variables, $\mrm{FV}$]
\label{ex:free-variables1}
\begin{eqnarray*}[c]
  \FV_{exp} \left(
    \begin{eqnalign}[Tl]
\begin{lstlisting}
fn Nil . Nil
  | Cons (x, xs) . Cons(f x, g xs)
\end{lstlisting}
    \end{eqnalign}
  \right) = \{\ttt{f}, \ttt{g} \} \\
%
  \FV_{match} \left(
    \begin{eqnalign}[Tl]
\begin{lstlisting}
Cons (x, Nil) . Cons(x, y)
\end{lstlisting}
    \end{eqnalign}
  \right) = \{\ttt{y}\} \\
%
  \FV_{pat} \left(
    \begin{eqnalign}[Tl]
\begin{lstlisting}
Cons (x, xs)
\end{lstlisting}
    \end{eqnalign}
  \right) = \{\ttt{x}, \ttt{xs}\} \\
\end{eqnarray*}
\end{example}


\subsection{Substitution}
We define substitution in expressions. An expression can be substituted for any
(sub)expression of an expressions, not just variables.

\begin{definition}[Substitution]\ \\
  If $e_1$, $e_2$ and $e_3$ are expressions we write $e_1[e_2/e_3]$ to be the
  result of substituting all occurrences of $e_3$ in $e_1$ with $e_2$.
  \begin{eqnarray}
    e_1[e_2/e_3] &=& e_2 \quad \mrm{if}\ e_1 = e_3 \label{eq:subst-sub}\\
    (e^1_1 e^2_1)[e_2/e_3] &=& e^1_1[e_2/e_3] e^2_1[e_2/e_3] \label{eq:subst-app}\\
    \fn p_1 \texttt{.} e_1 \texttt{|} m &=& \fn p_1 \texttt{.} e'_1
    \texttt{|} m' \label{eq:subst-lam}\\
    (c \texttt{(}e^1_1 \texttt{,} \ldots \texttt{,} e^1_n \texttt{)})[e_2/e_3]
    &=& c \texttt{(}e^1_1[e_2/e_3] \texttt{,} \ldots \texttt{,} e^1_n[e_2/e_3]
    \texttt{)} \label{eq:subst-con}
  \end{eqnarray}
Where in \fref{eq:subst-lam} we have
\begin{eqnarray*}[rlqTl]
  e'_1 &= e_1 & if $\kappa^{-1}(e_3) \subseteq p_1$\\
  e'_1 &= e_1[e_2/e_3]
\end{eqnarray*}
and
\[
(\fn m)[e_2/e_3] = \fn m'
\]

In \fref[plain]{eq:subst-app}, \fref[plain]{eq:subst-lam} and
\fref{eq:subst-con}  we require that \fref[plain]{eq:subst-sub} does not apply.

Note that in \fref[plain]{eq:subst-sub} we require $e_1$ and $e_3$ to be exactly
equal, not just alpha equivalent.
\end{definition}

\begin{example}[Substitution]
\label{ex:substituation1}

\begin{eqnarray*}[c]
\left(
  \begin{eqnalign}[Tl]
\begin{lstlisting}
fn x . x
  | y . x
\end{lstlisting}
  \end{eqnalign}
\right) \left[ \ttt{z}/\ttt{x} \right] \quad = \quad
  \begin{eqnalign}[Tl]
\begin{lstlisting}
fn x . x
  | y . z
\end{lstlisting}
  \end{eqnalign} \\
%
\left(
  \begin{eqnalign}[Tl]
\begin{lstlisting}
fn x . Cons (1, x)
\end{lstlisting}
  \end{eqnalign}
\right) \left[ \ttt{z}/\ttt{Cons (1, x)} \right] \quad = \quad
  \begin{eqnalign}[Tl]
\begin{lstlisting}
fn x . z
\end{lstlisting}
  \end{eqnalign} \\
\end{eqnarray*}

whereas this is not

\begin{eqnarray*}
\left(
  \begin{eqnalign}[Tl]
\begin{lstlisting}
fn x . x
\end{lstlisting}
  \end{eqnalign}
\right) [ \ttt{z}/\ttt{x} ] \quad \neq \quad
  \begin{eqnalign}[Tl]
\begin{lstlisting}
fn z . z
\end{lstlisting}
  \end{eqnalign}
\end{eqnarray*}
\end{example}

\subsection{Alpha equivalence}
\label{sec:alpha-equivalence}

We define alpha equivalence for expressions and for pattern-expression pairs. We
use the symbol $==a$ for both relations.

\begin{definition}[Alpha equivalence of expressions, $==a$]\ \\
\label{def:alpha-equivalence}
  First we define alpha equivalence given a mapping of bound variables:
  \begin{eqnarray}
    \sigma |- v_1 &==a& v_2 \label{eq:alpha-var} \\
    \sigma |- e^1_1e^1_2 &==a& e^2_1e^2_1 \label{eq:alpha-exp} \\
    \sigma |- \fn p^1_1 \texttt{.} e^1_1 \texttt{|} \ldots \texttt{|} p^1_n
    \texttt{.} e^1_n &==a& \fn p^2_1 \texttt{.} e^2_1 \texttt{|} \ldots \texttt{|} p^2_n
    \texttt{.} e^2_n \label{eq:alpha-match} \\
    \sigma |- c\ttt{(}e^1_1 \ttt{,} \ldots \ttt{,} e^1_n \ttt{)} &==a&
    c\ttt{(}e^2_1 \ttt{,} \ldots \ttt{,} e^2_n \ttt{)} \label{eq:alpha-con}
  \end{eqnarray}
where \fref{eq:alpha-var} holds
\begin{eqnarray*}[rlqTl]
\sigma (v_1) &= v_2 & if $v_1 \in \Dom(\sigma)$\\
v_1 &= v_2 & otherwise,
\end{eqnarray*}
\fref{eq:alpha-exp} holds if
\[
\sigma |- e^1_1 ==a e^2_1 \land \sigma |- e^1_2 ==a e^2_1,
\]
\fref{eq:alpha-match} holds if
\begin{eqnarray*}
  (p^1_1, e^1_1) &==a& (p^2_1, e^2_1)\\
  &\vdots&\\
  (p^1_n, e^1_n) &==a& (p^2_n, e^2_n)
\end{eqnarray*}
and \fref{eq:alpha-con} holds if
\[
\sigma |- e^1_1 ==a e^1_n \land \ldots \land \sigma |- e^2_1 ==a e^2_n.
\]

\begin{definition}[Alpha equivalence of pattern-expression pairs, $==a$]\ \\
\label{def:alpha-equivalence-patexp}
  Again we assume a mapping of bound variables. It is the case that
  \[
  \sigma |- (p_1, e_1) ==a (p_2, e_2)
  \]
  exactly when $p_1 ==_\pi p_2$ and
  \[
  \sigma ++ \pi |- e_1 ==a e_2
  \]
\end{definition}

If $e_1$ and $e_2$ are alpha equivalent expressions we write $e_1 ==a e_2$ which
is a shorthand for $[] |- e_1 ==a e_2$. Similarly for pattern-expression pairs.
\end{definition}

\begin{example}[Alpha equivalence, $==a$]\ \\
\label{ex:alpha-equivalence1}
  We have
  \begin{eqnarray*}[c]
    \begin{eqnalign}[Tl]
\begin{lstlisting}
fn Nil . Nil
  | Cons (x, xs) . Cons(f x, g xs)
\end{lstlisting}
    \end{eqnalign}
    ==a
    \begin{eqnalign}[Tl]
\begin{lstlisting}
fn Nil . Nil
  | Cons (y, ys) . Cons(f y, g ys)
\end{lstlisting}
    \end{eqnalign}
  \end{eqnarray*}
  
  whereas the example below is not, as the free variables are not the same
  
  \begin{eqnarray*}[c]
    \begin{eqnalign}[Tl]
\begin{lstlisting}
fn Nil . Nil
  | Cons (x, xs) . Cons(f x, g xs)
\end{lstlisting}
    \end{eqnalign}
    \not ==a
    \begin{eqnalign}[Tl]
\begin{lstlisting}
fn Nil . Nil
  | Cons (y, ys) . Cons(h y, j ys)
\end{lstlisting}
    \end{eqnalign}
  \end{eqnarray*}
\end{example}

\section{Semantic equivalence}
\label{sec:semantic-equivalence}
We write $e_1 \sim e_2$ if $e_1$ and $e_2$ are semantically equivalent. That is
if $e_2$ is substituted for $e_1$ (or vice versa) in any program $d$ to obtain
$d'$, then if $d$ evaluates to something in an environment $\sigma$ then $d'$
evaluates to that something in $\sigma$, and if $d$ diverges in $\sigma$ so does
$d'$.

\section{Orderings on patterns}
\label{sec:orderings-patterns}
We define a total ($<=$) relation on patterns, and a partial ($<='$) relation on
the quotient set of patterns by structural equivalence. Then we show that they
indeed are orderings.

We write $<$ and $<'$ for $<=$ and $<='$ strict (or irreflexive) counterparts
respectively.

As it turns out it is easier to define $<$ and $<='$ directly and then define
$<=$ and $<'$ in turn of those.

\begin{definition}[Strict total ordering, $<$]\ \\
  \label{def:pat-total-order-strict}
  Assume a total strict ordering $\lessdot$ on constructors and
  variables\footnote{For example let all constructors come before all variables
    and let variables and constructors be ordered lexicographically among
    themselves. Note that constructors and variables are not compared in the
    definition of $<$.}. We inductively define:
  \begin{eqnarray}
    v_1 &<& v_2 \quad \mrm{if}\ v_1 \lessdot v_2\label{eq:pat-total-order-strict-var}\\
    c\texttt{(}p_1\texttt{,} \ldots\texttt{,} p_n\texttt{)} &<& v\\
    c_1\texttt{(}p_1\texttt{,} \ldots\texttt{,} p_n\texttt{)} &<&
    c_2\texttt{(}p'_1\texttt{,} \ldots\texttt{,} p'_m\texttt{)}\label{eq:pat-total-order-strict-con}
  \end{eqnarray}
  Where \fref{eq:pat-total-order-strict-con} holds if
  \[
  c_1 \lessdot c_2 \lor (c_1 = c_2 \land ( p_1 < p'_1 \lor p_1 = p'_1 \land (\ldots p_n < p'_n \ldots )))
  \]
\end{definition}

\begin{example}[Strict total ordering on patterns, $<$]
  \begin{eqnarray*}[c]
    \begin{eqnalign}[Tl]
\begin{lstlisting}
Cons (y, z)
\end{lstlisting}
    \end{eqnalign}
    <
    \begin{eqnalign}[Tl]
\begin{lstlisting}
Cons (x, z)
\end{lstlisting}
    \end{eqnalign}
    <
    \begin{eqnalign}[Tl]
\begin{lstlisting}
Snoc (a, b)
\end{lstlisting}
    \end{eqnalign}
    <
    \begin{eqnalign}[Tl]
\begin{lstlisting}
a
\end{lstlisting}
    \end{eqnalign}
    <
    \begin{eqnalign}[Tl]
\begin{lstlisting}
b
\end{lstlisting}
    \end{eqnalign}
  \end{eqnarray*}
  Note: Keep in mind that the exact names of the variables are unimportant.
\end{example}

\begin{definition}[Total ordering, $<=$]\ \\
  \label{def:pat-total-order-weak}
  We define the reflexive cousin:
  \begin{eqnarray*}
    p_1 <= p_2 \Longleftrightarrow p_1 < p_2 \lor p_1 = p_2
  \end{eqnarray*}
\end{definition}



\begin{definition}[Partial ordering, $<='$]\ \\
  \label{def:pat-partial-order-weak}
  We say that $p_2$ weakly generalises $p_1$ or $p_1$ is at least as specific as
  $p_2$ and we write $p_1 <=' p_2$. Inductively defined.
  \begin{eqnarray}
    p &<='& v \label{eq:pat-partial-order-weak-var}\\
    c_1\texttt{(}p_1\texttt{,} \ldots\texttt{,} p_n\texttt{)} &<='&
    c_2\texttt{(}p'_1\texttt{,} \ldots\texttt{,} p'_m\texttt{)}
    \label{eq:pat-partial-order-weak-con}
  \end{eqnarray}
  Where \fref{eq:pat-partial-order-weak-con} holds if
  \begin{eqnarray*}
    c_1 &=& c_2 \quad \land\\
    p_1 &<='& p'_1 \quad \land\\
    &\ldots&\\
    p_n &<='& p'_n
  \end{eqnarray*}
\end{definition}



\begin{definition}[Strict partial ordering, $<'$]\ \\
  \label{def:pat-partial-order-strict}
  We define the strict counterpart of $<='$ by
  \begin{eqnarray*}
      p_1 <' p_2 \Longleftrightarrow p_1 <=' p_2 \land p_1 \not == p_2
  \end{eqnarray*}
\end{definition}

\begin{example}[Strict partial ordering on patterns, $<'$]
  \begin{eqnarray*}[c]
    \begin{eqnalign}[Tl]
\begin{lstlisting}
Cons (y, Nil)
\end{lstlisting}
    \end{eqnalign}
    <'
    \begin{eqnalign}[Tl]
\begin{lstlisting}
Cons (x, z)
\end{lstlisting}
    \end{eqnalign}
    <'
    \begin{eqnalign}[Tl]
\begin{lstlisting}
b
\end{lstlisting}
    \end{eqnalign}
  \end{eqnarray*}
  Keeping in mind that the exact names of the variables are unimportant.
\end{example}


\begin{lemma}[Total ordering]\ \\
  \label{lem:pat-total-orderings}
  The relation $<=$ is a total ordering, and $<$ is a strict total ordering on
  patterns.

  Proof is given in \fref{sec:proof-total-orderings}
\end{lemma}


\begin{lemma}[Partial ordering]\ \\
  \label{lem:pat-partial-orderings}
  The relation $<='$ is a partial ordering and $<'$ is a strict partial ordering
  on the equivalence classes of patterns modulo structural equivalence
  ($pat_{/_{==}}$).

  Proof is given in \fref{sec:proof-partial-orderings}
\end{lemma}

We write $p_1 > p_2$, $p_1 >= p_2$, $p_1 >' p_2$ and $p_1 >=' p_2$ to mean $p_2
< p_1$, $p_2 <= p_1$, $p_2 <' p_1$ and $p_2 <=' p_1$ respectively.

\begin{lemma}[]\ \\
  \label{lem:total-implies-partial}
  If two patterns $p_1$ and $p_2$ are ordered by the partial ordering then they
  are also ordered by the total one. That is
  \begin{eqnarray*}
    p_1 <' p_2 \Longrightarrow p_1 < p_2
  \end{eqnarray*}
\end{lemma}
\begin{proof}
  Straightforward using induction.
\end{proof}

\begin{definition}[Confusion, $||$]\ \\
  \label{def:pat-confusion}
  Let two patterns $p_1$ and $p_2$ be given. If it is the case that neither $p_1
  <=' p_2$ nor $p_1 >=' p_2$ we say that $p_1$ and $p_2$ are confused and we
  write $p_1 || p_2$.
\end{definition}

\begin{lemma}[Unique relation]\ \\
  \label{lem:unique-rel}
  Given two patterns $p_1$ and $p_2$ exactly one of the following hold
  \begin{eqnarray*}
    p_1 &==& p_2\\
    p_1 &<'& p_2\\
    p_1 &>'& p_2\\
    p_1 &||& p_2
  \end{eqnarray*}
\end{lemma}
\begin{proof}
  Immediately by inspection.
\end{proof}

\begin{lemma}[]\ \\
  \label{lem:more-specific-confused}
  If $p_1 <' p_2$ and $p_2 || p_3$, then $p_1 || p_3$.

  Proof is given in \fref{sec:proof-partial-orderings}
\end{lemma}


\section{Eliminating unused patterns}
A function is simply a match. And a match is a list of pairs of patterns and
corresponding bodys.

The input to a function is tried against the patterns from top to bottom. An
unused pattern is a pattern that will never see a value which it matches.

\paragraph{Elimination reasons}\ \\
This can happen for two reasons.
\begin{enumerate}
\item The pattern will never be tried against the input because the input
  matches an earlier pattern. \label{item:unused-reason-1}
\item The pattern is only tried against inputs it doesn't match. \label{item:unused-reason-2}
\end{enumerate}

\subsection{Cover}
\label{sec:cover}
\fixme{det her skal vidst lige luges lidt}

We define a cover to be a set of patterns such that for every input at least one
of the patterns will match that input and we write $Cov(P)$ if $P$ is a cover.

\[
  m = p_1\texttt{.}e_1 \texttt{|} \ldots \texttt{|} p_n\texttt{.}e_n
\]

There are two cases if we only look at nullary constructors
\begin{enumerate}
\item Trivially we have a cover with $P = \{p_1,\ldots,p_i\}$ for $1 \leq i \leq
  n$ if $p_i$ is a variable
  
  
\item Or we have a cover with $P = \{p_1,\ldots,p_j\}$ for $1 \leq j \leq n$ if
  the set $C$ formed by the constructors of $P$ and the set $D$ formed of all
  the constructors belonging to the data type of $p_1$ are equal. The set $D$
  can be found by looking up the data type of $p_1$ in $\rho$ and then looking
  up the data type in $\psi$ to get $D$.

\end{enumerate}

\noindent
where the cover $Cov(P)$ will be of which ever set is the smallest.

For constructors of arity $k$, the arguments to a constructor must also form a
cover, resulting in more than one pattern for a given constructor is needed to
form its own cover.

\begin{lemma}\ \\
  Any pattern following a cover is unused because of elimination reason
  \ref{item:unused-reason-1}.
\end{lemma}

\subsection{Shadowed patterns}
\label{sec:shadowed-patterns}
If a pattern is unused because of elimination reason \ref{item:unused-reason-2} we say that
it is shadowed.
\begin{definition}[Shadowed]\ \\
  Let
  \[
  m = p_1\texttt{.}e_1 \texttt{|} \ldots \texttt{|} p_n\texttt{.}e_n
  \]
  If $p_j <=' p_i$ for some $1 \leq i < j \leq n$, then $p_j$ is shadowed (by
  $p_i$).
\end{definition}

\subsection{Elimination}

We can now define the elimination of unused patterns.
\begin{definition}[Elimination, $->e$]\ \\
\label{def:shadowed-patterns-1}
  We define a reduction relation $->e$ that expresses the
  elimination of exactly one pattern from a match.

  Let
  \[
  m = p_1\texttt{.}e_1 \texttt{|} \ldots \texttt{|} p_n\texttt{.}e_n
  \]
  If there exist a $p_i$ such that $\{p_1, \ldots, p_{i-1}\}$ is a cover or
  $p_i$ is shadowed, then it is unused and can be eliminated. The resulting
  match is
  \[
  m' = p_1\texttt{.}e_1 \texttt{|} \ldots \texttt{|}
  p_{j-1}\texttt{.}e_{j-1} \texttt{|} p_{j+1}\texttt{.}e_{j+1} \texttt{|}
  \ldots \texttt{|} p_n\texttt{.}e_n,
  \]
  and we write $m ->e m'$.
\end{definition}

\begin{example}[Elimination, $->e$]\ \\
  The first two patterns make a cover (assuming the only constructors are
  \ttt{Cons} and \ttt{Nil}) so the last pattern is eliminated.
  \begin{eqnarray*}[c]
    \begin{eqnalign}[Tl]
\begin{lstlisting}
  Cons(x, xs) . Cons (x, xs)
| Nil . Nil
| x . x
\end{lstlisting}
    \end{eqnalign}
    ->e
    \begin{eqnalign}[Tl]
\begin{lstlisting}
  Cons(x, xs) . Cons (x, xs)
| Nil . Nil
\end{lstlisting}
    \end{eqnalign}
  \end{eqnarray*}
  The second pattern in the example below is shadowed by the first pattern and
  is thus eliminated.
  \begin{eqnarray*}[c]
    \begin{eqnalign}[Tl]
\begin{lstlisting}
  Cons (x, y) . Cons (x, y)
| Cons (Cons (x, y), z) . Cons (Cons (x, y), z)
| Nil
\end{lstlisting}
    \end{eqnalign}
    ->e
    \begin{eqnalign}[Tl]
\begin{lstlisting}
  Cons (x, y) . Cons (x, y)
| Nil
\end{lstlisting}
    \end{eqnalign}
  \end{eqnarray*}

\end{example}

\begin{lemma}[Preservation]\ \\
  If an unused pattern is removed from a program, then the resulting program is
  semantically equivalent.

  That is
  \[
  m ->e m' ==> \fn m \sim \fn m'
  \]
\end{lemma}

\begin{proof}\ \\
  Trivial (as if).
\end{proof}

\section{Generalisiation}
\fixme{Fatal: This doesn't hold! But it isn't too hard to fix. The problem is
  that generalising a pattern extends its domain: We can only do that when all
  following patterns are confused or shadowed.}

Sometimes patterns get unnecessary complex. If for example a pattern (or one of
its subpatterns) is a constructor pattern whose subpatterns are all variables,
and those variables are only used as arguments to the same constructor (in the
same order) in the function body, then the constructor could simply be replaced
by a fresh variable in pattern and body. That is generalisation of the pattern.

Sometimes the generalisation of a pattern makes it equivalent to another pattern
in the match. And sometimes the two patterns corresponding bodys will merge
seamlessly, such that two patterns can be made to one.
\\[1em]
First we need some auxiliary definitions.

\subsection{Partially ordered form}
\begin{definition}\ \\
  \label{def:part-order-form}
  A match $m = p_1\texttt{.}e_1\texttt{|}\ldots\texttt{|}p_n\texttt{.}e_n$ is in
  partially ordered form if
  \[
  \forall i \in \{1, \ldots, n\} : p_j \not <=' p_i \quad \textnormal{where $j > i$}
  \]
  Note that every match $m$ can be transformed to an equivalent match $m'$ such
  that $m'$ is in partially ordered form, by repeated elimination of shadowed
  patterns (\fref{sec:shadowed-patterns}).
\end{definition}

\subsection{Generalisation of patterns}
We define generalisation of a single pair of a pattern and its body. We write
$(p, e) |> (p', e')$ to mean that the pattern $p$ with its body $e$ generalises
to the pattern $p'$ with the body $e'$.

\begin{definition}[Generalisation of single pattern-body pairs, $|>$]\ \\
  \label{def:gener-patt}
  Inductively defined:
  \begin{eqnarray}
    (c \texttt{(} p_1 \texttt{,} \ldots \texttt{,} p_n \texttt{)} , e) &|>& (x , e[x
    / \kappa (c \texttt{(} p_1 \texttt{,} \ldots \texttt{,} p_n \texttt{)} )]) 
    \label{eq:single-gen-1}\\
    (c \texttt{(} p_1 \texttt{,} \ldots \texttt{,} p_i \texttt{,} \ldots
    \texttt{,} p_n \texttt{)}, e) &|>&
    (c \texttt{(} p_1 \texttt{,} \ldots \texttt{,} p'_i \texttt{,} \ldots
    \texttt{,} p_n \texttt{)}, e') \label{eq:single-gen-2}
  \end{eqnarray}
  Where \fref{eq:single-gen-1} holds when $x$ is a fresh variable and
  \[
  FV_{pat}(c\texttt{(}p_1\texttt{,}\ldots\texttt{,}p_n\texttt{)}) \cap FV_{exp}(e[x/\kappa
  (c\texttt{(}p_1\texttt{,}\ldots\texttt{,}p_n\texttt{)})]) = \emptyset
  \]
  and \fref{eq:single-gen-2} holds when \fref{eq:single-gen-1} does not and
  \[
  (p_i , e) |> (p'_i , e')
  \]
\end{definition}

\begin{lemma}\ \\
  \label{lem:single-gen-imp-gen}
  If a pattern $p$ (and some body) is generalised to $p'$ (and some other
  body), then $p'$ strictly generalises $p$. In other words
  \[
  (p, e) |> (p', e') ==> p <' p'.
  \]
\end{lemma}
\begin{proof}\ \\
  Straightforward induction proof.
\end{proof}

\begin{example}\ \\
  This example shows a match with an unnecessary complex pattern that reverses
  the components of a pair.

  \begin{eqnarray*}[c]
    \begin{eqnalign}[Tl]
\begin{lstlisting}
Pair (Cons (x, xs), zs) . Pair (zs, Cons (x, xs))
\end{lstlisting}
    \end{eqnalign}
    |>
    \begin{eqnalign}[Tl]
\begin{lstlisting}
Pair (xs, zs) . Pair (zs, xs)
\end{lstlisting}
    \end{eqnalign}
  \end{eqnarray*}
\end{example}

\subsection{Generalisation of matches}
For the generalisation of a match $m$, we require $m$ to be in partially ordered
form.

When generalising a pattern several things might happen. Assume

\begin{eqnarray*}[rqTcql]
  m = p_1 \texttt{.} e_1 \texttt{|} \ldots \texttt{|} p_i \texttt{.} e_i
  \texttt{|} \ldots \texttt{|} p_n \texttt{.} e_n & and & (p_i, e_i) |> (p'_i,
  e'_i).
\end{eqnarray*}

 \fixme{n >= i >= 1?}

Now, perhaps $m$ can be generalised if we substitute $p'_i$ for $p_i$ and $e'_i$
for $e_i$. We would like the resulting match to be partially ordered too, so we
must be cautious. Since we know from \fref{lem:single-gen-imp-gen} that $p'_i >'
p_i$ the first part of the match $p_1 \texttt{.} e_1 \texttt{|} \ldots
\texttt{|} p'_i \texttt{.} e'_i$ must still be partially ordered. So we consider
the patterns $p_j$ for $j > i$.

Note that by \fref{lem:unique-rel} we know that either $p_i <' p_j$ or $p_i ||
p_j$ due to $m$ being on partially ordered form.

\paragraph{Generalisation scenarios} \ \\
Four scenarios arise
\begin{enumerate}
\item $p'_i$ and $p_j$ are equivalent. The pattern was generalised to one that
  already existed. Now the only hope is that $e'_i$ and $e_j$ merge. By this we
  mean $(p'_i, e'_i) ==a (p_j, e_j)$ (\Fref{def:alpha-equivalence-patexp}).

  If this is the case then either $p'_i \texttt{.} e'_i$ or $p_j \texttt{.} e_j$
  shall be removed. \label{item:gen-scen-1}
\item $p'_i$ relates to $p_j$ in the same way that $p_i$ does. So $p_i <' p_j
  \Rightarrow p'_i <' p_j$ and $p_i || p_j \Rightarrow p'_i || p_j$. In this
  case nothing must be done. \label{item:gen-scen-2}
\item $p_i || p_j$ and $p'_i >' p_j$. So now $p'_i$ ``steals'' $p_j$s input. But
  because $p_i$ and $p_j$ were confused we know that $p_j$ will not steal any
  input originally intended for $p_i$. So we move $p_j$ and its body up, in the
  match so they come before $p'_i$. \label{item:gen-scen-3}
\item $p_i <' p_j$ and $p'_i >' p_j$. This means that $p'_i$ will match input
  intended for $p_j$ but we can not move $p_j$ above $p'_i$ for then it will
  steal input originally intended for $p_i$. So in this case $m$ can not be
  generalised. \label{item:gen-scen-4}
\end{enumerate}

\begin{definition}[Generalisation, $->g$]\ \\
\label{def:gener-match}
  We define a reduction relation $->g$ that expresses the
  generalisation of exactly one pattern from a match.

  Let
  \begin{eqnarray*}[rqTcql]
    m = p_1 \texttt{.} e_1 \texttt{|} \ldots \texttt{|} p_i \texttt{.} e_i
    \texttt{|} \ldots \texttt{|} p_n \texttt{.} e_n & and & (p_i, e_i) |> (p'_i,
    e'_i).
  \end{eqnarray*}
  and assume that a generalisation as described above can be done. Then the
  resulting match is
  \begin{eqnarray*}[rclqqqTl]
    m' &=& p_1 \texttt{.} e_1 \texttt{|} \ldots \texttt{|} p_{i-1} \texttt{.}
    e_{i-1} & (Untouched)\label{eq:gen-1}\\
    &\texttt{|}& p_{m_1} \texttt{.} e_{m_1} \texttt{|} \ldots \texttt{|} p_{m_k}
    \texttt{.} e_{m_k} & (Scenario \ref{item:gen-scen-3})\label{eq:gen-2}\\
    (&\texttt{|}& p'_i \texttt{.} e'_i \ \ ) 
    & (Perhaps scenario \ref{item:gen-scen-1})\label{eq:gen-3}\\
    &\texttt{|}& p_{s_1} \texttt{.} e_{s_1} \texttt{|} \ldots \texttt{|} p_{s_l}
    \texttt{.} e_{s_l} & (Scenario \ref{item:gen-scen-2})\label{eq:gen-4}
  \end{eqnarray*}

  Where $m_1 < \ldots < m_k$ and $s_1 < \ldots < s_l$.

  The third line is put in parentheses because it should be deleted in the case
  of scenario \ref{item:gen-scen-1}.

  And we write $m ->g m'$.

  \begin{lemma}\ \\
    If $m ->g m'$ then $m'$ is in partially ordered form.
  \end{lemma}
  \begin{proof}\ \\
    Assume
    \begin{eqnarray*}[rqTcql]
      m = p_1 \texttt{.} e_1 \texttt{|} \ldots \texttt{|} p_i \texttt{.} e_i
      \texttt{|} \ldots \texttt{|} p_n \texttt{.} e_n & and & (p_i, e_i) |> (p'_i,
      e'_i).
    \end{eqnarray*}

    Since every $p_j <' p'_i$ for $j > i$ is moved in front of $p'_i$ the only
    thing that can break the partial ordering of $m'$ is if $p_j <' p_k$ for
    $i < j < k$ and $p_k$ moves, but $p_j$ does not. For this to happen $p_j$
    must fall into scenario \ref{item:gen-scen-2} and $p_k$ must fall into
    scenario \ref{item:gen-scen-3}.

    Either $p'_i <' p_j$ or $p'_i || p_j$. In the first case we have by
    transitivity that $p'_i <' p_k$, so $p_k$ can not fall into category
    \ref{item:gen-scen-3}, which is a contradiction.

    In the latter case we have by assumption that $p'_i || p_j$ and $p_k <'
    p'_i$. But then we get by lemma \fref{lem:more-specific-confused} that $p_k
    || p_j$, which is a contradiction.
  \end{proof}

\end{definition}

\begin{example}[Generalisation, $->g$]\ \\
Assume that the constructor environment is $C = \{list \mapsto \{\mathtt{Cons},
\mathtt{Nil} \}, pair \mapsto \{\mathtt{Pair}\} \}$ which maps data types into ,
and that there exists a function \texttt{+}. \fixme{Do this right according to
  the new environment definitions}

Consider

\begin{sml}
val rec addodd = 
fn Cons (Cons (Pair (a, b), Pair (c, d)), xs) .
     Cons (Cons (Pair (a, b), + (Pair (c, d))), addodd xs)
 | Cons (Pair (a, b), Nil) . Cons (Pair (a, b), Nil)
 | Nil . Nil
 | x . x
\end{sml}

As the match is not in partially ordered form (due to the pattern \texttt{x})
our only choice is to eliminate. Luckily we can, because the three first
patterns forms a cover, so we can eliminate the fourth. We get
\begin{sml}
val rec addodd = 
fn Cons (Cons (Pair (a, b), Pair (c, d)), xs) .
     Cons (Cons (Pair (a, b), + (Pair (c, d))), addodd xs)
 | Cons (Pair (a, b), Nil) . Cons (Pair (a, b), Nil)
 | Nil . Nil
\end{sml}
which is in partially ordered form. The second clause can be generalised to
\smlinline{x . x}. This means that the clause \smlinline{Nil . Nil} should be moved
up. We get
\begin{sml}
val rec addodd = 
fn Cons (Cons (Pair (a, b), Pair (c, d)), xs) .
     Cons (Cons (Pair (a, b), + (Pair (c, d))), addodd xs)
 | Nil . Nil
 | x . x
\end{sml}
Next we can generalise \smlinline{Nil . Nil} to \smlinline{x . x} which coincide with
the third clause, so we remove it.
\begin{sml}
val rec addodd = 
fn Cons (Cons (Pair (a, b), Pair (c, d)), xs) .
     Cons (Cons (Pair (a, b), + (Pair (c, d))), addodd xs)
 | x . x
\end{sml}
The first clause can be generalised to

\begin{sml}
Cons (Cons (x, y), xs) . Cons (Cons (x, + y), addodd xs)
\end{sml}
\noindent (by generalising the two \texttt{Pair} patterns), so we get

\begin{sml}
val rec addodd = 
fn Cons (Cons (x, y), xs) .
     Cons (Cons (x, + y), addodd xs)
 | x . x
\end{sml}
which cannot be generalised or eliminated upon. So it is a normal form.

\end{example}

\section{Normal form}
We say that the function $\fn m$ is a normal form if there does not exist an
$m'$ such that $m ->e m'$ or $m ->g m'$.

\subsection{Reducing to normal form}
Consider a function $\fn m$. It can be converted to a normal form by repeatedly
eliminating and generalising patterns.

Note that $m$ should be in partially ordered form (\fref{def:part-order-form}) in
order to generalise its patterns. Luckily we can convert it to a partially
ordered form by repeated elimination.

\begin{lemma}\ \\
  If a function $f$ can be converted to the normal form $f'$, then $f$ and $f'$
  are semantically equivalent.
\end{lemma}

\begin{lemma}\ \\
  We expect the normal form to be unique (in some sense).
\end{lemma}

\begin{example}\ \\
Consider 
\begin{sml}
val f = 
fn A (A (a, b), c) . A (c, A (a, b))
 | A (a, b)        . A (b, a)
 | B               . B
 | x               . x
 | A (a, A (b, c))  . A (A (c, b), a)
\end{sml}
Assume that the only constructors are \texttt{A} and \texttt{B}.

As the match is not in partially ordered form (due to the last pattern being
more specific than the pattern \texttt{x}) our only choice is to
eliminate. Luckily we can, because the four (three actually) first patterns
forms a cover. With the last clause eliminated we have

\begin{sml}
val f = 
fn A (A (a, b), c) . A (c, A (a, b))
 | A (a, b)        . A (b, a)
 | B               . B
 | x               . x
\end{sml}

From here we can choose to eliminate or generalise. We generalise and it is the
case that
\[
(\texttt{A (A (a, b), c)}, \texttt{A (c, A (a, b))}) |> (\texttt{A (x, y)},
\texttt{A (y, x)})
\]

We arrive at generalisation scenario \ref{item:gen-scen-1} since $\texttt{A (a, b)} ==
\texttt{A (x, y)}$. So we must have 
\[ 
(\texttt{A (a, b)}, \texttt{A (b, a)}) 
==a (\texttt{A (x, y)}, \texttt{A (y, x)})
\]
which is the case. So the first clause can be deleted. We now have

\begin{sml}
val f = 
fn A (a, b) . A (b, a)
 | B        . B
 | x        . x
\end{sml}

Since the two first patterns form a cover the last clause can be eliminated. We
end up with

\begin{sml}
val f = 
fn A (a, b) . A (b, a)
 | B        . B
\end{sml}

which cannot be generalised or eliminated further, so it is a normal form.
\end{example}


\section{Weak unification}
\fixme{To do or not to do?}



%%% Local Variables: 
%%% mode: latex
%%% reftex-fref-is-default: t
%%% TeX-master: "../report"
%%% End: 


\subsection{Examples}
\label{sec:eval-normalform-examples}

Some of the verbose text prints parts of the AST. This is done in the same way
as the \ttt{SourceExplorer} application. Indentation is used to mark children of
a node and numbering is used to indicate ordering. Thus the function call
\ttt{foo (xs, ys)} would be written as

\begin{sml}
Exp_App
  1 Exp_Var foo(id: 34765, is: Val)
  2 Exp_Tuple
      1 Exp_Var ys(id: 34768, is: Val)
      2 Exp_Var xs(id: 34767, is: Val)  
\end{sml}

where the tuple \ttt{(id: 34765, is: Val)} given for all \ttt{Exp_Var} indicates
the variable has (unique) id 34765 and is a value. It can either be a value,
constructor or an exception. The id is used to decide whether two variables are
the same (in effect a unique renaming of all variables).

\begin{example}[Non possible rewriting -- foo]\
  \label{ex:eval-map-rewrite-foo}\\

  \begin{center}
    \begin{tabular}{|l|}
      \hline
      \textbf{Original function}
      \\\hline
      \begin{sml}
fun foo (x :: xs, ys) = x :: foo (ys, xs)
  | foo (nil, _) = nil
      \end{sml}
      \\\hline
      \textbf{Normalised function}
      \\\hline
      \begin{sml}
fun foo (x :: xs, ys) = x :: foo (ys, xs)
  | foo y = nil        
      \end{sml}
      \\\hline
    \end{tabular}
  \end{center}

  \noindent
  This function swaps the \ttt{xs} and \ttt{ys} arguments for each recursive
  call and thus does not match any of the defined \tsf{map} rewriting
  rules. This is seen when $\ol{xs}$ is inserted into the hole of the context
  $\mathcal{C}$ and we don't get a recursive call that matches that of the
  original code.

  \begin{narrow}{-3em}{0em}
    \setlength{\linewidth}{1.2\linewidth}
    \footnotesize
    
    \begin{multicols}{2}           
      \begin{sml}
Normalising and rewriting: test.sml
self -> foo(id: 34765, is: Val)
Number of found contexts: 1

(@\cmt{AST of the original recursive call}@)
Original recursive call:
Exp_App
  1 Exp_Var foo(id: 34765, is: Val)
  2 Exp_Tuple
      1 Exp_Var ys(id: 34768, is: Val)
      2 Exp_Var xs(id: 34767, is: Val)


Recursive call with xs inserted into the hole
of the context:
(@\cmt{It is seen that the resulting recursive call\ 
       has xs and ys switched and thus does not match}@)
Exp_App
  1 Exp_Var foo(id: 34765, is: Val)
  2 Exp_Tuple
      1 Exp_Var xs(id: 34767, is: Val)
      2 Exp_Var ys(id: 34768, is: Val)
Recursive call does NOT match original.

Before:
fun foo (x :: xs, ys) = x :: foo (ys, xs)
  | foo (nil, _) = nil

(@\cmt{Thus we are not able to rewrite the function\ 
       but only return its normal form.}@)
After:
fun foo (x :: xs, ys) = x :: foo (ys, xs)
  | foo y = nil
      \end{sml}
    \end{multicols}
  \end{narrow} 
\end{example}


\begin{example}[Rewriting non exhaustive with two contexts -- bar]\
  \label{ex:eval-map-rewrite-bar}\\

  \begin{center}
    \begin{tabular}{|l|}
      \hline
      \textbf{Original function}
      \\\hline
      \begin{sml}
fun bar (xs as x :: _, y :: ys) = y + x :: bar (xs, ys)
  | bar (_ :: _, nil) = nil
      \end{sml}
      \\\hline
      \textbf{Normalised function}
      \\\hline
      \begin{sml}
fun bar (xs as x :: _, y :: ys) = y + x :: bar (xs, ys)
  | bar (_ :: _, nil) = nil        
      \end{sml}
      \\\hline
    \end{tabular}
  \end{center}

  \noindent
  This function is obviously not exhaustive, as also noted by the
  \ttt{RewriteMap} application. This does however not influence the validity of
  the rewriting produced in this case as the two clauses actually does cover the
  same domain\footnote{Remember that this implementation does not check the
    \ttt{where} condition of the map rule as described in
    \fref[plain]{sec:using-current-features}.}.

  In this function we have deliberately chosen that the \ttt{ys} variable is the
  one being changed in the recursive call and thus this is where the hole should
  be in the context. The algorithm that tries to find the correct context places
  the holes from left to right. The effect of this is that it will find an
  invalid context at first, but second time it will find a valid one.

  \begin{narrow}{-3em}{0em}
    \setlength{\linewidth}{1.2\linewidth}
    \footnotesize
    
    \begin{multicols}{2}           
      \begin{sml}
Normalising and rewriting: test1.sml
Match is not exhaustive
  (x :: z, y :: ys) = y + x :: bar ((x :: z), ys)
  (x :: y, nil) = nil
self -> bar(id: 34769, is: Val)
(@\cmt{As described two contexts are found}@)
Number of found contexts: 2

Original recursive call:
Exp_App
  1 Exp_Var bar(id: 34769, is: Val)
  2 Exp_Tuple
      1 Exp_Par
          1 Exp_App
              1 Exp_Var ::(id: 4, is: Con)
              2 Exp_Tuple
                  1 Exp_Var x(id: 34770, is: Val)
                  2 Exp_Var z(id: 34774, is: Val)
      2 Exp_Var ys(id: 34773, is: Val)
Recursive call with xs inserted into the hole
of the context:
(@\cmt{Here the wrong context is found, as when inserting\ 
  $\ol{xs}$ into the hole it generates a recursive call that\
  is not a match to the original.}@)
Exp_App
  1 Exp_Var bar(id: 34769, is: Val)
  2 Exp_Tuple
      1 Exp_Var z(id: 34774, is: Val)
      2 Exp_App
          1 Exp_Var ::(id: 4, is: Con)
          2 Exp_Tuple
              1 Exp_Var y(id: 34772, is: Val)
              2 Exp_Var ys(id: 34773, is: Val)
Recursive call does NOT match original.

Original recursive call:
Exp_App
  1 Exp_Var bar(id: 34769, is: Val)
  2 Exp_Tuple
      1 Exp_Par
          1 Exp_App
              1 Exp_Var ::(id: 4, is: Con)
              2 Exp_Tuple
                  1 Exp_Var x(id: 34770, is: Val)
                  2 Exp_Var z(id: 34774, is: Val)
      2 Exp_Var ys(id: 34773, is: Val)
Recursive call with xs inserted into the hole
of the context:
(@\cmt{Where as the second context found is a correct match}@)
Exp_App
  1 Exp_Var bar(id: 34769, is: Val)
  2 Exp_Tuple
      1 Exp_App
          1 Exp_Var ::(id: 4, is: Con)
          2 Exp_Tuple
              1 Exp_Var x(id: 34770, is: Val)
              2 Exp_Var z(id: 34774, is: Val)
      2 Exp_Var ys(id: 34773, is: Val)
Recursive call match original.

Before:
fun bar (xs as x :: _, y :: ys) = y + x :: bar (xs, ys)
  | bar (_ :: _, nil) = nil

(@\cmt{And thus we can rewrite the function according to the map rule}@)
After:
fun bar (x :: z, ys) = map (fn y => y + x) ys
      \end{sml}
    \end{multicols}
  \end{narrow} 
\end{example}

%%% Local Variables: 
%%% mode: latex
%%% TeX-master: "../../report"
%%% End: 


\section{Further work}
\label{sec:further-work}
\begin{description}
\item[Formal Proofs] We have provided some proofs, but properties concerned with
  especially semantics (and preserving thereof) are left unproven.

\item[Extend normal forms to full SML] In our implementation we are only concerned
  with \ttt{fun ...}-declarations and we do not handle records (flexible or
  otherwise). To fully handle records one would have to perform type inference
  (see \fref{sec:preparation}).

\item[Modules] Turtledove does currently not implement all the described
  components of the design (see \fref{fig:turtledove-design}). 
  \begin{description}
  \item[Project manager] This has almost been implemented, but could do with a
    complete rewrite. Currently it does not compile due to changes in MyLib's
    JSON module. Also it only parses turtledove project files.

  \item[Communication bridge] Needs implementing.

  \item[Source Management] Needs implementing and needs investigation of cashing
    possibilities. Example could be not to re parse files that haven't changed
    such as the basis library (which rarely changes). Use the parser to do
    incremental parsing such that only newly inserted code is parsed.

  \item[SML parser] Better reporting of erroneous SML code. For example when
    multiple clauses of a function is defined using different number of
    arguments because parenthesis are missing around constructors or lists,
    which is a common by novice programmers.
  \end{description}

\item[Tools] Create useful tools that can work with the AST and interact the
  user and the development environment.

  \begin{description}
  \item[Todo] Merge the already created todo application into a tool, so the
    user may get a list of ``TODO'''s in currently open project.

  \item[Refactoring] Implement various refactoring commands such as renaming.

  \item[Auto completion] Use all information available in the environment to
    ease code creation for the developer. Connect this with ML-Doc to also show
    documentation of auto completed functions and others.

  \item[Source explorer] Merge the already created source explorer such that the
    user may get this information displayed, mimicking the behaviour of a ``class
    explorer'' from the development environment's of object oriented languages. 
  \end{description}

\item[Rewriting tool] Implement rewriting as a tool which will load a set of
  rewriting rules from files when started, such that rules may be added or
  removed without changing the source code. This includes figuring out how to
  automaticly find the transformers and contexts and the following:
  \fixme{transformers and contexts}
  \begin{description}
  \item[Rule Parser] The rule parser does currently not compile against the
    latest AST, due to some changes of the wrap. Also meta variables needs to be
    parsed and added to the BNF. We imagine a syntax of \ttt{_xs} (normal
    identifier prefixed with an underscore) to mean $\ol{xs}$.

  \item[Rules] Currently there are only defined a few rewriting rules (map and
    fold). For proper use a wide variety of rules needs to be defined. This
    includes the set of simplification (expression) rules.
  \end{description}

\item[Development Environment] Write plugins/extensions to existing development
  environments such that they may integrade Turtledove. Our original idea was to
  write bindings to emacs.


\item[AST] Currently every node in the AST have a copy of the environment as it
  looks at this point in the code. This means that the basis library is stored
  quite a lot of places which may not be the best way of doing it. Just as an
  indicator, running the current source explorer application on basis.mlb (The
  basis library) consumes roughly 300MB of memory and roughly 1GB on MyLib.mlb.

\item[Threading] Look at threading the tools and the various parses such that
  they don't influence the responsiveness of Turtledove when communicating with
  the development environment.


\item[Documentation] Document the internals of Turtledove and write examples and
  comments for the MyLib library.

\end{description}

%%% Local Variables: 
%%% mode: latex
%%% TeX-master: "../report"
%%% End: 



%%% Local Variables: 
%%% mode: latex
%%% TeX-master: "../report"
%%% End: 



%%%%%%%%%%%%%%%%%%%%%%%%%%%%%%%%%%%%%%%%%%%%%%%%
\chapter{Conclusions}

%%% Local Variables: 
%%% mode: latex
%%% TeX-master: "../report"
%%% End: 



\bibliographystyle{bibliography/theseurl}
\bibliography{bibliography/bibliography}

\appendix


%%%%%%%%%%%%%%%%%%%%%%%%%%%%%%%%%%%%%%%%%%%%%%%%
\chapter{Figures}


\section{Protocols}
\label{sec:protocols}

\subsection{JSON}
\label{sec:protocol-json}


\begin{nonfloatingfigure}


  \begin{grammar}
    <JSON-value> ::= \[[
    \begin{stack}
      <JSON-string> \\
      <JSON-number> \\
      <JSON-object> \\
      <JSON-array> \\
      "true" \\
      "false" \\
      "null"
    \end{stack}
    \]]

    <JSON-string> ::= \[[
    "\""
    \begin{stack}
      \\
      \fbox {
        \begin{minipage}{1,6in}
          \begin{center}
            Any\ unicode\ char\ except\ '{\char 34}',\ '$\backslash$'\
            and\ any\ control\ chars
          \end{center}
        \end{minipage}
      }
      \\
      "\\"
      \begin{stack}
        "\"" \\
        "\\" \\
        "/" \\
        "b" \\
        "f" \\
        "n" \\
        "r" \\
        "t" \\
        "u"
      \end{stack}
    \end{stack}
    "\""
    \]]

    <JSON-object> ::= \[[
    "{"
      \begin{stack}
        \\

        \begin{rep}
          <JSON-string> ":" <JSON-value> \\
          ","
        \end{rep}
      \end{stack}
      "}"
    \]]

    <JSON-array> ::= \[[
    "["
    \begin{stack}
      \\
      \begin{rep}
        <JSON-value> \\
        ","
      \end{rep}
    \end{stack}
    "]"
    \]]

    <JSON-number> ::= \[[
    \begin{stack}
      \\
      "-"
    \end{stack}
    \begin{stack}
      "0" \\
      1--9
      \begin{rep}
        \\
        0--9
      \end{rep}
    \end{stack}
    \begin{stack}
      \\
      "."
      \begin{rep}
        \\
        0--9
      \end{rep}
    \end{stack}
    \begin{stack}
      \\
      \begin{stack}
        e \\
        E
      \end{stack}
      \begin{stack}
        \\
        "+" \\
        "-"
      \end{stack}
      \begin{rep}
        \\
        0--9
      \end{rep}
    \end{stack}
    \]]

  \end{grammar}

  \caption{Definition of JSON as described 10th of January 2010 at
    \url{www.json.org} \cite{json}}
  \label{fig:protocol-json}
\end{nonfloatingfigure}


%%% Local Variables: 
%%% mode: latex
%%% TeX-master: "../../../report"
%%% reftex-fref-is-default: t
%%% End: 



%%% Local Variables: 
%%% mode: latex
%%% TeX-master: "../../report"
%%% End: 



\chapter{...}



\end{document}

%%% Local Variables: 
%%% mode: latex
%%% TeX-master: t
%%% reftex-fref-is-default: t
%%% End: 
