\subsection{Project file}
\label{sec:protocol-project-file}


\begin{nonfloatingfigure}

% \grammarparsep % amount of space between productions
\grammarindent11em % the indent. Needs to be wider if productions get bigger.
\begin{grammar}

<Strings-Array> ::= \[[
    "["
    \begin{stack}
      \\
      \begin{rep}
        <JSON-String> \\
        ","
      \end{rep}
    \end{stack}
    "]"
    \]]

<Project-File-Object> ::= "{" \\
  "  " "\"ProjectName\"" ":" <JSON-String> "," \\
  "  " "\"Properties\"" ":" <JSON-Object> "," \\
  "  " "\"ProjectValue\"" ":" <Value-Object> \\
  "}"

<Value-Object> ::= "{" \\
  "  " "\"Exposes\"" ":" <Strings-Array> "," \\
  "  " "\"Depends\"" ":" <Depends-Array> "," \\
  "  " "\"Nodes\"" ":" <Nodes-Array> \\
  "}"

<Depends-Array> ::= \[[
    "["
    \begin{stack}
      \\
      \begin{rep}
        <Depend-Object> \\
        ","
      \end{rep}
    \end{stack}
    "]"
    \]]

<Depend-Object> ::= "{" \\
  "  " "\"Name\"" ":" <JSON-String> "," \\
  "  " "\"Depends\"" ":" <Strings-Array> "," \\
  "}"

<Nodes-Array> ::= \[[
    \begin{stack}
      <String-Array> \\
      <Node-Groups-Array>
    \end{stack}
    \]]

<Node-Groups-Array> ::= \[[
    "["
    \begin{stack}
      \\
      \begin{rep}
        <Node-Group-Object> \\
        ","
      \end{rep}
    \end{stack}
    "]"
    \]]

<Node-Group-Object> ::= "{" \\
  "  " "\"Name\"" ":" <JSON-String> "," \\
  "  " "\"Value\"" ":" <Value-Object> "," \\
  "}"
\end{grammar}


\caption{Definition of the project file used in Turtledove}
\label{fig:protocol-project-file}
\end{nonfloatingfigure}

Change ``ProjectValue'' to be a ``ProjectNode'' insted as This restrains all
file to be grouped and thus have a group name as of where to add/remove files
to/from.

Remove ``Projectname'' as this then will be defined as the outer most node
(``ProjectNode'') with ``name'' = ``ProjectName'' .


\subsubsection{Examples}

\begin{example}\ 

  Project file with three individual files (x.sml, y.sml and z.sml) and two
  groupings (A and B) where y.sml depends on z.sml, z.sml on x.sml and group A
  on group B.

  This examples demonstrates how the \textit{<Nodes>} field can be both a string
  list of filenames and it can be a object list of Node-Groups

...

"ProjectNode" :
{
        "Name"  : "FooBar",
        "Value" :
        {
          "Exposes" : [ "A" ]
          "Depends" : [ { "Name": "A", "Depends" : ["B"] } ],
          "Nodes"   : 
          [
      {
        "Name"  : "A",
        "Value" :
        {
          "Exposes" : ["x.sml"]
          "Depends" : []
          "Nodes"   : ["x.sml"]
        }
      },
      {
        "Name"  : "B",
        "Value" :
        {
          "Exposes" : ["y.sml", "z.sml"]
          "Depends" : 
          [ 
            { "Name" : "y.sml", "Depends" : ["z.sml"]},
            { "Name" : "z.sml", "Depends" : ["x.sml"]} 
          ]
          "Nodes"   : ["y.sml", "z.sml"]
        }
      }          
          ]
        }
},


\begin{lstlisting}
{
  "ProjectName"   : "FooBar",
  "Properties"    : { "version" : "0.0 alfa" },
  "ProjectValue" :
  {
    "Exposes" : ["A"],
    "Depends" : [ { "Name": "A", "Depends" : ["B"] } ],
    "Nodes"   : 		
    [
      {
        "Name"  : "A",
        "Value" :
        {
          "Exposes" : ["x.sml"]
          "Depends" : []
          "Nodes"   : ["x.sml"]
        }
      },
      {
        "Name"  : "B",
        "Value" :
        {
          "Exposes" : ["y.sml", "z.sml"]
          "Depends" : 
          [ 
            { "Name" : "y.sml", "Depends" : ["z.sml"]},
            { "Name" : "z.sml", "Depends" : ["x.sml"]} 
          ]
          "Nodes"   : ["y.sml", "z.sml"]
        }
      }
    ]
}
\end{lstlisting}

\end{example}


%%% Local Variables: 
%%% mode: latex
%%% TeX-master: "../../../report"
%%% End: 
