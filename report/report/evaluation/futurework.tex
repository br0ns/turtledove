\section{Future work}

\begin{description}
\item[Formal Proofs] We have provided some proofs, but properties concerned with
  especially semantics (and preserving thereof) are left unproven.

\item[Extend normal forms to full SML] In our implementation we are only concerned
  with \ttt{fun ...}-declarations and we do not handle records (flexible or
  otherwise). To fully handle records one would have to perform type inference
  (see \fref{sec:preparation}).

\item[Modules] Turtledove does currently not implement all the described
  components of the design (see \fref{fig:turtledove-design}). 
  \begin{description}
  \item[Project manager] This has almost been implemented, but could do with a
    complete rewrite. Currently it does not compile due to changes in MyLib's
    JSON module. Also it only parses turtledove project files.

  \item[Communication bridge] Needs implementing.

  \item[Source Management] Needs implementing and needs investigation of cashing
    possibilities. Example could be not to re parse files that haven't changed
    such as the basis library (which rarely changes). Use the parser to do
    incremental parsing such that only newly inserted code is parsed.

  \item[SML parser] Better reporting of erroneous SML code. For example when
    multiple clauses of a function is defined using different number of
    arguments because parenthesis are missing around constructors or lists,
    which is a common by novice programmers.
  \end{description}

\item[Tools] Create useful tools that can work with the AST and interact the
  user and the development environment.

  \begin{description}
  \item[Todo] Merge the already created todo application into a tool, so the
    user may get a list of ``TODO'''s in currently open project.

  \item[Refactoring] Implement various refactoring commands such as renaming.

  \item[Auto completion] Use all information available in the environment to
    ease code creation for the developer. Connect this with ML-Doc to also show
    documentation of auto completed functions and others.

  \item[Source explorer] Merge the already created source explorer such that the
    user may get this information displayed, mimicking the behaviour of a ``class
    explorer'' from the development environment's of object oriented languages. 
  \end{description}

\item[Rewriting tool] Implement rewriting as a tool which will load a set of
  rewriting rules from files when started, such that rules may be added or
  removed without changing the source code. This includes figuring out how to
  automaticly find the transformers and contexts and the following:
  \fixme{transformers and contexts}
  \begin{description}
  \item[Rule Parser] The rule parser does currently not compile against the
    latest AST, due to some changes of the wrap. Also meta variables needs to be
    parsed and added to the BNF. We imagine a syntax of \ttt{_xs} (normal
    identifier prefixed with an underscore) to mean $\ol{xs}$.

  \item[Rules] Currently there are only defined a few rewriting rules (map and
    fold). For proper use a wide variety of rules needs to be defined. This
    includes the set of simplification (expression) rules.
  \end{description}

\item[Development Environment] Write plugins/extensions to existing development
  environments such that they may integrade Turtledove. Our original idea was to
  write bindings to emacs.


\item[AST] Currently every node in the AST have a copy of the environment as it
  looks at this point in the code. This means that the basis library is stored
  quite a lot of places which may not be the best way of doing it. Just as an
  indicator, running the current source explorer application on basis.mlb (The
  basis library) consumes roughly 300MB of memory and roughly 1GB on MyLib.mlb.

\item[Threading] Look at threading the tools and the various parses such that
  they don't influence the responsiveness of Turtledove when communicating with
  the development environment.


\item[Documentation] Document the internals of Turtledove and write examples and
  comments for the MyLib library.

\end{description}

%%% Local Variables: 
%%% mode: latex
%%% TeX-master: "../report"
%%% End: 
