\subsection{Examples}
\label{sec:eval-normalform-examples}

Some of the verbose text prints parts of the AST. This is done in the same way
as the \ttt{SourceExplorer} application. Indentation is used to mark children of
a node and numbering is used to indicate ordering. Thus the function call
\ttt{foo (xs, ys)} would be written as

\begin{sml}
Exp_App
  1 Exp_Var foo(id: 34765, is: Val)
  2 Exp_Tuple
      1 Exp_Var ys(id: 34768, is: Val)
      2 Exp_Var xs(id: 34767, is: Val)  
\end{sml}

where the tuple \ttt{(id: 34765, is: Val)} given for all \ttt{Exp_Var} indicates
the variable has (unique) id 34765 and is a value. It can either be a value,
constructor or an exception. The id is used to decide whether two variables are
the same (in effect a unique renaming of all variables).

\begin{example}[Non possible rewriting -- foo]\
  \label{ex:eval-map-rewrite-foo}\\

  \begin{center}
    \begin{tabular}{|l|}
      \hline
      \textbf{Original function}
      \\\hline
      \begin{sml}
fun foo (x :: xs, ys) = x :: foo (ys, xs)
  | foo (nil, _) = nil
      \end{sml}
      \\\hline
      \textbf{Normalised function}
      \\\hline
      \begin{sml}
fun foo (x :: xs, ys) = x :: foo (ys, xs)
  | foo y = nil        
      \end{sml}
      \\\hline
    \end{tabular}
  \end{center}

  \noindent
  This function swaps the \ttt{xs} and \ttt{ys} arguments for each recursive
  call and thus does not match any of the defined \tsf{map} rewriting
  rules. This is seen when $\ol{xs}$ is inserted into the hole of the context
  $\mathcal{C}$ and we don't get a recursive call that matches that of the
  original code.

  \begin{narrow}{-3em}{0em}
    \setlength{\linewidth}{1.2\linewidth}
    \footnotesize
    
    \begin{multicols}{2}           
      \begin{sml}
Normalising and rewriting: test.sml
self -> foo(id: 34765, is: Val)
Number of found contexts: 1

(@\cmt{AST of the original recursive call}@)
Original recursive call:
Exp_App
  1 Exp_Var foo(id: 34765, is: Val)
  2 Exp_Tuple
      1 Exp_Var ys(id: 34768, is: Val)
      2 Exp_Var xs(id: 34767, is: Val)


Recursive call with xs inserted into the hole
of the context:
(@\cmt{It is seen that the resulting recursive call\ 
       has xs and ys switched and thus does not match}@)
Exp_App
  1 Exp_Var foo(id: 34765, is: Val)
  2 Exp_Tuple
      1 Exp_Var xs(id: 34767, is: Val)
      2 Exp_Var ys(id: 34768, is: Val)
Recursive call does NOT match original.

Before:
fun foo (x :: xs, ys) = x :: foo (ys, xs)
  | foo (nil, _) = nil

(@\cmt{Thus we are not able to rewrite the function\ 
       but only return its normal form.}@)
After:
fun foo (x :: xs, ys) = x :: foo (ys, xs)
  | foo y = nil
      \end{sml}
    \end{multicols}
  \end{narrow} 
\end{example}


\begin{example}[Rewriting non exhaustive with two contexts -- bar]\
  \label{ex:eval-map-rewrite-bar}\\

  \begin{center}
    \begin{tabular}{|l|}
      \hline
      \textbf{Original function}
      \\\hline
      \begin{sml}
fun bar (xs as x :: _, y :: ys) = y + x :: bar (xs, ys)
  | bar (_ :: _, nil) = nil
      \end{sml}
      \\\hline
      \textbf{Normalised function}
      \\\hline
      \begin{sml}
fun bar (xs as x :: _, y :: ys) = y + x :: bar (xs, ys)
  | bar (_ :: _, nil) = nil        
      \end{sml}
      \\\hline
    \end{tabular}
  \end{center}

  \noindent
  This function is obviously not exhaustive, as also noted by the
  \ttt{RewriteMap} application. This does however not influence the validity of
  the rewriting produced in this case as the two clauses actually does cover the
  same domain\footnote{Remember that this implementation does not check the
    \ttt{where} condition of the map rule as described in
    \fref[plain]{sec:using-current-features}}.

  In this function we have deliberately chosen that the \ttt{ys} variable is the
  one being changed in the recursive call and thus this is where the hole should
  be in the context. The algorithm that tries to find the correct context places
  the holes from left to right. The effect of this is that it will find an
  invalid context at first, but second time it will find a valid one.

  \begin{narrow}{-3em}{0em}
    \setlength{\linewidth}{1.2\linewidth}
    \footnotesize
    
    \begin{multicols}{2}           
      \begin{sml}
Normalising and rewriting: test1.sml
Match is not exhaustive
  (x :: z, y :: ys) = y + x :: bar ((x :: z), ys)
  (x :: y, nil) = nil
self -> bar(id: 34769, is: Val)
(@\cmt{As described two contexts are found}@)
Number of found contexts: 2

Original recursive call:
Exp_App
  1 Exp_Var bar(id: 34769, is: Val)
  2 Exp_Tuple
      1 Exp_Par
          1 Exp_App
              1 Exp_Var ::(id: 4, is: Con)
              2 Exp_Tuple
                  1 Exp_Var x(id: 34770, is: Val)
                  2 Exp_Var z(id: 34774, is: Val)
      2 Exp_Var ys(id: 34773, is: Val)
Recursive call with xs inserted into the hole
of the context:
(@\cmt{Here the wrong context is found, as when inserting\ 
  $\ol{xs}$ into the hole it generates a recursive call that\
  is not a match to the original.}@)
Exp_App
  1 Exp_Var bar(id: 34769, is: Val)
  2 Exp_Tuple
      1 Exp_Var z(id: 34774, is: Val)
      2 Exp_App
          1 Exp_Var ::(id: 4, is: Con)
          2 Exp_Tuple
              1 Exp_Var y(id: 34772, is: Val)
              2 Exp_Var ys(id: 34773, is: Val)
Recursive call does NOT match original.

Original recursive call:
Exp_App
  1 Exp_Var bar(id: 34769, is: Val)
  2 Exp_Tuple
      1 Exp_Par
          1 Exp_App
              1 Exp_Var ::(id: 4, is: Con)
              2 Exp_Tuple
                  1 Exp_Var x(id: 34770, is: Val)
                  2 Exp_Var z(id: 34774, is: Val)
      2 Exp_Var ys(id: 34773, is: Val)
Recursive call with xs inserted into the hole
of the context:
(@\cmt{Where as the second context found is a correct match}@)
Exp_App
  1 Exp_Var bar(id: 34769, is: Val)
  2 Exp_Tuple
      1 Exp_App
          1 Exp_Var ::(id: 4, is: Con)
          2 Exp_Tuple
              1 Exp_Var x(id: 34770, is: Val)
              2 Exp_Var z(id: 34774, is: Val)
      2 Exp_Var ys(id: 34773, is: Val)
Recursive call match original.

Before:
fun bar (xs as x :: _, y :: ys) = y + x :: bar (xs, ys)
  | bar (_ :: _, nil) = nil

(@\cmt{And thus we can rewrite the function according to the map rule}@)
After:
fun bar (x :: z, ys) = map (fn y => y + x) ys
      \end{sml}
    \end{multicols}
  \end{narrow} 
\end{example}

%%% Local Variables: 
%%% mode: latex
%%% TeX-master: "../../report"
%%% End: 
