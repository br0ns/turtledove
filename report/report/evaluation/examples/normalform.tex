\subsection{Examples}
\label{sec:eval-normalform-examples}


\begin{example}[Complex function getting simpler -- f]\
  \label{ex:eval-normal-example-f}\\
  
  \begin{center}
    \begin{tabular}{|l|}
      \hline
      \textbf{Original function}
      \\\hline
      \begin{sml}
fun f (SOME (SOME (a, b), c)) = SOME (c, SOME (a, b))
  | f (SOME (a, b)) = SOME (b, a)
  | f NONE = NONE
  | f x = x
  | f (SOME (a, SOME (b, c))) = SOME (SOME (c, b), a)        
      \end{sml}
      \\\hline
    \end{tabular}
  \end{center}
  
  This rather complex function, that switches the pair $(a,\ b)$ to $(b,\ a)$ inside a
  \ttt{SOME} constructor, can be drastically simplified.
  
  \begin{narrow}{-3em}{0em}
    \setlength{\linewidth}{1.2\linewidth}
    \footnotesize
    
    \begin{multicols}{2}           
      \begin{sml}
Normalising: f.sml
Parsing mlb/sml: 
  User: 0.536, System: 0.196, Total: 0.732
Resolving: 
  User: 5.100, System: 0.660, Total: 5.760
(@\cmt{The normalisation of f begins}@)
Normalizing
  (SOME (SOME (a, b), c)) = SOME (c, SOME (a, b))
  (SOME (a, b)) = SOME (b, a)
  NONE = NONE
  x = x
  (SOME (a, SOME (b, c))) = SOME (SOME (c, b), a)
(@\cmt{These tree clauses forms a cover}@)
These are a cover:
  (SOME (SOME (a, b), c)) = SOME (c, SOME (a, b))
  (SOME (a, b)) = SOME (b, a)
  NONE = NONE
(@\cmt{So these can be dropped}@)
Dropping:
  (SOME (a, SOME (b, c))) = SOME (SOME (c, b), a)
  x = x
(@\cmt{and we end up with these}@)
After elimination:
  (SOME (SOME (a, b), c)) = SOME (c, SOME (a, b))
  (SOME (a, b)) = SOME (b, a)
  NONE = NONE
(@\cmt{The generalisation process is bottom up, thus it starts with the 3rd clause}@)
Generalising
  NONE = NONE
(@\cmt{Which can be generalised to}@)
becomes
  x = x
Now have
  x = x

(@\cmt{Generalisation of the 2nd clause}@)
Generalising
  (SOME (a, b)) = SOME (b, a)
(@\cmt{which can't be generalised}@)
becomes
  (SOME (a, b)) = SOME (b, a)
Now have
  (SOME (a, b)) = SOME (b, a)
  x = x

(@\cmt{Generalisation of the 1st clause}@)
Generalising
  (SOME (SOME (a, b), c)) = SOME (c, SOME (a, b))
(@\cmt{Which can be generalised to}@)
becomes
  (SOME (d, c)) = SOME (c, d)
(@\cmt{And now it can be removed as it unifies with}@)
Deleted through unification:
  (SOME (a, b)) = SOME (b, a)
(@\cmt{We end up with these two clauses as the result}@)
Now have
  (SOME (d, c)) = SOME (c, d)
  x = x

Before:
fun f (SOME (SOME (a, b), c)) = SOME (c, SOME (a, b))
  | f (SOME (a, b)) = SOME (b, a)
  | f NONE = NONE
  | f x = x
  | f (SOME (a, SOME (b, c))) = SOME (SOME (c, b), a)

After:
fun f (SOME (d, c)) = SOME (c, d)
  | f x = x      
    \end{sml}
    \end{multicols}
  \end{narrow}
\end{example}

\begin{example}[]\
  \label{ex:eval-normal-example-flipodd}\\
  
  \begin{center}
    \begin{tabular}{|l|}
      \hline
      \textbf{Original functions}
      \\\hline
      \begin{sml}
fun flip (a, b) = (b, a)
fun flipodd ((a, b) :: (c, d) :: xs) = (a, b) :: flip (c, d) 
                                              :: flipodd xs
  | flipodd [(a, b)] = [(a, b)]
  | flipodd [] = []
  | flipodd _ = nil        
      \end{sml}
      \\\hline
    \end{tabular}
  \end{center}
  
  some text
  
  \begin{narrow}{-3em}{0em}
    \setlength{\linewidth}{1.2\linewidth}
    \footnotesize
    
    \begin{multicols}{2}           
      \begin{sml}
Normalising: flipodd.sml
Parsing mlb/sml: 
  User: 0.508, System: 0.224, Total: 0.732
Resolving: 
  User: 5.188, System: 0.612, Total: 5.800
(@\cmt{First the \ttt{flip} function is normalised}@)
Normalizing
  (a, b) = (b, a)
After elimination:
  (a, b) = (b, a)
Generalising
  (a, b) = (b, a)
becomes
  (a, b) = (b, a)
Now have
  (a, b) = (b, a)


(@\cmt{Then the \ttt{flipodd} function is normalised}@)
Normalizing
  ((a, b) :: (c, d) :: xs) = (a, b) :: flip (c, d) 
                                    :: flipodd xs
  [(a, b)] = [(a, b)]
  [] = []
  _ = nil
These are a cover:
  ((a, b) :: (c, d) :: xs) = (a, b) :: flip (c, d) 
                                    :: flipodd xs
  ((a, b) :: nil) = ((a, b) :: nil)
  (nil) = (nil)
Dropping:
  x = nil
After elimination:
  ((a, b) :: (c, d) :: xs) = (a, b) :: flip (c, d) 
                                    :: flipodd xs
  ((a, b) :: nil) = ((a, b) :: nil)
  (nil) = (nil)
(@\cmt{Generalisation starts bottom up}@)
Generalising
  (nil) = (nil)
becomes
  x = x
Now have
  x = x

Generalising
  ((a, b) :: nil) = ((a, b) :: nil)
becomes
  x = x
Deleted through unification:
  x = x
Now have
  x = x

Generalising
  ((a, b) :: (c, d) :: xs) = (a, b) :: flip (c, d) 
                                    :: flipodd xs
becomes
  (e :: h :: xs) = e :: flip h :: flipodd xs
Now have
  (e :: h :: xs) = e :: flip h :: flipodd xs
  x = x

Before:
fun flip (a, b) = (b, a)
fun flipodd ((a, b) :: (c, d) :: xs) = (a, b) 
                                       :: flip (c, d) 
                                       :: flipodd xs
  | flipodd [(a, b)] = [(a, b)]
  | flipodd [] = []
  | flipodd _ = nil

After:
fun flip (a, b) = (b, a)
fun flipodd (e :: h :: xs) = e :: flip h 
                               :: flipodd xs
  | flipodd x = x      
    \end{sml}
    \end{multicols}
  \end{narrow}
\end{example}

\begin{example}[]\
  \label{ex:eval-normal-example-g}\\
  
  \begin{center}
    \begin{tabular}{|l|}
      \hline
      \textbf{Original functions}
      \\\hline
      \begin{sml}
fun g1 (NONE, _) = NONE
  | g1 (SOME x, _) = SOME x

fun g2 (SOME x, _) = SOME x
  | g2 (NONE, _) = NONE

fun g3 (NONE, _) = NONE
  | g3 _ = NONE      
      \end{sml}
      \\\hline
    \end{tabular}
  \end{center}

  When looking at the generalisation of \ttt{g1} and \ttt{g2} one might think
  that they both ought to be generalised to \smlinline{fun g (x, y) = x}. This
  is however not the case as can be seen below. As \ttt{SOME x} and \ttt{NONE}
  are confused with each other, they will be generalised in alphabetic order and
  thus \ttt{(NONE, _)} will always be generalised first. Again one might think
  that \smlinline{(NONE, _) = NONE} would be generalised to \smlinline{(x, _) =
    x} but that is not the case, as there is nothing on the right hand side that
  relies on the left hand side, and thus the biggest generalisation possible is
  \smlinline{x = NONE}. This has the side effect that the two resulting clauses
  can't be unified.

  \begin{narrow}{-3em}{0em}
    \setlength{\linewidth}{1.2\linewidth}
    \footnotesize
    
    \begin{multicols}{2}           
      \begin{sml}
Normalising: g.sml
Parsing mlb/sml: 
  User: 0.500, System: 0.224, Total: 0.724
Resolving: 
  User: 5.068, System: 0.648, Total: 5.716
(@\cmt{g1}@)
Normalizing
  (NONE, _) = NONE
  (SOME x, _) = SOME x
After elimination:
  (NONE, x) = NONE
  (SOME x, y) = SOME x
Generalising
  (NONE, x) = NONE
becomes
  y = NONE
Now have
  y = NONE

Generalising
  (SOME x, y) = SOME x
becomes
  (a, y) = a
Now have
  (a, y) = a
  y = NONE

(@\cmt{g2}@)
Normalizing
  (SOME x, _) = SOME x
  (NONE, _) = NONE
After elimination:
  (SOME x, y) = SOME x
  (NONE, x) = NONE
Generalising
  (NONE, x) = NONE
becomes
  y = NONE
Now have
  y = NONE

Generalising
  (SOME x, y) = SOME x
becomes
  (a, y) = a
Now have
  (a, y) = a
  y = NONE

(@\cmt{g3}@)
Normalizing
  (NONE, _) = NONE
  _ = NONE
After elimination:
  (NONE, x) = NONE
  x = NONE
Generalising
  x = NONE
becomes
  x = NONE
Now have
  x = NONE

Generalising
  (NONE, x) = NONE
becomes
  y = NONE
Deleted through unification:
  x = NONE
Now have
  y = NONE

Before:
fun g1 (NONE, _) = NONE
  | g1 (SOME x, _) = SOME x
fun g2 (SOME x, _) = SOME x
  | g2 (NONE, _) = NONE
fun g3 (NONE, _) = NONE
  | g3 _ = NONE

After:
fun g1 (a, y) = a
  | g1 y = NONE
fun g2 (a, y) = a
  | g2 y = NONE
fun g3 y = NONE
      \end{sml}
    \end{multicols}
  \end{narrow}
\end{example}

\begin{example}[Not exhaustive function -- bastard]\
  \label{ex:eval-normal-example-bastard}\\

  \begin{center}  
  \begin{tabular}{|l|}
    \hline
    \textbf{Original function}
    \\ \hline
\begin{sml}
fun bastard (xs as x :: _, y :: ys) = (x, y) :: bastard (xs, ys)
  | bastard (_, nil) = nil    
\end{sml}
    \\\hline    
  \end{tabular}
  \end{center}

  As seen below this function is not normalised, as
  the \ttt{bastard} function is not exhaustive.

  \begin{narrow}{-3em}{0em}
    \setlength{\linewidth}{1.2\linewidth}
    \footnotesize
    
    \begin{multicols}{2}           
      \begin{sml}
Normalising: bastard.sml
Parsing mlb/sml: 
  User: 0.496, System: 0.212, Total: 0.708
Resolving: 
  User: 4.892, System: 0.696, Total: 5.588
(@\cmt{Normalisation starts}@)
Normalizing
  (xs as x :: _, y :: ys) = (x, y) :: bastard (xs, ys)
  (_, nil) = nil
(@\cmt{But terminates as it is not exhaustive}@)
Match is not exhaustive
  (x :: z, y :: ys) = (x, y) :: bastard ((x :: z), ys)
  (x, nil) = nil
(@\cmt{Thus the same before and after code are written}@)
Before:
fun bastard (xs as x :: _, y :: ys) = (x, y) 
                                      :: bastard (xs, ys)
  | bastard (_, nil) = nil

After:
fun bastard (xs as x :: _, y :: ys) = (x, y) 
                                      :: bastard (xs, ys)
  | bastard (_, nil) = nil    
      \end{sml}
    \end{multicols}
  \end{narrow}
\end{example}



%%% Local Variables: 
%%% mode: latex
%%% TeX-master: "../../report"
%%% End: 
