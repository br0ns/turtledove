\section{About Turtledove}

Turtledove is written entirely in SML and depends heavily on the SML library
MyLib. The MyLib library is the result of all the extensions to the basis
library that we created for Turtledove but later found to be general purpose
enough to be a library for itself. The source code for both projects can be
found on \href{http://github.com/mortenbp/}{GitHub}\footnote{Turtledove and
  MyLib can be found on GitHub at \url{http://github.com/mortenbp/}}

There are two main folders within the Turtledove repository: \ttt{code} and
\ttt{report}. The code folder contains all the source code for Turtledove and
its folder structure resembles that of the original design idea (see
\fref{fig:turtledove-design}). The \ttt{app} folder contain various examples
applications that uses different features of Turtledove and the \ttt{test}
folder contains small test ``applications'' that was created along the way to
test new features of Turtledove. The \ttt{report} folder contains various
documents that have been created along the way but of main importance, it
contains this report.


\subsection{Installing and compiling Turtledove}

Turtledove has been written with MLton 20100608 as the target compiler, and the
\ttt{utils/path/Path.sml} file contains MLton specific MLB variable paths that
are hard coded into MLton or given as arguments given to MLton at compile
time. Quite a few of there paths are architecture specific (32/64bit) and thus
we have hard coded these in the \ttt{Path.sml} depending on the result from
running \ttt{uname -m}. Currently we support ``i686'' and ``x86_64'' for the
32bit and 64bit paths respectively. If \ttt{uname -m} returns another string, it
will have to be added in \ttt{Path.sml} together with the paths if they are
different from those that are already added.  

The MLB variable paths are used by the MLton basis library MLB files and as such
would have to be completely changed for any other interpreter/compiler than
MLton.

To test that the system is set up correctly the command ``mlton Turtledove.mlb''
can be run, and if it compiles without any errors, everything should
compile. Remember that MyLib must be installed before attempting to compile any
Turtledove related code.

The root code folder contains a make file that will build all the applications
and the test applications. All individual applications and test applications
also have their own make files as well. Some of these make files are even
extended with shorthands for running the executable on a predefined set of input
files.


\subsection{Installing MyLib}

As MyLib is designed as an SML library and as such we have chosen to place it in
\ttt{\$(SML_LIB)/mylib/MyLib.mlb}. We have supplied an ``install'' option to the
make file, in MyLib, that will create a symbolic link between MyLib and
\ttt{/usr/lib/mlton/sml/} if it is issued in the root folder of MyLib.

Currently MyLib will work with MLton and MosML. The two \ttt{MyLib.ui/.uo} files
are produced for MosML, which also has an ``install'' option that will create a
symbolic link between the two files and the MosML library folder
\ttt{/usr/lib/mosml/}. This ``install'' feature also requires that the current
working directory is the root of the MyLib folder.

%%% Local Variables: 
%%% mode: latex
%%% TeX-master: "../report"
%%% End: 
