\chapter{Introduction}\label{chap:introduction}
Our goal for this project has been to create a system --- which we have named
\textit{Turtledove} --- for ``Tool assisted programming in SML, with special
emphasis on semi-automatic rewriting to predefined standard forms''.

We wanted the system to make a clear distinction between the back-end and the
front-end in order to make it easy to port to different editors and IDE's. Our
own preference goes towards Emacs.

Even though this project focuses on rewriting of functions to standard forms we
wanted the system to be extensible in order for it to provide other tools to
programmers\footnote{Or as it were: ``... developers, developers, developers.''}
(e.g., list of TODO's, auto completion or renaming/refactoring).

To accommodate this Turtledove should consist of a few key components and a
plug-in system for implementing individual tools. The key components include
\begin{itemize}
\item A parser for MLB and SML files inspired by the work of \cite{mbp08}.
\item Resolving of source file dependencies, variable infix status and program
  environment. Type inference would also be useful for some tools but that is
  beyond the scope of this project.
\item A project manager for managing the users source files, and any changes
  done to them by the user (outside Turtledove) or by Turtledove itself.
\item A communication bridge to handle communication between Turtledove and the
  user's development environment.
\item An SML pretty printer for converting abstract syntax trees to source code.
\end{itemize}

Rewriting of programs will be done trough a rewriting tool. This tool should
have its own (rule) parser that extends on the SML parser, for reading in rules
at run time. Thus new rules can be added by the user without any changes to the
internals of Turtledove. One should be able to define rewriting rules for both
expressions and functions:

\begin{description}
\item[Expressions] These rules were mainly intend to function as a set of
  simplification rules but are not limited to that.

  Such simplifications could be

  \begin{center}
    {\allowdisplaybreaks
      \begin{tabular}{c@{$\quad ->\quad $}c}
        \mathsml{"" \^ s} & \mathsml{s}\\
        \mathsml{n + 0} & \mathsml{n}\\
        \mathsml{0 * n} & \mathsml{0}\\
        \mathsml{n * 1} & \mathsml{1}\\
        \mathsml{l @ []} & \mathsml{l}\\
        \mathsml{true orelse b} & \mathsml{true}\\
        \mathsml{if b then true else false} & \mathsml{b}\\
        \mathsml{if b then true else t} & \mathsml{b orelse t}
      \end{tabular}
    }
  \end{center}

\item[Functions] These are more high level and are intended to rewrite entire
  functions and are the main focus of this project. An example of a function
  that can be rewritten is

\begin{minipage}{1.0\linewidth}
  \begin{tabular}{lcl}
\begin{sml}
fun foo (x::xs) = x+1 :: foo xs
  | foo []      = []
\end{sml}
  & $->$ &
\begin{sml}
fun foo xs = map op+ xs
\end{sml}
  \end{tabular}
\end{minipage}
\end{description}

We define three rules for rewriting functions to utilise the list
combinators \ttt{map}, \ttt{foldl} and \ttt{foldr}.

Unfortunately the goals were not fully met. The achievements of the project
are

\begin{itemize}

\item A discussion of the implemented software as well as suggestions for
improvements and suggestions for futher work.
\item An MLB and SML parser.
\item Environment and infix resolving on SML AST's.
\item A general purpose library with auxiliary functions and data structures.
\item A partial SML AST pretty printer (mainly missing signatures, structures
  and functors).
\item Stand alone application that can normalise an SML AST
\item Stand alone application that implement a hard coded map rewriter.
\item A few stand alone applications that implement various features of intended
  tools such as todo's and total project code size.
\end{itemize}

For guidelines on using and installing Turtledove see \fref{chap:using-turtledove}.

\section{Motivation}
\label{sec:motivation}
Suppose that we want to find code that could be written shorter using the
\texttt{map} function.
Here is an obvious example:
\begin{sml}
fun add (x :: xs) = x + 1 :: add xs
  | add nil       = nil
\end{sml}
can be rewritten to
\begin{sml}
val add = map (fn x => x + 1)
\end{sml}
But suppose that the first function was instead
\begin{sml}
fun add nil       = nil
  | add (x :: xs) = x + 1 :: add xs
\end{sml}
or even\footnote{We have seen novice SML programmers write functions similar to
  this.}
\begin{sml}
fun add (x :: xs) = x + 1 :: add xs
  | add [x]       = x + 1 :: add nil
  | add nil       = nil
\end{sml}
Of course all three examples can be rewritten to the same form. So should we
have three rewriting rules? Infinitely many? No.

Our problem here is that equivalent matches (a match is a list of clauses) can
take many forms.

Therefore we define a normal form for matches in a language similar to SML. We
also give an algorithm for computing normal forms.

We then define a domain specific language for defining rewriting rules on
matches which have been converted to normal form.

\paragraph{Further work.}
The reader might find it odd that the second line in the last example above ends
in \smlinline{:: add nil}. Novice programmer or not, real code probably does not
look like this. The reason is that if \smlinline{:: add nil} is left out, the three
versions of the function \smlinline{add} does not have the same normal form.

We would like to determine that
\begin{sml}
fun add (x :: xs) = x + 1 :: add xs
  | add [x]       = [x + 1]
  | add nil       = nil
\end{sml}
is indeed equivalent to
\begin{sml}
fun add (x :: xs) = x + 1 :: add xs
  | add nil       = nil
\end{sml}
Conversion to normal form will not capture this, but we give some suggestions on
how it might in \fref{sec:furth-gener}.

\section{Related work}
There exists several tools for performing rewriting of program code. Of
relevance to this project we find HaRe (The Haskell Refactorer, see \cite{HARE})
and Stratego (see \cite{stratego}) most notable.

HaRe performs a number of refactorings in Haskell, among which are function
lifting and variable renaming.

Stratego is a general framework for expressing rewritings in any programming
language. Thus one must first describe the grammar of the language for which one
will define rewriting rules.\\

To our knowledge there exists no tool that implement rewriting of whole
functions.

\section{Readers prerequisites}
Readers of this text should, as a minimal prerequisite, be familiar with
Standard ML. Some knowledge of compiler design and programming language theory,
is also recommended. Also some algorithmic and mathematically maturity will be
of help. Any to-be computer scientist with a few years experience should be able
to either directly understand this text, or be able to easily acquire the needed
knowledge.\\

\noindent
Literature that may prove helpful:

\begin{itemize}
\item The definition of Standard ML revised, \cite{SML97}.
\item Commentary on Standard ML, \cite{SMLCOMM}. Very helpful when reading
  \cite{SML97}.
\item A Lexical Analyzer Generator for Standard ML, \cite{MLLEX, ml-lex-yacc}. A description
  of, and a tutorial to, the software ML-Lex.
\item ML-Yac Users Manual, \cite{MLYACC, ml-lex-yacc}. A description of, and a tutorial to,
  the software ML-Yacc. 
\end{itemize}

For general ideas on program refactoring in funcational programming languages
\cite{HARE} is a good source.

\section{Structure outline}
\begin{description}
\item[Chapter 1 -- Introduction.]
  This outline concludes the introduction.
\item[Chapter 2 -- Normal form.]
  We construct a small language similar to SML and define a normal form on
  matches in this language. We conclude by sketching an extension to SML.
\item[Chapter 3 -- Rewriting rules.]
  We define a language for describing rewriting rules. We give a system of
  inference rules that define how rewriting should be carried out. We conclude
  the chapter by giving an intuitive understanding of the rules and some further
  suggestions.
\item[Chapter 4 -- Concrete examples.]
  We define three concrete rules and use them to rewrite a suite of example
  code.
\item[Chapter 5 -- Design.]
  We give a high level description of our design goals and ideas.
\item[Chapter 6 -- Implementation.]
  Here we describe the system that was actually implemented.
\item[Chapter 7 -- Using Turtledove.]
  We give guidelines on how to get, compile and use Turtledove.
\item[Chapter 8 -- Evaluation.]
  This is a summary of the implemented system along with representative examples.
\item[Chapter 9 -- Conclusion.]
  We summarise our experiences during the project and assess on the result as a
  whole.
\end{description}


%%% Local Variables: 
%%% mode: latex
%%% TeX-master: "../report"
%%% End: 
