\subsection{\textsf{foldl}}
\begin{example}\
  \label{ex:fold-instance-composelist}\\
  In the 2008 exam set, the students were asked to implement a \ttt{composelist}
  function which given a list of functions \ttt{[$f_1$, $f_2$, $\ldots$, $f_n$]}
  and a value $v$ would compute the result \ttt{$f_1$($f_2$($\cdots$($f_n\
    v$)$\cdots$))}.

  Among all the solutions nearly everyone handed in one very similar to one of
  the two below. The functions are taken from the students' solutions as-is ---
  hence the questionable choice of variable names in function \#2.
  \begin{center}
    \begin{tabular}{|l|l|}
      \hline
      \textbf{Original Function \#1}
      & 
      \textbf{Normalised Function \#1}
      \\\hline
\begin{sml}
fun composelist [] x = x
  | composelist (f :: []) x = f(x)
  | composelist (f :: fs) x =
      f(composelist fs x)
\end{sml}
      &
\begin{sml}
fun composelist (f :: []) x = f (x)
  | composelist (f :: fs) x =
      f (composelist fs x)
  | composelist y x = x
\end{sml}
      \\\hline
      \textbf{Original Function \#2}
      &
      \textbf{Normalised Function \#2}
      \footnotesize{(see \fref{tr:trace-normalise-composelist})}
      \\\hline
\begin{sml}
fun composelist [] x = x
  | composelist (y::ys) x =
     y (composelist (ys) x)
\end{sml}
      &
\begin{sml}
fun composelist (y :: ys) x =
      y (composelist (ys) x)
  | composelist y x = x
\end{sml}
      \\\hline
    \end{tabular}
  \end{center}
  The first version of the \ttt{composelist} function can't be matched any of
  the \tsf{fold}-rules because the first clause can't be generalised during the
  conversion to normal form. We discuss how this might be done in
  \fref{sec:furth-gener}.

  Writing an extra clause (when it is not necessary) for the special case where
  there is exactly one element in a list is a typical novice mistake.

  The first function can't be rewritten using our rules. The second however can
  be rewritten using the \textsf{foldr} rule.
  \\

  \noindent
  The rule succeeds with the following environments.
  \begin{center}
    \begin{tabular}{c | Mr @{} Ml Mr @{} Ml Mr @{} Ml !{\hspace{3em}} Mr @{} Ml}
      \textbf{Clause}
      & \multicolumn{6}{c !{\hspace{3em}}}{\textbf{Metapatterns/-variables}}
      & \multicolumn{2}{c}{\textbf{Transformers}}
      \\ \hline
      \#1
      &\mathcal{C} \mathrel{} & \mapsto \ttt{($\diamond_1$, x)}
      & \ol{x} \mathrel{} & \mapsto \ttt{y}
      & &
      & \mathbb{D} \mathrel{} & \mapsto (\mathsml{((y, x), a)}, \mathsml{y a})
      \\
      % same clause
      & \textsf{self} \mathrel{} & \mapsto (\ttt{composelist}, 2)
      &&
      &&
      &&
      \\ \hline
      \#2
      & \mathcal{D} \mathrel{} & \mapsto \mathsml{(y, x)}
      &&
      &&
      & \mathbb{E} \mathrel{} & \mapsto (\mathsml{(y, x)}, \mathsml{x})
      \\
    \end{tabular}
  \end{center}
  And we get the result
  \begin{center}
    \begin{tabular}{|l|}
      \hline
      \textbf{Rewritten function} \\ \hline
      \begin{sml}
fun composelist ys x = foldr (fn (y, a) => y a) (x) ys
      \end{sml} \\ \hline
    \end{tabular}
  \end{center}
\end{example}

\begin{example}\
  \label{ex:fold-instance-sumlige}\\
  In the 2009 exam set the students were asked to rewrite the \ttt{sumlige}
  function using higher order functions.

  \begin{center}
    \begin{tabular}{|l|l|}
      \hline
      \textbf{Original function}
      &
      \textbf{Normalised function}
      \footnotesize{(see \fref{tr:trace-normalise-sumlige})}
      \\\hline
  \begin{sml}
fun sumlige [] = 0
  | sumlige (x :: xs) =
      if x mod 2 = 0
      then x div 2 + sumlige xs
      else sumlige xs
  \end{sml}
      &
  \begin{sml}
fun sumlige (x :: xs) =
      if x mod 2 = 0
      then x div 2 + sumlige xs
      else sumlige xs
  | sumlige x = 0
  \end{sml}
  \\\hline
    \end{tabular}
  \end{center}
Now the \textsf{foldr}-rule succeeds with the environments
  \begin{center}
    \begin{tabular}{c | Mr @{} Ml Mr @{} Ml Mr @{} Ml !{\hspace{3em}} Mr @{} Ml}
      \textbf{Clause}
      & \multicolumn{6}{c !{\hspace{3em}}}{\textbf{Metapatterns/-variables}}
      & \multicolumn{2}{c}{\textbf{Transformers}}
      \\ \hline
      \#1
      &\mathcal{C} \mathrel{} & \mapsto \diamond_1
      & \ol{x} \mathrel{} & \mapsto \ttt{x}
      & &
      % & \mathbb{D} \mathrel{} & \mapsto (\mathsml{(x, a)}, \mathsml{(if x mod 2
        % = 0 then x div 2 else 0})
      & \multirow{3}{*}{$\mathbb{D}\mathrel{} \mapsto$}
      & \multirow{3}{*}{
        $\left(\ttt{(x, a)},\
        \begin{tabular}{>{\ttfamily}l}
          if x mod 2 = 0\\
          then x div 2 + a\\
          else a
        \end{tabular}
        \right)$
      }
      \\
      &&
      &&
      &&
      &&
      \\
      % same clause
      & \textsf{self} \mathrel{} & \mapsto (\ttt{sumlige}, 1)
      &&
      &&
      &&
      \\ \hline
      \#2
      & \mathcal{D} \mathrel{} & \mapsto \mathsml{x}
      &&
      &&
      & \mathbb{E} \mathrel{} & \mapsto (\mathsml{x}, \mathsml{0})
      \\
    \end{tabular}
  \end{center}
  And we get the result
  \begin{center}
    \begin{tabular}{|l|}
      \hline
      \textbf{Rewritten function} \\ \hline
      \begin{sml}
fun sumlige xs = foldr (fn (x, a) =>
                           if x mod 2 = 0 then
                             x div 2 + a
                           else
                             a
                       ) 0 xs
      \end{sml} \\ \hline
    \end{tabular}
  \end{center}
\end{example}


%%% Local Variables: 
%%% mode: latex
%%% TeX-master: "../report"
%%% End: 
