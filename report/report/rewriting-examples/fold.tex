\section{Fold}

Exam answers from 2008

This can't be used because of the 2nd clause. This could have been fixed if the
weak unification was implemented.

fun composelist [] x        = x
  | composelist (f :: []) x = f(x)
  | composelist (f :: fs) x = f(composelist fs x);

However we did receive lots of answers like

fun composelist [] x      = x
  | composelist (y::ys) x = y (composelist (ys) x);


fun composelist f x = foldr (fn (f,n) => f n) x f;



\begin{example}
eksamen 2009
  \begin{sml}
fun sumlige [] = 0
  | sumlige (x :: xs) = if x mod 2 = 0 then x div 2 + sumlige xs else sumlige xs

  \end{sml}
\end{example}



\begin{example}[\textsf{fold}-instans]\
\begin{sml}
fun sum nil       = 0
  | sum (x :: xs) = sum xs + x
\end{sml}

normaliseres til

\begin{sml}
fun sum (x :: xs) = sum xs + x
  | sum _         = 0
\end{sml}

De to klausuler er en instans af \textsf{foldl}-reglen med

\begin{eqnarray*}[rl]
  \mathcal{C} &\mapsto \diamond\\
  \overline{x} &\mapsto \texttt{x}\\
  \overline{xs} &\mapsto \texttt{xs}\\
  \mathbb{D}(\texttt{($\overline{a}$, $\overline{b}$)}) &=
  \kappa(\overline{b}) \texttt{ + } \kappa(\overline{a})\\
  \mathcal{D} &\mapsto \texttt{\_}\\
  \mathbb{E}(\overline{a}) &= \texttt{0}
\end{eqnarray*}

Så de skrives om til
  \begin{eqnarray*}[c]
    \texttt{$\mathcal{C}[xs]$ => foldl (fn ($a$, $b$) =>
      $\mathbb{D}($($\mathcal{C}[a]$, $b$)$)$) $\mathbb{E}(\mathcal{C}[$nil$])$
      $xs$}\\
    =\\
    \texttt{a => foldl (fn (a, b) => b + a) 0 a}
  \end{eqnarray*}

Og det endelige resultat er

\begin{sml}
fun sum a = foldl (fn (a, b) => b + a) 0 a
\end{sml}


  % \begin{eqnarray*}
  %   (y::ys) &=>& y + \mathtt{self} \left( ys \right) \\
  %   nil &=>& 0
  % \end{eqnarray*}

  % Hvilket er normaliseret så det bliver ikke omskrevet.

  % \begin{enumerate}
  % \item Antal klausuler matcher.

  % \item 

  % \item $\mathcal{C}$ instantieres til $\diamond$ og $\mathcal{G} = 0$

  %   \begin{description}
  %   \item[Klausul 1] $\overline{x} \mapsto y, \overline{xs} \mapsto ys$
      
  %   \item[Klausul 2] $\overline{x} \mapsto nil$
  %   \end{description}
    
    
  % \item $\mathbb{D}(\overline{n}, \overline{m}) = \kappa(\overline{n}) +
  %   \kappa(\overline{m})$
  % \end{enumerate}

  % Vi kan derfor anvende denne skabelon og lave omskrivningen, hvor der haves
  % $\floor{\mathcal{C}} = \diamond, \floor{\mathcal{G}} = 0$

  % \begin{eqnarray*}[c]
  %    \floor{\mathcal{C}} [xs]  => \textsf{fold } \left( \textsf{fn } (a,b) =>
  %     \mathbb{D}\left( \floor{\mathcal{C}}[a],\ b \right) \right) \
  %   \floor{\mathcal{G}} \ xs \\
  %   %
  %   \Downarrow \\
  %   % 
  %   xs => \textsf{fold } \left( \textsf{fn } (a,b) => a+b
  %   \right) \ 0 \ xs
  % \end{eqnarray*}
  
\end{example}

%%% Local Variables: 
%%% mode: latex
%%% TeX-master: "../report"
%%% End: 
