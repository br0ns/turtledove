\section{Map}

\begin{example}[\textsf{map}-instans]\
\begin{sml}
fun foo (nil, f)     = nil
  | foo ([x], g)     = [g x]
  | foo (x :: xs, f) = f x :: foo (xs, f)
\end{sml}

bliver normaliseret til

\begin{sml}
fun foo ([x], g)     = [g x]
  | foo (x :: xs, f) = f x :: foo (xs, f)
  | foo _            = nil
\end{sml}

De to sidste to klausuler er en instans af \textsf{map}-reglen med
\begin{eqnarray*}[rl]
\mathcal{C} &\mapsto \texttt{($\diamond$, $\overline{y}$)}\\
\overline{x} &\mapsto \texttt{x}\\
\overline{xs} &\mapsto \texttt{xs}\\
\mathbb{D}(\texttt{($\overline{a}$, $\overline{b}$)}) &= \kappa(\overline{b})\
\kappa(\overline{a})\\
\mathcal{D} &\mapsto \texttt{\_}
\end{eqnarray*}

Så de skrives om til
  \begin{eqnarray*}[c]
    \texttt{$\mathcal{C}[xs]$ => map (fn $m$ =>
      $\mathbb{D}(\mathcal{C}[m])$) $xs$}\\
    =\\
    \texttt{(a, b) => map (fn a => b a) a}
  \end{eqnarray*}
  
Og det endelige resultat er

\begin{sml}
fun foo ([x], g) = [g x]
  | foo (a, b)   = map (fn a => b a) a
\end{sml}

\end{example}

\begin{example}[\textsf{map}-instans]\
\begin{sml}
fun foo nil       = nil
  | foo (x :: xs) = x + 1 :: foo xs
\end{sml}

bliver normaliseret til

\begin{sml}
fun foo (x :: xs) = x + 1 :: foo xs
  | foo _         = nil
\end{sml}

Resultatet bliver

\begin{sml}
fun foo a = map (fn a => a + 1) a
\end{sml}


% \begin{eqnarray*}
%     (\mathsf{nil}) &=>& \mathsf{nil} \\
%     (x::xs) &=>& x+1 :: \mathsf{foo} xs
%   \end{eqnarray*}
  
%   Bliver normaliseret til 

%   \begin{eqnarray*}
%     (x::xs) &=>& x+1 :: \mathsf{foo} xs \\
%     x &=>& \mathsf{nil}
%   \end{eqnarray*}

%   \begin{enumerate}
%   \item Antal klausuler matcher.
    
%   \item Funktions kroppe kan matche med tilsvarende klausuls kroppe.
    
%   \item $\mathcal{C}$ instantieres til $\diamond$
    
%     \begin{description}
%     \item[Klausul 1] 
%       \begin{itemize}
%       \item På venstresiden bindes $\overline{x} \mapsto x, \overline{xs} \mapsto xs$.
        
%       \item På højresiden fås derfor $\mathcal{C}[\overline{x}] = x$ og
%         $\mathcal{C}[\overline{xs}] = xs$        
%       \end{itemize}
        
      
%     \item[Klausul 2]
%       \begin{itemize}
%       \item På venstresiden bindes $\overline{x} \mapsto
%         x$.
        
%       \item På højresiden er der ikke brugt nogle patvars så der sker ikke
%         noget.
%       \end{itemize}
      
%     \end{description}
      
%   \item $\mathbb{D}$ genererer funktionen der for input
%     $\mathcal{C}[\overline{x}] = x$ returnerer $x + 1$ da dette er udtrykket i
%     funktionen som matcher $\mathbb{D}\left( \mathcal{C}[\overline{x}] \right)$
%     i skabelonen. Altså $\mathbb{D}(\overline{x}) = \kappa(\overline{x}) + 1$ 
%   \end{enumerate}

%   Vi har altså nu at funktionen matcher skabelonen og derfor kan omskrives til
%   følgende ud fra skabelonenes omskrivning regel

%   \begin{eqnarray*}[c]
%       \mathcal{C}|[\mathcal{X\!S}|] => \textsf{map } \left( \textsf{fn } m =>
%     \mathbf{D}\left( \mathcal{C}|[m|] \right) \right) \mathcal{X\!S} \\
%   %
%   \Downarrow \\
%   %
%   xs => \textsf{map } \left( \textsf{fn } m =>
%     \left( \textsf{fn } x => x + 1 \right) m \right) xs \\
%   \end{eqnarray*}

\end{example}


%%% Local Variables: 
%%% mode: latex
%%% TeX-master: "../report"
%%% End: 
