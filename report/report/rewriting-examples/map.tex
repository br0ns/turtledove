\subsection{\textsf{map}}
\begin{example}[Simple \textsf{map}-instance]\
  \label{ex:map-instance-foo}\\
  A short warming up example.
  \begin{center}
    \begin{tabular}{|l|l|}
      \hline \textbf{Original function}
      & \textbf{Normalised function}
      \footnotesize{(see \fref{tr:trace-normalise-foo})}
      \\ \hline
\begin{sml}
fun foo nil       = nil
  | foo (x :: xs) = x + 1 :: foo xs
\end{sml}
      &
\begin{sml}
fun foo (x :: xs) = x + 1 :: foo xs
  | foo x = x
\end{sml}
      \\ \hline
    \end{tabular}
  \end{center}
  The second clause is normalised so we need to use the rule \textsf{map$_\textsf{b}$}.

  The rewriting rules succeeds with the environments
  \begin{center}
    \begin{tabular}{c | Mr @{} Ml Mr @{} Ml Mr @{} Ml !{\hspace{3em}} Mr @{} Ml}
      \textbf{Clause}
      & \multicolumn{6}{c !{\hspace{3em}}}{\textbf{Metapatterns/-variables}}
      & \multicolumn{2}{c}{\textbf{Transformers}}
      \\ \hline
      \#1
      &\mathcal{C} \mathrel{} & \mapsto \diamond_1
      & \ol{x} \mathrel{} & \mapsto \ttt{x}
      & &
      & \mathbb{D} \mathrel{} & = (\ttt{x}, \mathsml{x + 1})
      \\
      % same clause
      & \textsf{self} \mathrel{} & \mapsto (\ttt{foo}, 1)
      &&
      &&
      &&
      \\ \hline
      \#2
      &&
      &&
      &&
      &&
      \\
    \end{tabular}
  \end{center}
  under which
  \begin{center}
    \begin{tabular}{|l|}
      \hline
      \textbf{Rewritten function}
      \footnotesize{(see \fref{tr:trace-rewrite-map-foo})}
      \\ \hline
      \begin{sml}
fun foo xs = map (fn x => x + 1) xs
      \end{sml} \\ \hline
    \end{tabular}
  \end{center}
  is a result.
\end{example}

\begin{example}[Simple \textsf{map}-instance with extra argument]\
  \label{ex:map-instance-evallist}\\
  A group assignment in 2009 and 2010 asked for a function \ttt{evallist} which
  given a list of functions and a value will apply that value to all the
  functions in the list and return the resulting list.

  The below \ttt{evallist} function was written as part of this group assignment
  by four students.
  \begin{center}
    \begin{tabular}{|l|l|}
      \hline
      \textbf{Original function}
      &
      \textbf{Normalised function}
      \footnotesize{(see \fref{tr:trace-normalise-evallist})}
      \\ \hline
      \begin{sml}
fun evallist (f :: fs, v) =
      f v :: evallist(fs, v)
  | evallist ([], _) = []
      \end{sml}
      &
      \begin{sml}
fun evallist (f :: fs, v) =
      f v :: evallist (fs, v)
  | evallist y = nil
      \end{sml}
      \\ \hline
    \end{tabular}
  \end{center}
  Generalisation demands that the largest possible subpattern is generalised
  (see \fref{def:gener-patt}) which is why the \ttt{nil} on the right hand side
  is kept. That means that this time we need the \textsf{map$_\textsf{a}$} rule.

  We get these environments:
  \begin{center}
    \begin{tabular}{c | Mr @{} Ml Mr @{} Ml Mr @{} Ml !{\hspace{3em}} Mr @{} Ml}
      \textbf{Clause}
      & \multicolumn{6}{c !{\hspace{3em}}}{\textbf{Metapatterns/-variables}}
      & \multicolumn{2}{c}{\textbf{Transformers}}
      \\ \hline
      \#1
      &\mathcal{C} \mathrel{} & \mapsto \ttt{($\diamond_1$, v)}
      & \ol{x} \mathrel{} & \mapsto \ttt{f}
      & &
      & \mathbb{D} \mathrel{} & = (\ttt{(f, v)}, \ttt{f v})
      \\
      % same clause
      & \textsf{self} \mathrel{} & \mapsto (\ttt{evallist}, 1)
      &&
      &&
      &&
      \\ \hline
      \#2
      & \mathcal{D} \mathrel{} & \mapsto \ttt{y}
      &&
      && 
      &&
      \\
    \end{tabular}
  \end{center}
  and we get the result
  \begin{center}
    \begin{tabular}{|l|}
      \hline
      \textbf{Rewritten function}
      \footnotesize{(see \fref{tr:trace-rewrite-map-evallist})}
      \\ \hline
      \begin{sml}
fun evallist (fs, v) = map (fn f => f v) fs
      \end{sml} \\ \hline
    \end{tabular}
  \end{center}
\end{example}

\begin{example}[Advanced \textsf{map}-instance]\
  \label{ex:map-instance-collatz}\\
  From the exam set of 2009, the students was asked to take the below
  function and then rewrite it using higher order functions. Since the recursion
  is inside the \ttt{if}-statement, it will not directly match the
  \textsf{map}-rule. To be able to match it we need to modify it slightly such
  that the recursion is outside the \ttt{if}-statement.

  \paragraph{Note.} This modification could just as well have been done by
  a rewriting rule (on expressions).
  \begin{center}
    \begin{tabular}{|l|l|}
      \hline
      \textbf{Original function}
      &
      % Empty cell
      \\ \hline
      \begin{sml}
fun collatz [] = []
  | collatz (x :: xs) =
      if x mod 2 = 0 then
        x div 2 :: collatz xs
      else
        3 * x + 1 :: collatz xs
      \end{sml}
      &
      % Empty cell
      \\ \hline
      % New row
      \textbf{Modified function}
      &
      \textbf{Normalised function}
      \footnotesize{(see \fref{tr:trace-normalise-collatz})}
      \\ \hline
      \begin{sml}
fun collatz [] = []
  | collatz (x :: xs) =
      (if x mod 2 = 0 then
       x div 2 else 3 * x + 1)
      :: collatz xs
      \end{sml}
      &
      \begin{sml}
fun collatz (x :: xs) =
      (if x mod 2 = 0 then
       x div 2 else 3 * x + 1)
      :: collatz xs
  | collatz x = x
      \end{sml}
      \\ \hline
    \end{tabular}
  \end{center}
  The \textsf{map$_\textsf{b}$} rule succeeds withe the following environments:
  \begin{center}
    \begin{tabular}{c | Mr @{} Ml Mr @{} Ml Mr @{} Ml !{\hspace{3em}} Mr @{} Ml}
      \textbf{Clause}
      & \multicolumn{6}{c !{\hspace{3em}}}{\textbf{Metapatterns/-variables}}
      & \multicolumn{2}{c}{\textbf{Transformers}}
      \\ \hline
      \#1
      &\mathcal{C} \mathrel{} & \mapsto \diamond_1
      & \ol{x} \mathrel{} & \mapsto \ttt{x}
      & &
      & \multirow{3}{*}{$\mathbb{D}\mathrel{} =$}
      & \multirow{3}{*}{
        $\left(\ttt{x},\
        \begin{tabular}{>{\ttfamily}l}
          if x mod 2 = 0\\
          then x div 2\\
          else 3 * x + 1
        \end{tabular}
        \right)$
      }
      \\
      % same clause
      & \textsf{self} \mathrel{} & \mapsto (\ttt{collatz}, 1)
      &&
      &&
      &&
      \\
      % same clause
      &&
      &&
      &&
      & & %
      \\ \hline
      \#2
      & \mathcal{D} \mathrel{} & \mapsto \ttt{x}
      &&
      &&
      &&
      \\
    \end{tabular}
  \end{center}
  and we get the result
  \begin{center}
    \begin{tabular}{|l|}
      \hline
      \textbf{Rewritten function}
      \footnotesize{(see \fref{tr:trace-rewrite-map-collatz})}
      \\ \hline
      \begin{sml}
fun collatz xs =
      map (fn x => if x mod 2 = 0 then
                     x div 2
                   else
                    3 * x + 1)
          xs
      \end{sml} \\ \hline
    \end{tabular}
  \end{center}
\end{example}

\begin{example}[Advanced \textsf{map}-instance with multiple arguments]\
  \label{ex:map-instance-mapfil}\\
  One of the assignments from the 2008 exam set, which however didn't end up in
  the final exam set, was to create a function \ttt{mapfil} that goes through a
  list and applies a function \ttt{f} on each element if the predicate function \ttt{p}
  is true and leaves the element unchanged if false.

  The \ttt{mapfil} function below has been created by us and by the same
  reasoning as the \ttt{collatz} function (see \fref{ex:map-instance-collatz}) we
  have explicitly factored out the recursive call.


  \begin{center}
    \begin{tabular}{|l|l|}
      \hline
      \textbf{Original function}
      &
      \textbf{Normalised function}
      \footnotesize{(see \fref{tr:trace-normalise-mapfil})}
      \\ \hline
      \begin{sml}
fun mapfil (f, p, []) = []
  | mapfil (f, p, x :: xs) =
      (if p x then f x else x)
      :: mapfil (f, p, xs)
      \end{sml}
      &
      \begin{sml}
fun mapfil (f, p, x :: xs) =
      (if p x then f x else x)
      :: mapfil (f, p, xs)
  | mapfil x = nil
      \end{sml}
      \\ \hline
    \end{tabular}
  \end{center}
  This is an instance of the \textsf{map$_\textsf{a}$} rule. We get the environments
  \begin{center}
    \begin{tabular}{c | Mr @{} Ml Mr @{} Ml Mr @{} Ml !{\hspace{3em}} Mr @{} Ml}
      \textbf{Clause}
      & \multicolumn{6}{c !{\hspace{3em}}}{\textbf{Metapatterns/-variables}}
      & \multicolumn{2}{c}{\textbf{Transformers}}
      \\ \hline
      \#1
      &\mathcal{C} \mathrel{} & \mapsto \ttt{(f, p, $\diamond_1$)}
      & \ol{x} \mathrel{} & \mapsto \ttt{x}
      & &
      & \multirow{3}{*}{$\mathbb{D}\mathrel{} =$}
      & \multirow{3}{*}{
        $\left(\ttt{(f, p, x)},\
        \begin{tabular}{>{\ttfamily}l}
          if p x\\
          then f x\\
          else x
        \end{tabular}
        \right)$
      }
      \\
      % same clause
      & \textsf{self} \mathrel{} & \mapsto (\ttt{collatz}, 1)
      &&
      &&
      &&
      \\
      % same clause
      &&
      &&
      &&
      & & %
      \\ \hline
      \#2
      & \mathcal{D} \mathrel{} & \mapsto \ttt{x}
      &&
      &&
      &&
      \\
    \end{tabular}
  \end{center}
  and we get the result
  \begin{center}
    \begin{tabular}{|l|}
      \hline
      \textbf{Rewritten function}
      \footnotesize{(see \fref{tr:trace-rewrite-map-mapfil})}
      \\ \hline
      \begin{sml}
fun mapfil (f, p, xs) => map (fn x => if p x then f x else x) xs
      \end{sml} \\ \hline
    \end{tabular}
  \end{center}
\end{example}



%%% Local Variables: 
%%% mode: latex
%%% TeX-master: "../report"
%%% End: 
