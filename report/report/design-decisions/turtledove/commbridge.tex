\subsection{Communication bridge}

The intercommunication links in Turtledove are all controlled by the
communication manager including communication to ``outside'' of Turtledove such
as a development environment. 

% Used below in the cite
\def \protocoljson {\fref[plain]{fig:protocol-json}}

The protocol has been chosen to be very simple and versatile with almost no
limitations as it can then be used for both internal and external
communication. The protocol utilises JSON (JavaScript Object
Notation)\cite[\protocoljson]{json} which makes it easy to serialise different types of values
to and from the desired tools inside Turtledove.


\begin{nonfloatingfigure}
{ % so the \angleit command doesn't contaminate the whole environment.
  \newcommand{\angleit}[1]{$\langle$\textnormal{\textit{#1}}$\rangle$}
\begin{lstlisting}
 (@\angleit{request}@) ::= "{ \"Meta\" : (@\angleit{JSON-value}@), \"Dest\" : (@\angleit{JSON-string}@),
                \"Data\" : (@\angleit{JSON-value}@) } \n\n"

(@\angleit{response}@) ::= "{ \"Dest\" : (@\angleit{JSON-string}@), \"Meta\" : (@\angleit{JSON-value}@),
                \"Data\" : (@\angleit{JSON-value}@) } \n\n"
\end{lstlisting}    
}
  
  \caption{Definition of the communication protocol between Turtledove (and its
    tools) and the development evironment.}
  \label{fig:intercom-protocol-def}
\end{nonfloatingfigure}


The protocol is defined (see \fref{fig:intercom-protocol-def}) as a JSON-object
encoded string with two newlines at the end as the stop delimiter. The JSON
object contains three fields, ``Meta'' (Metadata), ``Dest'' (Destination) and
``Data'' (Data payload) which is described below.

\begin{description}
\item[Meta] is a JSON value specified by the development environment. This value
  is not used in any way by Turtledove but is returned in the response. This
  string is intended for internal bookkeeping in the development environment for
  example to distinguish which file/buffer and/or at which line and column the
  response originated from.

  As there is a possibility of Turtledove getting multi threaded, this is a good
  way of handling multiple request to Turtledove that may have different
  response times (One request waiting for a new complete reparse of the project
  code and another requesting a static lookup of some data).

  If this feature is not needed by the development environment it can set this
  field to the JSON value \texttt{null} but it is not a requirement.

  Some of the tools may report back when they are done doing something, without
  the development environment has invoked it. In these cases the JSON value
  \texttt{null} will be used.

\item[Dest] is a non empty JSON string that needs to match a named destination in
  Turtledove. Examples of such a named destination, also including name of
  tools:

  \begin{itemize}
  \item ``Turtledove'': This is for communication with the main program. This
    includes enabling/disabling of individual tools, status information and
    closing down the server gracefully.

  \item ``ProjectManager'': This is for communication with the project
    manager. This include actions such as modifying text, creation/deletion of
    files and projects.
  \end{itemize}
\end{description}


It is important to remember that the resulting string sent to and from
Turtledove must not contain any double newlines except the ones that are defined by
the protocol. This is a requirement, even though the definition of JSON allows
whitespace between a pair of tokens, as it would then render it near to
impossible to determine when the request/response is done without incrementally
parsing the JSON encoded string as it is received.

We have chosen the double newline character as the stop signal as it can't be used
inside the structure of the JSON object and when present in a JSON string it
must be escaped. 

This simple constraint of not having any double newlines inside the JSON encoded
string should not restrain any practical usability of the JSON protocol.


\paragraph{Examples}

\begin{example} If a file needs to be added then the following string could be
  sent to the ``ProjectManager'':
\begin{verbatim}
"{ \"Meta\" : null, \"Dest\" : \"ProjectManager\",
   \"Data\" : { \"command\" : \"addfile\", 
                \"file\" : \"myfoo.sml\"}} \n\n"
\end{verbatim}
\end{example}


%%% Local Variables: 
%%% mode: latex
%%% TeX-master: "../../report"
%%% End: 
