\subsection{Communication bridge}

The intercommunication links in Turtledove are all controlled by the
communication manager including communication to ``outside'' of Turtledove such
as a development environment. 

% Used below in the cite to refer to the appendix definition of json.
\def \protocoljson {\Fref[plain]{fig:protocol-json}}

The protocol has been chosen to be very simple and versatile with almost no
limitations as it can then be used for both internal and external
communication. The protocol utilises JSON (JavaScript Object
Notation)\cite[\protocoljson]{json} which makes it easy to serialise different types of values
to and from the desired tools inside Turtledove.


\begin{nonfloatingfigure}
{ % so the \angleit command doesn't contaminate the whole environment.
  \newcommand{\angleit}[1]{$\langle$\textnormal{\textit{#1}}$\rangle$}
\begin{lstlisting}
 (@\angleit{request}@) ::= "{ \"Meta\" : (@\angleit{JSON-value}@), \"Dest\" : (@\angleit{JSON-string}@),
                \"Data\" : (@\angleit{JSON-value}@) } \n\n"

(@\angleit{response}@) ::= "{ \"Meta\" : (@\angleit{JSON-value}@), \"Orig\" : (@\angleit{JSON-string}@),
                \"Data\" : (@\angleit{JSON-value}@) } \n\n"
\end{lstlisting}    
}
  
  \caption{Definition of the communication protocol between Turtledove (and its
    tools) and the development evironment.}
  \label{fig:intercom-protocol-def}
\end{nonfloatingfigure}


The protocol is defined (see \fref{fig:intercom-protocol-def}) as a JSON-object
encoded string with two newlines at the end as the stop delimiter. The JSON
object contains three fields, ``Meta'' (Metadata), ``Dest'' (Destination) or
``Orig'' (Origin) and ``Data'' (Data payload) which is described below.

\begin{description}
\item[Meta] is a JSON value specified by the development environment. This value
  is not used in any way by Turtledove but is returned in the response. This
  string is intended for internal bookkeeping by the development environment for
  example to distinguish which file/buffer and/or at which line and column the
  request came from.

  If this feature is not needed by the development environment it can set this
  field to the JSON value \texttt{null} but it is not a requirement.

  Some of the tools may report back when they are done doing something, without
  the development environment has invoked it. In these cases the JSON value
  \texttt{null} will be used.

\item[Dest/Orig] is a non empty JSON string that needs to match a named destination in
  Turtledove. Examples of such a named destinations, also include names of
  tools.
\end{description}


It is important to remember that the resulting string sent to and from
Turtledove must not contain any double newlines except the ones that are defined by
the protocol. This is a requirement, even though the definition of JSON allows
whitespace between a pair of tokens, as it would then render it near to
impossible to determine when the request/response is done without incrementally
parsing the JSON encoded string as it is received.

%%% Local Variables: 
%%% mode: latex
%%% TeX-master: "../../report"
%%% End: 
