\subsection{Project manager}

When turtledove is to be used in collaboration with a development environment
and wants to take advantage of all the features it has to offer, the project
manager must be used. The project manager is ultimately responsible for handling
project files which in the end will be converted into an MLB file. 

The reason why project files aren't just MLB files is that they don't enforce
enough limitations in how they are written, which would make it practically
impossible to write a descent tool to manipulate them. Whereas it makes it quite
easy for the user to hack something together.

Dependencies are properly the most challenging thing for a tool to handle as
these may depend on other files and so fourth. 

\begin{example}\ \\
  If we have 3 files: \textit{Foo.sig}, \textit{Foo.sml}, \textit{Main.sml}
  where the main file depends on the two foo files and obviously the foo
  implementation depends on the foo signature. 

  \begin{sml}
local
  $(SML_LIB)/basis/basis.mlb
  $(SML_LIB)/mylib/MyLib.mlb

  Foo.sig Foo.sml
in
  Main.sml
end    
  \end{sml}
\end{example}

\noindent
From the above example it is practically impossible for a system to know that
\textit{Foo.sml} depends on \textit{Foo.sig} and so forth. It would however be
easy to make up a scheme\footnote{We have in fact implemented such a scheme, see
  \fref{sec:xx}} of how to write MLB files in a ``modular'' way such that
dependencies could be easily processed by a tool, but that would not stop a user
from making valid modifications to the MLB file that would break this scheme.

\fixme{fix the footnote reference}

Thus we will use a more restricted format for project files but still with the
same functionality as can be expressed in MLB files. See
\fref{sec:appendix-project-manager} for a in depth description of the
functionality.



%%% Local Variables: 
%%% mode: latex
%%% TeX-master: "../../report"
%%% End: 
