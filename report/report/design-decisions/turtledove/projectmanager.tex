\subsection{Project manager}

When turtledove is to be used in collaboration with a development environment
and wants to take advantage of all the features it has to offer, the project
manager must be used. The project manager is ultimately responsible for handling
project files which in the end will be converted into an MLB file. 

The reason why project files aren't just MLB files is that they don't enforce
enough limitations in how they are written, which would make it practically
impossible to write a descent tool to manipulate them. Whereas it makes it quite
easy for the user to hack something together.

Dependencies are properly the most challenging thing for a tool to handle as
these may depend on other files and so fourth. 

\begin{example}\ \\
  If we have 3 files: \textit{Foo.sig}, \textit{Foo.sml}, \textit{Main.sml}
  where the main file depends on the two foo files and obviously the foo
  implementation depends on the foo signature. 

  \begin{sml}
local
  $(SML_LIB)/basis/basis.mlb
  $(SML_LIB)/mylib/MyLib.mlb

  Foo.sig Foo.sml
in
  Main.sml
end    
  \end{sml}
\end{example}

\noindent
From the above example it is practically impossible for a system to know that
\textit{Foo.sml} depends on \textit{Foo.sig} and so fourth. It would however be
easy to make up a scheme\footnote{We have in fact implemented such a scheme, see
  \fref{sec:xx}} of how to write MLB files in a ``modular'' way such that
dependencies could be easily processed by a tool, but that would not stop a user
from making valid modifications to the MLB file that would break this scheme.

\fixme{fix the footnote reference}

Thus we will use a more restricted format for project files but still with the
same functionality as can be expressed in MLB files:

\begin{description}
\item[Files] added, removed or renamed. \\

  A file is any valid file that one would normally reference in an MLB file.
  There is however the restriction that a file may only be added once, and can
  then be referenced as a dependency multiple times.
  
\item[Groupings] added, removed or renamed. \\

  We want to introduce a form of logical grouping of files to ease dependency
  references by using the name of the grouping instead of all the files inside
  the grouping. Groupings should also assist in controlling what visible to the
  outside of a grouping by the means of exposed files which is exactly what a
  local declaration in MLB files does.

  A grouping should be able to contain other groupings, such that smaller
  logical groupings can be combined to one or more larger groupings. 

\item[Dependencies] added or removed. \\

  Dependencies make sure that files are parsed in the correct order. A common
  use of dependencies is that structures must depend upon the signatures that
  they implement such that the signature is parsed before the structure.

  One might ask why this is not auto generated? In some (if not most) cases it
  can't be determined, most obviously if there exists two modules with the same
  name but with different meaning. However in most cases a fair guess can be
  generated, though it leaves the few cases unsolved where it is not possible
  and it would be complicated and time consuming to implement. So the other
  solution has been chosen which also offer more flexibility to the user at the
  cost of being a bit more complicated to use.

\item[Exposure] added or removed. \\

  It is often desired to have control of which files are visible to other files
  by writing local declarations in MLB files, so the environment aren't
  polluted.  

  For example a grouping containing two files \textit{FooUtils.sml} and
  \textit{Foo.sml}, here we don't necessarily want to expose the functions
  defined in the utility file as these may only have meaning to the functions
  defined in the foo file. In this case we would also have a dependency from
  \textit{Foo.sml} to \textit{FooUtils.sml} such that the utility file is parsed
  before the foo file.


\item[Properties] added or removed. \\

  A project files should have a list of key-value pairs that forms some
  predefined properties for this specific project. Such properties could be
  Turtledove specific user preferences.

\end{description}

\paragraph{Note.} All names of files and groups must be unique as no separate
unique id fields are used as dependency/exposure references. Thus the name of a
file is chosen to be its path and the name of groupings are just the name they
are given. This way dependency and exposure references will be this unique name.

Files contains the path and a list of what it depends on and groupings contains
their name and lists of containing files/groupings, dependent files/groupings
and exposed files/groupings.

As a project must consist of some sort of top level element (the project
itself), we have kept it simple and chosen to use a grouping for this. Thus a
project file is made up of a list of properties and a top level project
grouping.

As a grouping may consists sub groupings, only the first level of sub groupings
are allowed in exposure references. In dependency, references are
only allowed from any first level element in a grouping (file or exposure of sub grouping)
to any first level element within the group or to any first level element in the
parent group. The below example illustrates this:

\begin{example}[Project description for a project named ``Posidon'']\ \\
  \label{ex:Sample-project-file-posidon}  
  The \texttt{Posidon} project contains the following 
  
  \begin{description}
  \item[Files]\ \\
    \texttt{u.sml} and \texttt{j.sml}.
    
  \item[Groupings]\
    \begin{itemize}
    \item Group \texttt{A} which
      \begin{itemize}
      \item Contains the file \texttt{x.sml}.
      \item Exposes the file \texttt{x.sml}.
      \item Doesn't depend on anything.
      \end{itemize}
      
    \item Group \texttt{B} which
      \begin{itemize}
      \item Contains the files \texttt{y.sml} and \texttt{z.sml}.
      \item Exposes the file \texttt{z.sml}.
      \item Has dependencies
        \begin{itemize}
        \item \texttt{y.sml} depends on \texttt{x.sml}.
        \item \texttt{z.sml} depends on \texttt{y.sml}.
        \end{itemize}
      \end{itemize}
      
    \item Group \texttt{C} which
      \begin{itemize}
      \item Contains the files \texttt{n.sml} and \texttt{m.sml}.
      \item Exposes the files \texttt{n.sml} and \texttt{m.sml}.
      \item Doesn't depend on anything.
      \end{itemize}
    \end{itemize}


  \item[Exposes]\ \\
    Group \texttt{C} and file \texttt{j.sml}
    
  \item[Dependencies]\
    \begin{itemize}
    \item \texttt{C} depends on \texttt{B}
    \item \texttt{j.sml} depends on \texttt{u.sml}
    \item \texttt{j.sml} depends on \texttt{B}.
    \end{itemize} 
  \end{description}

\end{example}


There is one problem with this design, that it doesn't emulate exactly the
expressive power of MLB files, however we believe that it lets the user emulate
exactly the need that a novice programmer need. One of the drawbacks of this
design is that it doesn't currently allow for structure, signature or functor
replication and exposure. We have personally used this in some of our code (see
\fref{ex:parsers.mlb}) but we have newer come across this need/usage in our
teaching.

If Turtledove at some point were extended to give back a list of
signatures, structures or functors in a given set of files, then it wouldn't be
that big a deal extending the design to not only expose files or groups but also
the name and or replicating of such a structure, signature or functor in
cooperation with the Editor.

\subsubsection{Input files}

The project manager should be able to handle three types of input files: MLB files,
Project files (turtledove project files) and stand alone SML files. This will
make it possible for advanced users to create their own MLB but however loose
most of the project managers facilities. It should also be possible to use just
a single SML file, as this is what novice programmers often tend to use, by either
wrapping it in an MLB template that includes the basis library or an empty Turtledove
project file template. 

%%% Local Variables: 
%%% mode: latex
%%% TeX-master: "../../report"
%%% End: 
