\subsection{Parsers}
\label{sec:design-parsers}

One of the most basic capabilities of Turtledove is to parse a project. A
project is described by one or more ML Basis files and consists of a collection
of source files containing SML code. Thus we need to be able to parse these two
kinds of data.

We use ML-Lex and ML-Yacc for that task. At the time of writing those tools are
really the only option there is when working in Standard ML, apart from writing
the parsers from scratch. The shortcomings of ML-Lex and ML-Yacc is more than
paid for in that we can let us inspire by existing parsers that use those
tools. The MLKit and MLTon compilers contains parsers for both MLB and SML and
SML/NJ contains a SML parser all written using ML-Lex and ML-Yacc.

We have chosen to mimic the parsers found in MLTon. That is we have copied the
rules of the lexer and the productions of the grammar with very minor
exceptions. The code for each rule and production is our own though.
\\[1em]

A major part of a parser is to decide upon a way to represent the data one is
parsing. In this regard we take an unorthodox approach to representing SML.
\\[1em]

The structure of the data one wishes to parse is often described by a BNF
grammar. That is also the case for MLB and SML. In any case a LALR(1) grammar
has to be produced for ML-Yacc.

The most common way to represent parsed data is to define a set of mutually
recursive datatypes; one for each category of the BNF (or nonterminal of the
LALR(1) grammar). This approach has several benefits:

\begin{enumerate}
\item It is easy to design the representation. It can be written directly of the
      BNF. Another side effect of this is that the representation is already
      documented.
\item The type system ensures that it is only possible to represent data with a
      structure defined by the BNF. Other invariants like a particular variable
      naming scheme for example can naturally not be ensured by the type
      system. However some type safety is better than none.
\item \ldots
\item \ldots
\item Profit!
\end{enumerate}

However the type safety offered is also a limitation; where ever one wants to
walk through the representation one needs to define a set of mutually recursive
functions, one for each datatype. This represents a significant overhead, when
working with the parsed data. Oftentimes one is only interested in a small
selection of tokens when working through a syntax tree. Consider for example the
task of collecting the names of all defined functions in a SML program.

Another limitation is that ``wrapper constructors'' are often needed. It happens
when a rule of a category is just the name of another category. In other words,
one category is a subset of the other. Take for example the grammar
\begin{quote}
\textit{a} ::= \underline{A} | \textit{b}\\
\textit{b} ::= \underline{B} | \underline{C}
\end{quote}
If we didn't have the type system values of \textit{b} would also be values of
\textit{a}. But we do have a type system and we model values as mutually
recursive datatypes. This means that the type of values of \textit{a} and the
type of values of \textit{b} are distinct, and so we need a \textit{a}
constructor to hold \textit{b} values. A representation could be
\begin{quote}
\begin{verbatim}
datatype a = a_A
           | a_b
     and b = b_B
           | b_C
\end{verbatim}
\end{quote}

When we work with more complex grammars, as the one for SML, wrapping values for
the sake of the type system gets quite obtrusive.

From the work with ...\fixme{Reference til Twist bachproj}, the authors know SML
modelled by mutually recursive datatypes is difficult to work with.


\subsubsection{Payload}


\subsubsection{Representing MLB}
Lets first review the BNF grammar for ML Basis files.
\begin{quote}
\begin{verbatim}
foobar ::= foo
         | bar
\end{verbatim}
\end{quote}

We take the common approach of mutually recursive datatypes for representing
MLB. There are two reasons for this.
\begin{enumerate}
\item The BNF is quite small. There are only two categories, namely
      \texttt{basdec} and \texttt{basexp}. But one will probably also want to
      define functions for handling lists of \texttt{basdec}s and lists of
      bindings (pairs of identifiers and \texttt{basexp}s). So four mutually
      recursive functions need to be defined for each operation on an MLB
      representation. This number is acceptable.
\item Turtledove will not need to manipulate MLB. Since we write a project
      manager that use its on representation of projects, and generate MLB from
      that we only ever need to read MLB files.
\end{enumerate}

\subsubsection{Representing SML}

%%% Local Variables: 
%%% mode: latex
%%% TeX-master: "../../report"
%%% End: 
