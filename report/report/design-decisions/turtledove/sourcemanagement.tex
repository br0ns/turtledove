\subsection{Source Management}

Parsing of non well formed code is a major challenge and will always be present
when parsing code that is currently being written. Whether it being a new
function that is currently being written or one or more parts of non well formed
code it will inhibit the SML parser for giving an AST of the entire file. 

Parsing of entire projects is a relative costly operation, the source management
should thus only make the SML parser re parse as small fragments of code as
possible to maintain low response times. 


Multiple steps are taken to make sure these two issues are handled. 

\begin{itemize}
\item Remove non well formed code chunks from the source such that an AST can be
  produced.
  \begin{itemize}
  \item Make sure that position information is not invalid because of any
    temporary removed code in the produced AST
  \item Extract as much information from the non well formed code as possible
    and add this to the environment. For example: In function definitions the
    name and number of arguments could be extracted and added to the environment
    such that syntax colouring of recursive calls can be done and possible error
    reporting of wrong number of arguments in succeeding clauses.
  \end{itemize}

\item Take an old AST, a chunk of code and merge the parsed code into
  the old AST such that a new and update AST is produced.

\item Report errors from the SML parser to the development environment.

\end{itemize}



%%% Local Variables: 
%%% mode: latex
%%% TeX-master: "../../report"
%%% End: 
