
\section{Parsers}

\subsection{ML Basis files}

The MLB parser parses the files in two steps, First is a regular parse and the
second parse pulls out information about which files were included, etc. so any
MLB files referenced may also be parsed and their AST inserted.  ... bla blah

\fixme{Write this section}

\subsection{Standard ML}

Some clever stuff?

\fixme {Write this section}

\subsection{Rule}

The Rule parser is based on the SML parser with some added tokens to the lexer
and rules to the grammar to handle the rule syntax (see
\fref{tab:rule-grammar}).

\subsubsection{Unicode}

The rule syntax uses the symbols \texttt{£} and \texttt{§} to denote
transformers and meta patterns respectively. These were a bit tricky to
implement as they are not ASCII characters. By default SML-Lex\cite{ml-lex-yacc}
only supports 8-bit characters and thus if the upper parts of UTF-8 characters
are needed they must be specially handled. One way to handle it would be to
convert the input file into some fixed length encoding (i.e., UTF-32), but as
SML only uses ASCII chars that would demand a total remake of the Rule lexer
definition. Instead we went with a UTF-8 solution which only need to be
specially fitted for the transformer and meta patterns as it seems that most
editors now a days use UTF-8 as default encoding (seems to be the case for emacs
on linux). Implementing the two UTF-8 values in the lexer was then just a matter
of defining a named expression containing the conjunction of the two decimal
values making up each of the UTF-8 values (see \fref{tab:utf8-rule-values}).

\begin{table}
  \centering
  \begin{tabular}{|l|c|c|c|}
    \hline
    \textbf{Letter} & \textbf{UTF-8} & \textbf{Hex} & \textbf{Decimal} \\ \hline
    Pound sign (£)   & U+00A3 & 0xC2 0xA3 &  $194$ $163$ \\ \hline
    Section sign (§) & U+00A7 & 0xC2 0xA7 & $194$ $167$ \\ \hline
  \end{tabular}

  \caption{Table of UTF-8 and hex values of \texttt{£} and \texttt{§}}
  \label{tab:utf8-rule-values}
\end{table}

As we are using the internal SML-Lex position feature \texttt{yypos} we have to
decrement it by one each time a transformer or meta pattern is encountered as it
is counted as two characters, where it actually is just one (composite)
character.


%%% Local Variables: 
%%% mode: latex
%%% TeX-master: "../report"
%%% End: 

