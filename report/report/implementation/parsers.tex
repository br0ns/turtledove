
\section{Parsers}

\subsection{ML Basis files}

Parsing an MLB file is a two step process. First the MLB file is parsed by the
MLB parser and afterwards all references to external files are swapped with the
AST gotten from parsing that particular file (MLB or SML).

The lex and yacc files for the MLB parser are heavily inspired by MLton's MLB
parser.

\subsection{SML and rules}

The SML and rule parser shares the same AST as to make it easier for the
matching process later on. This means that the AST is extended with some rule
specific tokens.

The lex and yacc files for the SML parser, and thus also the rule parser, are
heavily inspired by MLton's SML parser.

\subsubsection{Rule parser}

As of this writing the rule parser only supports a subset of SML expressions and
patterns. The rule parser extends upon the SML parser adding a \ttt{ruleProgram}
non terminal as the first rule (no \ttt{\%start} symbol is defined). We have
defined \ttt{spat} as \ttt{pat} and \ttt{sexp} as \ttt{exp}, where all the
extensions introduced by \synth{spat} and \synth{sexp} are added directly to the
\ttt{pat} and \ttt{exp} non terminals. Currently exceptions and type annotations
are commented out for simplicity.

\subsubsection{Unicode in rule parser}

The rule syntax uses the symbols \texttt{£} and \texttt{§} to denote
transformers and meta patterns respectively. These need some special handling as
the lexer\cite{ml-lex-yacc} just reads 8-bit characters.

One way to handle it would be to convert the input file into some fixed length
encoding (f.x., UTF-32), but that would demand a total remake of the lexer
definition, which would not just be a major job but also be error prone.

Instead we just handle it as UTF-8\footnote{We remember that UTF-8 is a variable
  length encoding from one to four 8-bit characters}. Implementing the two
UTF-8 values in the lexer was then just a matter of defining a named expression
containing the conjunction of the two decimal values making up each of the UTF-8
values (see \fref{tab:utf8-rule-values}).

\begin{table}
  \centering
  \begin{tabular}{|l|c|c|c|}
    \hline
    \textbf{Letter} & \textbf{UTF-8} & \textbf{Hex} & \textbf{Decimal} \\ \hline
    Pound sign (£)   & U+00A3 & 0xC2 0xA3 &  $194$ $163$ \\ \hline
    Section sign (§) & U+00A7 & 0xC2 0xA7 & $194$ $167$ \\ \hline
  \end{tabular}

  \caption{Table of UTF-8, hex and decimal values of \texttt{£} and \texttt{§}}
  \label{tab:utf8-rule-values}
\end{table}

As we are using the internal lexer position \texttt{yypos} we have to decrement
it by one each time a transformer or meta pattern is encountered as it is
counted as two characters, where it actually is just one composite character.


%%% Local Variables: 
%%% mode: latex
%%% TeX-master: "../report"
%%% End: 

