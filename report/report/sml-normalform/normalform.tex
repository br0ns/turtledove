\section{Grammar}

\newcommand{\fn}{\ttt{fn}\ }
\newcommand{\rec}{\ttt{rec}\ }
\newcommand{\clause}[2]
{
  #1\mathrel{\texttt{.}}#2
}

An SML like grammar, without types, is presented below which will be the basis
of defining normal forms for SML functions.

\begin{eqnarray*}[rqcql:Tl]
  var & ::= & \ttt{[a-z]$^{+}$}                        & Identifiers\\
  con & ::= & \ttt{[A-Z][a-z]$^{*}$ | [0-9]$^{+}$}      & Constructors\\
% Matches
  match & ::= & \epsilon                              & Empty match\\
  & & pat\texttt{.}exp\ \texttt{|}\ match             & Clause\\
% Patterns
  pat & ::= & var                                     & Variable\\
  & & con                                             & Constructor of arity $0$\\
  & & con\texttt{(}pat_1\texttt{,} \ldots\texttt{,} pat_n\texttt{)} & Constructor of arity $n$\\
% Expressions
  exp & ::= & var                                     & Variable\\
  & & exp_1\ exp_2                                     & Application\\
  & & \fn match                                       & Function\\
  & & con                                             & Constructor of arity $0$\\
  & & con\texttt{(}exp_1\texttt{,} \ldots\texttt{,} exp_n\texttt{)} & Constructor of arity $n$\\
% Declarations
  dec & ::= & var \mapsto exp                         & value binding\\
  & & \rec var \mapsto exp                            & Recursive value binding\\
  & & dec_1 \ttt{;} \cdots \ttt{;} dec_n              & Sequence, $n \geq 2$\\
  & & \epsilon                                        & Empty program\\
\end{eqnarray*}

We use (with superfixes, subfixes and primes) $v$, $c$, $m$, $p$, $e$ and $d$ to
range over $var$, $con$, $match$, $pat$, $exp$ and $dec$ respectively.

\paragraph{Data types.} Every constructor belongs to exactly one data type. For
simplicity it is not possible to define data types and constructors in the
language but we will assume a fixed function $\psi$ mapping constructors to the
set of constructors of its own data type. This information is used solely when
determining coverage (see \fref[plain]{sec:cover}).

For the sake of examples assume that $list = \{\texttt{Nil}, \texttt{Cons}\}$,
$pair = \{\texttt{Pair}\}$ and $number = \{\texttt{1}, \texttt{2}, \ldots\}$ are
data types and that the arity of \texttt{Cons} and \texttt{Pair} are 2 and that
the arity of \texttt{Nil} and all $numbers$ are 0.

\paragraph{Well-formedness.}
\begin{itemize}
\item Expressions must not contain unbound variables.
\item No pattern may contain a given variable more than once.
\item In a match, corresponding nodes in the patterns should either be variables
  or constructors of the same data type.
\end{itemize}

\section{Semantics}
We do not define the semantics of the language formally. We have chosen not to
do this as the work in this chapter is later extended to SML, thus rendering
proofs etc. about semantics useless, and because the exact semantics of the
language is of no importance to most of the work in this chapter.

\section{List of symbols}

Below is a summarised list of symbols that will be introduced in this chapter
\\

\begin{tabular}{| >{$}c<{$} | p{18em} | l|}
  \hline
  \textbf{Symbol} & \textbf{Description} & \textbf{Defined at} \\ \hline
%
  ++ & Union mappings.
  & \Fref[plain]{sec:auxil-defin}  \\ \hline
%
  = & Syntactic equality. & \\ \hline
%
  ==_\pi & Equivalence of two patterns.
  & \Fref[plain]{def:equivalence-patterns} \\ \hline
%
  ==a & Alpha equivalence of two expressions or clauses.
  & \Fref[plain]{def:alpha-equivalence} \\ \hline
%
  \lessdot & Some total ordering on constructors and variables.
  & \Fref[plain]{def:pat-total-order-strict} \\ \hline
%
  <, \ <=& Total ordering of patterns.
  & \Fref[plain]{lem:pat-total-orderings} \\ \hline
%
  <', \ <='& Partial ordering of patterns.
  & \Fref[plain]{lem:pat-partial-orderings} \\ \hline
%
  || & Confusion of two patterns (when they don't partially order). &
  \Fref[plain]{def:pat-confusion} \\ \hline
%
  ->e & Elimination of one unused pattern from a match.
  & \Fref[plain]{def:shadowed-patterns-1} \\ \hline
%
  |> & Generalisation of a single clause.
  & \Fref[plain]{def:gener-patt} \\ \hline
%
  ->g & Generalisation of one pattern in a match.
  & \Fref[plain]{def:gener-match} \\ \hline
\end{tabular}


\section{A note about evaluation}
We expect programs to be run in an environment containing predefined functions
(that is variables bound to predefined functions) and constructors. Thus the
program
\begin{quote}
% \begin{verbatim}
\ttt{x $\mapsto$ plus (Pair (1, 8))}
% \end{verbatim}
\end{quote}
might make perfect sense (if in particular \ttt{plus} is a variable bound to a
suitable function (perhaps addition), and \texttt{Pair}, \texttt{1} and
\texttt{8} are constructors of arity 2, 0 and 0, respectively).
\section{Auxiliary definitions}
\label{sec:auxil-defin}

In the following we define what we mean by equivalence of patterns (with a
permutation of variables), free variables (for expressions, matches and
patterns), substitution (in expressions) and alpha equivalence (of expressions).

\paragraph{Note.}
\begin{enumerate}
\item
\label{item:note-plusplus}
If $f : A -> B$ and $g : A -> B$ are arbitrary mappings then
\begin{eqnarray*}[rlqTl]
  (g ++ f)(x) &= f(x) & if $x \in \Dom(f)$\\
  (g ++ f)(x) &= g(x) & otherwise
\end{eqnarray*}
and
\[
  \Dom (g ++ f) = \Dom (g) \cup \Dom (f).
\]


\item
We write $p \sqsubseteq p'$ to mean that $p$ is a subpattern of $p'$. More
precisely this is the case if $p = p'$ or if $p' = c \ttt{(} p_1 \ttt{,} \ldots
\ttt{,} p_n \ttt{)}$ and $p \sqsubseteq p_i$ for some $i \in \{1, \ldots, n\}$.

In particular we have\footnote{See \fref{sec:free-variables}
  for the definition of $\FV_{pat}$.} $x \sqsubseteq p$ exactly when $x \in
\FV_{pat}(p)$. The relation is obviously reflexive.

\begin{example}\ \label{ex:suppattern1}\\
  Recall that lowercase identifiers are variables, and uppercase ones are
  constructors. Variables as subpatterns:
  \begin{eqnarray*}
    \ttt{x} \sqsubseteq \ttt{A(x,y)} \qquad
    \ttt{y} \sqsubseteq \ttt{A(x,y)} \qquad
    \ttt{z} \not \sqsubseteq \ttt{A(x,y)}
  \end{eqnarray*}
  Patterns as subpatterns:
  \begin{eqnarray*}
    \ttt{A(x,y)} \sqsubseteq \ttt{B(A(x,y),z)} \qquad
    \ttt{A(x,y)} \not \sqsubseteq \ttt{A(A(a,b),c)} \qquad
  \end{eqnarray*}
\end{example}

\item
The syntactic category $pat$ is a proper subset of $exp$. Let $\kappa : pat ->
exp$ be the canonical mapping from $pat$ to $exp$. It is injective so it has a
left inverse $\kappa^{-1} : exp -> pat$. $\kappa^{-1}$ is clearly not total.
\end{enumerate}

\subsection{Equivalence of patterns}
\label{sec:equivalence-patterns}
We say that two patterns are equivalent if they can be transformed into
each other by a suitable renaming of the variables.

If $p_1$ and $p_2$ are equivalent we write $==_\pi$ where $\pi$ is a permutation
of variables, such that for each variable $x$ in $p_1$ its counterpart in $p_2$
is $\pi(x)$.

For example we have $A(x,y) ==_\pi A(z,x)$ where $\pi = \{x \mapsto z, y \mapsto x\}$.

\begin{definition}[Equivalence of patterns, $==_\pi$]
\label{def:equivalence-patterns}
  \begin{eqnarray}[rlqTl]
    v_1 &==_{\pi} v_2  & where $\pi = \{v_1 \mapsto v_2\}$ \label{eq:struct-eq-var} \\
    c\ttt{(}p^1_1 \ttt{,} \ldots \ttt{,} p^1_n \ttt{)} & ==_{\pi}
    c\ttt{(}p^2_1 \ttt{,} \ldots \ttt{,} p^2_n \ttt{)} & \label{eq:struct-eq-con}
  \end{eqnarray}
where \fref{eq:struct-eq-con} holds if
\begin{eqnarray*}[c]
  p^1_1 ==_{\pi_1} p^2_1 \\
  \vdots \\
  p^1_n ==_{\pi_n} p^2_n
\end{eqnarray*}
and $\pi = \pi_1 ++ \ldots ++ \pi_n$.

Note that the domains of each of the $\pi_i$'s are disjoint because no variable
can occur more than once in a pattern (by the definition of the syntax).

We write $==$ to mean $==_\pi$ (with a suitable non-fixed $\pi$) where $\pi$ has no
interest. Equivalence of patterns is defined by $==_\pi$, \emph{only} when $\pi$
is not fixed.
\end{definition}

\begin{example}[Equivalence of patterns, $==_{\pi}$]
  \label{ex:pattern-equiv1}
  \begin{eqnarray*}
    \ttt{A(x,y)} &==_{\pi}& \ttt{A(f,g)}\\
    \ttt{A(f,g)} &==_{\pi'}& \ttt{A(x,y)}
  \end{eqnarray*}
with
  \begin{eqnarray*}
    \pi &=& \{\ttt{x}\mapsto \ttt{f}, \ttt{y} \mapsto \ttt{g}\} \\
    \pi' = \pi^{-1} &=& \{\ttt{f} \mapsto \ttt{x}, \ttt{g} \mapsto \ttt{y}\}
\end{eqnarray*}

Whereas these patterns are not equivalent

  \begin{eqnarray*}
    \ttt{A(x,y)} &\not ==& \ttt{B(h,j)} \\
    \ttt{1} &\not ==& \ttt{2}
  \end{eqnarray*}
\end{example}

\begin{lemma}[Reflexivity, symmetry and transitivity]\ \\
  The relation $==$ is a equivalence relation; it is reflexive, symmetric and
  transitive.

  \begin{proof}
    Straightforward by induction.
  \end{proof}
\end{lemma}

\subsection{Free variables}\label{sec:free-variables}

We denote the free variables of expressions, matches and patterns with the three
functions $\FV_{exp}$, $\FV_{match}$ and $\FV_{pat}$, respectively.

\begin{definition}[Free variables of expressions, $\FV_{exp}$]\ \\
  Inductively defined:
  \begin{eqnarray}
    \FV_{exp} (v) &=& \{v\} \\
    \FV_{exp} (\fn m) &=& \FV_{match} (m) \\
    \FV_{exp} (e_1e_2) &=& \FV_{exp} (e_1) \cup \FV_{exp} (e_2) \\
    \FV_{exp} (c\ttt{(}e_1\ttt{,} \ldots \ttt{,} e_n \ttt{)}) &=& \FV_{exp}
    (e_1) \cup \ldots \cup \FV_{exp} (e_n)
  \end{eqnarray}
\end{definition}

\begin{definition}[Free variables of matches, $\FV_{match}$]\ \\
  Inductively defined:
  \begin{eqnarray}
    \FV_{match} (\epsilon) &=& \emptyset \\
    \FV_{match} (p\ttt{.}e\ \ttt{|}\ m) &=& \left( \FV_{exp}(e) \setminus
      \FV_{pat}(p) \right) \cup \FV_{match} (m)
  \end{eqnarray}
\end{definition}

\begin{definition}[Free variables of patterns, $\FV_{pat}$] \ \\
  Inductively defined:
  \begin{eqnarray}
    \FV_{pat} (v) &=& \{v\} \\
    \FV_{pat} (c\ttt{(}p_1\ttt{,} \ldots \ttt{,} p_n\ttt{)}) &=& \FV_{pat} (p_1)
    \cup \ldots \cup \FV_{pat} (p_n)
  \end{eqnarray}
\end{definition}

\begin{example}[Free variables, $\mrm{FV}$]
\label{ex:free-variables1}
\begin{eqnarray*}[c]
  \FV_{exp} \left(
    \begin{eqnalign}[Tl]
\begin{lstlisting}
fn Nil . Nil
  | Cons (x, xs) . Cons(f x, g xs)
\end{lstlisting}
    \end{eqnalign}
  \right) = \{\ttt{f}, \ttt{g} \} \\
%
  \FV_{match} \left(
    \begin{eqnalign}[Tl]
\begin{lstlisting}
Cons (x, Nil) . Cons(x, y)
\end{lstlisting}
    \end{eqnalign}
  \right) = \{\ttt{y}\} \\
%
  \FV_{pat} \left(
    \begin{eqnalign}[Tl]
\begin{lstlisting}
Cons (x, xs)
\end{lstlisting}
    \end{eqnalign}
  \right) = \{\ttt{x}, \ttt{xs}\} \\
\end{eqnarray*}
\end{example}


\subsection{Substitution}
We define substitution in expressions. An expression can be substituted for any
(sub)expression of an expressions, not just variables.

\begin{definition}[Substitution]\ \\
  If $e_1$, $e_2$ and $e_3$ are expressions we write $e_1[e_2/e_3]$ to be the
  result of substituting all occurrences of $e_3$ in $e_1$ with $e_2$.
  \begin{eqnarray}
    e_1[e_2/e_3] &=& e_2 \quad \mrm{iff}\ e_1 = e_3 \label{eq:subst-sub}\\
    (e^1_1 e^2_1)[e_2/e_3] &=& e^1_1[e_2/e_3] e^2_1[e_2/e_3] \label{eq:subst-app}\\
    \fn p_1 \texttt{.} e_1 \texttt{|} m &=& \fn p_1 \texttt{.} e'_1
    \texttt{|} m' \label{eq:subst-lam}\\
    (c \texttt{(}e^1_1 \texttt{,} \ldots \texttt{,} e^1_n \texttt{)})[e_2/e_3]
    &=& c \texttt{(}e^1_1[e_2/e_3] \texttt{,} \ldots \texttt{,} e^1_n[e_2/e_3]
    \texttt{)} \label{eq:subst-con}
  \end{eqnarray}
Where in \fref{eq:subst-lam} we have
\begin{eqnarray*}[rlqTl]
  e'_1 &= e_1 & if $FV_{exp}(e_3) \cap FV_{pat}(p_1) = \emptyset$\\
  e'_1 &= e_1[e_2/e_3] & otherwise
\end{eqnarray*}
and
\[
(\fn m)[e_2/e_3] = \fn m'
\]

In \fref[plain]{eq:subst-app}, \fref[plain]{eq:subst-lam} and
\fref{eq:subst-con}  we require that \fref[plain]{eq:subst-sub} does not apply.

Note that in \fref[plain]{eq:subst-sub} we require $e_1$ and $e_3$ to be exactly
equal, not just alpha equivalent.
\end{definition}

\begin{example}[Substitution]
\label{ex:substituation1}

\begin{eqnarray*}[c]
\left(
  \begin{eqnalign}[Tl]
\begin{lstlisting}
fn x . x
 | y . x
\end{lstlisting}
  \end{eqnalign}
\right) \left[ \ttt{z}/\ttt{x} \right] \quad = \quad
  \begin{eqnalign}[Tl]
\begin{lstlisting}
fn x . x
 | y . z
\end{lstlisting}
  \end{eqnalign} \\
%
\left(
  \begin{eqnalign}[Tl]
\begin{lstlisting}
fn x . Cons (1, x)
\end{lstlisting}
  \end{eqnalign}
\right) \left[ \ttt{z}/\ttt{Cons (1, x)} \right] \quad = \quad
  \begin{eqnalign}[Tl]
\begin{lstlisting}
fn x . z
\end{lstlisting}
  \end{eqnalign} \\
\end{eqnarray*}

whereas

\begin{eqnarray*}
\left(
  \begin{eqnalign}[Tl]
\begin{lstlisting}
fn x . x
\end{lstlisting}
  \end{eqnalign}
\right) [ \ttt{z}/\ttt{x} ] \quad \neq \quad
  \begin{eqnalign}[Tl]
\begin{lstlisting}
fn z . z
\end{lstlisting}
  \end{eqnalign}
\end{eqnarray*}
\end{example}

\subsection{Alpha equivalence}
\label{sec:alpha-equivalence}

We define alpha equivalence on expressions and clauses. We use the symbol $==a$
for both relations.

\begin{definition}[Alpha equivalence of expressions, $==a$]\
\label{def:alpha-equivalence}\\
  First we define alpha equivalence given a mapping of bound variables:
  \begin{eqnarray}
    \sigma |- v_1 &==a& v_2 \label{eq:alpha-var} \\
    \sigma |- e^1_1e^1_2 &==a& e^2_1e^2_1 \label{eq:alpha-exp} \\
    \sigma |- \fn p^1_1 \texttt{.} e^1_1 \texttt{|} \ldots \texttt{|} p^1_n
    \texttt{.} e^1_n &==a& \fn p^2_1 \texttt{.} e^2_1 \texttt{|} \ldots \texttt{|} p^2_n
    \texttt{.} e^2_n \label{eq:alpha-match} \\
    \sigma |- c\ttt{(}e^1_1 \ttt{,} \ldots \ttt{,} e^1_n \ttt{)} &==a&
    c\ttt{(}e^2_1 \ttt{,} \ldots \ttt{,} e^2_n \ttt{)} \label{eq:alpha-con}
  \end{eqnarray}
where \fref{eq:alpha-var} holds when
\begin{eqnarray*}[rlqTl]
\sigma (v_1) &= v_2 & if $v_1 \in \Dom(\sigma)$\\
v_1 &= v_2 & otherwise,
\end{eqnarray*}
\fref{eq:alpha-exp} holds if
\[
\sigma |- e^1_1 ==a e^2_1 \land \sigma |- e^1_2 ==a e^2_1,
\]
\fref{eq:alpha-match} holds if
\begin{eqnarray*}
  \clause{p^1_1}{e^1_1} &==a& \clause{p^2_1}{e^2_1}\\
  &\vdots&\\
  \clause{p^1_n}{e^1_n} &==a& \clause{p^2_n}{e^2_n}
\end{eqnarray*}
and \fref{eq:alpha-con} holds if
\[
\sigma |- e^1_1 ==a e^1_n \land \ldots \land \sigma |- e^2_1 ==a e^2_n.
\]

\begin{definition}[Alpha equivalence of clauses, $==a$]\
\label{def:alpha-equivalence-patexp}\\
  Again we assume a mapping of bound variables. It is the case that
  \[
  \sigma |- \clause{p_1}{e_1} ==a \clause{p_2}{e_2}
  \]
  exactly when $p_1 ==_\pi p_2$ and $\sigma ++ \pi |- e_1 ==a e_2$.
\end{definition}

If $e_1$ and $e_2$ are alpha equivalent expressions we write $e_1 ==a e_2$ which
is a shorthand for $[] |- e_1 ==a e_2$. Similarly for clauses.
\end{definition}

\begin{example}[Alpha equivalence, $==a$]\
\label{ex:alpha-equivalence1}\\
  These are alpha equivalent
  \begin{eqnarray*}[c]
    \begin{eqnalign}[Tl]
\begin{lstlisting}
fn Nil . Nil
 | Cons (x, xs) . Cons(f x, g xs)
\end{lstlisting}
    \end{eqnalign}
    ==a
    \begin{eqnalign}[Tl]
\begin{lstlisting}
fn Nil . Nil
 | Cons (y, ys) . Cons(f y, g ys)
\end{lstlisting}
    \end{eqnalign}
  \end{eqnarray*}

  whereas these are not, as the free variables are not the same

  \begin{eqnarray*}[c]
    \begin{eqnalign}[Tl]
\begin{lstlisting}
fn Nil . Nil
 | Cons (x, xs) . Cons(f x, g xs)
\end{lstlisting}
    \end{eqnalign}
    \not ==a
    \begin{eqnalign}[Tl]
\begin{lstlisting}
fn Nil . Nil
 | Cons (y, ys) . Cons(h y, j ys)
\end{lstlisting}
    \end{eqnalign}
  \end{eqnarray*}
\end{example}

% \section{Semantic equivalence}
% \label{sec:semantic-equivalence}
% We write $e_1 \sim e_2$ if $e_1$ and $e_2$ are semantically equivalent. That is
% if $e_2$ is substituted for $e_1$ (or vice versa) in any program $d$ to obtain
% $d'$, then if $d$ evaluates to something in an environment $\sigma$ then $d'$
% evaluates to that something in $\sigma$, and if $d$ diverges in $\sigma$ so does
% $d'$.

\section{Orderings on patterns}
\label{sec:orderings-patterns}
We define a total relation ($<=$) on patterns, and a partial relation ($<='$) on
the quotient set of patterns by $\equiv$. Then we show that they indeed are
orderings.

We write $<$ and $<'$ for $<=$'s and $<='$'s strict (or irreflexive) counterparts
respectively.

As it turns out it is easier to define $<$ and $<='$ directly and then define
$<=$ and $<'$ in turn of those.

\begin{definition}[Strict total ordering, $<$]\
  \label{def:pat-total-order-strict}\\
  Assume a total strict ordering $\lessdot$ on constructors and
  variables\footnote{For example let all constructors order less than all
    variables and let variables and constructors be ordered lexicographically
    among themselves. Note that constructors and variables are not compared in
    the definition of $<$.}. We inductively define:
  \begin{eqnarray}
    v_1 &<& v_2 \quad \mrm{if}\ v_1 \lessdot v_2\label{eq:pat-total-order-strict-var}\\
    c\texttt{(}p_1\texttt{,} \ldots\texttt{,} p_n\texttt{)} &<& v\\
    c_1\texttt{(}p_1\texttt{,} \ldots\texttt{,} p_n\texttt{)} &<&
    c_2\texttt{(}p'_1\texttt{,} \ldots\texttt{,} p'_m\texttt{)}\label{eq:pat-total-order-strict-con}
  \end{eqnarray}
  Where \fref{eq:pat-total-order-strict-con} hold if
  \[
  c_1 \lessdot c_2 \lor (c_1 = c_2 \land ( p_1 < p'_1 \lor p_1 = p'_1 \land (\ldots p_n < p'_n \ldots )))
  \]
\end{definition}

\begin{example}[Strict total ordering on patterns, $<$]
  \begin{eqnarray*}[c]
    \begin{eqnalign}[Tl]
\begin{lstlisting}
Cons (x, z)
\end{lstlisting}
    \end{eqnalign}
    <
    \begin{eqnalign}[Tl]
\begin{lstlisting}
Cons (y, z)
\end{lstlisting}
    \end{eqnalign}
    <
    \begin{eqnalign}[Tl]
\begin{lstlisting}
Snoc (a, b)
\end{lstlisting}
    \end{eqnalign}
    <
    \begin{eqnalign}[Tl]
\begin{lstlisting}
a
\end{lstlisting}
    \end{eqnalign}
    <
    \begin{eqnalign}[Tl]
\begin{lstlisting}
b
\end{lstlisting}
    \end{eqnalign}
  \end{eqnarray*}
  Note: Keep in mind that the exact names of the variables are unimportant.
\end{example}

\begin{definition}[Total ordering, $<=$]\
  \label{def:pat-total-order-weak}\\
  We define the reflexive cousin:
  \begin{eqnarray*}
    p_1 <= p_2 \Longleftrightarrow p_1 < p_2 \lor p_1 = p_2
  \end{eqnarray*}
\end{definition}



\begin{definition}[Partial ordering, $<='$]\
  \label{def:pat-partial-order-weak}\\
  We say that $p_2$ weakly generalises $p_1$ or $p_1$ is at least as specific as
  $p_2$ and we write $p_1 <=' p_2$. Inductively defined.
  \begin{eqnarray}
    p &<='& v \label{eq:pat-partial-order-weak-var}\\
    c_1\texttt{(}p_1\texttt{,} \ldots\texttt{,} p_n\texttt{)} &<='&
    c_2\texttt{(}p'_1\texttt{,} \ldots\texttt{,} p'_m\texttt{)}
    \label{eq:pat-partial-order-weak-con}
  \end{eqnarray}
  Where \fref{eq:pat-partial-order-weak-con} hold if
  \[
  \begin{eqnalign}[ccc@{\quad}l]
    c_1 &=& c_2 & \land\\
    p_1 &<='& p'_1 & \land\\
    &\ldots& & \land\\
    p_n &<='& p'_n &
  \end{eqnalign}
  \]
\end{definition}



\begin{definition}[Strict partial ordering, $<'$]\
  \label{def:pat-partial-order-strict}\\
  We define the strict counterpart of $<='$ by
  \begin{eqnarray*}
      p_1 <' p_2 \Longleftrightarrow p_1 <=' p_2 \land p_1 \not == p_2
  \end{eqnarray*}
\end{definition}

\begin{example}[Strict partial ordering on patterns, $<'$]
  \label{ex:orderings-patterns-1}
  \begin{eqnarray*}[c]
    \begin{eqnalign}[Tl]
\begin{lstlisting}
Cons (y, Nil)
\end{lstlisting}
    \end{eqnalign}
    <'
    \begin{eqnalign}[Tl]
\begin{lstlisting}
Cons (x, z)
\end{lstlisting}
    \end{eqnalign}
    <'
    \begin{eqnalign}[Tl]
\begin{lstlisting}
b
\end{lstlisting}
    \end{eqnalign}
  \end{eqnarray*}
  Keeping in mind that the exact names of the variables are unimportant.
\end{example}


\begin{lemma}[Total ordering]\
  \label{lem:pat-total-orderings}\\
  The relation $<=$ is a total ordering, and $<$ is a strict total ordering on
  patterns.

  Proof is given in \fref{sec:proof-total-orderings}.
\end{lemma}


\begin{lemma}[Partial ordering]\
  \label{lem:pat-partial-orderings}\\
  The relation $<='$ is a partial ordering and $<'$ is a strict partial ordering
  on the equivalence classes of patterns modulo structural equivalence
  ($pat_{/_{==}}$).

  Proof is given in \fref{sec:proof-partial-orderings}.
\end{lemma}

We write $p_1 > p_2$, $p_1 >= p_2$, $p_1 >' p_2$ and $p_1 >=' p_2$ to mean $p_2
< p_1$, $p_2 <= p_1$, $p_2 <' p_1$ and $p_2 <=' p_1$ respectively.

\begin{lemma}[]\
  \label{lem:total-implies-partial}\\
  If two patterns $p_1$ and $p_2$ are ordered by the partial ordering then they
  are also ordered by the total one. That is
  \begin{eqnarray*}
    p_1 <' p_2 \Longrightarrow p_1 < p_2
  \end{eqnarray*}
\end{lemma}
\begin{proof}
  Straightforward using induction.
\end{proof}

\begin{definition}[Confusion, $||$]\
  \label{def:pat-confusion}\\
  Let two patterns $p_1$ and $p_2$ be given. If it is the case that neither $p_1
  <=' p_2$ nor $p_1 >=' p_2$ we say that $p_1$ and $p_2$ are confused and we
  write $p_1 || p_2$.
\end{definition}

\begin{lemma}[Unique relation]\
  \label{lem:unique-rel}\\
  Given two patterns $p_1$ and $p_2$ exactly one of the following hold
  \begin{eqnarray*}
    p_1 &==& p_2\\
    p_1 &<'& p_2\\
    p_1 &>'& p_2\\
    p_1 &||& p_2
  \end{eqnarray*}
\end{lemma}
\begin{proof}
  Immediately by inspection.
\end{proof}

\begin{lemma}[]\
  \label{lem:more-specific-confused}\\
  If $p_1 <' p_2$ and $p_2 || p_3$, then $p_1 || p_3$.

  Proof is given in \fref{sec:proof-partial-orderings}.
\end{lemma}


\section{Eliminating unused patterns}
A function is simply a match. And a match is a list of pairs of patterns and
corresponding bodies.

During pattern matching the input to a function is tried against the patterns
from top to bottom. An unused pattern is a pattern that will never see a value
which it matches.

This can happen for two reasons.
\begin{enumerate}
\item The pattern will never be tried against the input because the input
  matches an earlier pattern\label{item:unused-reason-1}.
\item The pattern is only tried against inputs it doesn't
  match\label{item:unused-reason-2}.
\end{enumerate}

\subsection{Cover}
\label{sec:cover}
We define a cover to be a set of patterns such that for every input at least one
of the patterns will match that input and we write $Cov(P)$ if $P$ is a cover.

Let
\[
P = \{p_1,\ldots,p_n\}
\]

Now $P$ can be a cover for two reasons

\begin{enumerate}
\item At least one of the patterns is a variable.
\item If none of the patterns is a variable they must all have a constructor in
  their root node. First we determine the data type associated with the
  constructors. It must be the case that
  \[
  p_1 = c \texttt{(} p_{11} \texttt{,} \ldots \texttt{,} p_{1m}\texttt{)}
  \quad \textrm{or} \quad
  p_1 = c
  \]
  In either case the set of constructors associated with the data type is
  \[
  \{c_1, \ldots, c_l\} = \psi(c).
  \]
  Partition $P$ into $l$ subsets, $P_1, \ldots, P_l$, such that the root node of
  all the patterns in $P_i$ is the constructor $c_i$. Note that every pattern
  belongs to a subset because of the well-formedness criterion.

  Coverage is defined recursively
  \[
  Cov(P) <=> \forall i. Cov(P_i)
  \]
\end{enumerate}

\begin{lemma}\ \\
  Any pattern following a cover is unused because of elimination reason
  \ref{item:unused-reason-1}.
\end{lemma}

\subsection{Shadowed patterns}
\label{sec:shadowed-patterns}
If a pattern is unused because of elimination reason \ref{item:unused-reason-2} we say that
it is shadowed.
\begin{definition}[Shadowed]\ \\
  Let
  \[
  m = p_1\texttt{.}e_1 \texttt{|} \ldots \texttt{|} p_n\texttt{.}e_n
  \]
  If $p_j <=' p_i$ for some $1 \leq i < j \leq n$, then $p_j$ is shadowed (by
  $p_i$).
\end{definition}

\subsection{Elimination}

We can now define the elimination of unused patterns.
\begin{definition}[Elimination, $->e$]\
\label{def:shadowed-patterns-1}\\
  We define a reduction relation $->e$ that expresses the
  elimination of exactly one pattern from a match.

  Let
  \[
  m = p_1\texttt{.}e_1 \texttt{|} \ldots \texttt{|} p_n\texttt{.}e_n
  \]
  If there exist a $p_i$ such that $\{p_1, \ldots, p_{i-1}\}$ is a cover or
  $p_i$ is shadowed, then it is unused and can be eliminated. The resulting
  match is
  \[
  m' = p_1\texttt{.}e_1 \texttt{|} \ldots \texttt{|}
  p_{j-1}\texttt{.}e_{j-1} \texttt{|} p_{j+1}\texttt{.}e_{j+1} \texttt{|}
  \ldots \texttt{|} p_n\texttt{.}e_n,
  \]
  and we write $m ->e m'$.
\end{definition}

\begin{example}[Elimination, $->e$]\ \\
  The first two patterns make a cover so the last pattern is eliminated.
  \begin{eqnarray*}[c]
    \begin{eqnalign}[Tl]
\begin{lstlisting}
  Cons(x, xs) . Cons (x, xs)
| Nil . Nil
| x . x
\end{lstlisting}
    \end{eqnalign}
    ->e
    \begin{eqnalign}[Tl]
\begin{lstlisting}
  Cons(x, xs) . Cons (x, xs)
| Nil . Nil
\end{lstlisting}
    \end{eqnalign}
  \end{eqnarray*}
  The second pattern in the example below is shadowed by the first pattern and
  is thus eliminated.
  \begin{eqnarray*}[c]
    \begin{eqnalign}[Tl]
\begin{lstlisting}
  Cons (x, y) . Cons (x, y)
| Cons (Cons (x, y), z) . Cons (Cons (x, y), z)
| Nil
\end{lstlisting}
    \end{eqnalign}
    ->e
    \begin{eqnalign}[Tl]
\begin{lstlisting}
  Cons (x, y) . Cons (x, y)
| Nil
\end{lstlisting}
    \end{eqnalign}
  \end{eqnarray*}

\end{example}

% \begin{lemma}[Preservation]\ \\
%   If an unused pattern is removed from a program, then the resulting program is
%   semantically equivalent.

%   That is
%   \[
%   m ->e m' ==> \fn m \sim \fn m'
%   \]
% \end{lemma}

% \begin{proof}\ \\
%   Trivial (as if).
% \end{proof}

\section{Generalisation}\label{sec:generalisiation}
Sometimes patterns get unnecessary complex. If for example a pattern (or one of
its subpatterns) is a constructor pattern whose subpatterns are all variables,
and those variables are only used as arguments to the same constructor (in the
same order) in the function body, then the constructor could simply be replaced
by a fresh variable in pattern and body. That is generalisation of the clause.

Sometimes the generalisation of a clause makes one or more other clauses in the
match superfluous because their meaning is captured by the generalised one. When
this happens they can be eliminated.
\\[1em]
First we need some auxiliary definitions.

\subsection{Partially ordered form}
\begin{definition}\
  \label{def:part-order-form}\\
  A match $m = p_1\texttt{.}e_1\texttt{|}\ldots\texttt{|}p_n\texttt{.}e_n$ is
  partially ordered if
  \[
  \forall i \in \{1, \ldots, n\} : p_j \not <=' p_i \quad \textnormal{where $j > i$}
  \]
  Note that every match $m$ can be transformed to an equivalent match $m'$ such
  that $m'$ is in partially ordered form, by repeated elimination of shadowed
  patterns (\fref{sec:shadowed-patterns}).
\end{definition}

\subsection{Partial ordering with mapping}
If $p$ is less general than $p'$, that is $p <' p'$, it is because $p'$ has
variables where $p$ has constructors in one or more places. We informally extend
the relation $<'$ to regard this mapping of variables to subpatterns, and we
write $p <'_\pi p'$ if $\pi$ is such a mapping from the variables of $p'$ to
subpatterns of $p$.

\begin{example}[Strict partial ordering on patterns with mapping, $<'_\pi$]\ \\
  We repeat \fref{ex:orderings-patterns-1}.
  \begin{eqnarray*}[c]
    \begin{eqnalign}[Tl]
\begin{lstlisting}
Cons (y, Nil)
\end{lstlisting}
    \end{eqnalign}
    <'_\pi
    \begin{eqnalign}[Tl]
\begin{lstlisting}
Cons (x, z)
\end{lstlisting}
    \end{eqnalign}
    <'_\tau
    \begin{eqnalign}[Tl]
\begin{lstlisting}
b
\end{lstlisting}
    \end{eqnalign}
  \end{eqnarray*}
  where
  \[
  \pi =
  \left\{
    \begin{array}{l@{$\ \mapsto\ $}r}
      \ttt{x} & \ttt{y}\\
      \ttt{z} & \ttt{Nil}
    \end{array}
  \right\}
  \quad \textrm{and} \quad
  \tau =
  \left\{
    \begin{array}{l@{$\ \mapsto\ $}r}
      \ttt{b} & \ttt{Cons (x, z)}
    \end{array}
  \right\}.
  \]
\end{example}

We write $p <='_\pi p'$ to mean $p <'_\pi p' \lor p \equiv_\pi p'$.

\subsection{Substitution}\label{sec:substitution}
We extend the substitution in expressions of variables to whole patterns. We
write
\[
e[p'/p] :\equiv e[\pi(x_i)/x_i]
\]
where $p' <='_\pi p$ and the $x_i$'s range over $dom(\pi)$.

\begin{example}[Pattern substitution]
  Let
  \[
  e = \ttt{Pair (x, y)} \ , \quad
  p = \ttt{Pair (y, x)} \ , \quad
  p' = \ttt{Pair (Cons (a, b), x)}
  \]
  then
  \[
  e[p'/p] = \ttt{Pair (x, Cons (a, b))}
  \]
  with
  \[
  p' <='_\pi p \ , \quad
  \pi =
  \left\{
    \begin{array}{l@{$\ \mapsto\ $}r}
      \ttt{y} & \ttt{Cons (a, b)}\\
      \ttt{x} & \ttt{x}
    \end{array}
  \right\}
  \]
\end{example}

Substitution is important for our purposes because it makes it possible to
decide whether the meaning of one clause contains the meaning of another.
Consider two clauses $c = \clause{p}{e}$ and $c' = \clause{p'}{e'}$ such that
$p' <' p$.

We would like to know if the meaning of $c'$ is contained in $c$. So we
substitute the pattern of $c$ for the pattern of $c'$ to obtain
\[
\hat{e} = e[p'/p]
\]
This is the body of $c$ specialised to input intended for $c'$. Now comparing
\[
\hat{e} \stackrel{?}{=}_\alpha e'
\]
we can decide whether $c'$ is contained in $c$ or not.

\subsection{Generalisation of patterns}
We define generalisation of a single clause. We write $\clause{p}{e} |>
\clause{p'}{e'}$ to mean that the pattern $p$ with its body $e$ generalises to
the pattern $p'$ with the body $e'$.

\begin{definition}[Generalisation of single pattern-body pairs, $|>$]\
  \label{def:gener-patt}\\
  Inductively defined:
  \begin{eqnarray}
    (c \texttt{(} p_1 \texttt{,} \ldots \texttt{,} p_n \texttt{)} , e) &|>& (x , e[x
    / \kappa (c \texttt{(} p_1 \texttt{,} \ldots \texttt{,} p_n \texttt{)} )]) 
    \label{eq:single-gen-1}\\
    (c \texttt{(} p_1 \texttt{,} \ldots \texttt{,} p_i \texttt{,} \ldots
    \texttt{,} p_n \texttt{)}, e) &|>&
    (c \texttt{(} p_1 \texttt{,} \ldots \texttt{,} p'_i \texttt{,} \ldots
    \texttt{,} p_n \texttt{)}, e') \label{eq:single-gen-2}
  \end{eqnarray}
  Where \fref{eq:single-gen-1} holds when $x$ is a fresh variable and
  \[
  FV_{pat}(c\texttt{(}p_1\texttt{,}\ldots\texttt{,}p_n\texttt{)}) \cap FV_{exp}(e[x/\kappa
  (c\texttt{(}p_1\texttt{,}\ldots\texttt{,}p_n\texttt{)})]) = \emptyset
  \]
  (the daunting equation above simply says that every variable of the subpattern
  should be eliminated from the body during generalisation), and
  \fref[plain]{eq:single-gen-2} holds when \fref[plain]{eq:single-gen-1} does not and
  \[
  \clause{p_i}{e} |> \clause{p'_i}{e'}
  \]
\end{definition}

\begin{lemma}\
  \label{lem:single-gen-imp-gen}\\
  If a pattern $p$ (and some body) is generalised to $p'$ (and some other
  body), then $p'$ strictly generalises $p$. In other words
  \[
  \clause{p}{e} |> \clause{p'}{e'} ==> p <' p'.
  \]
\end{lemma}
\begin{proof}\ \\
  Straightforward induction proof.
\end{proof}

\begin{example}\ \\
  This example shows a clause with an unnecessary complex pattern.
  \begin{eqnarray*}[c]
    \begin{eqnalign}[Tl]
\begin{lstlisting}
Pair (Cons (x, xs), zs) . Pair (zs, Cons (x, xs))
\end{lstlisting}
    \end{eqnalign}\\
    |>\\
    \begin{eqnalign}[Tl]
\begin{lstlisting}
Pair (xs, zs) . Pair (zs, xs)
\end{lstlisting}
    \end{eqnalign}
  \end{eqnarray*}
\end{example}

\subsection{Generalisation of matches}
For the generalisation of a match $m$, we require $m$ to be in partially ordered
form.

When generalising a pattern several things might happen. Assume

\begin{eqnarray*}[rqTcql]
  m = p_1 \texttt{.} e_1 \texttt{|} \ldots \texttt{|} p_i \texttt{.} e_i
  \texttt{|} \ldots \texttt{|} p_n \texttt{.} e_n & and & \clause{p_i}{e_i} |> \clause{p'_i}{e'_i}.
\end{eqnarray*}

Now, perhaps $m$ can be generalised if we substitute $p'_i$ for $p_i$ and $e'_i$
for $e_i$. We would like the resulting match to be partially ordered too, so we
must be cautious. Since we know from \fref{lem:single-gen-imp-gen} that $p'_i >'
p_i$ the first part of the match $p_1 \texttt{.} e_1 \texttt{|} \ldots
\texttt{|} p'_i \texttt{.} e'_i$ must still be partially ordered. So we consider
the patterns $p_j$ for $j > i$.

Note that by \fref{lem:unique-rel} we know that either $p_i <' p_j$ or $p_i ||
p_j$ due to $m$ being in partially ordered form.

\paragraph{Generalisation scenarios} \ \\
Four scenarios arise
\begin{enumerate}
\item $p'_i$ and $p_j$ are equivalent. The pattern was generalised to one that
  already existed.

  Since $p'_i \equiv p_j$ we know that $p_i <' p_j$ and thus $\hat{e_j} =
  e_j[p_i/p_j]$ exists. If $\hat{e_j} ==a e_i$ we know that the meaning of
  $\clause{p_i}{e_i}$ and thus $\clause{p'_i}{e'_i}$ is contained in
  $\clause{p_j}{e_j}$. Therefore $\clause{p_i}{e_i}$ can simply be
  removed.\label{item:gen-scen-1}
\item $p'_i$ relates to $p_j$ in the same way that $p_i$ does. So $p_i <' p_j
  \Rightarrow p'_i <' p_j$ and $p_i || p_j \Rightarrow p'_i || p_j$. In this
  case nothing must be done. \label{item:gen-scen-2}
\item $p_i || p_j$ and $p'_i >' p_j$. So now $p'_i$ ``steals'' $p_j$'s input. But
  because $p_i$ and $p_j$ were confused we know that $p_j$ will not steal any
  input originally intended for $p_i$. So we move $p_j$ and its body up in the
  match so they placed before $p'_i$. \label{item:gen-scen-3}
\item $p_i <' p_j$ and $p'_i >' p_j$. This means that $p'_i$ will match input
  intended for $p_j$.

  If the meaning of $\clause{p'_i}{e'_i}$ is the same as for $\clause{p_j}{e_j}$
  for the inputs that $p_j$ matches we can remove $\clause{p_j}{e_j}$. This is
  the case when $e'_i[p_j/p'_i] ==a e_j$. If it is not the case $m$ can not be
  generalised. \label{item:gen-scen-4}
\end{enumerate}

\begin{definition}[Generalisation, $->g$]\
\label{def:gener-match}\\
  We define a reduction relation $->g$ that expresses the
  generalisation of exactly one clause in a match.

  Let
  \begin{eqnarray*}[rqTcql]
    m = p_1 \texttt{.} e_1 \texttt{|} \ldots \texttt{|} p_i \texttt{.} e_i
    \texttt{|} \ldots \texttt{|} p_n \texttt{.} e_n & and & \clause{p_i}{e_i} |> \clause{p'_i}{e'_i}.
  \end{eqnarray*}
  and assume that a generalisation as described above can be done. Then the
  resulting match is
  \begin{eqnarray*}[rclqqqTl]
    m' &=& p_1 \texttt{.} e_1 \texttt{|} \ldots \texttt{|} p_{i-1} \texttt{.}
    e_{i-1} & (Untouched)\label{eq:gen-1}\\
    &\texttt{|}& p_{m_1} \texttt{.} e_{m_1} \texttt{|} \ldots \texttt{|} p_{m_k}
    \texttt{.} e_{m_k} & (Scenario \ref{item:gen-scen-3})\label{eq:gen-2}\\
    (&\texttt{|}& p'_i \texttt{.} e'_i \ \ ) 
    & (Perhaps scenario \ref{item:gen-scen-1})\label{eq:gen-3}\\
    &\texttt{|}& p_{s_1} \texttt{.} e_{s_1} \texttt{|} \ldots \texttt{|} p_{s_l}
    \texttt{.} e_{s_l} & (Scenario \ref{item:gen-scen-2})\label{eq:gen-4}
  \end{eqnarray*}

  Where $m_1 < \ldots < m_k$ and $s_1 < \ldots < s_l$.

  The third line is put in parentheses because it should be deleted in the case
  of scenario \ref{item:gen-scen-1}. Note also that clauses can be removed
  because of scenario \ref{item:gen-scen-4}.

  And we write $m ->g m'$.

  \begin{lemma}\ \\
    If $m ->g m'$ then $m'$ is in partially ordered form.
  \end{lemma}
  \begin{proof}\ \\
    Assume
    \begin{eqnarray*}[rqTcql]
      m = p_1 \texttt{.} e_1 \texttt{|} \ldots \texttt{|} p_i \texttt{.} e_i
      \texttt{|} \ldots \texttt{|} p_n \texttt{.} e_n & and & \clause{p_i}{e_i} |> \clause{p'_i}{e'_i}.
    \end{eqnarray*}

    Since every $p_j <' p'_i$ for $j > i$ is moved in front of $p'_i$ the only
    thing that can break the partial ordering of $m'$ is if $p_j <' p_k$ for
    $i < j < k$ and $p_k$ moves, but $p_j$ does not. For this to happen $p_j$
    must fall into scenario \ref{item:gen-scen-2} and $p_k$ must fall into
    scenario \ref{item:gen-scen-3}.

    Either $p'_i <' p_j$ or $p'_i || p_j$. In the first case we have by
    transitivity that $p'_i <' p_k$, so $p_k$ can not fall into category
    \ref{item:gen-scen-3}, which is a contradiction.

    In the latter case we have by assumption that $p'_i || p_j$ and $p_k <'
    p'_i$. But then we get by lemma \fref{lem:more-specific-confused} that $p_k
    || p_j$, which is a contradiction.
  \end{proof}

\end{definition}

\begin{example}[Generalisation, $->g$]\ \\
Assume the existence of a function \texttt{flip}.

Consider

\begin{sml}
val rec flipodd =
fn Cons (Cons (Pair (a, b), Pair (c, d)), xs) .
     Cons (Cons (Pair (a, b), flip (Pair (c, d))), flipodd xs)
 | Cons (Pair (a, b), Nil) . Cons (Pair (a, b), Nil)
 | Nil . Nil
 | x . x
\end{sml}

As the match is not in partially ordered form (due to the pattern \texttt{x})
our only choice is to eliminate. Luckily we can, because the three first
patterns forms a cover, so we can eliminate the fourth. We get
\begin{sml}
val rec flipodd =
fn Cons (Cons (Pair (a, b), Pair (c, d)), xs) .
     Cons (Cons (Pair (a, b), flip (Pair (c, d))), flipodd xs)
 | Cons (Pair (a, b), Nil) . Cons (Pair (a, b), Nil)
 | Nil . Nil
\end{sml}
which is in partially ordered form. The second clause can be generalised to
\smlinline{x . x}. This means that the clause \smlinline{Nil . Nil} should be removed
because of scenario \ref{item:gen-scen-4}. We get
\begin{sml}
val rec flipodd =
fn Cons (Cons (Pair (a, b), Pair (c, d)), xs) .
     Cons (Cons (Pair (a, b), flip (Pair (c, d))), flipodd xs)
 | x . x
\end{sml}
The first clause can be generalised to

\begin{sml}
Cons (Cons (x, y), xs) . Cons (Cons (x, flip y), flipodd xs)
\end{sml}
\noindent (by generalising the two \texttt{Pair} patterns), so we get

\begin{sml}
val rec flipodd =
fn Cons (Cons (x, y), xs) .
     Cons (Cons (x, flip y), flipodd xs)
 | x . x
\end{sml}
which cannot be generalised or eliminated upon. So it is a normal form.

In \fref{ex:eval-normal-example-flipodd} we use Turtledove to find the normal
form of a similar SML function.

\end{example}

\section{Normal form}
We say that the function $\fn m$ is a normal form if there does not exist an
$m'$ such that $m ->e m'$ or $m ->g m'$ and $m$ is ordered by patterns according
to $<$.

Notice that a partially ordered match (which $m$ must be) can be ordered by $<$
according to \fref{lem:more-specific-confused} and
\fref{lem:total-implies-partial}.

\subsection{Reducing to normal form}
Consider a function $\fn m$. It can be converted to a normal form by repeatedly
eliminating and generalising patterns.

Note that $m$ should be in partially ordered form (\fref{def:part-order-form}) in
order to generalise its patterns. Luckily we can convert it to a partially
ordered form by repeated elimination.

\begin{lemma}\ \\
  If a function $f$ can be converted to the normal form $f'$, then $f$ and $f'$
  are semantically equivalent.

  Since we have not defined the semantics of our toy language this lemma can
  naturally not be proven. The gist of it is that we believe it to be true when
  the normal form conversion is extended to SML. But we have not proven this.
\end{lemma}

\begin{lemma}\ \\
  We expect the normal form to be unique under alpha equivalence. That is if
  $f$  has normal forms
  \[
  f' = \fn \clause{p'_1}{e'_1} \texttt{|} \ldots \texttt{|} \clause{p'_n}{e'_n}
  \quad \textrm{and} \quad
  f'' = \fn \clause{p''_1}{e''_1} \texttt{|} \ldots \texttt{|} \clause{p''_m}{e''_m}
  \]
  then
  \[
  n = m \land \clause{p'_1}{e'_1} ==a \clause{p''_1}{e''_1} \land \cdots \land
  \clause{p'_n}{e'_n} ==a \clause{p''_m}{e''_m}
  \]

  We have not proven this (hopefully only) because of lack of time.
\end{lemma}

\begin{example}\ \\
Assume that $t = \{\texttt{A}, \texttt{B}\}$ is a data type and that the arity
of \texttt{A} and \texttt{B} are 2 and 0 respectively.

Consider
\begin{sml}
val f =
fn A (A (a, b), c) . A (c, A (a, b))
 | A (a, b)        . A (b, a)
 | B               . B
 | x               . x
 | A (a, A (b, c)) . A (A (c, b), a)
\end{sml}
As the match is not in partially ordered form (due to the last pattern being
more specific than the pattern \texttt{x}) our only choice is to
eliminate. Luckily we can, because the four (three actually) first patterns
form a cover. With the last clause eliminated we have

\begin{sml}
val f =
fn A (A (a, b), c) . A (c, A (a, b))
 | A (a, b)        . A (b, a)
 | B               . B
 | x               . x
\end{sml}

From here we can choose to eliminate or generalise. We generalise and it is the
case that
\[
\clause{\texttt{A (A (a, b), c)}}{\texttt{A (c, A (a, b))}}
|>
\clause{\texttt{A (x, y)}}{\texttt{A (y, x)}}
\]

We arrive at generalisation scenario \ref{item:gen-scen-1} since $\texttt{A (a, b)} ==
\texttt{A (x, y)}$. We have
\[
\ttt{A (b, a)}[\ttt{A (A (a, b), c)}/\ttt{A (a, b)}] = \ttt{A (c, A (a, b))}
\]
We require that
\[
\ttt{A (c, A (a, b))}
==a
\ttt{A (c, A (a, b))}
\]
which is obviously the case. So the first clause can be deleted. We now have

\begin{sml}
val f =
fn A (a, b) . A (b, a)
 | B        . B
 | x        . x
\end{sml}

Since the two first patterns form a cover the last clause can be eliminated. We
get

\begin{sml}
val f =
fn A (a, b) . A (b, a)
 | B        . B
\end{sml}

And lastly the second clause can be generalised. We end up with

\begin{sml}
val f =
fn A (a, b) . A (b, a)
 | x        . x
\end{sml}

which cannot be generalised or eliminated further, so it is a normal form.

We demonstrate Turtledove on a similar function in
\fref{ex:eval-normal-example-f}.
\end{example}

\section{Further generalisation}\label{sec:furth-gener}
During generalisation we sometimes need to decide whether two clause bodies have
the same meaning (see generalisation scenario \ref{item:gen-scen-1} in
\fref{sec:generalisiation}). We require the them to be alpha equivalent. It is
certainly true that alpha equivalent clauses have the same meaning but maybe we
could be more lenient.

We have not worked towards this but we have some ideas. In particular we have
considered two techniques:
\begin{description}
\item[Simplification.] Clause bodies that have the same meaning can take many
  forms. One could devise a set of rules for simplifying clause bodies.
\item[Inlining of function calls.] The inlining of function calls where the
  argument(s) is a value can sometimes simplify a clause body as we show below.
\end{description}

Consider the function\footnote{We have seen novice SML programmers make very
  similar (indeed alpha equivalent) mistakes surprisingly often.} (assume that
\texttt{plus} exists)
\begin{sml}
val rec tricky =
fn Cons (x, Nil) . Cons (plus (x, 42), Nil)
 | Cons (x, xs)  . Cons (plus (x, 42), tricky xs)
 | Nil           . Nil
\end{sml}

It's normal form is

\begin{sml}
val rec tricky =
fn Cons (x, Nil) . Cons (plus (x, 42), Nil)
 | Cons (x, xs)  . Cons (plus (x, 42), tricky xs)
 | x             . x
\end{sml}

We would like to be able to tell that the first clause is superfluous. But it
cannot be removed through generalisation:

It can be generalised to

\begin{sml}
Cons (x, xs) . Cons (plus (x, 42), xs)
\end{sml}

We arrive at generalisation scenario \ref{item:gen-scen-1}. If we substitute
the second pattern for the first in the second body we get
\begin{eqnarray*}
\ttt{Cons (plus (x, 42), tricky xs)}[\ttt{Cons (x, xs)}/\ttt{Cons (x, Nil)}] =\\
\ttt{Cons (plus (x, 42), tricky Nil)}
\end{eqnarray*}

which is different from

\[
\ttt{Cons (plus (x, 42), Nil)}
\]

But if we inline \ttt{tricky Nil} we get \ttt{Nil} as desired. And thus we could
arrive at the desired function

\begin{sml}
val rec tricky =
fn Cons (x, xs)  . Cons (plus (x, 42), tricky xs)
 | x             . x
\end{sml}

A major problem with this kind of approach is the inlining of the function
definition -- obviously it makes finding a normal form undecidable.

%%% Local Variables: 
%%% mode: latex
%%% reftex-fref-is-default: t
%%% TeX-master: "../report"
%%% End: 
