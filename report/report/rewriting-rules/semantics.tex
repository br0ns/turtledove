
\section{Semantics}
A rewriting rule ``works'' on a list of SML-clauses. SML clauses are found in
function declarations, recursive value declarations, anonymous functions, case
expressions and exception handlers. A rewriting rule works on a sublist (not
necessarily proper) of such a list of clauses.

% \paragraph{Remark.} In the scheme bodies we allow the keyword \textsf{self},
% which we take to mean a recursive call.

\subsection{Inference rules}
\def\TheTrueColour{BrickRed}
% \def\TheTrueColour{RubineRed}
% \def\TheTrueColour{ProcessBlue}
% \def\TheTrueColour{RoyalBlue}

\newcommand{\cc}[1]{{\color{\TheTrueColour}#1}}
\newcommand{\subspat}[3]{\ensuremath{#1\cc{(}#2\cc{)}\mathrel{\cc{=}}#3}}
\newcommand{\matchpat}[3]{\ensuremath{\cc{{\color{Black}#1}\mathrel{:}\left\langle{\color{Black}#2},\mathrel{ }{\color{Black}#3}\right\rangle}}}
\newcommand{\matchbody}[4]{\ensuremath{#1\cc{,}\mathrel{ }#2\mathrel{\cc{|-}}#3\mathrel{\cc{:}}#4}}
\newcommand{\matchclause}[3]{\ensuremath{#1\mathrel{\cc{|-}}#2\mathrel{\cc{:}}#3}}
\newcommand{\rewrite}[4]{\ensuremath{#1\mathrel{\cc{,}}#2\mathrel{\cc{|-}}#3\mathrel{\cc{\curvearrowright}}#4}}
\newcommand{\becomesthrough}[3]{\ensuremath{#1\mathrel{\textsf{\cc{becomes}}}#2\mathrel{\textsf{\cc{through}}}#3}}

Along with the judgement form we give ``type specifications'' for each variable,
in the sense that any rule of the declared form has an implicit side condition
requiring values to belong to the stated set except for a few places; for
variables and maps we rather give representatives of the desired sets. That is
$\overline{x}$ means ``a pattern variable'' and not ``the particular pattern
variable $\overline{x}$'', and $\sigma$ means a map of the type described above.

Also we have \cc{coloured} the syntax of each judgement form.

\paragraph{Judgement} \fbox{\subspat{\sigma}{spat}{mpat}} \\

Substitute leafs of a scheme pattern according to the environment $\sigma$. Note
how holes are handled (rule 2 and 3).

\[
\inference
{
  \subspat{\sigma}{spat_1}{mpat_1} &
  \cdots &
  \subspat{\sigma}{spat_n}{mpat_n}
}
{
  \subspat
  {\sigma}
  {con_1 (spat_1, \ldots, spat_n)}
  {con_2 (mpat_1, \ldots, mpat_n)}
}
{
  con_1 = con_2
}
\]

\[
\inference
{
}
{
  \subspat{\sigma}
  {\mathcal{C}[mpat_1]\cdots[mpat_n]}
  {mpat_\diamond[mpat_1/\diamond_1]\cdots[mpat_n/\diamond_n]}
}
{
  \sigma ( \mathcal{C} ) = mpat_\diamond
}
\]

\[
\inference
{
}
{
  \subspat{\sigma}{\diamond_i}{\diamond_i}
}
\]

\[
\inference
{
}
{
  \subspat{\sigma}{\ol{a}}{\ol{a}}
}
\]

\paragraph{Judgement} \fbox{\matchpat{pat}{mpat}{\theta}} \\

Expresses that $pat$ is an instance of $mpat$ with the mapping $\theta$ in the
sense that if every pattern variable in $mpat$ is replaced with its image in
$\theta$ then $pat$ is obtained.

\[
\inference
{
  \matchpat{pat_1}{mpat_1}{\theta_1} &
  \cdots &
  \matchpat{pat_n}{mpat_n}{\theta_n} &
}
{
  \matchpat
  {con_1 (pat_1, \ldots, pat_n)}
  {con_2 (mpat_1, \ldots, mpat_n)}
  {\theta_1 ++ \cdots ++ \theta_n}
}
{
  con_1 = con_2
}
\]

\[
\inference
{
}
{
  \matchpat
  {pat}
  {\overline{x}}
  {\{\overline{x} \mapsto pat\}}
}
\]

\paragraph{Judgement} \fbox{\matchbody{\sigma}{\theta}{sexp}{exp}}
\label{sec:match-body}\\

Expresses that $exp$ is an instance of $sexp$ under the environments $\sigma$
and $\theta$. We have omitted the rules that deal with the structure of
expressions and scheme expressions, as it is unchanged.

The binding of SML variables is taken into account, e.g. if
$\matchbody{\sigma}{\theta}{sexp}{exp_1}$ and $exp_1 ==a exp_2$ then $\matchbody{\sigma}{\theta}{sexp}{exp_2}$.

We write $exp[exp'/pat]$ for the substitution of $exp'$ in $exp$ according to
the pattern $pat$ (see \ref{sec:substitution}).

\[
\inference
{
  \matchbody
  {\sigma}
  {\theta}
  {sexp}
  {exp}
}
{
  \matchbody
  {\sigma}
  {\theta}
  {\textsf{self}\ sexp}
  {(id\ x_1\ \cdots\ x_n)[exp/\ttt{($x_1$, $\ldots$, $x_n$)}]}
}
{
  \sigma(\textsf{self}) = (id, n)
}
\]

\[
\inference
{
}
{
  \matchbody
  {\sigma}
  {\theta}
  {\overline{x}}
  {\texttt{x}}
}
{
  \theta(\overline{x}) = \texttt{x}
}
\]

\[
\inference
{
  \matchbody{\sigma}{\theta}{sexp_1}{exp_1} &
  \cdots &
  \matchbody{\sigma}{\theta}{sexp_n}{exp_n}
}
{
  \matchbody
  {\sigma}
  {\theta}
  {\mathbb{E}(sexp_1, \ldots, sexp_n)}
  {exp[\ttt{($exp_1$, $\ldots$, $exp_n$)}/pat]}
}
{
  \sigma(\mathbb{E}) = (pat, exp)
}
\]

The set of hole free meta patterns is a subset of the set of scheme
expressions. Notice that this means that there are exactly $n$ holes in $mpat_\diamond$.

\[
\inference
{
  \matchbody
  {\sigma}
  {\theta}
  {mpat_\diamond[sexp_1/\diamond_1]\cdots[sexp_n/\diamond_n]}
  {exp}
}
{
  \matchbody
  {\sigma}
  {\theta}
  {\mathcal{C}[sexp_1]\cdots[sexp_n]}
  {exp}
}
{
  mpat_\diamond = \sigma(\mathcal{C})
}
\]

\paragraph{Judgement} \fbox{\matchclause{\sigma}{clause}{sclause}} \\

\[
\inference
{
  \subspat{\sigma}{spat}{mpat} &
  \matchpat{pat}{mpat}{\theta} &
  \matchbody{\sigma}{\theta}{sexp}{exp}
}
{
  \matchclause{\sigma}{pat \texttt{ => } exp}{spat => sexp}
}
\]

\paragraph{Judgement} \fbox{\matchclause{\sigma}{clause^{+}}{sclause^{+}}} \\

\[
\inference
{
  \matchclause{\sigma}{clause_1}{sclause_1} &
  \cdots &
  \matchclause{\sigma}{clause_n}{sclause_n}
}
{
  \matchclause
  {\sigma}
  {
    \begin{tabular}{Mc}
      clause_1\\
      \vdots\\
      clause_n\\
    \end{tabular}
  }
  {
    \begin{tabular}{Mc}
      sclause_1\\
      \vdots\\
      sclause_n\\
    \end{tabular}
  }
}
\]
Recall that $\rho$ is an environment of relations on $mpat$'s (see
\fref{sec:definitions}).

\[
\inference
{
  \matchclause{\sigma}{clause^{+}}{sclause^{+}}
}
{
  \matchclause
  {\sigma}
  {clause^{+}}
  {sclause^{+} \texttt{ where }
    \begin{tabular}{Mc}
      \textsf{a}(\mathcal{A}_1, \ldots, \mathcal{A}_n)\\
      \vdots \\
      \textsf{z}(\mathcal{Z}_1, \ldots, \mathcal{Z}_m)
    \end{tabular}
  }
}
{
  \begin{tabular}{Mc}
    \rho(\textsf{a})(\sigma(\mathcal{A}_1), \ldots, \sigma(\mathcal{A}_n))\\
    \vdots \\
    \rho(\textsf{z})(\sigma(\mathcal{Z}_1), \ldots, \sigma(\mathcal{Z}_m))
  \end{tabular}
}
\]

\paragraph{Judgement} \fbox{\rewrite{\sigma}{rule}{clause^{+}}{clause^{+}}} \\

There may be zero or more initial clauses $clause^{*}_1$ before the clauses
$clause^{+}_2$ that match a scheme $scheme$ and then zero or more preceding
clauses $clause^{*}_3$ afterwards. The number of clauses in $clause^{+}_2$ and
$scheme$ needs to be the same as required by
$\matchclause{\sigma}{clause^{+}_2}{scheme}$.

\[
\inference
{
  \matchclause{\sigma}{clause^{+}_2}{scheme} &
  \matchclause{\sigma}{clause'^{+}_2}{sclause^{+}}
}
{
  \rewrite
  {\sigma}
  {scheme \Downarrow sclause^{+}}
  {
    \begin{tabular}{Ml}
      clause^{*}_1\\
      clause^{+}_2\\
      clause^{*}_3\\
    \end{tabular}
  }
  {
    \begin{tabular}{Ml}
      clause^{*}_1\\
      clause'^{+}_2\\
      clause^{*}_3\\
    \end{tabular}
  }
}
{\ \scalebox{1.5}{$(\dagger)$}}
\]
\begin{description}
\item[\scalebox{1.0}{$(\dagger)$}] When designing rewriting rules one must be
  certain that recursive calls are handled by the clauses concerning the rules
  itself. Consider for example
  \begin{sml}
fun foo [x]       = [x + 1]
  | foo (x :: xs) = x + 2 :: foo xs
  | foo nil       = nil
  \end{sml}
  if we didn't have a side condition a rewriting rule could rewrite the function to
  \begin{sml}
fun foo [x] = [x + 1]
  | foo xs  = map (fn x => x + 2) xs
  \end{sml}
  which would be wrong (the argument \smlinline{[1,2]} is a counter example). So
  we require that the patterns of the clauses $clause^{*}_1$ are pairwise
  confused (see \fref{def:pat-confusion}) with the patterns of the clauses
  $clause^{+}_2$.
\end{description}

\paragraph{Judgement} \fbox{\becomesthrough{exp_1}{exp_2}{rule}}\\

Recall that the \smlinline{fun}-syntax is just semantic sugar (see
\cite[appendix B]{SML97}). Rule 2 and 3 are concerned with this.
\[
\inference
{
  \rewrite
  {\sigma[\textsf{self} \mapsto (id, 1)]}
  {rule}
  {clause^{+}}
  {clause'^{+}}
}
{
  \becomesthrough
  {\texttt{val rec $id$ = fn $clause^{+}$}}
  {\texttt{val rec $id$ = fn $clause'^{+}$}}
  {rule}
}
\]

\[
\inference
{
  \rewrite
  {\sigma[\textsf{self} \mapsto (id, n)]}
  {rule}
  {clause^{+}}
  {clause'^{+}}
}
{
  \begin{array}{c}
    \begin{array}{l}
      \texttt{val rec $id$ = fn $pat_1$ => $\ldots$ fn $pat_n$ =>}\\
      \texttt{case ($pat_1$, $\ldots$, $pat_n$) of $clause^{+}$}
    \end{array}\\
    \textsf{\color{\TheTrueColour}\ becomes\ }\\
    \begin{array}{l}
      \texttt{val rec $id$ = fn $pat_1$ => $\ldots$ fn $pat_n$ =>}\\
      \texttt{case ($pat_1$, $\ldots$, $pat_n$) of $clause'^{+}$}
    \end{array}
  \end{array}
  \textsf{\color{\TheTrueColour}\ through\ }
  rule
}
\]

\[
\inference
{
  \rewrite
  {\sigma[\textsf{self} \mapsto \textsf{undef}]}
  {rule}
  {clause^{+}}
  {clause'^{+}}
}
{
  \becomesthrough
  {\texttt{case $exp$ of $clause^{+}$}}
  {\texttt{case $exp$ of $clause'^{+}$}}
  {rule}
}
\]

\[
\inference
{
  \rewrite
  {\sigma[\textsf{self} \mapsto \textsf{undef}]}
  {rule}
  {clause^{+}}
  {clause'^{+}}
}
{
  \becomesthrough
  {\texttt{handle $exp$ of $clause^{+}$}}
  {\texttt{handle $exp$ of $clause'^{+}$}}
  {rule}
}
\]

%%% Local Variables: 
%%% mode: latex
%%% TeX-master: "../report"
%%% End: 
