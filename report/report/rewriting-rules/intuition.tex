\section{Intuition}
In this section we will try to give an intuitive understanding of the rules
given in the previous section, and why we have made them the way they are.

\subsection{Scheme patterns, contexts and meta patterns}
The $\sigma$-environment binds variables of the form $\mathcal{C}$ to meta
patterns with zero or more holes. They can be thought of as contexts since other
(hole free) meta patterns can inhabit the holes. A (variable denoting a) meta
pattern applied to as many hole free meta patterns as it has holes is called a
context. If $\sigma(\mathcal{C}) = \diamond_1\ \ttt{::}\ \ol{ys}$ then --- as an
example --- this is a context:
\[
\mathcal{C}[\ttt{SOME}\ \ol{a}]
\]

Scheme patterns are patterns over contexts and meta variables. So with the same
$\sigma$ as above this could be a scheme pattern:
\[
\ttt{($\ol{a}$, $\mathcal{C}[\ol{y}]$)}
\]

Meta patterns can be instantiated to concrete SML patterns in which case the
pattern variables are bound to subpatterns of the SML pattern. Consider the SML
pattern
\begin{sml}
((a, b), x :: y :: xs)
\end{sml}
recall that \ttt{::} is right associative. Thus the resulting tree is this:
\begin{center}
\begin{tikzpicture}[]
  \node (outer-pair)                               {\smlinline{(, )}};
  \node (dummy)      [below right of = outer-pair] {};
  \node (inner-pair) [below left of = outer-pair]  {\smlinline{(, )}};
  \node (A)          [below left of = inner-pair]  {\smlinline{a}};
  \node (B)          [below right of = inner-pair] {\smlinline{b}};
  \node (fst-cons)   [right of = dummy]            {\smlinline{::}};
  \node (snd-cons)   [below right of = fst-cons]   {\smlinline{::}};
  \node (X)          [below left of = fst-cons]    {\smlinline{x}};
  \node (Y)          [below left of = snd-cons]    {\smlinline{y}};
  \node (XS)         [below right of = snd-cons]   {\smlinline{xs}};
  \path (outer-pair) edge (inner-pair)
        (outer-pair) edge (fst-cons)

        (inner-pair) edge (A)
        (inner-pair) edge (B)

        (fst-cons) edge (X)
        (fst-cons) edge (snd-cons)

        (snd-cons) edge (Y)
        (snd-cons) edge (XS);
\end{tikzpicture}
\end{center}

First a scheme pattern is turned into a meta pattern under the environment
$\sigma$. With the same $\sigma$ as above we would have
\[
\inference
{
  \inference
  {}
  {
    \subspat
    {\sigma}
    {\ol{a}}
    {\ol{a}}
  } &
  \inference
  {}
  {
    \subspat
    {\sigma}
    {\mathcal{C}[\ol{y}]}
    {(\diamond_1\ \ttt{::}\ \ol{ys})[\ol{y}/\diamond_1]}
  }
}
{
  \subspat
  {\sigma}
  {\ttt{($\ol{a}$, $\mathcal{C}[\ol{y}]$)}}
  {\ttt{($\ol{a}$, $\ol{y}\ $::$\ \ol{ys}$)}}
}
\]
and then

\begin{narrow}{-5em}{0em}
\[
\inference
{
  \inference
  {}
  {
    \matchpat
    {\mathsml{(a, b)}}
    {\ol{a}}
    {
      \left\{
        \ol{a} \mapsto \mathsml{(a, b)}
      \right\}
    }
  } &
  \inference
  {
    \inference
    {}
    {
      \matchpat
      {\mathsml{x}}
      {\ol{y}}
      {
        \left\{
          \ol{y} \mapsto \mathsml{x}
        \right\}
      }
    } &
    \inference
    {}
    {
      \matchpat
      {\mathsml{y :: xs}}
      {\ol{ys}}
      {
        \left\{
          \ol{ys} \mapsto \mathsml{y :: xs}
        \right\}
      }
    }
  }
  {
    \matchpat
    {\mathsml{x :: y :: xs}}
    {\ttt{$\ol{y}\ $::$\ \ol{ys}$}}
    {
      \left\{
        \begin{tabular}{Mc@{$\ \mapsto\ $}Ml}
          \ol{y} & \ttt{x}\\
          \ol{ys} & \mathsml{y :: xs}
        \end{tabular}
      \right\}
    }
  }
}
{
  \matchpat
  {\mathsml{((a, b), x :: y :: xs)}}
  {\ttt{($\ol{a}$, $\ol{y}\ $::$\ \ol{ys}$)}}
  {
    \left\{
      \begin{tabular}{Mc@{$\ \mapsto\ $}Ml}
        \ol{a} & \ttt{(a, b)}\\
        \ol{y} & \ttt{x}\\
        \ol{ys} & \mathsml{y :: xs}
      \end{tabular}
    \right\}
  }
}
\]
\end{narrow}

In this case we get
\[
\theta =
\left\{
  \begin{tabular}{Mc@{$\ \mapsto\ $}Ml}
    \ol{a} & \ttt{(a, b)}\\
    \ol{y} & \ttt{x}\\
    \ol{ys} & \mathsml{y :: xs}
  \end{tabular}
\right\}
\]

\subsection{Transformers}
In the bodies of scheme clauses, that is scheme expressions, we want to express that
\textit{some} computation takes place without being too specific about
\textit{what} it actually is. We use transformers to that end.

A transformer is simply a pair of a (concrete) SML pattern and an SML
expression. On can think of it as a single clause, or a function definition
without the benefits of pattern matching.

Consider the function definition
\begin{sml}
fun foo ((a, b) :: xs) = f (b, a) :: foo xs
  | foo nil            = nil
\end{sml}
It is an obvious target for at \textsf{map}-rule. During rewriting we would get
a $\theta$ that binds a variable to \smlinline{(a, b)}. Now we need to express
the relationship between \smlinline{(a, b)} and \smlinline{f (b, a)}. We do that
simply by storing the pattern and the expression.

\subsection{The ``\textsf{self}'' keyword}
Recursion is an important feature of SML. Therefore we need to be able to
express recursive calls in rewriting rules. We use the keyword \textsf{self} to
refer to the function being defined (if any).

It is possible to use curried arguments with the \smlinline{fun}-syntax. But a
\smlinline{fun} declaration is rewritten to a series of anonymous functions with
an inner case expression where the curried arguments is arranged in a
tuple. This means that the arguments are given in curried form in recursive
calls but in tuple form in the patterns.

This detail should not affect rewriting rules. Our solution is to always use
tuples in rewriting rules. Currying the arguments is then done during rewriting
if needed (see the $\fbox{\matchbody{\sigma}{\theta}{sexp}{exp}}$-judgement
above).

\subsection{Schemes}
Schemes are lists of scheme clauses. A scheme clause is essentially a pair of a
scheme pattern and a scheme expression written as
\[
spat => sexp
\]
A concrete SML clause can be an instance of a scheme clause under a particular
$\sigma$-environment. The $\theta$-environment is build on an per clause basis
as we saw above. Let
\[
\sigma =
\left\{
  \begin{tabular}{Mc@{$\ \mapsto\ $}Mr}
    \mathcal{C} & \diamond_1\\
    \mathbb{E} & (\ttt{(v, w)}, \ttt{w + v})\\
    \textsf{self} & (\ttt{foo}, 1)
  \end{tabular}
\right\}
\]
Consider the SML clauses
\begin{eqnarray*}[ll]
  \mathsml{(a, b) :: xs}\ttt{ }&\mathsml{= b + a :: foo xs}\\
  \mathsml{nil}\ttt{ }&\mathsml{= nil}
\end{eqnarray*}
and the scheme clauses
\begin{eqnarray*}[lcl]
  \mathcal{C}[\ol{x}\ \ttt{::}\ \ol{xs}] &=&
\mathbb{E}(\mathcal{C}[\ol{x}])\ \ttt{::}\ \textsf{self}(\mathcal{C}[\ol{xs}])\\
\mathcal{C}[\ttt{nil}] &=& \ttt{nil}
\end{eqnarray*}
Instantiating the meta patterns we get
\[
\inference
{}
{
  \subspat
  {\sigma}
  {\mathcal{C}[\ol{x}\ \ttt{::}\ \ol{xs}]}
  {\ttt{$\ol{x}$\ \ttt{::}\ $\ol{xs}$}}
}
\quad \textrm{and} \quad
\inference
{}
{
  \subspat
  {\sigma}
  {\mathcal{C}[\ttt{nil}]}
  {\ttt{nil}}
}
\]
We see that the SML patterns are instances of the meta patterns:
\[
\inference
{\vdots}
{
  \matchpat
  {\mathsml{(a, b) :: xs}}
  {\ttt{$\ol{x}$\ \ttt{::}\ $\ol{xs}$}}
  {\theta_1}
}
\quad \textrm{and} \quad
\inference
{\vdots}
{
  \matchpat
  {\mathsml{nil}}
  {\ttt{nil}}
  {\theta_2}
}
\]
where
\[
\theta_1 =
    \left\{
      \begin{tabular}{Mc@{$\ \mapsto\ $}Ml}
        \ol{x} & \ttt{(a, b)}\\
        \ol{xs} & \ttt{xs}
      \end{tabular}
    \right\}
    \ , \quad
\theta_2 = \{\}
\]
And finally we can see that the SML bodies are instances of the scheme ditto:
\[
\inference
{
  \inference
  {
    \inference
    {
      \inference
      {
        \inference
        {}
        {
          \matchbody
          {\sigma}
          {\theta_1}
          {\ol{x}}
          {\ttt{(a, b)}}
        }
      }
      {
        \matchbody
        {\sigma}
        {\theta_1}
        {\diamond_1[\ol{x}/\diamond_1]}
        {\ttt{(a, b)}}
      }
    }
    {
      \matchbody
      {\sigma}
      {\theta_1}
      {\mathcal{C}[\ol{x}]}
      {\mathsml{(a, b)}}
    }
  }
  {
    \matchbody
    {\sigma}
    {\theta_1}
    {\mathbb{E}(\mathcal{C}[\ol{x}])}
    {(\mathsml{w + v})[\mathsml{(a, b)}/\mathsml{(v, w)}]}
  } &
  \inference
  {
    \inference
    {
      \inference
      {
        \inference
        {}
        {
          \matchbody
          {\sigma}
          {\theta_1}
          {\ol{xs}}
          {\ttt{xs}}
        }
      }
      {
        \matchbody
        {\sigma}
        {\theta_1}
        {\diamond_1[\ol{xs}/\diamond_1]}
        {\ttt{xs}}
      }
    }
    {
      \matchbody
      {\sigma}
      {\theta_1}
      {\mathcal{C}[\ol{xs}]}
      {\mathsml{xs}}
    }
  }
  {
    \matchbody
    {\sigma}
    {\theta_1}
    {\textsf{self}(\mathcal{C}[\ol{xs}])}
    {(\mathsml{foo x})[\mathsml{xs}/\mathsml{x}]}
  }
}
{
  \matchbody
  {\sigma}
  {\theta_1}
  {
    \mathbb{E}(\mathcal{C}[\ol{x}])\ \ttt{::}\
    \textsf{self}(\mathcal{C}[\ol{xs}])
  }
  {
    \mathsml{b + a :: foo xs}
  }
}
\]
and
\[
\inference
{}
{
  \matchbody
  {\sigma}
  {\theta_2}
  {\ttt{nil}}
  {\ttt{nil}}
}
\]

\subsection{The $\sigma$-environment}
The reader has probably noticed that the inference rules describe how to compute
the $\theta$'s but not the $\sigma$.

As described the $\sigma$ environment contains three things:
\begin{enumerate}
\item \textsf{self}, that is the name and arity of the function being defined.
\item Contexts.
\item Transformers.
\end{enumerate}

We compute each in the following manner.
\begin{description}
\item[\textsf{self}.] The function being defined and its arity is readily
  apparent. The rules of the judgement
  $\fbox{\becomesthrough{exp_1}{exp_2}{rule}}$ describe the value of
  \textsf{self}.
\item[Contexts.] The real problem about $\sigma$ is finding the contexts. In our
  implementation of the \textsf{map}-rule we use a backtracking algorithm for
  finding all candidates.

  Consider the SML clause
  \[
  \mathsml{(x :: xs, y :: ys) = x + y :: foo (x :: xs, ys)}
  \]
  and the scheme clause
  \[
  \mathcal{C}[\ol{z}\ \ttt{::}\ \ol{zs}] = \cdots
  \]
  Should the $\mathcal{C}$ be bound to $\ttt{($\diamond_1$, \smlinline{y ::
      ys})}$ or $\ttt{(\smlinline{x :: xs}, $\diamond_1$)}$? We try both in
  turn. If the first results in the rewriting rule being successful we use that
  one. Else we try the other.
\item[Transformers.] When the contexts are fixed the transformers are easy to
  determine. One can simply use the first clause in which a transformer is
  mentioned as a prototype for it. Since both the contexts and the \textsf{self}
  is known at this point the pattern and the body of the transformer can easily
  be computed once the $\theta$ is known.
\end{description}

\subsection{The $\rho$-environment}
We express restrictions on a scheme with the keyword \textsf{where} followed by
the restrictions. At this time we have only considered one relation, namely
\textsf{samedom}.

Consider the SML functions
\begin{sml}
fun foo (x :: xs, y) = x + y :: foo (xs, y)
  | foo _ = nil
\end{sml}
,
\begin{sml}
fun bar (x :: xs, SOME y) = x + y :: bar (xs, SOME y)
  | bar (_, SOME _) = nil
  | bar _ = raise Fail "foobar"
\end{sml}
and
\begin{sml}
fun baz (x :: xs, SOME y) = x + y :: bar (xs, SOME y)
  | baz _ = nil
\end{sml}

The two first can be rewritten to \ttt{map} forms but not the last. The problem
is that if the last one is rewritten its patterns no longer form a cover.

What we need to know is that \ttt{($\diamond_1$, y)} and \ttt{_} (first
function) and \mbox{\ttt{($\diamond_1$, SOME y)}} and \mbox{\ttt{(_, SOME _)}}
(second function) match the same set of values and that
\mbox{\ttt{($\diamond_1$, SOME y)}} and \ttt{_} does not\footnote{We regard
  $\diamond_n$ as a wildcard pattern.}.

We use the relation \textsf{samedom} to express that two meta patterns match the
same set of values.

\subsubsection{Other relations}
If the rewriting system is extended to handle expressions other useful relations
arise.

Examples include
\begin{description}
\item[\textsf{pure}$(exp)$] Determine if an expression is pure. Because of the
  side-effects and exceptions in SML one cannot in general reorder
  expression. It is for example not the case that
  \[
  exp_1\ \ttt{+}\ exp_2 = exp_2\ \ttt{+}\ exp_1.
  \]
  The (unary) relation $\textsf{pure}(exp)$ would express that $exp$ is free of
  side-effects and exceptions. This is of course undecidable but it is not too
  hard to come up with heuristics that cover a lot of code in practice.
\item[\textsf{eval}$(exp_1, exp_2)$] Try to evaluate an expression. This could be
  useful for live testing of code inside the editor.
\item[\textsf{simp}$(exp_1, exp_2)$] Simplify an expression. Apart from just
  making code simpler this could also make rewriting rules more powerful.
\end{description}

Notice that any rewriting rule is in itself a binary relation (on matches or
expression). Thus one rule could use other rules through constraints.

We have not pursued these ideas in practice.

\section{Some rewriting rules}
We have written rules for \ttt{map}, \ttt{foldl} and \ttt{foldr} forms.

We give the rules and some examples of use in the next chapter.

% \subsection{Relations on meta patterns (the $\rho$-environment)}

% \subsection{Example}

% Introduction...

% \begin{example}[Matching / use of pattern variables] \
% \label{ex:intuition-example}

%   \fixme{Find ordenligt eksempel navn/tekst}

%   \fixme{this example uses dom. Rearrange the text so dom is explained before}

%   \fixme{add reference to the map rule definition, instead of hiding that this
%     is a map example.}
  
%   Given the below function \texttt{foo}, we will try and match it up with the
%   \textsf{map}-rule

% \begin{sml}
% fun foo (a, b :: xs) = b + a
%   | foo _            = 0
% \end{sml}

%   The first clause in \texttt{foo} is an instance of the first
%   scheme in the \textsf{map}-rule

%   \[
%   |\ \mathcal{C}[\overline{x}\texttt{ :: }\overline{xs}] =>
%   \mathbb{D}(\mathcal{C}[\overline{x}])
%   \]

%   with

%   \begin{eqnarray*}[rl]
%     \mathcal{C} &\mapsto \texttt{(}\overline{y}\texttt{,} \diamond\texttt{)}\\
%     \overline{a} &\mapsto \texttt{a}\\
%     \overline{x} &\mapsto \texttt{b}\\
%     \overline{xs} &\mapsto \texttt{xs}\\
%     \mathbb{D}(\texttt{(}\overline{a}\texttt{,} \overline{b}\texttt{)}) &=
%     \kappa(\overline{b}) \texttt{ + } \kappa(\overline{a})
%   \end{eqnarray*}

%   The second clause is an instance of the scheme

%   \[
%   |\ \mathcal{D} => \mathbb{E}(\mathcal{D})
%   \]

%   with

%   \begin{eqnarray*}[rl]
%     \mathcal{D} &\mapsto \overline{a}\\
%     \overline{a} &\mapsto \texttt{\_}\\
%     \mathbb{E}(\overline{a}) &= \texttt{0}
%   \end{eqnarray*}
  
%   Finally, if $dom(\lfloor\mathcal{D}\rfloor) =
%   dom(\lfloor\mathcal{C[\texttt{\_}]}\rfloor)$, which is the case here, then the
%   function \texttt{foo} is an instance of the $\mathsf{map}$-rule and may be
%   rewritten accordingly:

%   \fixme{maybe some more text about the dom}

%   \begin{eqnarray*}[c]
%     \texttt{fun foo $\lfloor \mathcal{C} \rfloor [\lfloor xs \rfloor]$ = map (fn $x$ =>
%       $\mathbb{D}(\lfloor \mathcal{C} \rfloor[\lfloor x \rfloor])$) $\lfloor xs
%       \rfloor$}\\
%     =\\
%     \texttt{fun foo (a, b) = map (fn b => b + a) b}
%   \end{eqnarray*}
  
%   We have explicitly written the canonical instantiation, but as noted before
%   this will not always be done.
% \end{example}



%%% Local Variables: 
%%% mode: latex
%%% TeX-master: "../report"
%%% End: 
