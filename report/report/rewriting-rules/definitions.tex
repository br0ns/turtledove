\section{Definitions}\label{sec:definitions}
% \subsection{Patterns}
A subset of patterns in SML (see \fref{sec:preparation}) can be thought of as
trees with constructors as nodes and variables as leafs (where nullary
constructors are seen as nodes with no leafs).

In the following chapter we will be generalising patterns to be parameterised
over their leafs, meaning that SML patterns are patterns over (SML) variables.

We follow the convention that constants can be thought of as being nullary
constructors\footnote{As in our toy language of
  \fref[plain]{chap:normal-form}.}.

We write concrete SML in typewriter font as this:
\begin{sml}
fun id x = x
\end{sml}

In the following we use the names given in \cite{SML97} for the different
syntactic categories, we assume a fixed SML environment $E = (S\!E, T\!E, V\!E)$
and a set of $n$-ary relations over $mpat$ (defined below), $\rho$.

\begin{definition}
  We write $pat^\alpha$ to mean the subset of the term algebra
  \[
  T(con, \alpha)
  \]
  where the signature $con$ consists of the constructors of $E$ and the set of
  record constructors, and each variable occurs at most once.
\end{definition}

\paragraph{Note.} One can think of $pat^\alpha$ as SML-patterns generalised to
any kind of ``variables''. The constructors of datatypes are either constants or
unary functions. The record constructors form a countably infinite set of
functions with finite arity. To distinguish constructors from the surrounding
program text we write them with an initial capital letter (with very few
exceptions like \codeinline{::} and \codeinline{nil}). We respect the infix
status of the symbols given by $E$.

\begin{description}
\item[Record constructors] The record constructor given by
  \[
  (a, b) \mapsto \texttt{\{foo = $a$, bar = $b$\}}
  \]
  is denoted by $\mathfrak{R}_{\texttt{foo},\texttt{bar}}$.

  Recall that the tuple constructors are just syntactic sugar for a subset of
  the record constructors. We write $\mathfrak{T}_n$ to mean
  $\mathfrak{R}_{\texttt{1},\ldots,\texttt{n}}$.
\end{description}

\paragraph{Syntax}\ \\
\renewcommand{\arraystretch}{1.5}
\begin{tabular}{lMlMcMl}
  \textbf{Kind} & \textbf{Category} && \textbf{Is}\\
  (SML) identifiers & id, con & ::= & \mathtt{x} \mid
  \mathtt{xs} \mid \mathtt{A} \mid \mathtt{B} \mid \ldots \\
  Pattern variables & patvar & ::= & \overline{x} \mid \overline{y} \mid \ldots\\
  Holes & hole & ::= & \diamond_1 \mid \diamond_2 \mid \ldots \\

  Meta patterns & mpat & ::= & pat^{patvar}\\
  & mpat_\diamond & ::= & pat^{patvar \cup hole}\\

  Contexts & ctx & ::= & ctxvar[mpat]^{*}\\
  Context variables & ctxvar & ::= & \mathcal{A} \mid \mathcal{B} \mid
  \ldots\\

  Scheme patterns & spat & ::= & pat^{ctx \cup patvar}\\
  Scheme expressions & sexp & ::= & \textrm{ (see \fref{tab:scheme-expressions})}
  \\
  Scheme clauses & sclause & ::= & spat => sexp \\
  Schemes & scheme & ::= & sclause^{+}\\
  & & \mid & sclause^{+} \textsf{ where } cstrn^{+}\\

  Transformers & trans & ::= & transvar\ sexp\\
  Transformer variables & transvar & ::= & \mathbb{D} \mid \mathbb{E} \mid \ldots\\

  Constraint &  cstrn & ::= & rel(mpatvar^{+}) \\
  Relation & rel & ::= & \textsf{a} \mid \textsf{b} \mid \ldots\\

  (Rewriting) Rules & rule & ::= & scheme \Downarrow sclause^{+}\\
\end{tabular}

\paragraph{Scheme expressions.}
Ordinary SML expression without types (as explained in \cite[section
6.1]{SML97}) are extended with pattern variables, transformers and
meta expressions of the form
\[
mexp ::= mpatvar[sexp]^{*}
\]

See \fref{tab:rule-grammar} for the full grammar.

% \paragraph{Aliases}\ \\
% \begin{tabular}{lMlMl}
%   \textbf{Kind} & \textbf{Alias} & \textbf{Is}\\
%   Patterns & pat & = pat^{id}\\
%   % Meta patterns & mpat & = pat^{patvar \cup hole} \qquad (\dagger)\\
%   Scheme patterns & spat & = pat^{mpat_0}
% \end{tabular}\\
% $(\dagger)$ We write $mpat_0$ for the subclass of meta patterns with no holes,
% $mpat_1$ for the subclass with exactly one hole, etc.

\paragraph{Environment.}
We use two kinds of maps during rewriting. We denote them $\sigma$ and
$\theta$. $\sigma$ is global to a particular rewriting and $\theta$ is local to
each clause.

Maps support three operations: binding ($[a \mapsto 42]$), lookup ($\psi(a)$)
and union ($\psi_1 ++ \psi_2$). For union the second map takes precedence if a
variable is bound in both maps (see note \ref{item:note-plusplus} in
\fref{sec:auxil-defin}). We write $\psi[a \mapsto 42]$ as a shorthand for $\psi
++ [a \mapsto 42]$.

Below is listed the different kinds of variables and their domains. $\sigma$
maps $recvar$'s, $transvar$'s and $mvar$'s, and $\theta$ maps $patvar$'s.

\paragraph{Variables}\ \\
\begin{tabular}{lMlMlMl}
  \textbf{Over} & \textbf{Category}& \textbf{Notation} & \textbf{Domain}\\
  Recursion & recvar & \textsf{self} & (id \times \mathbb{N}) \cup \textsf{undef}\\
  % Constructors & convar & \mathfrak{a}, \mathfrak{b}, \ldots & con\\
  Pattern variables & patvar & \overline{a}, \overline{b}, \ldots & pat\\
  % Patterns & patvar & \overline{x}, \overline{y}, \ldots & pat^{id} \\
  Transformers & transvar &  \mathbb{D}, \mathbb{E}, \ldots & pat \times exp\\
  Contexts & ctxvar & \mathcal{A}, \mathcal{B}, \ldots & mpat_\diamond \\
  % Scheme patterns & spatvar & \alpha, \beta, \ldots & spat \\

\end{tabular}
\renewcommand{\arraystretch}{1}

%%% Local Variables: 
%%% mode: latex
%%% TeX-master: "../report"
%%% End: 
