\section{Intuition}

This section will try and give an intuition into how why and what the
definitions above are all good for \fixme{proper introduction}


\subsection{Meta patterns}
\fixme{Reference to the definition}
Meta patterns ($\mathcal{C}$) are patterns over the union of patternvariables
($\overline{x}$) and holes ($\diamond$). A meta pattern can be thought of as
higher-order in the sense that it may contain one or more holes. Every hole in a
meta pattern can be replaced by another meta pattern and we write
$\mathcal{C}[\texttt{SOME } \overline{x}]$ to insert the meta pattern
$\texttt{SOME } \overline{x}$ at the first\footnote{with a suitable definition
  of ``first''.} holes place in $\mathcal{C}$.

Meta patterns can, as SML patterns, bee seen as a tree with constructors in the
nodes and pattern variables and holes in the liefs. Thus this concrete SML
pattern \texttt{((a,\ b),\ x\ ::\ y\ ::\ xs)} is an instance of the following
meta pattern (shown graphically)


\begin{tikzpicture}[]
  \node (pair)     {$(,)$};
  
  \node (X)  [below left of = pair] {$\overline{x}$};
  
  \node (cons)  [below right of = pair] {$::$};
  
  \node (Y)  [below left of = cons] {$\overline{y}$};
  \node (Z)  [below right of = cons] {$\overline{z}$};
  
  \path (pair) edge (X)
        (pair) edge (cons)
    
        (cons) edge (Y)
        (cons) edge (Z);
\end{tikzpicture}


\subsection{Schemes}

\fixme{Is this the correct syntax used in the mathematical definitions?}

\fixme{Add a ref to the definition}

Schemes are lists of pairs of scheme patterns and scheme expressions (also
called a clause) where we write $|\ spat => sexp$ to mean the pair $(spat,
sexp)$. A piece of concrete SML code can be an instance of a scheme, where the meta
patterns gets instantiated accordingly. 

Pattern variables in a scheme pattern gets instantiated per clause and they are
implicitly replaced by the concrete SML pattern they are bound to from the left
hand side of the clause (scheme pattern) when used on the right hand side of the
clause (scheme expression), see \fref{ex:intuition-example}

whereas meta patterns gets instantiated across the whole
scheme.


\subsubsection{Scheme patterns}
\fixme{Reference to the definition}
Scheme patterns can be seen as patterns over meta patterns. If this was not the
case and we defined them as just being meta patterns then we would be able to
express patterns as $(\overline{x}, \overline{y})$ and $\mathcal{C}[\overline{x}
:: \overline{xs}]$, however we would not be able to express patterns as
$(\overline{x}, \mathcal{C}[\overline{y} :: \overline{ys}])$ since the liefs of
meta patterns may only be pattern variables or holes.

By defining scheme patterns as patterns over meta patterns we allow before
mentioned pattern and thus gains more expressive power

\subsubsection{Scheme expressions}
\fixme{Reference to the definition}
blalblalbla

\subsection{Rewriting rules}

\fixme{This is utterly wrong.}
\fixme{Reference to the definition}


A rewriting rule is defined from the clauses of the function that it must match,
thus it doesn't define any meta information such as the functions name. However
to be able to match recursive functions we introduce the keyword \textsf{self}
as a placeholder for the name of the function that is being matched.

\fixme{Her kommer normalformen på noget af en opgave, men det ser ud til at
  virke. Tjek det!}

A rule may match partially in the sense that if the clauses $n$ through $n+1$ of
a function is an instance of a rule with $i$ schemes, then the rewriting may still be
done for those specific clauses.

\fixme{example hereof}

\fixme{This might be broken up and then the canonical instantiation may be brought
  up and the rest of the text below in a subsubsection}. 

% subsubsection{simplicity of the syntax and why it looks as it does}

\fixme{Most of the below is more aimed at the schemes}

We can use the same rule syntax for both curried and uncurried functions by not
expressing more information about the function or its parameters in the rule.

It might be remembered that the curried function below

\begin{sml}
fun add n nil       = nil
  | add n (x :: xs) = n + x :: add n xs
\end{sml}

is desugared (in multiple steps) to

\begin{sml}
val rec add = fn n => 
              fn xs => (fn (n, nil) => nil
                         | (n, x::xs) => n + x :: add n xs) (n, xs)
\end{sml} 

and the same uncurried function 

\begin{sml}
fun add' (n, nil)     = nil
  | add' (n, x :: Xs) = n + x :: add' (n, xs)
\end{sml}

becomes

\begin{sml}
val rec add' = fn (n, nil) = []
                | (n, x::xs) = n + x :: add' (n, xs)
\end{sml}

               

As it is seen, both cases end up with an anonymous function. If there are more
than one then only the inner most is of interest as this anonymous
function gets the two arguments as a tuple. This anonymous function has a number
of clauses with pattern and body which match that of the original function

We have chosen to define our schemes in such a way that they look similar to the
above. Alle clauses in a scheme starts on $|$ (which except for the first clause
also apply for SML), since we have partial matching and thus don't know if the
the first function clause will match the first scheme clause.

We believe this makes up the most natural way of expressing schemes that will
match SML and thus a scheme is a list of $n$ scheme clauses

\begin{eqnarray*}[rqrl]
| & spat_1 & => sexp_1 \\
  &  & \vdots \\
| & spat_n & => sexp_n
\end{eqnarray*}

where $spat_n$ is a scheme pattern and $sexp_n$ is a scheme expression.

\fixme{Der beskrives slet ikke hvordan og hvorledes med resultat
  omskrivningen. Der beskrives kun den matchende del af skabelonen.}

\subsubsection{Restrictions, predicates and meta functions}

\fixme{This is a lot of ideas of what we might need. It should be more concrete}

We express restrictions on a scheme with the keyword \textsf{where} followed by
the restrictions \fixme{insert ref to the syntax definition}. 

As we saw in the \textsf{map} example above it may be necessary to introduce
restrictions on some of the meta patterns as a function of others, such as
$dom$.

Predicates and functions over expressions (and with extension over meta
functions) \fixme{why over metafunctions?} might come in handy, such as a
predicate that may determine whether an expression has side effects.

blabla....

Metafunktionerne implementeres nok kønnest ``uden for''
omskrivningssystemet. F.eks. omformes erklæringer til normalform før de
forsøges omskrevet, og man kunne forestille sig et forsimplingstrin
efterfølgende (eta-reduktion vil være ret nyttigt). Andre ting som
\texttt{let}-løftning bør også overvejes.

Bemærk at metafunktionerne kan have en nytteværdi i sig selv, og ikke blot
sammen med omskrivningssystemet.


\subsubsection{Canonical instantiation}
\fixme{Reference to the definition}
The canonical instantiation of a meta pattern ($\floor{\mathcal{C}}$) is when
all pattern variables are instantiated to concrete SML variables with capture
avoidance. 

\fixme{add something about using of meta patterns from the schemes}

This is used in the resulting part of a rewriting rule, because pattern
variables aren't bound to any concrete SML code.

Example: $\floor{\mathcal{C}}$ $\mathcal{C} = \texttt{(}\diamond\texttt{,} \overline{y}\texttt{)}$:
$\floor{\mathcal{C}} = \floor{\texttt{(}\diamond\texttt{,} \overline{y}\texttt{)}} = \texttt{(}\diamond\texttt{,}
\floor{\overline{y}}\texttt{)} = \texttt{(}\diamond\texttt{,} \texttt{a)}$.

\paragraph{Note.} The canonical instantiation only occurs in the right hand side
of rewriting rules and since they are mandatory we will let it be implicit in the
rest of the document.

\subsection{Example}

Introduction...

\begin{example}[Matching / use of pattern variables] \
\label{ex:intuition-example}

  \fixme{Find ordenligt eksempel navn/tekst}

  \fixme{this example uses dom. Rearrange the text so dom is explained before}

  \fixme{add reference to the map rule definition, instead of hiding that this
    is a map example.}
  
  Given the below function \texttt{foo}, we will try and match it up with the
  \textsf{map}-rule

\begin{sml}
fun foo (a, b :: xs) = b + a
  | foo _            = 0
\end{sml}

  The first clause in \texttt{foo} is an instance of the first
  scheme in the \textsf{map}-rule

  \[
  |\ \mathcal{C}[\overline{x}\texttt{ :: }\overline{xs}] =>
  \mathbb{D}(\mathcal{C}[\overline{x}])
  \]

  with

  \begin{eqnarray*}[rl]
    \mathcal{C} &\mapsto \texttt{(}\overline{y}\texttt{,} \diamond\texttt{)}\\
    \overline{a} &\mapsto \texttt{a}\\
    \overline{x} &\mapsto \texttt{b}\\
    \overline{xs} &\mapsto \texttt{xs}\\
    \mathbb{D}(\texttt{(}\overline{a}\texttt{,} \overline{b}\texttt{)}) &=
    \kappa(\overline{b}) \texttt{ + } \kappa(\overline{a})
  \end{eqnarray*}

  The second clause is an instance of the scheme

  \[
  |\ \mathcal{D} => \mathbb{E}(\mathcal{D})
  \]

  with

  \begin{eqnarray*}[rl]
    \mathcal{D} &\mapsto \overline{a}\\
    \overline{a} &\mapsto \texttt{\_}\\
    \mathbb{E}(\overline{a}) &= \texttt{0}
  \end{eqnarray*}
  
  Finally, if $dom(\lfloor\mathcal{D}\rfloor) =
  dom(\lfloor\mathcal{C[\texttt{\_}]}\rfloor)$, which is the case here, then the
  function \texttt{foo} is an instance of the $\mathsf{map}$-rule and may be
  rewritten accordingly:

  \fixme{maybe some more text about the dom}

  \begin{eqnarray*}[c]
    \texttt{fun foo $\lfloor \mathcal{C} \rfloor [\lfloor xs \rfloor]$ = map (fn $x$ =>
      $\mathbb{D}(\lfloor \mathcal{C} \rfloor[\lfloor x \rfloor])$) $\lfloor xs
      \rfloor$}\\
    =\\
    \texttt{fun foo (a, b) = map (fn b => b + a) b}
  \end{eqnarray*}
  
  We have explicitly written the canonical instantiation, but as noted before
  this will not always be done.
\end{example}



%%% Local Variables: 
%%% mode: latex
%%% TeX-master: "../report"
%%% End: 
