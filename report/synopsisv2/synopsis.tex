\documentclass[a4paper,oneside]{article}

\usepackage[T1]{fontenc}
\usepackage[utf8]{inputenc}
\usepackage[english]{babel}

\usepackage[format=hang]{caption,subfig}
\usepackage{graphicx}
\usepackage[draft]{fixme}     % Indsæt "fixme" noter i drafts.
\usepackage{hyperref}  % Indsæter links (interne og eksterne) i PDF

\setcounter{secnumdepth}{1} % Sæt overskriftsnummereringsdybde. Disable = -1.


\author{Morten Brøns-Pedersen \and Jesper Reenberg}
\title{Synopsis}

\begin{document}

\maketitle

\fixme{Search and replace ``programmer assisting tools'' with PATs}
\fixme{Search and replace ``plugins'' with new word ``tools''}

\section{Title}
``Turtledove: Tool assisted programming in SML, with special emphasis on semi-automatic rewriting of
declarations to predefined standard forms.''

\section{Motivation}
Today functional programming languages are not very widespread outside academic circles. A reason
for this is that not many PATs (programmer assisting tools) exist for functional programming languages
compared to certain imperative languages.

We wish to make functional programming, specifically in SML, easier to approach for the beginner as
well as more advantageous for the veteran. The means by which we hope to accomplish this task is the
development of a framework for implementing PATs, and one or more actual
tools.

Functional programming languages has a wide variety of advantages that makes them specially suited
for quick and correct statical analysis. The effect of this is that it will be possible to make more
complex tools for functional programming languages than for imperative ones.
\\

We emphasise on a single tool. Namely semi-automatic rewriting of terms to predefined standard
forms. \fixme{Hvorfor lige dette værktøj?}

\section{Elaboration}
The project can be divided into three parts.
\begin{description}
\item[Front end] The front end is responsible for communicating with the programmer. This
communication goes both ways. Some tools will work automatically, waiting for the programmer to
accept corrections and/or suggestions. Some will be initiated by the user.
\item[Back end] The back end will perform tasks common to most tools. This includes communicating
with the front end in a suitable format, reading and possibly writing project files, reading and
parsing source files, defining a representation of syntax trees, keeping an internal representation
of the source code up to date and defining auxiliary functions (e.g. functions for converting syntax
trees to source code, converting data to a format suitable for communication with the front end,
etc.).
\item[Tools] The tools will be initiated by the back end from which they will get a representation
of the source code as a syntax tree or the source code itself. Messages to the individual tools from
the front end will be delivered via the back end. The tools can send messages back to the front end
through the back end. In this project we will focus on a single tool. Se section
\ref{primary_goals:a_refactoring_tool} below.
\end{description}

The back end and the tools are split up because most tools will have a lot of tasks in common as
described above.  So while the back end and the tools are actually the same piece of software, the
distinction is an important one.

The importance of this distinction implies that we develop a well defined API for communication
between the back end and the tools.

We describe the project in greater detail in the next section.
\fixme{Denne sætning passer måske ikke så godt ind?}

\section{Primary goals}
We focus our work on the back end and a single tool. A simple front end (e.g. for Emacs) would make
testing a lot easier, but it is not a requirement for the success of this project.

As we wish to develop the back end and tool(s) separately from the front end, we choose to let all
communication between the two be through plain text.

The back end and the tool(s) compiles into a single program. We name that program and it's source
code ``Turtledove'', and we use the name interchangeably.

\subsection{Turtledove: tool assisted programming for SML} 
Turtledove will be used for two purposes; i) tool assisted programming, and ii) implementation of
new tools.

Since we intent that Turtledove should grow to include more than one tool, the
protocol over which Turtledove and a compliant front end communicate needs to be
flexible enough to accommodate this.

We choose the most flexible protocol whose only feature is to guide information to and from a
specific tool. This implies that each tool will have to develop it's own protocol, and that the
front end will have to support each tool individually. We don't believe this to be a serious
limitation in practice.

Therefore we will develop a small framework for serializing (e.g. as JSON) and unserializing data as
a part of Turtledove.
\\

Turtledove will communicate with the front end through standard streams (i.e., stdin and stdout).
\\

Turtledove itself needs to be programmed in a modular way, as to make it easy to implement new tools
later. The tools shall each be confined and only interact with the rest of the program through an
API and/or the inclusion of libraries.

The API will provide the tools with functions for;
\begin{itemize}
\item retrieving (up-to-date) syntax trees and raw code. No tool should access files on it's own as
      Turtledove might track chances in files that has not yet been saved.
\item receiving data from the front end or Turtledove (e.g. ``do this'' or ``this file has changed'').
\item sending data to the front end.
\end{itemize}

The functionality of Turtledove is split between what Turtledove itself does --- the core
functionality --- and the framework that is available to programmers who write tools.
\subsubsection{Core functionality}
The core functionality is quite limited. Turtledove does two things:
\begin{enumerate}
\item Turtledove will take the path of a SML file or project file either as a command line argument
      or through the input stream after it has been started. It then reads and parses the file or
      files contained in the project. We choose SML basis files as project files.

      The front end should signal when a file is changed and possibly what has changed within that
      file. The one extreme is to signal a file change only when a file is saved. The other is to
      signal every key stroke. Testing will reveal the right balance.

      Tools shall be signaled when the source code has changed.

\item The communication between the front end and a tool is carried by Turtledove. Turtledove will
      determine the correct receiver (possibly itself) and guide data to it.

      The execution of tools will be initiated by Turtledove and done in a threaded manner. It will
      also carry data from the tools to the front end.
\end{enumerate}

\subsubsection{Framework}
Turtledove will contain a framework that helps tools perform common tasks.

This includes but is not limited to the following

\begin{description}\fixme{Skriv...}
\item[Parsing] This is crucial for the refactoring tool to.......

\item[Syntax tree representation] A way to represent syntax trees in a clever and reusable way
  that will also benefit tools to come.

\item[Pretty printing] To convert syntax tree to code

\item[Serializing] The communication bridge and protocol that links the back
  end with the front end and manages inter communication.
\end{description}

% Derudover skal der lavet en del funktionalitet som værktøjer kan bruge i form af
% biblioteker. Eksempler:
% \begin{itemize}
% \item SML-parser. Det kan f.eks. være snedigt at lagre vores ``standardformer'' som SML-kode, og så
%       indlæse det på den måde.
%
%    Hvilke "standardformer" ?
%
% \item Repræsentation af syntaxtræer.
% \item Syntaxtræ-til-kode-omformer.
% \item JSON-bibliotek, samt passende datastrukturer.
% \end{itemize}

\subsection{A refactoring tool}
\label{primary_goals:a_refactoring_tool}
Sådan som vi aftalte...

\subsection{Summary}
To summarize: the primary goals of Turtledove are
\begin{enumerate}
\item Read messages from the front end and send them to the appropriate tool or
      act directly upon them.
  
\item Send messages replies back to the front end.

\item Read and possibly write SML basis files.
\footnote{Some of the most popular SML implementations support SML basis files:
\href{http://mlton.org/}{MLton} and \href{http://www.itu.dk/research/mlkit}{MLKit} support them
directly and there is a tool that converts SML basis files to the CM files used by
\href{http://smlnj.org}{SML/NJ} automatically.}

\item Read source code and construct syntax trees.

\item Notify tools when the source code has been changed. This should be done as a minimum when
      source files are saved and preferably more often.

\item Implement a framework that help tools do common tasks.
\end{enumerate}

\section{Secondary goals}

\subsection{Project management}
Accepting changes to a project from the front end and then write those changes to the SML basis file
containing the project.

\subsection{A nice front end}
For Emacs obviously.

\section{Limitations}
\fixme{Nothing has to work as intended.}

\begin{itemize}
\item Turtledove need not be able to read CM project files.

\item The SML parser need not handle syntax errors.

\item Changes to unsaved files need not be reflected inside Turtledove.

\item No user documentation of software will be produced (e.g., manual, user guide).
\end{enumerate}

\section{Learning objectives}

\begin{enumerate}
\item To develop structured, flexible and well-documented software for
      manipulation of source code.
\item To understand and describe source code transformations formally, and to implement such
      transformations.
\end{enumerate}

\section{Tasks and timeline}

\begin{description}
\item[About six months] Do the damn thing.
\item[2 days] Drink our brains out.
\item[Fix the timeline]
\end{description}

\section{Expansion (``Nice to have'')}
\fixme{Fjern ``modulnavne''. De hører ikke hjemme her}

\begin{enumerate}

\item An incremental \texttt{SML parser}

\item That the \texttt{SML Parser} can handle syntax errors. That is,
  by skipping the source code until next valid expression and then
  continue to parse. The skipped part could be reported to the
  front-end and then handled appropriately (e.g., underlined).

  This functionality might be available through SML-Yacc.

\item The \texttt{Source Management} implements a patch system, so
  when a component request the content of a file, which is is edited
  in the IDE and not saved to the file system, it gets the content
  from the file system but where a line is edited in the editor it
  gets the line from the editor instead.


\item Support of \texttt{DocString} by the \texttt{Communication
    Bridge} so when the front end hovers a function call it can show
  the \texttt{DocString} describing parameters and what the function
  does.

\item Multiple levels of parsing of the syntax tree for example to gradually
  remove syntactic sugar.
  \begin{itemize}

  \item A notion of ``run levels'' could be added, so at the end of each parsing
    all plugins that hook this level will be executed and given access to this
    specific version of the parse tree. While the plugins execute the next parse
    of the syntax tree should continue if there is plugins depending of
    one of the next run levels.

  \end{itemize}

\item The \texttt{Source Managemnt} needs to implement some sort of patch
  system, such that when a component request the content of a file and the user
  has made changes in the IDE that is not save to the file, these changes are
  also reflected in the content returned to the component.

\end{enumerate}

\end{document}
