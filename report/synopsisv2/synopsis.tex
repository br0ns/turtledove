\documentclass[a4paper,11pt,fleqn]{article}
\usepackage[english]{babel}
\usepackage[utf8]{inputenc}
\usepackage{amsfonts}
\usepackage{amsmath}
\usepackage{amssymb}
\usepackage{amsthm}

\author{Morten Brøns-Pedersen \and Jesper Reenberg}
\title{Synopsis}

\begin{document}

\maketitle

\section{Title}
``Programmer tools based on static semantic analysis of SML programs, with special emphasis on
semi-automatic rewriting of declarations to predefined forms.''

\section{Motivation}
Today functional programming languages are not very widespread outside academic circles. A reason
for this is that not many programmer assisting tools exist for functional programming languages
compared to certain imperative languages.

We wish to make functional programming, specifically in SML, easier to approach for the beginner as
well as more advantageous for the veteran. The means by which we hope to accomplish this task is the
development of a framework for implementing programmer assisting tools, and one or more actual
tools.

Functional programming languages has a wide variety of advantages that make them specially suited
for quick and correct statical analysis. The effect of this is that it will be possible to make more
complex tools for functional programming languages than for imperative ones.

\section{Elaboration}
The project can be divided into three parts.
\begin{description}
\item[Front end] The front end is responsible for communicating with the programmer. This
communication goes both ways. Some tools will work automatically waiting for the programmer to
accept corrections and/or suggestions. Some will be initiated by the user.
\item[Back end] The back end will perform tasks common to most tools. This includes communicating
with the front end in a suitable format, reading and possibly writing project files, reading and
parsing source files, defining a representation of syntax trees, keeping an internal representation
of the source code up to date and defining auxiliary functions (e.g. functions for converting syntax
trees to source code, converting data to a format suitable for communication with the front end,
etc.).
\item[Tools] The tools will be initiated by the back end from which they will get a representation
of the source code as a syntax tree or the source code itself. Messages to the individual tools from
the front end will be delivered via the back end. The tools can send messages back to the front end
through the back end. In this project we will focus on a single tool. Se section
\ref{primary_goals:a_refactoring_tool} below.
\end{description}

The back end and the tools are split up because most tools will have a lot of tasks in common as
described above.

So while the back end and the tools are actually the same piece of software, the distinction is an
important one.

The importance of this distinctions implies that we develop a well defined API for communication
between the back end and the tools.

\section{Primary goals}

\subsection{Turtledove: a framework for implementing tools based on semantic analysis}
Hear ye, here ye! All behold the great Turtledove.

\subsection{A refactoring tool}\label{primary_goals:a_refactoring_tool}
Proof of concept, and Jakob's pet project.

\section{Secondary goals}

\subsection{A nice front end}
For Emacs obviously.

\section{Limitation}
Nothing has to work as intended.

\section{Tasks and timeline}
\begin{quote}
\begin{description}
\item[\bf About six months] Do the damn thing.
\item[\bf 2 days] Drink our brains out.
\end{description}
\end{quote}

\end{document}
