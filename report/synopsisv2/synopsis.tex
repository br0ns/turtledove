\documentclass[a4paper,11pt,fleqn]{article}
\usepackage[english]{babel}
\usepackage[utf8]{inputenc}
\usepackage{amsfonts}
\usepackage{amsmath}
\usepackage{amssymb}
\usepackage{amsthm}
\usepackage[draft]{fixme}

\author{Morten Brøns-Pedersen \and Jesper Reenberg}
\title{Synopsis}

\begin{document}

\maketitle

\section{Title}\fixme{``Turtledove'' er en pladsholder ind til videre. Jesper synes om
``Seagull''. Jeg har selv foreslået ``Albatros''.}
``Turtledove: Tool assisted programming in SML, with special emphasis on semi-automatic rewriting of
declarations to predefined standard forms.''

\section{Motivation}
Today functional programming languages are not very widespread outside academic circles. A reason
for this is that not many programmer assisting tools exist for functional programming languages
compared to certain imperative languages.

We wish to make functional programming, specifically in SML, easier to approach for the beginner as
well as more advantageous for the veteran. The means by which we hope to accomplish this task is the
development of a framework for implementing programmer assisting tools, and one or more actual
tools.

Functional programming languages has a wide variety of advantages that make them specially suited
for quick and correct statical analysis. The effect of this is that it will be possible to make more
complex tools for functional programming languages than for imperative ones.
\\

We emphasize on a single tool. Namely semi-automatic rewriting of terms to predefined standard
forms. \fixme{Hvorfor lige dette værktøj?}

\section{Elaboration}
The project can be divided into three parts.
\begin{description}
\item[Front end] The front end is responsible for communicating with the programmer. This
communication goes both ways. Some tools will work automatically, waiting for the programmer to
accept corrections and/or suggestions. Some will be initiated by the user.
\item[Back end] The back end will perform tasks common to most tools. This includes communicating
with the front end in a suitable format, reading and possibly writing project files, reading and
parsing source files, defining a representation of syntax trees, keeping an internal representation
of the source code up to date and defining auxiliary functions (e.g. functions for converting syntax
trees to source code, converting data to a format suitable for communication with the front end,
etc.).
\item[Tools] The tools will be initiated by the back end from which they will get a representation
of the source code as a syntax tree or the source code itself. Messages to the individual tools from
the front end will be delivered via the back end. The tools can send messages back to the front end
through the back end. In this project we will focus on a single tool. Se section
\ref{primary_goals:a_refactoring_tool} below.
\end{description}

The back end and the tools are split up because most tools will have a lot of tasks in common as
described above.  So while the back end and the tools are actually the same piece of software, the
distinction is an important one.

The importance of this distinction implies that we develop a well defined API for communication
between the back end and the tools.

We describe the project in greater detail in the next section.

\section{Primary goals}
We focus our work on the back end and a single tool. A simple front end (e.g. for Emacs) would make
testing a lot easier, but it is not a requirement for the success of this project.

As we wish to develop the back end and tool(s) separately from the front end, we choose to let all
communication between the two be through plain text.

The back end and the tool(s) compiles into a single program. We name that program and it's source
code ``Turtledove'', and we use the name interchangeably.

\subsection{Turtledove: modular tool assisted programming for SML} \fixme{Hear ye, hear ye! Behold
the Turtledove.}
Turtledove will be used for two purposes; i) tool assisted programming, and ii) implementation of
new tools.

Since we intent that Turtledove should grow to include more than one tool, the protocol over which
Turtledove and a compliant front end communicate need to be flexible enough to accommodate this.

Turtledove will communication with the front end through stdin and stdout. It shall be possible to
send data to a specific tool and determine from which tool outgoing data originates.
\\

Turtledove itself needs to be programmed in a modular way, as to make it easy to implement new tools
later. The tools shall each be confined and only interact with the rest of the program through an
API and/or the inclusion of libraries.
\\

Jeg synes jeg er begyndt at væve og klokken er mange nu, så jeg ridser resten op på dansk.

API'et skal gøre det muligt for værktøjer at
\begin{itemize}
\item hente (up-to-date) syntaxtræer og rå kode. Men de bør ikke selv tilgå filer da Turtledove kan
      være i stand til at afspejle ændringer som endnu ikke er gemt.
\item modtage beskeder fra front-end'en.
\item sende beskeder til front-end'en.
\end{itemize}

Derudover skal der lavet en del funktionalitet som værktøjer kan bruge i form af
biblioteker. Eksempler:
\begin{itemize}
\item SML-parser. Det kan f.eks. være snedigt at lagre vores ``standardformer'' som SML-kode, og så
      indlæse det på den måde.
\item Repræsentation af syntaxtræer.
\item Syntaxtræ-til-kode-omformer.
\item JSON-bibliotek, samt passende datastrukturer.
\end{itemize}

Kernefunktionaliteten i Turtledove bliver:
\begin{itemize}
\item Læs beskeder fra front-end'en og send dem til det rigtige værktøj eller agér direkte på dem.
\item Send beskeder fra de forskellige værktøjer til front-enden.
\item Læs og eventuelt skriv projektfiler. Vi vil begrænse os til .mlb-filer (bruges af blandt andre
      kittet og mlton).
\item Indlæs kode og parse til syntaxtræer.
\item Giv værktøjer besked når koden ændrer sig.
\end{itemize}

\subsection{A refactoring tool}\label{primary_goals:a_refactoring_tool}\fixme{Proof of concept, and Jakob's pet project.}
For at være helt ærlig tror jeg endnu ikke at der er fuldstændig konsensus i gruppen om hvad der
egentlig skal ske her. Så i stedet for at skrive hvad jeg mener, tror jeg det er bedst hvis vi får
diskuteret det først.

\section{Secondary goals}

\subsection{A nice front end}
For Emacs obviously.

\section{Limitation}
Nothing has to work as intended.

\section{Tasks and timeline}
\begin{quote}
\begin{description}
\item[\bf About six months] Do the damn thing.
\item[\bf 2 days] Drink our brains out.
\end{description}
\end{quote}

\end{document}
