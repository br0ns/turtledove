\section*{Intercommunication}


\subsection*{Protocol}


The communication protocol between the development environment and Turtledove is
very simple and versatile with almost no limitations. The protocol utilises JSON
(JavaScript Object Notation) \footnote{\url{http://www.json.org/}} for the data
payload which makes it easy to serialise different types of values to and from
the desired tools inside Turtledove.

Besides the data payload the protocol consists of a ``CallerID'' string and a
``DestinationID'' string which is described below

\begin{description}
\item[CallerID] is a string specified by the development environment. This
  string is not used in any way by Turtledove but is returned in the response
  string. This string is intended for internal bookkeeping in the development
  environment for example to distinguish which file/buffer and/or at which line
  and column the response originated from.

  As there is a possibility of Turtledove getting multi threaded, this is a good
  way of handling multiple request to Turtledove that may have different
  response times (One request waiting for a new complete reparse of the project
  code and another requesting a static lookup of some data).

  If the development environment doesn't need the use of this string an empty
  string can be passed to Turtledove.

  Some of the tools may report back when they are done doing something, without
  the development environment has invoked it. In these cases an empty string
  will be used as ``CallerID''.

\item[DestinationID] is a string that needs to match a named destination in
  Turtledove. Examples of such a named destination, also including name of
  tools:

  \begin{itemize}
  \item ``Turtledove'': This is for communication with the main program. This
    includes enabling/disabling of individual tools, status information and
    closing down the server gracefully.

  \item ``ProjectManager'': This is for communication with the project
    manager. This include actions such as modifying text, creation/deletion of
    files and projects.
  \end{itemize}
\end{description}


\begin{nonfloatingfigure}
  \begin{grammar}
    <request> ::= \[[ "CallerID" "\\n" "DestinationID" "\\n" <JSON-value> "\\n\\n" \]]
    
    <response> ::= \[[ "DestinationID" "\\n" "CallerID" "\\n" <JSON-value> "\\n\\n" \]]
  \end{grammar}
  
  \caption{Definition of the communication protocol between Turtledove (and its
    tools) and the editor.}
  \label{fig:intercom-protocol-definition}

\end{nonfloatingfigure}


It is important to remember that the resulting string sent to and from
Turtledove must not contain any newlines except the ones that are mandatory by
the protocol. This is a requirement even though the definition of JSON allows
whitespace between a pair of tokens, as it would then render it impossible to
determine when the request/response is done.

We have chosen the newline character as a separator of the fields as it
makes it easy for debug pretty printing and since a newline character must be
escaped i a name/id (and besides it is unlikely to be used here anyway).

The same argument applies to the decision of using double newline chars to mark
the end of a request or response. 


\paragraph{Examples}



\begin{example} If a file needs to be added then the following string could be
  sent to the ``Projectmanager'':
\begin{verbatim}
"some_id \n ProjectManager \n {\"command\" : \"addfile\", 
                                  \"file\" : \"myfoo.sml\"} \n\n"
\end{verbatim}
\end{example}


\subsection*{Useful commands}

 Below is a draft of which commands might be useful to implement in the final system. 

\subsubsection*{Turtledove}

\begin{description}
\item[quit] This command closes down the server in a graceful way.
\end{description}

\subsubsection*{ProjectManager }

\begin{description}
\item[OpenProject] Makes the project manager open a project. If the
  project doesn't exists, then a new project with the name is created.

  If another project is currently opened, then this project is closed, if told
  to do so.
\item[CloseProject] makes the project manager close the currently opened project,
  saving any unsaved changes if told to do so. 
\item[DeleteProject] Makes the project manager delete the currently opened project
  with all its files.
\item[AddFile] Adds a new file to the project.
\item[DeleteFile] Deletes a file contained in the project
\item[ChangeFile] Add or delete content in a open file contained in the
  project. This command should be used when ever the user adds or deletes text
  from any file contained in the project, so the project manager can maintain an
  up to date version of the file, without having to reread the file again and
  thus also at some point do incremental parsing of the changes only.
\end{description}


\paragraph{Get names in scope}

Returns a list of identifiers visible from the current scope. This is particular
handy for doing auto completion and 


\paragraph{Goto definition}

Gets the filename and position of the definition of the specified function

\paragraph{Rename}

Rename a function or identifier together with its definition and all its uses.


%%% Local Variables: 
%%% mode: latex
%%% reftex-fref-is-default: t
%%% TeX-master: "../../report"
%%% End: 
