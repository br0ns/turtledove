\section{Intercommunication}


\subsection*{Protocol}

\fixme{Lav protocol til at være en JSON object med meta, dest og data. Beskriv
  at meta kan være vilkårlig (json-value) men dest skal være en string og data
  skal være en json-value. keep the two newlines}


The communication protocol between the development environment and Turtledove is
very simple and versatile with almost no limitations. The protocol utilises JSON
(JavaScript Object Notation) \footnote{\url{http://www.json.org/}} which makes
it easy to serialise different types of values to and from the desired tools
inside Turtledove. 

\begin{nonfloatingfigure}
  \newcommand{\angleit}[1]{$\langle$\textnormal{\textit{#1}}$\rangle$}
\begin{lstlisting}
 (@\angleit{request}@) ::= "{ \"Meta\" : (@\angleit{JSON-value}@), \"Dest\" : (@\angleit{JSON-string}@),
                \"Data\" : (@\angleit{JSON-value}@) } \n\n"

(@\angleit{response}@) ::= "{ \"Dest\" : (@\angleit{JSON-string}@), \"Meta\" : (@\angleit{JSON-value}@),
                \"Data\" : (@\angleit{JSON-value}@) } \n\n"
\end{lstlisting}    
  
  \caption{Definition of the communication protocol between Turtledove (and its
    tools) and the development evironment.}
  \label{fig:intercom-protocol-def}
\end{nonfloatingfigure}

The protocol is defined (see \fref{fig:intercom-protocol-def}) as a JSON-object
encoded string with two newlines at the end as the stop delimiter. The JSON
object contains three fields, ``Meta'' (Metadata), ``Dest'' (Destination) and
``Data'' (Data payload) which is described below.

\begin{description}
\item[Meta] is a JSON value specified by the development environment. This value
  is not used in any way by Turtledove but is returned in the response. This
  string is intended for internal bookkeeping in the development environment for
  example to distinguish which file/buffer and/or at which line and column the
  response originated from.

  As there is a possibility of Turtledove getting multi threaded, this is a good
  way of handling multiple request to Turtledove that may have different
  response times (One request waiting for a new complete reparse of the project
  code and another requesting a static lookup of some data).

  If this feature is not needed by the development environment it can set this
  field to the JSON value \texttt{null} but it is not a requirement.

  Some of the tools may report back when they are done doing something, without
  the development environment has invoked it. In these cases the JSON value
  \texttt{null} will be used.

\item[Dest] is a non empty JSON string that needs to match a named destination in
  Turtledove. Examples of such a named destination, also including name of
  tools:

  \begin{itemize}
  \item ``Turtledove'': This is for communication with the main program. This
    includes enabling/disabling of individual tools, status information and
    closing down the server gracefully.

  \item ``ProjectManager'': This is for communication with the project
    manager. This include actions such as modifying text, creation/deletion of
    files and projects.
  \end{itemize}
\end{description}


It is important to remember that the resulting string sent to and from
Turtledove must not contain any double newlines except the ones that are defined by
the protocol. This is a requirement, even though the definition of JSON allows
whitespace between a pair of tokens, as it would then render it near to
impossible to determine when the request/response is done without incrementally
parsing the JSON encoded string as it is received.

We have chosen the double newline character as the stop signal as it can't be used
inside the structure of the JSON object and when present in a JSON string it
must be escaped. 

This simple constraint of not having any double newlines inside the JSON encoded
string should not restrain any practical usability of the JSON protocol.


\paragraph{Examples}

\begin{example} If a file needs to be added then the following string could be
  sent to the ``ProjectManager'':
\begin{verbatim}
"{ \"Meta\" : null, \"Dest\" : \"ProjectManager\",
   \"Data\" : { \"command\" : \"addfile\", 
                \"file\" : \"myfoo.sml\"}} \n\n"
\end{verbatim}
\end{example}


\subsection*{Useful commands}

 Below is a draft of which commands might be useful to implement in the final system. 

\subsubsection*{Turtledove}

\begin{description}
\item[quit] This command closes down the server in a graceful way.
\end{description}

\subsubsection*{ProjectManager }

\begin{description}
\item[OpenProject] Makes the project manager open a project. If the
  project doesn't exists, then a new project with the name is created.

  If another project is currently opened, then this project is closed, if told
  to do so.
\item[CloseProject] makes the project manager close the currently opened project,
  saving any unsaved changes if told to do so. 
\item[DeleteProject] Makes the project manager delete the currently opened project
  with all its files.
\item[AddFile] Adds a new file to the project.
\item[DeleteFile] Deletes a file contained in the project
\item[ChangeFile] Add or delete content in a open file contained in the
  project. This command should be used when ever the user adds or deletes text
  from any file contained in the project, so the project manager can maintain an
  up to date version of the file, without having to reread the file again and
  thus also at some point do incremental parsing of the changes only.
\end{description}


\subsubsection*{Utils}

\begin{description}
\item[GotoDef] Returns the filename and position of the definition of the
  specified function
\item[Rename] Rename a function or identifier together with its definition and
  all its uses.
\item[NamesInScope] Returns a list of identifiers visible from the current
  scope. This is particular handy for doing auto completion and
\end{description}



%%% Local Variables: 
%%% mode: latex
%%% reftex-fref-is-default: t
%%% TeX-master: "../../report"
%%% End: 
