\section{Definitions}
We write concrete SML i typewriter font as this:
\begin{code}
fun id x = x
\end{code}

\begin{definition}
In the following we write $pat^\alpha$ to mean the term algebra
\[
T(con)\ \text{over}\ \alpha
\]
of the constructors of some fixed environment $E = (S\!E, T\!E, V\!E)$ (Structure
Environment, Type Environment and Value Environment) and the set of record
constructors over $\alpha$. \fixme{And the identity?}

\end{definition}

\paragraph{Note.} One can think of $pat^\alpha$ as SML-patterns generalised to
any kind of ``variables''. The constructors of datatypes are either constants or
unary functions. The record constructors form a countably infinite set of
functions with finite arity. To distinguish constructors from the surrounding
program text we write them with at initial capital letter (with very few
exceptions like \codeinline{::} \codeinline{nil}). We respect the infix status
of the symbols given by $E$.

We write $\mathfrak{a}, \mathfrak{b}, \mathfrak{c}, \ldots$ for variables ranging
over $con$.

\begin{definition}[Record constructors] \ \\
  The record constructor given by
  \[
  (\alpha, \beta) \mapsto \texttt{\{foo = $\alpha$, bar = $\beta$\}}
  \]
  is denoted by $\mathfrak{R}_{\texttt{foo},\texttt{bar}}$.

  Recall that the tuple constructors are just syntactic sugar for a subset of
  the record constructors. We write $\mathfrak{T}_n$ to mean
  $\mathfrak{R}_{\texttt{1},\ldots,\texttt{n}}$.
\end{definition}

\begin{definition}[(SML) Variables] \ \\
  Is the set $var$ of all valid SML variables, which we denote by any text in
  typewriter font as with any concrete SML. For example $\mathtt{x},\
  \mathtt{y},\ \mathtt{xs}$
\end{definition}

\begin{definition}[Simple (SML) expressions]
  Is the set of $simp\_exp$ which is SML expressions without type annotation and
  infix status. \fixme[inline, margin=false]{Is this ok or shoud it be inlined down in the scheme
    body?}
\end{definition}

\begin{definition}[Meta variables] \ \\
  Is the set of $mvar = \{x,\ y,\ xs,\ \ldots \}$. Meta variable range
  over $var$ and wild-cards.
\end{definition}

\begin{definition}[Pattern variables] \ \\
  Is the set of $pvar = \{ \overline{x},\ \overline{y},\
  \overline{xs} \}$. Pattern variables range over $pat^{var}$ .
\end{definition}

\begin{definition}[Holes] \ \\
  Is the countably infinite set, $holes$, of constants named $\diamond_1, \diamond_2,
  \ldots$, and we write $\diamond$ to mean $\diamond_1$.
\end{definition}

\begin{definition}[Meta patterns] \ \\
  Is the set of meta patterns $mpat = pat^{pvar \cup holes}$. The subset $mpat_n
  \subset mpat$ is the meta patterns with exactly $n$ holes. We write
  $\mathcal{A},\mathcal{B}, \mathcal{C}, \ldots$ for variables ranging over
  $mpat$.
\end{definition}

\begin{definition}[Scheme patterns]
  Is the set $spat = pat^{mpat_0} = \{\alpha, \beta, \ldots \}$
\end{definition}

\begin{definition}[Scheme bodies] \ \\
  Is the set $sexp$. The grammar and semantics is described in the next section.
\end{definition}

\begin{definition}[Scheme clauses] \ \\
  Is the set $clauses = spat \times sbod$. We write $|\ \alpha => \beta$ for the pair
  $(\alpha, \beta)$
\end{definition}

\begin{definition}[Schemes] \ \\
  Is the set $schemes = clauses^{+}$
\end{definition}

\begin{definition}[(Rewriting) rules] \ \\
  Is set $rules = schemes \times sbod$. We write
  \[
  \shortstack[ccc]{$\alpha$ \\ $\Downarrow$ \\ $\beta$}
  \]
  for the pair $(\alpha, \beta)$.
\end{definition}

\begin{definition}[Transformers]
  Is the set $trans$ which is a subset of the partial functions
  \[
  pat^{var \cup holes} \rightharpoonup exp
  \]
\end{definition}

%%% Local Variables: 
%%% mode: latex
%%% TeX-master: "../rewriting-syntax"
%%% End: 
