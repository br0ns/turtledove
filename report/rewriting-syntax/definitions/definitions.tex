\section{Definitions}
We write concrete SML in typewriter font as this:
\begin{sml}
fun id x = x
\end{sml}

We use the names given in \cite{SML97} for the different syntactic categories.

\begin{definition}
  In the following we write $pat^\alpha$ to mean the subset of the term algebra
  \[
  T(con)\ \text{over}\ \alpha
  \]
  of the constructors of some fixed environment $E = (S\!E, T\!E, V\!E)$
  (Structure Environment, Type Environment and Value Environment) and the set of
  record constructors over $\alpha$, where each variable occurs at most
  once. \fixme{And the identity?}

\end{definition}

\paragraph{Note.} One can think of $pat^\alpha$ as SML-patterns generalised to
any kind of ``variables''. The constructors of datatypes are either constants or
unary functions. The record constructors form a countably infinite set of
functions with finite arity. To distinguish constructors from the surrounding
program text we write them with at initial capital letter (with very few
exceptions like \codeinline{::} \codeinline{nil}). We respect the infix status
of the symbols given by $E$.

We write $\mathfrak{a}, \mathfrak{b}, \mathfrak{c}, \ldots$ for variables
ranging over $con$.

\fixme{Note about non-equality of wildcards.} 

\begin{tabular}{lMlMlMl}
  (SML) Variables & & var & ::= \mathtt{x} \mid \mathtt{y} \mid
  \mathtt{xs} \mid \ldots \\
  Holes & & hole & ::= \diamond \mid \diamond_1 \mid \diamond_2, \mid \ldots \\
  Meta variables & a,\ b, \ldots & mvar & ::= var \mid \texttt{_} \\ 
  Pattern variables & \overline{x},\ \overline{y},\ \ldots & pvar & ::=
  pat^{var} \\
  Transformers &  \mathbb{D},\ \mathbb{E},\ \ldots & trans & ::=
  pat^{var \cup hole} = sexp \\
  Meta patterns & \mathcal{A},\ \mathcal{B},\ \ldots & mpat &::= pat^{pvar \cup
    holes} \\
  Scheme patterns & \alpha,\ \beta,\ \ldots & spat & ::= pat^{mpat_0} \\
  Scheme expressions & & sexp & \textrm{ (see \fref{fig:scheme-expressions})}
  \\
  Scheme clauses & & sclause & ::= spat \texttt{ => }
  sexp \\
  Schemes & & scheme & ::= caluse+ \\ 
  (Rewriting) Rules & & rule & ::= scheme where
  \Downarrow scheme \\
  Matching condition & & where & ::= .... \\
\end{tabular}


\fixme{Klassedeling $mpat_k$}


We fix some definitions and terminology for the rest of this chapter. Let T be a
term algebra in some signature. A term rewriting system for T is a set of
ordered pairs of elements of T of the form (l, r). Viewed as a set of relations,
the rewriting system determines a presentation for a quotient algebra A of T.

When we take into account the fact that the relations are expressed as ordered
pairs, we have a way of reducing the elements of T. Suppose an element u of T
has a subword l and (l,r) is a rule of the rewriting system, then we can replace
the subterm l of u by the term r and obtain a new word v. We say that we have
rewritten u as v. Note that u and v represent the same element of A. If u can
not be rewritten using any rule of the rewriting system we sat that u is
reduced.

\begin{description}
\item[Record constructors] The record constructor given by
  \[
  (a, b) \mapsto \texttt{\{foo = $a$, bar = $b$\}}
  \]
  is denoted by $\mathfrak{R}_{\texttt{foo},\texttt{bar}}$.

  Recall that the tuple constructors are just syntactic sugar for a subset of
  the record constructors. We write $\mathfrak{T}_n$ to mean
  $\mathfrak{R}_{\texttt{1},\ldots,\texttt{n}}$.


\item[(SML) Variables] Is the set $var$ of all valid SML variables, which we
  denote by any text in typewriter font as with any concrete SML. For example
  $\mathtt{x},\ \mathtt{y},\ \mathtt{xs}$.


  % \item[Simple (SML) expressions]
  %   Is the set of $simp\_exp$ which is SML expressions without type annotation
  %   and infix status. \fixme[inline, margin=false]{Is this ok or shoud it be
  %     inlined down in the scheme body?}
  % \end{definition}

\item[Meta variables] Is the set of $mvar = \{x,\ y,\ xs,\ \ldots \}$. Meta
  variable range over $var \cup \{\texttt{_}\}$.


\item[Pattern variables] Is the set of $pvar = \{ \overline{x},\ \overline{y},\
  \overline{xs} \}$. Pattern variables range over $pat^{var}$ .


\item[Holes] Is the countably infinite set, $holes$, of constants named
  $\diamond_1, \diamond_2, \ldots$, and we write $\diamond$ to mean
  $\diamond_1$.


\item[Meta patterns] Is the set of meta patterns $mpat = pat^{pvar \cup
    holes}$. The subset $mpat_n \subset mpat$ is the meta patterns with exactly
  $n$ holes. We write $\mathcal{A},\mathcal{B}, \mathcal{C}, \ldots$ for
  variables ranging over $mpat$.


\item[Scheme patterns] Is the set $spat = pat^{mpat_0} = \{\alpha, \beta, \ldots
  \}$


\item[Scheme bodies] Is the set $sexp$. The grammar and semantics is described
  in the next section. We write $\alpha, \beta, \gamma, \ldots$ for variables
  ranging over $sexp$.


\item[Scheme clauses] Is the set $clauses = spat \times sexp$. We write $|\
  \alpha => \beta$ for the pair $(\alpha, \beta)$


\item[Schemes] Is the set $schemes = clauses^{+}$


\item[(Rewriting) rules] Is set $rules = schemes \times sexp$. We write
  \[
  \shortstack[ccc]{$\alpha$ \\ $\Downarrow$ \\ $\beta$}
  \]
  for the pair $(\alpha, \beta)$.


\item[Transformers] Is the set $trans = pat^{var \cup holes} \times sexp$. We
  write $\mathbb{D}, \mathbb{E}, \ldots$ to range over transformers.
  % Is the set $trans$ which is a subset of the partial functions
  % \[
  % pat^{var \cup holes} \rightharpoonup exp
  % \]

\end{description}


%%% Local Variables: 
%%% mode: latex
%%% TeX-master: "../rewriting-syntax"
%%% End: 
