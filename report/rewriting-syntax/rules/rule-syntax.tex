\subsection{Rewriting rule syntax}

When specifying rewriting rules we 

\begin{nonfloatingfigure}

  \setlength{\grammarindent}{7.3em}
  \begin{grammar} 
    
    <rule-program> ::= <rule>$^{*}$
    
    <rule> ::= "rule" <rule-header> <scheme> "becomes" <clause>$^{+}$ "end"
    
    <rule-header> ::= <rule-type> <rule-name>
    
    <rule-type> ::= "clauses" 
    \alt "expression" 
    
    <rule-name> ::= <longid>
    
    <rule-self> ::= "self" <sexp>
    
    <scheme> ::= <clause>$^{+}$ [<cstrns>]
    
    <clause> ::= "|" <spat> "=>" <sexp>
    
    <cstrns> ::= "where" <cstrn-body> 
    
    <cstrn-body> ::= <cstrn-rel> 
    \alt <cstrn-rel> "," <cstrn-body>
    
    <cstrn-rel> ::= <longid> "(" <cstrn-rel-body>$^{+}$ ")"
    
    <cstrn-rel-body> ::= <spat>   
    
    <transformer> ::= "£" <longid> <sexp>

    <meta-pattern-sexp> ::= "§" <longid>  <sexp>$^{*}$
    
    <meta-pattern-spat> ::= "§" <longid>  <spat>$^{*}$

  \end{grammar}
  
  \caption{Complete rule grammar. See \fref{fig:scheme-expressions} for the
    \synt{sexp} grammar and \fref{fig:scheme-patterns} for the \synt{spat}
    grammar which uses the transformers and meta patterns}
  \label{fig:rule-grammar}
\end{nonfloatingfigure}


%%% Local Variables:
%%% mode: latex
%%% TeX-master: "../rewriting-syntax"
%%% End: 
