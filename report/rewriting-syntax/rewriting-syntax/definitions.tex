\section{Definitions}
We write concrete SML in typewriter font as this:
\begin{sml}
fun id x = x
\end{sml}

We use the names given in \cite{SML97} for the different syntactic categories.

In the following we assume a fixed SML environment $E = (S\!E, T\!E, V\!E)$ and
a set of $n$-ary relations over $mpat$ (defined below), $\sigma$.

\begin{definition}
  In the following we write $pat^\alpha$ to mean the subset of the term algebra
  \[
  T(con, \alpha)
  \]
  of the constructors of $E$ and the set of record constructors over $\alpha$,
  where each variable occurs at most once. \fixme{And the identity?}

\end{definition}

\paragraph{Note.} One can think of $pat^\alpha$ as SML-patterns generalised to
any kind of ``variables''. The constructors of datatypes are either constants or
unary functions. The record constructors form a countably infinite set of
functions with finite arity. To distinguish constructors from the surrounding
program text we write them with an initial capital letter (with very few
exceptions like \codeinline{::} and \codeinline{nil}). We respect the infix
status of the symbols given by $E$.

We regard all occurences of the specieal variable \texttt{\_} (``wildcard'') as
different from any other variable.

\begin{description}
\item[Record constructors] The record constructor given by
  \[
  (a, b) \mapsto \texttt{\{foo = $a$, bar = $b$\}}
  \]
  is denoted by $\mathfrak{R}_{\texttt{foo},\texttt{bar}}$.

  Recall that the tuple constructors are just syntactic sugar for a subset of
  the record constructors. We write $\mathfrak{T}_n$ to mean
  $\mathfrak{R}_{\texttt{1},\ldots,\texttt{n}}$.
\end{description}

\paragraph{Syntax}\ \\
\renewcommand{\arraystretch}{1.5}
\begin{tabular}{lMlMrMl}
  \textbf{Kind} & \textbf{Category} & \textbf{Is}\\
  (SML) Constructors & con & ::= & \mathtt{Node} \mid \mathtt{Leaf} \mid
  \texttt{::} \mid \mathfrak{T}_{42} \mid \mathfrak{R}_{\mathtt{foo},
      \mathtt{bar}, \mathtt{baz}} \mid \ldots\\
  (SML) Variables & id & ::= & \mathtt{x} \mid \mathtt{y} \mid
  \mathtt{xs} \mid \ldots \mid \mathtt{\_}\\
  Pattern variables & patvar & ::= & \overline{x} \mid \overline{y} \mid \ldots\\
  Holes & hole & ::= & \diamond \mid \diamond_1 \mid \diamond_2 \mid \ldots \\

  Meta patterns & mpat & ::= & mpat[pat^{patvar \cup hole}]\\
  & & \mid & mpatvar\\
  Meta pattern variable & mpatvar & ::= & \mathcal{A} \mid \mathcal{B} \mid \ldots\\
  % & & \mid & mpatvar\\

  % Meta patterns & mpat & ::= & pat^{patvar \cup hole}\\
  % & & \mid & mpat[mpat]\\
  % & & \mid & \mathcal{A} \mid \mathcal{B} \mid \ldots\\

  Scheme patterns & spat & ::= & pat^{mpat}\\
  Scheme expressions & sexp & & \textrm{ (see \fref{tab:scheme-expressions})}
  \\
  Scheme clauses & sclause & ::= & spat => sexp \\
  Schemes & scheme & ::= & sclause^{+}\\
  & & \mid & sclause^{+} \textsf{ where } cstrn, \ldots\\

  Transformers & trans & ::= & trans[spat]\\
  & & \mid & \mathbb{D} \mid \mathbb{E} \mid \ldots\\
  % & & \mid & transvar\\

  Constraint &  cstrn & ::= & rel(mpatvar, \ldots) \\
  Relation & rel & ::= & \textsf{a} \mid \textsf{b} \mid \ldots\\
  (Rewriting) Rules & rule & ::= & scheme \Downarrow sclause^{+}\\
\end{tabular}

% \paragraph{Aliases}\ \\
% \begin{tabular}{lMlMl}
%   \textbf{Kind} & \textbf{Alias} & \textbf{Is}\\
%   Patterns & pat & = pat^{id}\\
%   % Meta patterns & mpat & = pat^{patvar \cup hole} \qquad (\dagger)\\
%   Scheme patterns & spat & = pat^{mpat_0}
% \end{tabular}\\
% $(\dagger)$ We write $mpat_0$ for the subclass of meta patterns with no holes,
% $mpat_1$ for the subclass with exactly one hole, etc.

\paragraph{Environment.}
An environment, $\sigma$, is constructed recursively. It is either empty,
$\sigma = []$, or it has at least one binding, $\sigma = x \mapsto y, \sigma'$,
where $x$ and $y$ are variables. There are a number of different kinds of
variables, which are listed below. Thus as an example we could have $\sigma =
\mathcal{A} \mapsto \ldots, \overline{x} \mapsto \ldots, \sigma'$.

\paragraph{Variables}\ \\
\begin{tabular}{lMlMlMl}
  \textbf{Over} & \textbf{Category}& \textbf{Looks like} & \textbf{Domain}\\
  Recursion & recvar & \textsf{self} & (id \times \mathbb{N}) \cup \textsf{undef}\\
  % Constructors & convar & \mathfrak{a}, \mathfrak{b}, \ldots & con\\
  Identifiers & idvar & a, b, \ldots & id\\
  % Patterns & patvar & \overline{x}, \overline{y}, \ldots & pat^{id} \\
  Transformers & transvar &  \mathbb{D}, \mathbb{E}, \ldots & pat \times exp\\
  Meta patterns & mpatvar & \mathcal{A}, \mathcal{B}, \ldots & pat^{patvar \cup hole} \\
  % Scheme patterns & spatvar & \alpha, \beta, \ldots & spat \\

 % mpat^{+} \rightharpoonup
 %  \{t\!t, f\!\!f\}\\
%   & var & &
% {
%   \begin{array}{l}
%     rec \cup vvar\\
%     \cup pvar\\
%     \cup tvar\\
%     \cup mpvar\\
%     \cup spvar\\
%     \cup rvar
% \end{array}
% }

\end{tabular}
\renewcommand{\arraystretch}{1}
\\

\fixme{Klassedeling $mpat_k$}

% \begin{description}
% \item[(SML) Variables] Is the set $var$ of all valid SML variables, which we
%   denote by any text in typewriter font as with any concrete SML. For example
%   $\mathtt{x},\ \mathtt{y},\ \mathtt{xs}$.


%   % \item[Simple (SML) expressions]
%   %   Is the set of $simp\_exp$ which is SML expressions without type annotation
%   %   and infix status. \fixme[inline, margin=false]{Is this ok or shoud it be
%   %     inlined down in the scheme body?}
%   % \end{definition}

% \item[Meta variables] Is the set of $mvar = \{x,\ y,\ xs,\ \ldots \}$. Meta
%   variable range over $var \cup \{\texttt{_}\}$.


% \item[Pattern variables] Is the set of $pvar = \{ \overline{x},\ \overline{y},\
%   \overline{xs} \}$. Pattern variables range over $pat^{var}$ .


% \item[Holes] Is the countably infinite set, $holes$, of constants named
%   $\diamond_1, \diamond_2, \ldots$, and we write $\diamond$ to mean
%   $\diamond_1$.


% \item[Meta patterns] Is the set of meta patterns $mpat = pat^{pvar \cup
%     holes}$. The subset $mpat_n \subset mpat$ is the meta patterns with exactly
%   $n$ holes. We write $\mathcal{A},\mathcal{B}, \mathcal{C}, \ldots$ for
%   variables ranging over $mpat$.


% \item[Scheme patterns] Is the set $spat = pat^{mpat_0} = \{\alpha, \beta, \ldots
%   \}$


% \item[Scheme bodies] Is the set $sexp$. The grammar and semantics is described
%   in the next section. We write $\alpha, \beta, \gamma, \ldots$ for variables
%   ranging over $sexp$.


% \item[Scheme clauses] Is the set $clauses = spat \times sexp$. We write $|\
%   \alpha => \beta$ for the pair $(\alpha, \beta)$


% \item[Schemes] Is the set $schemes = clauses^{+}$


% \item[(Rewriting) rules] Is set $rules = schemes \times sexp$. We write
%   \[
%   \shortstack[ccc]{$\alpha$ \\ $\Downarrow$ \\ $\beta$}
%   \]
%   for the pair $(\alpha, \beta)$.


% \item[Transformers] Is the set $trans = pat^{var \cup holes} \times sexp$. We
%   write $\mathbb{D}, \mathbb{E}, \ldots$ to range over transformers.
%   % Is the set $trans$ which is a subset of the partial functions
%   % \[
%   % pat^{var \cup holes} \rightharpoonup exp
%   % \]

% \end{description}


%%% Local Variables: 
%%% mode: latex
%%% TeX-master: "../rewriting-syntax"
%%% End: 
