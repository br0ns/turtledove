% xcolor=table is because it seems that beamer uses the xcolor package in a
% strange way and thus don't accept us giving arguments to the package.
\documentclass[slidestop,compress,mathserif, xcolor=dvipsnames]{beamer}

\usepackage[T1]{fontenc}
\usepackage[utf8]{inputenc}
\usepackage[danish]{babel}

\renewcommand{\ttdefault}{txtt} % Bedre typewriter font


% Use the NAT theme in uk (also possible in DK)
\usetheme[nat, uk, footstyle=low]{Frederiksberg}

% Make overlay sweet nice by having different transparancy depending on how
% "far" ahead the overlay is. AWSOME!!
\setbeamercovered{highly dynamic}
% possible to shift back, so they are just invisible untill they should overlay
% \setbeamercovered{invisible}

% Extend figures into either left or right margin
% Ex: \begin{narrow}{-1in}{0in} .. \end{narrow} will place 1in into left margin
\newenvironment{narrow}[2]{%
  \begin{list}{}{%
  \setlength{\topsep}{0pt}%
  \setlength{\leftmargin}{#1}%
  \setlength{\rightmargin}{#2}%
  \setlength{\listparindent}{\parindent}%
  \setlength{\itemindent}{\parindent}%
  \setlength{\parsep}{\parskip}}%
\item[]}{\end{list}}

\usepackage{mdwtab}
\usepackage{mathenv}
\usepackage{amsfonts}
\usepackage{amsmath}
\usepackage{amssymb}
\usepackage{amsthm}
\usepackage{semantic} % for the \mathlig function

\usepackage[final]{listings} % Make sure we show the listing even though we are
                             % making a final report.
\lstset{ %
language=ml,                % choose the language of the code
basicstyle=\ttfamily\tiny,        % the size of the fonts that are used for the code
basewidth=0.5em,
numbers=left,                   % where to put the line-numbers
numberstyle=\tiny,      % the size of the fonts that are used for the line-numbers
% stepnumber=2,                   % the step between two line-numbers. If it's 1 each line will be numbered
% numbersep=5pt,                  % how far the line-numbers are from the code
% backgroundcolor=\color{white},  % choose the background color. You must add \usepackage{color}
% showspaces=false,               % show spaces adding particular underscores
% showstringspaces=false,         % underline spaces within strings
% showtabs=false,                 % show tabs within strings adding particular underscores
% frame=single	                % adds a frame around the code
% tabsize=2,	                % sets default tabsize to 2 spaces
% captionpos=b,                   % sets the caption-position to bottom
% breaklines=true,                % sets automatic line breaking
% breakatwhitespace=false,        % sets if automatic breaks should only happen at whitespace
escapeinside={(@}{@)}          % if you want to add a comment within your code
}

\lstnewenvironment{sml}
{%\lstset{}
 } % starting code, ex a lstset that is more specific;
{} % ending code

% Write a short text to have that shown in the footer of slides other than the
% title slide.
\title[]{\ \ Turtledove}
% A possible subslide.
\subtitle{\tiny{Tool assisted programming in SML, with special emphasis on
    semi-automatic\\ rewriting to predefined standard forms.}}

\author{Morten Brøns-Pedersen}

% Only write DIKU in the footer of slides (except the title slide).
\institute[DIKU]{Department of Computer Science}

% Remove the date stamp from the footer of slides (except title slide) by giving
% it no short "text"
\date[]{\today}


\newcommand{\ttt}[1]{\texttt{#1}}
\newcommand{\tnm}[1]{\textnormal{#1}}
\newcommand{\tsf}[1]{\textsf{#1}}
\newcommand{\mrm}[1]{\mathrm{#1}}
\newcommand{\ol}[1]{\overline{#1}}

\def\TheTrueColour{purple}
\newcommand{\cc}[1]{{\color{\TheTrueColour}#1}}
\newcommand{\subspat}[3]{\ensuremath{#1\cc{(}#2\cc{)}\mathrel{\cc{=}}#3}}
\newcommand{\matchpat}[3]{\ensuremath{\cc{{\color{black}#1}\mathrel{:}\left\langle{\color{black}#2},\mathrel{ }{\color{black}#3}\right\rangle}}}
\newcommand{\matchbody}[4]{\ensuremath{#1\cc{,}\mathrel{ }#2\mathrel{\cc{|-}}#3\mathrel{\cc{:}}#4}}
\newcommand{\matchclause}[3]{\ensuremath{#1\mathrel{\cc{|-}}#2\mathrel{\cc{:}}#3}}
\newcommand{\rewrite}[4]{\ensuremath{#1\mathrel{\cc{,}}#2\mathrel{\cc{|-}}#3\mathrel{\cc{\curvearrowright}}#4}}
\newcommand{\becomesthrough}[3]{\ensuremath{#1\mathrel{\textsf{\cc{becomes}}}#2\mathrel{\textsf{\cc{through}}}#3}}
\newcommand{\mathsml}[1]{\textnormal{\lstinline{#1}}}

\begin{document}

\frame[plain]{\titlepage}


\begin{frame}
  \frametitle{Oversigt}

\end{frame}

\begin{frame}[fragile]
  \frametitle{Regler for \texttt{exists} og \texttt{all}}

  \begin{block}{$\textsf{exists}_{\textsf{a}}$}
    \scriptsize
    \begin{eqnarray*}[x]
      \begin{eqnalign}[lcl]
        \mathcal{C}[\ol{x}\ \ttt{::}\ \ttt{xs}] &=>& \mathbb{D}(\mathcal{C}[\ol{x}])\
        \ttt{orelse}\ \ttt{self}(\mathcal{C}[\ttt{xs}]) \\
        \mathcal{D} &=>& \ttt{false} \\
      \end{eqnalign}\\
      \textsf{where}\ \textsf{samedom} (\mathcal{C}, \mathcal{D}) \\
      %
      \Downarrow \\
      %
      \begin{eqnalign}[lcl]
        \mathcal{C}[\ttt{xs}] &=>& \ttt{List.exists (fn x => $\mathbb{D}(
          \mathcal{C}[$x$])$)} \ \ttt{xs} \\
      \end{eqnalign}
    \end{eqnarray*}
  \end{block}
  \begin{block}{$\textsf{all}_{\textsf{a}}$}
    \scriptsize
    \begin{eqnarray*}[x]
      \begin{eqnalign}[lcl]
        \mathcal{C}[\ol{x}\ \ttt{::}\ \ttt{xs}] &=>& \mathbb{D}(\mathcal{C}[\ol{x}])\
        \ttt{andalso}\ \ttt{self}(\mathcal{C}[\ttt{xs}]) \\
        \mathcal{D} &=>& \ttt{true} \\
      \end{eqnalign}\\
      \textsf{where}\ \textsf{samedom} (\mathcal{C}, \mathcal{D}) \\
      %
      \Downarrow \\
      %
      \begin{eqnalign}[lcl]
        \mathcal{C}[\ttt{xs}] &=>& \ttt{List.all (fn x => $\mathbb{D}(
          \mathcal{C}[$x$])$)} \ \ttt{xs} \\
      \end{eqnalign}
    \end{eqnarray*}
  \end{block}
\end{frame}

\begin{frame}[fragile]
  \frametitle{Eksempel}

  \begin{block}{}
    \begin{sml}
      fun member (_, nil) = false
        | member (x, y :: ys) = x = y orelse member (x, ys)

      fun sublist (nil, _) = true
        | sublist (x :: xs, ys) = member (x, ys) andalso sublist (xs, ys)
    \end{sml}
  \end{block}
  kan skrives om til
  \begin{block}{}
    \begin{sml}
      fun member (x, ys) = List.exists (fn y => x = y) ys

      fun sublist (xs, ys) = List.all (fn x => member (x, ys)) xs
    \end{sml}
  \end{block}
\end{frame}

\begin{frame}[fragile]
  \frametitle{Eksempel --- Normalform}

  \begin{block}{}
    \begin{sml}
      fun member (_, nil) = false
        | member (x, y :: ys) = x = y orelse member (x, ys)
    \end{sml}
  \end{block}
  \begin{block}{}
    \begin{sml}
      fun member (x, y :: ys) = x = y orelse member (x, ys)
        | member x = false
    \end{sml}
  \end{block}
  \vspace{5pt}\hrule
  \begin{block}{}
    \begin{sml}
      fun sublist (nil, _) = true
        | sublist (x :: xs, ys) = member (x, ys) andalso sublist (xs, ys)
    \end{sml}
  \end{block}
  \begin{block}{}
    \begin{sml}
      fun sublist (x :: xs, ys) = member (x, ys) andalso sublist (xs, ys)
        | sublist x = true
    \end{sml}
  \end{block}
\end{frame}

\begin{frame}[c]
  \frametitle{Eksempel --- Normalform}

  \begin{center}
    \Huge{Live demo}
  \end{center}
\end{frame}

\begin{frame}[fragile]
  \frametitle{Eksempel --- Omskrivning}

  Først undersøges om funktionen er en instans af reglen. For hvert klausulpar skal
  eksistere en afledning
  \begin{block}{}
    \scriptsize
    \[
    \inference
    { \mathcal{H} & \mathcal{I} & \mathcal{J} \\
      \subspat{\sigma}{spat}{mpat} &
      \matchpat{pat}{mpat}{\theta} &
      \matchbody{\sigma}{\theta}{sexp}{exp}
    }
    {
      \matchclause{\sigma}{pat \texttt{ => } exp}{spat => sexp}
    }
    \]
  \end{block}\ \\[1em]
  Jeg demonstrerer omskrivning af \texttt{member}.

  Lad
  {\scriptsize
    \[
    \sigma =
    \left\{
      \begin{tabular}{Mc@{$\ \mapsto\ $}Mr}
        \mathcal{C} & \ttt{($\ol{a}$, $\diamond_1$)}\\
        \mathcal{D} & \ol{a}\\
        \mathbb{E} & (\ttt{(v, w)}, \ttt{v = w})\\
        \textsf{self} & (\ttt{member}, 1)
      \end{tabular}
    \right\}
    \]
  }
\end{frame}

\begin{frame}[fragile]
  \frametitle{Eksempel --- Omskrivning}

  \begin{block}{}
    \begin{sml}
      fun member (@\color{purple}(x, y :: ys) = x = y orelse member (x, ys)@)
        | member x = false
    \end{sml}
  \end{block}

  \begin{block}{}
    \scriptsize
    \begin{eqnarray*}[x]
      \begin{eqnalign}[lcl]
        \color{purple}
        \mathcal{C}[\ol{x}\ \ttt{::}\ \ttt{xs}] &\color{purple}=>&
        \color{purple}
        \mathbb{D}(\mathcal{C}[\ol{x}])\ \ttt{orelse}\ \ttt{self}(\mathcal{C}[\ttt{xs}]) \\
        \mathcal{D} &=>& \ttt{false} \\
      \end{eqnalign}
    \end{eqnarray*}
  \end{block}
\end{frame}

\begin{frame}[fragile]
  \frametitle{Eksempel --- Omskrivning}

  \begin{block}{}
    \tiny
    \[
    \mathcal{H}:\
    \inference
    {
    }
    {
      \subspat
      {\sigma}
      {\mathcal{C}[\ttt{$\ol{x}$ :: xs}]}
      {\ttt{($\ol{a}$, $\diamond_1$)$[\ol{x}$ :: xs$ / \diamond_1]$}}
    }
    \]
  \end{block}
  \begin{block}{}
    \tiny
    \[
    \mathcal{I}:\
    \inference
    {
      \inference
      {
      }
      {
        \matchpat
        {\ttt{x}}
        {\ol{a}}
        {\{\ol{a} \mapsto \ttt{x}\}}
      } &
      \inference
      {
        \inference
        {
        }
        {
          \matchpat
          {\ttt{y}}
          {\ol{x}}
          {\{\ol{x} \mapsto \ttt{y}\}}
        } &
        \inference
        {
        }
        {
          \matchpat
          {\ttt{ys}}
          {\ttt{xs}}
          {\emptyset}
        }
      }
      {
        \matchpat
        {\ttt{y :: ys}}
        {\ttt{$\ol{x}$ :: xs}}
        {\{\ol{x} \mapsto \ttt{y}\}}
      }
    }
    {
      \matchpat
      {\ttt{(x, y :: ys)}}
      {\ttt{($\ol{a}$, $\ol{x}$ :: xs)}}
      {\{\ol{a} \mapsto \ttt{x}, \ol{x} \mapsto \ttt{y}\}}
    }
    \]
  \end{block}
\end{frame}

\begin{frame}[fragile]
  \frametitle{Eksempel --- Omskrivning}

  \center\tiny Lad $\theta = \{\ol{a} \mapsto \ttt{x}, \ol{x} \mapsto
  \ttt{y}\}$, og husk at \ttt{xs} identificeres med \ttt{ys}.
  \begin{block}{}
    \tiny
    \[
    \mathcal{J}_1:\
    \inference
    {
      \inference
      {
        \inference
        {
          \inference
          {
          }
          {
            \matchbody
            {\sigma}
            {\theta}
            {\ol{a}}
            {\ttt{x}}
          } &
          \inference
          {
          }
          {
            \matchbody
            {\sigma}
            {\theta}
            {\ol{x}}
            {\ttt{y}}
          }
        }
        {
          \matchbody
          {\sigma}
          {\theta}
          {\ttt{($\ol{a}$, $\diamond_1$)$[\ol{x}/\diamond_1]$}}
          {\ttt{(x, y)}}
        }
      }
      {
        \matchbody
        {\sigma}
        {\theta}
        {\mathcal{C}[\ol{x}]}
        {\ttt{(x, y)}}
      }
    }
    {
      \matchbody
      {\sigma}
      {\theta}
      {\mathbb{D}(\mathcal{C}[\ol{x}])}
      {\ttt{$($v = w$)[$(x, y)$/$(v, w)$]$}}
    }
    \]
  \end{block}
  \begin{block}{}
    \[
    \mathcal{J}_2:\
    \inference
    {
      \inference
      {
        \inference
        {
          \inference
          {
          }
          {
            \matchbody
            {\sigma}
            {\theta}
            {\ol{a}}
            {\ttt{x}}
          } &
          \inference
          {
          }
          {
            \matchbody
            {\sigma}
            {\theta}
            {\ttt{xs}}
            {\ttt{ys}}
          }
        }
        {
          \matchbody
          {\sigma}
          {\theta}
          {\ttt{($\ol{a}$, $\diamond_1$)$[$xs$/\diamond_1]$}}
          {\ttt{(x, y)}}
        }
      }
      {
        \matchbody
        {\sigma}
        {\theta}
        {\ttt{$\mathcal{C}[$xs$]$}}
        {\ttt{(x, ys)}}
      }
    }
    {
      \matchbody
      {\sigma}
      {\theta}
      {\ttt{self$(\mathcal{C}[$xs$])$}}
      {\ttt{$($member x$)[$(x, ys)$/$x$]$}}
    }
    \]
  \end{block}
  \begin{block}{}
    \tiny
    \[
    \mathcal{J}:\
    \inference
    {\mathcal{J}_1 & & & & & & & & \mathcal{J}_2}
    {
      \matchbody
      {\sigma}
      {\theta}
      {\ttt{$\mathbb{D}(\mathcal{C}[\ol{x}])$ orelse self$(\mathcal{C}[$xs$])$}}
      {\ttt{x = y orelse member (x, ys)}}
    }
    \]
  \end{block}
\end{frame}

\begin{frame}
  \frametitle{Eksempel --- Resultat}

  \begin{block}{}
    \tiny
    \begin{eqnarray*}[x]
      \begin{eqnalign}[lcl]
        \mathcal{C}[\ol{x}\ \ttt{::}\ \ttt{xs}] &=>& \mathbb{D}(\mathcal{C}[\ol{x}])\
        \ttt{orelse}\ \ttt{self}(\mathcal{C}[\ttt{xs}]) \\
        \mathcal{D} &=>& \ttt{false} \\
      \end{eqnalign}\\
      \textsf{where}\ \textsf{samedom} (\mathcal{C}, \mathcal{D}) \\
      %
      \Downarrow \\
      %
      \begin{eqnalign}[lcl]
        \color{purple}
        \mathcal{C}[\ttt{xs}] &\color{purple}=>&
        \color{purple}\ttt{List.exists (fn x => $\mathbb{D}(
          \mathcal{C}[$x$])$)} \ \ttt{xs} \\
      \end{eqnalign}
    \end{eqnarray*}
  \end{block}

  \begin{block}{}
    \tiny
    \[
    \inference
    {
      \inference
      {
        \inference
        {
          \inference
          {
            \inference
            {
            }
            {
              \matchbody
              {\sigma}
              {\theta}
              {\ol{a}}
              {\ttt{x}}
            } &
            \inference
            {
            }
            {
              \matchbody
              {\sigma}
              {\theta}
              {\ttt{x}}
              {\ttt{y}}
            }
          }
          {
            \matchbody
            {\sigma}
            {\theta}
            {\ttt{($\ol{a}$, $\diamond_1$)$[\ttt{x}/\diamond_1]$}}
            {\ttt{(x, y)}}
          }
        }
        {
          \matchbody
          {\sigma}
          {\theta}
          {\mathcal{C}[\ttt{x}]}
          {\ttt{(x, y)}}
        }
      }
      {
        \matchbody
        {\sigma}
        {\theta}
        {\mathbb{D}(\mathcal{C}[\ttt{x}])}
        {\ttt{$($v = w$)[$(x, y)$/$(v, w)$]$}}
      } &
      \inference
      {
      }
      {
        \matchbody
        {\sigma}
        {\theta}
        {\ttt{xs}}
        {\ttt{ys}}
      }
    }
    {
      \matchbody
      {\sigma}
      {\theta}
      {\ttt{List.exists (fn x => $\mathbb{D}(\mathcal{C}[$x$])$)} \ \ttt{xs}}
      {\ttt{List.exists (fn y => x = y) ys}}
    }
    \]
  \end{block}

\end{frame}


\begin{frame}
  \frametitle{Stratego}

\end{frame}

\end{document}

%%% Local Variables: 
%%% mode: latex
%%% TeX-master: t
%%% End: 
