% xcolor=table is because it seems that beamer uses the xcolor package in a
% strange way and thus don't accept us giving arguments to the package.
\documentclass[slidestop,compress,mathserif, xcolor=dvipsnames]{beamer}

\usepackage[T1]{fontenc}
\usepackage[utf8]{inputenc}
\usepackage[danish]{babel}

\renewcommand{\ttdefault}{txtt} % Bedre typewriter font


% Use the NAT theme in uk (also possible in DK)
\usetheme[nat, uk, footstyle=low]{Frederiksberg}

% Make overlay sweet nice by having different transparancy depending on how
% "far" ahead the overlay is. AWSOME!!
\setbeamercovered{highly dynamic}
% possible to shift back, so they are just invisible untill they should overlay
% \setbeamercovered{invisible}

% Extend figures into either left or right margin
% Ex: \begin{narrow}{-1in}{0in} .. \end{narrow} will place 1in into left margin
\newenvironment{narrow}[2]{%
  \begin{list}{}{%
  \setlength{\topsep}{0pt}%
  \setlength{\leftmargin}{#1}%
  \setlength{\rightmargin}{#2}%
  \setlength{\listparindent}{\parindent}%
  \setlength{\itemindent}{\parindent}%
  \setlength{\parsep}{\parskip}}%
\item[]}{\end{list}}

%\usepackage{movie15}

\usepackage[final]{listings} % Make sure we show the listing even though we are
                             % making a final report.

\lstset{ %
% language=Octave,                % choose the language of the code
basicstyle=\ttfamily,        % the size of the fonts that are used for the code
basewidth=0.5em,
% numbers=left,                   % where to put the line-numbers
% numberstyle=\footnotesize,      % the size of the fonts that are used for the line-numbers
% stepnumber=2,                   % the step between two line-numbers. If it's 1 each line will be numbered
% numbersep=5pt,                  % how far the line-numbers are from the code
% backgroundcolor=\color{white},  % choose the background color. You must add \usepackage{color}
% showspaces=false,               % show spaces adding particular underscores
% showstringspaces=false,         % underline spaces within strings
% showtabs=false,                 % show tabs within strings adding particular underscores
% frame=single	                % adds a frame around the code
% tabsize=2,	                % sets default tabsize to 2 spaces
% captionpos=b,                   % sets the caption-position to bottom
% breaklines=true,                % sets automatic line breaking
% breakatwhitespace=false,        % sets if automatic breaks should only happen at whitespace
escapeinside={(@}{@)}          % if you want to add a comment within your code
}

\lstnewenvironment{sml}
{%\lstset{}
 } % starting code, ex a lstset that is more specific;
{} % ending code

\newcommand{\smlinline}[1]{\lstinline{#1}}
\newcommand{\mathsml}[1]{\textnormal{\lstinline{#1}}}

\usepackage{mdwtab}
\usepackage{mathenv}
\usepackage{amsfonts}
\usepackage{amsmath}
\usepackage{amssymb}
\usepackage{amsthm}
\usepackage{semantic} % for the \mathlig function


\def\TheTrueColour{BrickRed}
\newcommand{\cc}[1]{{\color{\TheTrueColour}#1}}
\newcommand{\subspat}[3]{\ensuremath{#1\cc{(}#2\cc{)}\mathrel{\cc{=}}#3}}
\newcommand{\matchpat}[3]{\ensuremath{\cc{{\color{black}#1}\mathrel{:}\left\langle{\color{black}#2},\mathrel{ }{\color{black}#3}\right\rangle}}}
\newcommand{\matchbody}[4]{\ensuremath{#1\cc{,}\mathrel{ }#2\mathrel{\cc{|-}}#3\mathrel{\cc{:}}#4}}
\newcommand{\matchclause}[3]{\ensuremath{#1\mathrel{\cc{|-}}#2\mathrel{\cc{:}}#3}}
\newcommand{\rewrite}[4]{\ensuremath{#1\mathrel{\cc{,}}#2\mathrel{\cc{|-}}#3\mathrel{\cc{\curvearrowright}}#4}}
\newcommand{\becomesthrough}[3]{\ensuremath{#1\mathrel{\textsf{\cc{becomes}}}#2\mathrel{\textsf{\cc{through}}}#3}}

\newcommand{\ttt}[1]{\texttt{#1}}
\newcommand{\tnm}[1]{\textnormal{#1}}
\newcommand{\tsf}[1]{\textsf{#1}}
\newcommand{\mrm}[1]{\mathrm{#1}}
\newcommand{\ol}[1]{\overline{#1}}

\newcommand{\Cov}{\mathrm{Cov}}
\providecommand{\FV}{\mathrm{FV}}
\providecommand{\Dom}{\mathrm{Dom}}


\mathlig{||}{\parallel}
\mathlig{<'}{\prec}
\mathlig{>'}{\succ}
\mathlig{<='}{\preccurlyeq}
\mathlig{>='}{\succcurlyeq}
\mathlig{<=}{\leqslant}
\mathlig{>=}{\geqslant}
\mathlig{<>}{\neq}
\mathlig{|=}{\sqsubset}
\mathlig{=|}{\sqsupset}
\mathlig{==}{\equiv}
\mathlig{==a}{=_{\alpha}}
\mathlig{<|}{\lhd}
\mathlig{|>}{\rhd}
\mathlig{++}{\mathrel{\mbox{+\!\!\!+}}}
% ~>e or ~>g conflicts with the \cite command for some reason.
\mathlig{->e}{\stackrel{elim}{\leadsto}}
\mathlig{->g}{\stackrel{gen}{\leadsto}}
\mathlig{++}{\mathrel{\mbox{+\!\!\!+}}}

% Write a short text to have that shown in the footer of slides other than the
% title slide.
\title[]{Turtledove}
% A possible subslide.
\subtitle{\tiny{Tool assisted programming in SML, with special emphasis on
    semi-automatic\\ rewriting to predefined standard forms.}}

\author{Jesper Reenberg}

% Only write DIKU in the footer of slides (except the title slide).
\institute[DIKU]{Department of Computer Science}

% Remove the date stamp from the footer of slides (except title slide) by giving
% it no short "text"
\date[]{\today} 




\begin{document}

\frame[plain]{\titlepage}

\begin{frame}[c]
  \frametitle{Oversigt}  
  Af stører (nutidige) projekter findes:
  
  \begin{itemize}
  \item HaRe {\footnotesize{(Haskell)}}
    
  \item Værktøjer       
    \begin{itemize}
    \item Programatica {\footnotesize{(brugt i HaRe)}}
    \item Strafunski {\footnotesize{(brugt i HaRe)}}
    \item Stratego
    \end{itemize}       
  \end{itemize}

\end{frame}

\begin{frame}[c]
  \frametitle{Sammenligning -- HaRe 1/3}  

  \begin{columns}    
    \begin{column}{0.5\textwidth{}}
      \textbf{HaRe:}
      \begin{itemize}
      \item Kommentar og layout bevarelse 
        \begin{itemize}
        \item Haskell kan bruge layout regler
        \end{itemize}    
      \item Mulighed for brugerdefineret refaktorering og program transformationer.      
      \end{itemize}      
    \end{column}
    \begin{column}{0.5\textwidth{}}
      \textbf{Turtledove:}
      \begin{itemize}
      \item Ingen layout bevarelse 
        \begin{itemize}
        \item Ikke strengt nødvendigt
        \item Bruger en simpel pretty printer
        \end{itemize}
      \item Forberedt for brugerdefinerede omskrivnings regler.
      \end{itemize}
    \end{column}
  \end{columns}
\end{frame}

\begin{frame}[c]
  \frametitle{Sammenligning -- HaRe 2/3}  

  Refaktoreringer i HaRe (Ikke nødvendigvis begrænset til):

  \begin{itemize}
  \item Structural
    \begin{itemize}
    \item Rename
    \item Add/Remove argument to fun-definition
    \end{itemize}

  \item Modul
    \begin{itemize}
    \item Clean import lists
    \item Move definition between modules
    \item Add/Remove entity to module export list
    \end{itemize}

  \item Data-Oriented
    \begin{itemize}
    \item Concrete to abstract data type \\ 
      {\footnotesize{(hide the value constructors from the user)}}
    \item Create discriminator functions \\
      {\footnotesize{(isLeaf, isNode, ...)}}
    \item Create constructor functions  \\
      {\footnotesize{(mkLeaf, mkNode ...)}}
    \end{itemize}                                                        
  \end{itemize}
\end{frame}


\begin{frame}[c]
  \frametitle{Sammenligning -- HaRe 3/3}  

  Alle nævnte refaktoreringer fra HaRe kan implementeres som plug-ins i
  Turtledove.
  \\ \ \\

  Udviklede "`Apps"' er eksempler på simple plug-ins. 
  \\ \ \\
  
  \begin{center}
    Eksempel/Demo: Simpel omdøbning af funktions navne.
  \end{center}
  
\end{frame}


\begin{frame}[c]
  \frametitle{Programatica}  

  \begin{quote}
    "`A Haskell front-end, with functionality similar to what you find in a
    compiler front-end for Haskell, implemented in Haskell."'
    \\ \raggedleft \cite{programatica-features}
  \end{quote}

  \begin{columns}    
    \begin{column}{0.3\textwidth{}}
      Features:
      \begin{itemize}
      \item Lexer
      \item Parser
      \item Type checker
      \item Pretty printer
      \item etc.
      \end{itemize}
    \end{column}
    \begin{column}{0.5\textwidth{}}
      Ikke brugbar fordi:
      \begin{itemize}
      \item Haskell specifik
      \item Havde SML lexer/parser 
      \item Havde MLB lexer/parser
      \item Lille/ingen dokumentation
      \end{itemize}
    \end{column}
  \end{columns}
\end{frame}

\begin{frame}[c]
  \frametitle{Strafunski}

  \begin{quote}
    "`Strafunski is a Haskell-centred software bundle for implementing language
    processing components --- most notably program analyses and
    transformation"'\\ \raggedleft \cite{LV03-PADL}
  \end{quote}

  \ \\ \ \\

  \begin{columns}    
    \begin{column}{0.5\textwidth{}}
      Features:
      \begin{itemize}
      \item Generisk travasering og tranformation af AST's
        \begin{itemize}
        \item Via forskellige strategier
        \end{itemize}
      \end{itemize}
    \end{column}
    \begin{column}{0.5\textwidth{}}
      Ikke brugbar fordi:
      \begin{itemize}
      \item Alt arbejde skal udføres i Haskell
      \end{itemize}
    \end{column}

  \end{columns} 
\end{frame}


\begin{frame}[c]
  \frametitle{Stratego}

fooo

\end{frame}


\begin{frame}[c, fragile]
  \frametitle{Eksempel: Foldr}
  
  Hvis vi kikker på den inferens regel der "`matcher"' en funktions clausul og
  regel clausul: \fbox{\matchclause{\sigma}{clause}{sclause}} .
  
  {
    \scriptsize
    \begin{block}{Regel: Definition 22}    
      \begin{eqnarray*}[x]
        \begin{eqnalign}[lcl]
          \color{red}
          \mathcal{C}[\ol{x}\ \ttt{::}\ \ttt{xs}][\ol{a}]
          & \color{red} =>&
          \color{red} \ttt{self}(\mathcal{C}[\ol{x}][\mathbb{D}(\mathcal{C}[\ol{x}][\ol{a}])])
          \\
          \mathcal{C}[\ol{x}][\ol{a}] &=>& \ol{a}
        \end{eqnalign}\\
        % 
        \Downarrow \\
        % 
        \begin{eqnalign}[lcl]
          \mathcal{C}[\ttt{xs}][\ttt{a}] &=>& \ttt{foldl (fn (x, a) =>
            $\mathbb{D}(\mathcal{C}[$x$][$a$])$) \ttt{a} \ttt{xs}}
        \end{eqnalign}
      \end{eqnarray*}
    \end{block}
    
    \begin{block}{Norm fun: Eksempel 19, uden curry, forkortet navn}
      \begin{sml}
fun cmplist (@\color{red}(y :: ys, x) = y (cmplist (ys, x))@)
  | cmplist (y, x) = x
      \end{sml}    
    \end{block}
  }
Dog kun første clausul, da anden clausul er ikke spændene.
\end{frame}

\begin{frame}[c, fragile]
  \frametitle{Eksempel: Foldr}


  \begin{narrow}{-3em}{-3em}
  {
    \tiny   

    \textnormal{Lad} $\sigma(C) = (\diamond_1 \mathtt{::} \ol{ys},\
    \diamond_2)$:
    \\ \ \\

    $
    \subspat{\sigma}
    {\mathcal{C}[\ol{x}\ \ttt{::}\ \ttt{xs}][\ol{a}])}
    {\ttt{($\ol{y}$ :: $\ol{ys}$, $\ol{x}$)}} 
    \equiv
    $

    \[   
      \inference
      {}
      {
        \subspat{\sigma}
        {\mathcal{C}[\ol{x}\ \ttt{::}\ \ttt{xs}][\ol{a}])}
        {(\diamond_1 \mathsml{::} \ol{ys},\ \diamond_2)[\ol{y}/\diamond_1][\ol{x}/\diamond_2]}
      }
    \]      
 
    
    $\matchpat{\mathsml{(y :: ys, x)}}
    {\ttt{($\ol{y}$ :: $\ol{ys}$, $\ol{x}$)}}
    {\Theta}
    \equiv$

    \[
    \inference
    {
      \inference
      {}
      {
        \matchpat{\mathsml{y :: ys}}
        {\ttt{$\ol{y}$ :: $\ol{ys}$}}
        {         
          \left\{
            \begin{tabular}{Mc@{$\ \mapsto\ $}Ml}                
              \ol{y} & \ttt{y} \\
              \ol{ys} & \ttt{ys}
            \end{tabular}
          \right\}
        }
      }
      &
      \inference
      {}
      { 
        \matchpat{\mathsml{x}}
        {\ol{x}}
        {
          \{\ol{x} \mapsto \ttt{x}\}
        }
      }
    }
    {
      \matchpat{\mathsml{(y :: ys, x)}}
      {\ttt{($\ol{y}$ :: $\ol{ys}$, $\ol{x}$)}}
      {
        \Theta = 
        \left\{                       
          \begin{tabular}{Mc@{$\ \mapsto\ $}Ml}
            \ol{y} & \ttt{y} \\
            \ol{ys} & \ttt{ys} \\
            \ol{x} & \ttt{x}
          \end{tabular}
        \right\}
      }        
    }
    \]
    
    $\matchbody{\sigma}{\Theta}
    {a}
    {b}
    \equiv
    $

    \[
    \inference
    {}
    {
      \matchbody{\sigma}{\Theta}
      {\ttt{self}(\mathcal{C}[\ol{x}][\mathbb{D}(\mathcal{C}[\ol{x}][\ol{a}])])}
      {\ttt{cmplist} (...,...)}
    }
    {\sigma(\mathsf{self}) = (\ttt{cmplist}, 1)}
    \]
    
FOLDR!

    \[
    \inference
    {
      %
      % \sigma(spat) = mpat    
      %
      \inference
      {
      }
      {
        \subspat{\sigma}
        {\mathcal{C}[\ol{x}\ \ttt{::}\ \ttt{xs}][\ol{a}])}
        {\ttt{($\ol{y}$ :: $\ol{ys}$, $\ol{x}$)}} 
      }
      &
      % 
      % pat : <mpat, \Theta>
      %
      \inference
      {
      }
      {
        \matchpat{\mathsml{(y :: ys, x)}}
        {\ttt{($\ol{y}$ :: $\ol{ys}$, $\ol{x}$)}}
        {\Theta}
      }    
      &
      %
      % \sigma, \Theta |- sexp : exp
      % 
      \inference
      {}
      {
        \matchbody{\sigma}{\Theta}
        {sexp}
        {exp}
      }
    }
    {
      \matchclause{\sigma}
      {\mathsml{(y :: ys, x) => y (cmplist (ys, x))}}
      {\mathcal{C}[\ol{x}\ \ttt{::}\ \ttt{xs}][\ol{a}] =>
        \ttt{self}(\mathcal{C}[\ol{x}][\mathbb{D}(\mathcal{C}[\ol{x}][\ol{a}])])
      }
    }
    \]
  }
  \end{narrow}

\end{frame}


\begin{frame}[c]
  \frametitle{Konklusion}

Vi valgte at lave vores eget.
  Det kan være svært at sætte sig ind i andre frameworks på "`kort tid"'.
  

\end{frame}

\begin{frame}
  \frametitle{Bibliografi}

  \bibliographystyle{../report/bibliography/theseurl}
  \bibliography{../report/bibliography/bibliography}
  
\end{frame}
\end{document}

%%% Local Variables: 
%%% mode: latex
%%% TeX-master: t
%%% End: 
