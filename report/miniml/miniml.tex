% \documentclass[a4paper, oneside, final]{memoir} 
\documentclass[a4paper, oneside, draft]{memoir} 
\usepackage[T1]{fontenc}
\usepackage[utf8]{inputenc}
\usepackage[british]{babel}

% bedre orddeling Gør at der som minimum skal blive to tegn på linien ved
% orddeling og minimum flyttes to tegn ned på næste linie. Desværre er værdien
% anvendt af babel »12«, hvilket kan give orddelingen »h-vor«.
\renewcommand{\britishhyphenmins}{22} 

% Fix of fancyref to work with memoir. Makes references look
% nice. Redefines memoir \fref and \Fref to \refer and \Refer.
% \usepackage{refer}             %
% As we dont really have any use for \fref and \Fref we just undefine what
% memoir defined them as, so fancyref can define what it wants.
\let\fref\undefined
\let\Fref\undefined
\usepackage{fancyref} % Better reference. 

\usepackage{pdflscape} % Gør landscape-environmentet tilgængeligt
\usepackage{fixme}     % Indsæt "fixme" noter i drafts.
\usepackage{hyperref}  % Indsæter links (interne og eksterne) i PDF

\usepackage[format=hang]{caption,subfig}
\usepackage{graphicx}
\usepackage{stmaryrd}
\usepackage{amssymb}
\usepackage{listings}
\usepackage{ulem} % \sout - strike-through
\usepackage{tikz}

\usepackage{mdwtab}
\usepackage{mathenv}
\usepackage{amsthm}
\usepackage{semantic} % for the \mathlig function


\renewcommand{\ttdefault}{txtt} % Bedre typewriter font
% \usepackage[sc]{mathpazo}     % Palatino font
% \renewcommand{\rmdefault}{ugm} % Garamond
% \usepackage[garamond]{mathdesign}

% \overfullrule=5pt
% \setsecnumdepth{part}
\setcounter{secnumdepth}{1} % Sæt overskriftsnummereringsdybde. Disable = -1.
\chapterstyle{hangnum} % changes style of chapters, to look nice.

\theoremstyle{definition}
\newtheorem{judgement}{Judgement}
\newtheorem{definition}{Definition}
\newtheorem{lemma}{Lemma}
\newtheorem{theorem}{Theorem}
\newtheorem{corollary}{Corollary}


\mathlig{||}{\parallel}
\mathlig{<'}{\prec}
\mathlig{>'}{\succ}
\mathlig{<='}{\preccurlyeq}
\mathlig{>='}{\succcurlyeq}
\mathlig{<=}{\leq}
\mathlig{>=}{\geq}
\mathlig{<>}{\neq}
\mathlig{|=}{\sqsubset}
\mathlig{=|}{\sqsupset}
\mathlig{==}{\equiv}
\mathlig{==a}{=_{\alpha}}
\mathlig{<|}{\lhd}
\mathlig{|>}{\rhd}

\begin{document}
\chapter{Term rewriting in a subset of SML}
\section{Grammar}
\fixme{remember to introduce new relations with name.}
\begin{eqnarray*}[rqcql:Tl]
  var & = & (a-z)* & Identifiers\\
  con & = & (A-Z)(a-z)* & Constructors\\
  num & = & \mathbb{Z}    & Natural numbers\\
  match & ::= & \epsilon                            & Empty match\\
  & & pat\texttt{.}exp\ \texttt{|}\ match               & Pattern -> expression\\
  pat & ::= & var                                       & Variable\\
  & & con\texttt{(}pat_1\texttt{,} \ldots\texttt{,} pat_n\texttt{)} & Where con
  has arity $n$\\
  exp & ::= & var                                       & Variable\\
  & & exp_1 exp_2                                     & Application\\
  & & \lambda match                                  & Abstraction\\
  & & con\texttt{(}exp_1\texttt{,} \ldots\texttt{,} exp_n\texttt{)} & Where con
  has arity $n$\\
  dec & ::= & var \mapsto exp                         & Plain value binding\\
  & & \texttt{rec}\ var \mapsto \lambda match         & Recursive value binding\\
  & & con \texttt{:} \overline{num}                           & Constructor of arity $num$\\
  & & \epsilon                             & Empty program\\
  & & dec_1 \texttt{;} dec_2                                      & Sequencing
\end{eqnarray*}

\paragraph{Note.} In the following we write $==a$ to mean alpha equivalence. We
also use (with superfixes, subfixes and marks) $v$, $c$, $n$, $m$, $p$, $e$ and
$d$ to range over $var$, $con$, $num$, $match$, $pat$, $exp$ and $dec$
respectively.

\section{Total ordering on patterns}
\begin{definition}[$p_1 < p_2$]
  Assume a total transitive ordering $<$ on constructors and
  variables. Inductively defined:
  \begin{eqnarray}
    v_1 &<& v_2\\
    c\texttt{(}p_1\texttt{,} \ldots\texttt{,} p_n\texttt{)} &<& v\\
    c_1\texttt{(}p_1\texttt{,} \ldots\texttt{,} p_n\texttt{)} &<&
    c_2\texttt{(}p'_1\texttt{,} \ldots\texttt{,} p'_m\texttt{)}\label{eq:con-total-order-2}
  \end{eqnarray}
  Where \ref{eq:con-total-order-2} holds iff
  \[
  c_1 < c_2 \lor c_1 = c_2 \land ( p_1 < p'_1 \lor p_1 = p'_1 \land (\ldots p_n < p'_n \ldots ))
  \]
\end{definition}

\begin{lemma}[Transitivity of $<$]
  \[
  p_1 < p_2 \land p_2 < p_3 \Rightarrow p_1 < p_3
  \]
\end{lemma}
\begin{proof}
  If $p_3 = v$ then the result follows immediately. Therefore assume $p_3 = c_3
  \texttt{(}p^3_1 \texttt{,} \ldots \texttt{,} p^3_n\texttt{)}$.
  By the assumptions we have that $p_1 = c_1 \texttt{(}p^1_1 \texttt{,} \ldots
  \texttt{,} p^1_n\texttt{)}$, $p_2 = c_2 \texttt{(}p^2_1 \texttt{,} \ldots
  \texttt{,} p^2_n\texttt{)}$ and
  \begin{eqnarray}[c]
    c_1 < c_2 \lor c_1 = c_2 \land ( p^1_1 < p^2_1 \lor p^1_1 = p^2_1 \land
    (\ldots p^1_n < p^2_n \ldots )) \label{eq:trans-tot-proof-1}\\
    c_2 < c_3 \lor c_2 = c_3 \land ( p^2_1 < p^3_1 \lor p^2_1 = p^3_1 \land
    (\ldots p^2_n < p^3_n \ldots )) \label{eq:trans-tot-proof-2}
  \end{eqnarray}
  Combining \fref{eq:trans-tot-proof-1} and \fref{eq:trans-tot-proof-2} we get
  \begin{eqnarray*}[c]
    c_1 < c_2 \lor c_1 = c_2 \land c_2 < c_3 \lor c_2 = c_3 \land\\ ( p^1_1 < p^2_1
    \lor p^1_1 = p^2_1 \land p^2_1 < p^3_1 \lor p^2_1 = p^3_1 \land
    (\ldots p^1_n < p^2_n \land p^2_n < p^3_n \ldots ))
  \end{eqnarray*}
  And then using induction we get
  \begin{eqnarray*}
    c_1 < c_3 \lor c_1 = c_3 \land ( p^1_1 < p^3_1 \lor p^1_1 = p^3_1 \land
    (\ldots p^1_n < p^3_n \ldots ))
  \end{eqnarray*}
  and then by definition $p_1 < p_3$.
\end{proof}

\section{Partial ordering on patterns}
\begin{definition}[$p_1 <=' p_2$]
  $p_2$ generalises $p_1$ or $p_1$ is more specific than $p_2$. Inductively
  defined.
  \begin{eqnarray}
    p &<='& v\\
    c_1\texttt{(}p_1\texttt{,} \ldots\texttt{,} p_n\texttt{)} &<='&
    c_2\texttt{(}p'_1\texttt{,} \ldots\texttt{,} p'_m\texttt{)}\label{eq:con-partial-order-2}
  \end{eqnarray}
  Where \fref{eq:con-partial-order-2} holds iff
  \begin{eqnarray*}
    c_1 &=& c_2 \quad \land\\
    p_1 &<='& p'_1 \quad \land\\
    &\ldots&\\
    p_n &<='& p'_n
  \end{eqnarray*}
\end{definition}

\begin{definition}[$p_1 \prec p_2$]
  \[
  p_1 <' p_2 \Leftrightarrow p_1 <=' p_2 \land p_1 \not ==a p_2
  \]
\end{definition}

And we write $p_1 >=' p_2$ and $p_1 >' p_2$ to mean $p_2 <=' p_1$ and $p_2 <'
p_1$ respectively.

\begin{lemma}[Transitivity of $<'$ and $<='$]
  \begin{eqnarray*}
    p_1 <=' p_2 \land p_2 <=' p_3 \Rightarrow p_1 <=' p_3\\
    p_1 <' p_2 \land p_2 <' p_3 \Rightarrow p_1 <' p_3
  \end{eqnarray*}
  Since $<='$ is the reflexive extension of $<'$, then if $<='$ is transitive
  clearly so is $<'$. We prove the transitivity of $<='$.
  \begin{proof}
    If $p_3 = v$ the result follows immediately. So assume $p_3 = c_3
    \texttt{(}p^3_1 \texttt{,} \ldots \texttt{,} p^3_n\texttt{)}$.
    By the assumptions we have that $p_1 = c_1 \texttt{(}p^1_1 \texttt{,} \ldots
    \texttt{,} p^1_n\texttt{)}$, $p_2 = c_2 \texttt{(}p^2_1 \texttt{,} \ldots
    \texttt{,} p^2_n\texttt{)}$ and
    \begin{eqnarray}[c]
      c_1 = c_2 \land p^1_1 <=' p^2_1 \ldots p^1_n <=' p^2_n \label{eq:trans-part-proof-1}\\
      c_2 = c_3 \land p^2_1 <=' p^3_1 \ldots p^2_n <=' p^3_n \label{eq:trans-part-proof-2}
    \end{eqnarray}
    Combining \fref{eq:trans-part-proof-1} and \fref{eq:trans-part-proof-2} gives us
    \begin{eqnarray*}
      c_1 = c_2 \land c_2 = c_3 \land p^1 <=' p^2_1 \land p^2_1 <=' p^3_1 \ldots
      p^1_n <=' p^2_n \land p^2_n <=' p^3_n
    \end{eqnarray*}
    And then by induction we get
    \begin{eqnarray*}
      c_1 = c_3 \land p^1_1 <=' p^3_1 \ldots p^1_n <=' p^3_n
    \end{eqnarray*}
    which by definition gives us $p_1 <=' p_3$.
  \end{proof}
\end{lemma}

\begin{definition}[$p_1 || p_2$]
  \[
  p_1 || p_2 \Leftrightarrow p_1 \not <=' p_2 \land p_2 \not <=' p_1
  \]
\end{definition}

\begin{corollary}[Reflexivity of $||$]
  \[
  p_1 || p_2 \Leftrightarrow p_2 || p_1
  \]
\end{corollary}
\begin{proof}
  By definition.
\end{proof}


\begin{lemma}[Equivalence classes of patterns]
  \[
  \forall p_1, p_2 : p_1 ==a p_2 \lor p_1 <' p_2 \lor p_1 >' p_2 \lor p_1 || p_2
  \]
\end{lemma}
\begin{proof}
  Immediately by inspection.
\end{proof}

\begin{lemma}\label{part-imp-tot}
  \[
  p_1 <' p_2 \Rightarrow p_1 < p_2
  \]
\end{lemma}
\begin{proof}
  Immediately by induction.
\end{proof}

\section{Elimination of unused patterns}\label{sec:elim-unus-patt}
\begin{definition}[$m_1 =| m_2$]
  Let
  \[
  m = p_1\texttt{.}e_1 \texttt{|} \ldots \texttt{|} p_n\texttt{.}e_n
  \]
  If $p_j <=' p_i$ where $j > i$ then 
  \[
  m =| m'
  \]
  where
  \[
  m' = p_1\texttt{.}e_1 \texttt{|} \ldots \texttt{|}
  p_{j-1}\texttt{.}e_{j-1} \texttt{|} p_{j+1}\texttt{.}e_{j+1} \texttt{|}
  \ldots \texttt{|} p_n\texttt{.}e_n
  \]
  In other words; pattern $j$ and its body can be removed from $m$.
\end{definition}

\begin{lemma}(Preservation)
  \[
  m \sqsupset m' \Rightarrow \forall e : \texttt{(}\lambda m\texttt{)}e \downarrow
  v \Leftrightarrow \texttt{(}\lambda m'\texttt{)}e \downarrow v
  \]
\end{lemma}
\begin{proof}
  Trivial (as if).
\end{proof}

\section{Partially ordered form}
\begin{definition}
  A match $m = p_1\texttt{.}e_1\texttt{|}\ldots\texttt{|}p_n\texttt{.}e_n$ is in
  partially ordered form iff
  \[
  \forall i \in \{1, \ldots, n\} : p_j \not <=' p_i \quad \textnormal{where $j > i$}
  \]

  Note that every match $m$ can be transformed to an equivalent match $m'$ such
  that $m'$ is in partially ordered form, by repeated elimination of unused
  patterns (\fref{sec:elim-unus-patt}).
\end{definition}

\section{Generalisation of patterns}

In the following we define generalisation of matches using the operator $<|$ using the
auxiliary operator $<|'$, which generalises patterns.

REMOVE:\\
The syntactic category $pat$ is a proper subset of $exp$. Let $\kappa : pat
\rightarrow exp$ be the canonical mapping from $pat$ to $exp$.

Let the free variables of a pattern $FV_{pat} : pat \rightarrow 2^{var}$ be
defined by $FV_{pat} = FV_{exp} \circ \kappa$, where $FV_{exp}$ denotes the free
variables of an expression in the usual sense.\\
END REMOVE.

\begin{definition}[Generalisation of patterns $<|'$]
\begin{eqnarray}[c]
  (c\texttt{(}p_1\texttt{,}\ldots\texttt{,}p_n\texttt{)} , e)
  <|'
  (x , e[x/\kappa
  (c\texttt{(}p_1\texttt{,}\ldots\texttt{,}p_n\texttt{)})]) \label{eq:gen-1}\\
  \begin{eqnalign}
    (c\texttt{(}p_1\texttt{,}\ldots\texttt{,}&p_i&\texttt{,}\ldots\texttt{,}p_n\texttt{)}
    , e) <|' \\
    (c\texttt{(}p_1\texttt{,}\ldots\texttt{,}&p'_i&\texttt{,}\ldots\texttt{,}p_n\texttt{)},
    e')
  \end{eqnalign} \label{eq:gen-2}
\end{eqnarray}

Where \fref{eq:gen-1} holds when $x$ is a fresh variable and
\[
FV_{pat}(c\texttt{(}p_1\texttt{,}\ldots\texttt{,}p_n\texttt{)}) \cap FV_{exp}(e[x/\kappa
(c\texttt{(}p_1\texttt{,}\ldots\texttt{,}p_n\texttt{)})]) = \emptyset
\]
and \fref{eq:gen-2} holds when
\[
(p_i , e) <|' (p'_i , e')
\]
\end{definition}
\begin{lemma}[Generalisation] \label{lem:gener-patt}
  \[
  (p_1, e_1) <|' (p_2, e_2) \Rightarrow p_1 <' p_2
  \]
\end{lemma}
\begin{proof}
  Straight forward induction proof.
\end{proof}

\begin{definition}[Generalisation of matches $<|$]
Assume that the match is partially ordered, $p_i' <' p_j$, $p_i.e_i <|'
p'_i.e'_i$ then by \fref{lem:gener-patt} we have that $p_i <' p'_i$.

\begin{eqnarray}[c]
  \begin{eqnalign}
    p_1.e_1 \texttt{ |} \ldots \texttt{| } &p_i.e_i& \texttt{ |} \ldots
    \texttt{| } p_n.e_n <| \\ 
    p_1.e_1 \texttt{ |} \ldots \texttt{| } &p'_i.e'_i&
    \texttt{ |} \ldots \texttt{| } p_n.e_n
  \end{eqnalign} \label{eq:gen-all-stay} \\
  \begin{eqnalign}[Ll]
    p_1.e_1 \texttt{ |} \ldots \texttt{| } p_m.e_m \texttt{ |} \ldots \texttt{| }
    p_{q_1}.e_{q_1} \texttt{ |} \ldots \texttt{| } p_{q_k}.e_{q_k} \texttt{ |}
    \ldots \texttt{| } \\ 
    p_{q_l}.e_{q_l} \texttt{ |} \ldots \texttt{| } p_n.e_n <| & \\
    & p_1.e_1 \texttt{ |} \ldots \texttt{| } p_{q_1}.e_{q_1} \texttt{ | }
    p_{q_k}.e_{q_k} \texttt{ | } p_{q_l}.e_{q_l} \texttt{ | } p'_m.e'_m \texttt{
      |} \\ 
    & \ldots \texttt{| } p_n.e_n
  \end{eqnalign} \label{eq:gen-all-move}\\
  \begin{eqnalign}
    p_1.e_1 \texttt{ |} \ldots \texttt{| } p_m.e_m \texttt{ |} &\ldots&
    \texttt{| } p_q.e_q \texttt{ |} \ldots \texttt{| } p_n.e_n <| \\ 
    p_1.e_1 \texttt{ |} \ldots \texttt{| } &m_q.e_q&
    \texttt{ |} \ldots \texttt{| } p_n.e_n
  \end{eqnalign} \label{eq:gen-all-remove}
\end{eqnarray}


Where \ref{eq:gen-all-stay} holds when
\[
(p_i || p_j \land p_i' || p_j) \lor (p_i <' p_j \land p_i' <' p_j) \quad
\texttt{for $j \in \{i+1 \ldots n\}$ }
\]
and \ref{eq:gen-all-move} holds when
\[
p_i || p_j \land p_j <' p_i' \quad \texttt{for $j \in \{i+1 \ldots n\}$ }
\]
and \ref{eq:gen-all-remove} holds when
\[
p_i <' p_j \land p_i' = p_j \quad \texttt{for $j \in \{i+1 \ldots n\}$ }
\]

We name \ref{eq:gen-all-stay} as ``stay'', \ref{eq:gen-all-move} as ``move'' and
\ref{eq:gen-all-remove} as ``remove''.
\end{definition}


% \section{Generalisation with unification of bodys}



% \section{Weak unification}

\end{document}


%%% Local Variables: 
%%% mode: latex
%%% TeX-master: t
%%% reftex-fref-is-default: t
%%% End: 
