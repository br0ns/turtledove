%\documentclass[a4paper, oneside, final]{memoir} 
\documentclass[a4paper, oneside, draft]{memoir} 
\usepackage[T1]{fontenc}
\usepackage[utf8]{inputenc}
\usepackage[british]{babel}

% bedre orddeling Gør at der som minimum skal blive to tegn på linien ved
% orddeling og minimum flyttes to tegn ned på næste linie. Desværre er værdien
% anvendt af babel »12«, hvilket kan give orddelingen »h-vor«.
\renewcommand{\britishhyphenmins}{22} 

% Fix of fancyref to work with memoir. Makes references look
% nice. Redefines memoir \fref and \Fref to \refer and \Refer.
% \usepackage{refer}             %
% As we dont really have any use for \fref and \Fref we just undefine what
% memoir defined them as, so fancyref can define what it wants.
\let\fref\undefined
\let\Fref\undefined
\usepackage{fancyref} % Better reference. 

\usepackage{pdflscape} % Gør landscape-environmentet tilgængeligt
\usepackage{fixme}     % Indsæt "fixme" noter i drafts.
\usepackage{hyperref}  % Indsæter links (interne og eksterne) i PDF

\usepackage[format=hang]{caption,subfig}
\usepackage{graphicx}
\usepackage{stmaryrd}
\usepackage{listings}
\usepackage{ulem} % \sout - strike-through
\usepackage{tikz}

\usepackage{mdwtab}
\usepackage{mathenv}
\usepackage{amsfonts}
\usepackage{amsmath}
\usepackage{amssymb}
\usepackage{amsthm}
\usepackage{semantic} % for the \mathlig function


\renewcommand{\ttdefault}{txtt} % Bedre typewriter font
%% \usepackage[sc]{mathpazo}     % Palatino font
%% \renewcommand{\rmdefault}{ugm} % Garamond
%% \usepackage[garamond]{mathdesign}

% \overfullrule=5pt
% \setsecnumdepth{part}
\setcounter{secnumdepth}{1} % Sæt overskriftsnummereringsdybde. Disable = -1.
\chapterstyle{hangnum} % changes style of chapters, to look nice.

\theoremstyle{definition}
\newtheorem{judgment}{Judgment}
\newtheorem{definition}{Definition}
\newtheorem{lemma}{Lemma}
\newtheorem{theorem}{Theorem}
\newtheorem{corollary}{Corollary}
\newtheorem{example}{Example}

\newcommand*{\fancyrefdeflabelprefix}{def}
\fancyrefaddcaptions{english}{
  \newcommand*{\Frefdefname}{Definition}
  \newcommand*{\frefdefname}{\MakeLowercase{\Frefdefname}}
}
\frefformat{vario}{\fancyrefdeflabelprefix}{%
  \frefdefname\fancyrefdefaultspacing#1#3%
}
\Frefformat{vario}{\fancyrefdeflabelprefix}{%
  \Frefdefname\fancyrefdefaultspacing#1#3%
}

\newcommand*{\fancyreflemlabelprefix}{lem}
\fancyrefaddcaptions{english}{
  \newcommand*{\Freflemname}{Lemma}
  \newcommand*{\freflemname}{\MakeLowercase{\Freflemname}}
}
\frefformat{vario}{\fancyreflemlabelprefix}{%
  \freflemname\fancyrefdefaultspacing#1#3%
}
\Frefformat{vario}{\fancyreflemlabelprefix}{%
  \Freflemname\fancyrefdefaultspacing#1#3%
}
\frefformat{plain}{\fancyreflemlabelprefix}{%
  \freflemname\fancyrefdefaultspacing#1%
}
\Frefformat{plain}{\fancyreflemlabelprefix}{%
  \Freflemname\fancyrefdefaultspacing#1%
}

\newcommand*{\fancyrefthmlabelprefix}{thm}
\fancyrefaddcaptions{english}{
  \newcommand*{\Frefthmname}{Theorem}
  \newcommand*{\frefthmname}{\MakeLowercase{\Frefthmname}}
}
\frefformat{vario}{\fancyrefthmlabelprefix}{%
  \frefthmname\fancyrefdefaultspacing#1#3%
}
\Frefformat{vario}{\fancyrefthmlabelprefix}{%
  \Frefthmname\fancyrefdefaultspacing#1#3%
}

\newcommand*{\fancyrefcorlabelprefix}{cor}
\fancyrefaddcaptions{english}{
  \newcommand*{\Frefcorname}{Corollary}
  \newcommand*{\frefcorname}{\MakeLowercase{\Frefcorname}}
}
\frefformat{vario}{\fancyrefcorlabelprefix}{%
  \frefcorname\fancyrefdefaultspacing#1#3%
}
\Frefformat{vario}{\fancyrefcorlabelprefix}{%
  \Frefcorname\fancyrefdefaultspacing#1#3%
}

\newcommand*{\fancyrefexlabelprefix}{ex}
\fancyrefaddcaptions{english}{
  \newcommand*{\Frefexname}{Example}
  \newcommand*{\frefexname}{\MakeLowercase{\Frefexname}}
}
\frefformat{vario}{\fancyrefexlabelprefix}{%
  \frefexname\fancyrefdefaultspacing#1#3%
}
\Frefformat{vario}{\fancyrefexlabelprefix}{%
  \Frefexname\fancyrefdefaultspacing#1#3%
}

\newcommand{\ttt}[1]{\texttt{#1}}
\newcommand{\tnm}[1]{\textnormal{#1}}
\newcommand{\mrm}[1]{\mathrm{#1}}

\newcommand{\Cov}{\mathrm{Cov}}
\providecommand{\FV}{\mathrm{FV}}
\providecommand{\Dom}{\mathrm{Dom}}


\mathlig{||}{\parallel}
\mathlig{<'}{\prec}
\mathlig{>'}{\succ}
\mathlig{<='}{\preccurlyeq}
\mathlig{>='}{\succcurlyeq}
\mathlig{<=}{\leqslant}
\mathlig{>=}{\geqslant}
\mathlig{<>}{\neq}
\mathlig{|=}{\sqsubset}
\mathlig{=|}{\sqsupset}
\mathlig{==}{\equiv}
\mathlig{==a}{=_{\alpha}}
\mathlig{<|}{\lhd}
\mathlig{|>}{\rhd}
\mathlig{++}{\mathrel{\mbox{+\!\!\!+}}}
\mathlig{~>e}{\stackrel{elim}{\leadsto}}
\mathlig{~>g}{\stackrel{gen}{\leadsto}}

%%%%%%%%%%%%%%%%%%%%%%%%%%%%%%%%%%%%%%%%%%%%%%%%%%%%%%%%
%	    	     Forside
%%%%%%%%%%%%%%%%%%%%%%%%%%%%%%%%%%%%%%%%%%%%%%%%%%%%%%%%
\makeatletter % open mode for reading @ signed variables 
\def\maketitle{%
  \null
  \thispagestyle{empty}%
  \vfill
  \begin{center}\leavevmode
    \normalfont
    \Huge{\raggedleft \@title\par}%
    \hrulefill\par
    \Large{\raggedright \subtitle\par}%
    \vskip 2cm
    {\@date\par}%
  \end{center}%
  \vfill
  \begin{flushleft}
    {\large \@author } \\
    {\footnotesize \suplementInfo }
  \end{flushleft}
  \cleardoublepage % lave 1 ekstre side blank efter
  \clearpage % Terminates the page here. Everything else vil be placed on next page.
}
\makeatother % closing mode for reading @ signed variables
%%%%%%%%%%%%%%%%%%%%%%%%%%%%%%%%%%%%%%%%%%%%%%%%%%%%%%%%
%		Data til forside
%%%%%%%%%%%%%%%%%%%%%%%%%%%%%%%%%%%%%%%%%%%%%%%%%%%%%%%%
\title{Term rewriting in a subset of SML}
\def\subtitle{\footnotesize{A joined Smartypants inc. and Morning Wood Productions venture.}
\author{Morten Brøns-Pedersen and Jesper Reenberg}}
\def\suplementInfo{
  \kern 5pt \hrule width 11pc \kern 5pt % putter 5pt spacing oven over og neden under stregen
  Dept. of Computer Science \\
  University of Copenhagen}
% \date{} % used to set explicit dates

\begin{document}

\frontmatter

\maketitle
\thispagestyle{empty}

\begin{abstract}
....
\end{abstract}

\clearpage 
\chapter*{Preface}
This paper is part of a preparational exercise done to ease the making of a 22.5
ECTS point project about ``Term rewriting in SML'' at the Department of Computer
Science (DIKU), University of Copenhagen. Authers of both projects are Morten
Brøns-Pedersen and Jesper Reenberg and they are both supervised by Jakob Grue
Simonsen, Assistant professor at DIKU

\clearpage

\tableofcontents*

\mainmatter

\chapter{Introduction}

\chapter{Introduction}

\section{Problem Statement}

{\footnotesize [Huge explanation about the actual problem.]}

\section{Motivation}
\label{sec:motivation}
Suppose that we want to find code that could be written shorter using the
\texttt{map} function.
Here is an obvious example:
\begin{sml}
fun add (x :: xs) = x + 1 :: add xs
  | add nil       = nil
\end{sml}
can be rewritten to
\begin{sml}
val add = map (fn x => x + 1)
\end{sml}

But suppose that the first function was instead
\begin{sml}
fun add nil       = nil
  | add (x :: xs) = x + 1 :: add xs
\end{sml}
or even\footnote{We have seen novice SML programmers write functions similar to
  this.}
\begin{sml}
fun add (x :: xs) = x + 1 :: add xs
  | add [x]       = x + 1 :: add nil
  | add nil       = nil
\end{sml}
Of course all three examples can be rewritten to the same form. So should we
have three rewriting rules? Infinitely many? No.

Our problem here is that equivalent matches (a match is a list of pairs of
patterns and expressions) can take many forms.

In this paper we define a language similar to Core SML. We define a
normal form for matches. We then show how to obtain an equivalent normal form
from an arbitrary match.

\paragraph{Further work.}
The reader might find it odd that the second line in the last example above ends
in \smlinline{:: add nil}. Novice programmer or not, real code probably does not
look like this. The reason is that if \smlinline{:: add nil} is left out, the three
versions of the function \smlinline{add} does not have the same normal form.

We would like to determine that
\begin{sml}
fun add (x :: xs) = x + 1 :: add xs
  | add [x]       = [x + 1]
  | add nil       = nil
\end{sml}
is indeed equivalent to
\begin{sml}
fun add (x :: xs) = x + 1 :: add xs
  | add nil       = nil
\end{sml}
In the first function the patterns in line one and two is \smlinline{x :: xs}
and \smlinline{x :: nil}, so we can try to instantiate \smlinline{xs} to
\smlinline{nil} in the first function body, to try to eliminate the second line
of the function.

The first body becomes \smlinline{x + 1 :: add nil} which is not equal to the
second which is \smlinline{x + 1 :: nil}. But if we inline the definition of
\smlinline{add} in the first body we get \smlinline{x + 1 :: nil}. And so the
second line of the body can be eliminated.

Of course inlining function definitions to find a normal form, makes the
reduction to normal forms an undecidable problem.

For this reason we have decided to define a normal form that can always be
found, and then build on top of that.

Consider this other example:
\begin{sml}
fun add (x :: xs, b) = x + b :: add xs
  | add (nil, _)     = nil
\end{sml}
That too can be written using \smlinline{map}. But again we would need a new
rewriting rule for that. A solution could be to use meta patterns in rewriting
rules.

\fixme[inline,margin=false]{Ok, I'm just ranting here. Keeping the example for
  further brainstorm later on.}

\section{Readers prerequisites}


Readers of this text should, as a minimal prerequisite, be familiar with
Standard ML. Some knowledge of compiler design and programming language theory,
is also recommended. Also some algorithmic and mathematically maturity will be
of help. Any to-be computer scientist with a few years experience should be able
to either directly understand this text, or be able to easily acquire the needed
knowledge.

\section{Structure outline}


%%% Local Variables: 
%%% mode: latex
%%% TeX-master: "../report"
%%% End: 


\chapter{Rules for term rewriting}

\section{Grammar}
\newcommand{\fun}{\ttt{fun}\ }
\newcommand{\rec}{\ttt{rec}\ }

\fixme{remember to introduce new relations with name.}
\begin{eqnarray*}[rqcql:Tl]
  var & = & \ttt{[a-z]+} & Identifiers\\
  con & = & \ttt{[A-Z][a-z]* | [0-9]} & Constructors\\
  num & = & \mathbb{Z}    & Natural numbers\\
  match & ::= & \epsilon                            & Empty match\\
  & & pat\texttt{.}exp\ \texttt{|}\ match               & Pattern -> expression\\
  pat & ::= & var                                       & Variable\\
  & & con\texttt{(}pat_1\texttt{,} \ldots\texttt{,} pat_n\texttt{)} & Where con
  has arity $n$\\
  exp & ::= & var                                       & Variable\\
  & & exp_1 exp_2                                     & Application\\
  & & \fun match                                  & Abstraction\\
  & & con\texttt{(}exp_1\texttt{,} \ldots\texttt{,} exp_n\texttt{)} & Where con
  has arity $n$\\
  dec & ::= & var \mapsto exp                         & Plain value binding\\
  & & \rec var \mapsto \fun match         & Recursive value binding\\
  & & con \texttt{:} \overline{num}                           & Constructor of arity $num$\\
  & & \epsilon                             & Empty program\\
  & & dec_1 \texttt{;} dec_2
\end{eqnarray*}

No pattern may contain a given variable more than once.

\paragraph{Note.} In constructor declarations the arity is overlined to enhance
the difference between numerals and their representation. In practice we write
the syntactic representation of a numeral as the numeral set in typewriter
font.

We use (with superfixes, subfixes and primes) $v$, $c$, $n$, $m$, $p$, $e$
and $d$ to range over $var$, $con$, $num$, $match$, $pat$, $exp$ and $dec$
respectively.

\section{A note about evaluation}
We expect programs to be run in an environment containing predefined functions
(that is variables bound to predefined functions) and constructors. Thus the
program
\begin{quote}
% \begin{verbatim}
\ttt{x $\mapsto$ plus (pair (1, 8))}
% \end{verbatim}
\end{quote}
might make perfect sense (if in particular \ttt{plus} is a variable bound to a
suitable function (perhaps addition), and \texttt{pair}, \texttt{1} and
\texttt{8} are constructors of arity 2, 0 and 0, respectively).
\section{Auxiliary definitions}
\label{sec:auxil-defin}

In the following we define what we mean by equivalence of patterns (with a
permutation of variables), free variables (for expressions, matches and
patterns), substitution (in expressions) and alpha equivalence (of expressions).

\paragraph{Note.}
\begin{enumerate}
\item
If $f : A -> B$ and $g : A -> B$ are arbitrary mappings then
\begin{eqnarray*}[rlqTl]
  (g ++ f)(x) &= f(x) & if $x \in \Dom(f)$\\
  (g ++ f)(x) &= g(x) & otherwise
\end{eqnarray*}
and
\[
  \Dom (g ++ f) = \Dom (g) \cup \Dom (f).
\]

\item
We write $p \sqsubseteq p'$ to mean that $p$ is a subpattern of $p'$. More
precisely this is the case if $p = p'$ or if $p' = c \ttt{(} p_1 \ttt{,} \ldots
\ttt{,} p_n \ttt{)}$ and $p \subseteq p_i$ for some $i \in \{1, \ldots, n\}$.

In particular we have\footnote{See \fref{sec:free-variables}
  for the definition of $\FV_{pat}$.} $x \sqsubseteq p$ exactly when $x \in
\FV_{pat}(p)$. The relation is obviously reflexive.

\begin{example}
  \label{ex:suppattern1}
  Recall that lowercase identifiers are variables, and uppercase ones are
  constructors. Variables as subpatterns:
  \begin{eqnarray*}
    \ttt{x} \sqsubseteq \ttt{A(x,y)} \qquad
    \ttt{y} \sqsubseteq \ttt{A(x,y)} \qquad
    \ttt{z} \not \sqsubseteq \ttt{A(x,y)}
  \end{eqnarray*}
  Patterns as subpatterns:
  \begin{eqnarray*}
    \ttt{A(x,y)} \sqsubseteq \ttt{B(A(x,y),z)} \qquad
    \ttt{A(x,y)} \not \sqsubseteq \ttt{A(A(a,b),c)} \qquad
  \end{eqnarray*}
\end{example}

\item
The syntactic category $pat$ is a proper subset of $exp$. Let $\kappa : pat ->
exp$ be the canonical mapping from $pat$ to $exp$. It is injective so it has a
left inverse $\kappa^{-1} : exp -> pat$. $\kappa^{-1}$ is clearly not total.
\end{enumerate}

\subsection{Equivalence of patterns}
\label{sec:equivalence-patterns}
We say that two patterns are equivalent if they can be transformed into
each other by a suitable renaming of the variables.

If $p_1$ and $p_2$ are equivalent we write $==_\pi$ where $\pi$ is a permutation
of variables, such that for each variable $x$ in $p_1$ its counterpart in $p_2$
is $\pi(x)$.

For example we have $A(x,y) ==_\pi A(z,x)$ where $\pi = [x \mapsto z, y \mapsto x]$.

\fixme{maybe: show that sigma is a permutation of variables}

\begin{definition}[Equivalence of patterns, $==_\pi$]
\label{def:equivalence-patterns}
  \begin{eqnarray}[rlqTl]
    v_1 &==_{\pi} v_2  & where $\pi = [v_1 \mapsto v_2]$ \label{eq:struct-eq-var} \\
    c\ttt{(}p^1_1 \ttt{,} \ldots \ttt{,} p^1_n \ttt{)} & ==_{\pi}
    c\ttt{(}p^2_1 \ttt{,} \ldots \ttt{,} p^2_n \ttt{)} & \label{eq:struct-eq-con}
  \end{eqnarray}
where \fref{eq:struct-eq-con} holds if
\begin{eqnarray*}[c]
  p^1_1 ==_{\pi_1} p^2_1 \\
  \vdots \\
  p^1_n ==_{\pi_n} p^2_n
\end{eqnarray*}
and $\pi = \pi_1 ++ \ldots ++ \pi_n$.

Note that the domains of each of the $\sigma$s are disjoint because no variable
can occur more than once in a pattern (by the definition of the syntax).

We write $==$ to mean $==_\pi$ (with a suitable non-fixed $\pi$) where $\pi$ has no
interest. Equivalence of patterns is defined by $==_\pi$, \emph{only} when $\pi$
is not fixed.
\end{definition}

\begin{example}
  \label{ex:pattern-equiv1}
  We have
  \begin{eqnarray*}
    \ttt{A(x,y)} &==_{\pi}& \ttt{A(f,g)}\\
    \ttt{A(f,g)} &==_{\pi'}& \ttt{A(x,y)}
  \end{eqnarray*}
  with
  \begin{eqnarray*}
    \pi &=& [x\mapsto f, y \mapsto g] \\
    \pi' &=& [f \mapsto x, g \mapsto y]
  \end{eqnarray*}
  but not
  \begin{eqnarray*}
    \ttt{A(x,y)} &\not ==& \ttt{B(h,j)} \\
    \ttt{1} &\not ==& \ttt{2}
  \end{eqnarray*}
\end{example}

\fixme{Show equivalence relation: reflexive, symmetric and transitive.}
\fixme{maybe: show $p_1 ==_\pi p_2$ iff $p_2 ==_{\pi^{-1}} p_1$.}


\subsection{Free variables}\label{sec:free-variables}

We denote the free variables of expressions, matches and patterns with the tree
functions $\FV_{exp}$, $\FV_{match}$ and $\FV_{pat}$, respectively.

\begin{definition}[Free variables of expressions, $\FV_{exp}$] \ \\
  Inductively defined:
  \begin{eqnarray}
    \FV_{exp} (v) &=& \{v\} \\
    \FV_{exp} (\fun m) &=& \FV_{match} (m) \\
    \FV_{exp} (e_1e_2) &=& \FV_{exp} (e_1) \cup \FV_{exp} (e_2) \\
    \FV_{exp} (c\ttt{(}e_1\ttt{,} \ldots \ttt{,} e_n \ttt{)}) &=& \FV_{exp}
    (e_1) \cup \ldots \cup \FV_{exp} (e_n)
  \end{eqnarray}
\end{definition}

\begin{definition}[Free variables of matches, $\FV_{match}$]\ \\
  Inductively defined:
  \begin{eqnarray}
    \FV_{match} (\epsilon) &=& \emptyset \\
    \FV_{match} (p\ttt{.}e\ \ttt{|}\ m) &=& \left( \FV_{exp}(e) \setminus
      \FV_{pat}(p) \right) \cup \FV_{match} (m)
  \end{eqnarray}
\end{definition}

\begin{definition}[Free variables of patterns, $\FV_{pat}$] \ \\
  Inductively defined:
  \begin{eqnarray}
    \FV_{pat} (v) &=& \{v\} \\
    \FV_{pat} (c\ttt{(}p_1\ttt{,} \ldots \ttt{,} p_n\ttt{)}) &=& \FV_{pat} (p_1)
    \cup \ldots \cup \FV_{pat} (p_n)
  \end{eqnarray}
\end{definition}

\begin{example}
\label{ex:free-variables1}
\begin{eqnarray*}[c]
  \FV_{exp} \left(
    \begin{eqnalign}[Tl]
\begin{lstlisting}
fun Nil . Nil
  | Cons (x, xs) . Cons(f x, g xs)
\end{lstlisting}
    \end{eqnalign}
  \right) = \{\ttt{f}, \ttt{g} \} \\
%
  \FV_{match} \left(
    \begin{eqnalign}[Tl]
\begin{lstlisting}
Cons (x, Nil) . Cons(x, y)
\end{lstlisting}
    \end{eqnalign}
  \right) = \{\ttt{y}\} \\
%
  \FV_{pat} \left(
    \begin{eqnalign}[Tl]
\begin{lstlisting}
Cons (x, xs)
\end{lstlisting}
    \end{eqnalign}
  \right) = \{\ttt{x}, \ttt{xs}\} \\
\end{eqnarray*}
\end{example}

\subsection{Substitution}
We define substitution in expressions. An expression can be substituted for any
(sub)expression of an expressions, not just variables.

\begin{definition}[Substitution]
  If $e_1$, $e_2$ and $e_3$ are expressions we write $e_1[e_2/e_3]$ to be the
  result of substituting all occurrences of $e_3$ in $e_1$ with $e_2$.
  \begin{eqnarray}
    e_1[e_2/e_3] &=& e_2 \quad \mrm{if}\ e_1 = e_3 \label{eq:subst-sub}\\
    (e^1_1 e^2_1)[e_2/e_3] &=& e^1_1[e_2/e_3] e^2_1[e_2/e_3] \label{eq:subst-app}\\
    \fun p_1 \texttt{.} e_1 \texttt{|} m &=& \fun p_1 \texttt{.} e'_1
    \texttt{|} m' \label{eq:subst-lam}\\
    (c \texttt{(}e^1_1 \texttt{,} \ldots \texttt{,} e^1_n \texttt{)})[e_2/e_3]
    &=& c \texttt{(}e^1_1[e_2/e_3] \texttt{,} \ldots \texttt{,} e^1_n[e_2/e_3]
    \texttt{)} \label{eq:subst-con}
  \end{eqnarray}
Where in \fref{eq:subst-lam} we have
\begin{eqnarray*}[rlqTl]
  e'_1 &= e_1 & if $\kappa^{-1}(e_3) \subseteq p_1$\\
  e'_1 &= e_1[e_2/e_3]
\end{eqnarray*}
and
\[
(\fun m)[e_2/e_3] = \fun m'
\]

In \fref[plain]{eq:subst-app}, \fref[plain]{eq:subst-lam} and
\fref{eq:subst-con}  we require that \fref[plain]{eq:subst-sub} does not apply.

Note that in \fref[plain]{eq:subst-sub} we require $e_1$ and $e_3$ to be exactly
equal, not just alpha equivalent.
\end{definition}

\begin{example}
\label{ex:substituation1}
We have
\begin{eqnarray*}[c]
\left(
  \begin{eqnalign}[Tl]
\begin{lstlisting}
fun x . x
  | y . x
\end{lstlisting}
  \end{eqnalign}
\right) \left[ \ttt{z}/\ttt{x} \right] \quad = \quad
  \begin{eqnalign}[Tl]
\begin{lstlisting}
fun x . x
  | y . z
\end{lstlisting}
  \end{eqnalign} \\
%
\left(
  \begin{eqnalign}[Tl]
\begin{lstlisting}
fun x . Cons (1, x)
\end{lstlisting}
  \end{eqnalign}
\right) \left[ \ttt{z}/\ttt{Cons (1, x)} \right] \quad = \quad
  \begin{eqnalign}[Tl]
\begin{lstlisting}
fun x . z
\end{lstlisting}
  \end{eqnalign} \\
\end{eqnarray*}

but not

\begin{eqnarray*}
\left(
  \begin{eqnalign}[Tl]
\begin{lstlisting}
fun x . x
\end{lstlisting}
  \end{eqnalign}
\right) [ \ttt{z}/\ttt{x} ] \quad \neq \quad
  \begin{eqnalign}[Tl]
\begin{lstlisting}
fun z . z
\end{lstlisting}
  \end{eqnalign}
\end{eqnarray*}
\end{example}

\subsection{Alpha equivalence}
\label{sec:alpha-equivalence}

We define alpha equivalence for expressions and for pattern-expression pairs. We
use the symbol $==a$ for both relations.

\begin{definition}[Alpha equivalence of expressions, $==a$]
\label{def:alpha-equivalence}
  First we define alpha equivalence given a mapping of bound variables:
  \begin{eqnarray}
    \sigma |- v_1 &==a& v_2 \label{eq:alpha-var} \\
    \sigma |- e^1_1e^1_2 &==a& e^2_1e^2_1 \label{eq:alpha-exp} \\
    \sigma |- \fun p^1_1 \texttt{.} e^1_1 \texttt{|} \ldots \texttt{|} p^1_n
    \texttt{.} e^1_n &==a& \fun p^2_1 \texttt{.} e^2_1 \texttt{|} \ldots \texttt{|} p^2_n
    \texttt{.} e^2_n \label{eq:alpha-match} \\
    \sigma |- c\ttt{(}e^1_1 \ttt{,} \ldots \ttt{,} e^1_n \ttt{)} &==a&
    c\ttt{(}e^2_1 \ttt{,} \ldots \ttt{,} e^2_n \ttt{)} \label{eq:alpha-con}
  \end{eqnarray}
where \fref{eq:alpha-var} holds
\begin{eqnarray*}[rlqTl]
\sigma (v_1) &= v_2 & if $v_1 \in \Dom(\sigma)$\\
v_1 &= v_2 & otherwise,
\end{eqnarray*}
\fref{eq:alpha-exp} holds if
\[
\sigma |- e^1_1 ==a e^2_1 \land \sigma |- e^1_2 ==a e^2_1,
\]
\fref{eq:alpha-match} holds if
\begin{eqnarray*}
  (p^1_1, e^1_1) &==a& (p^2_1, e^2_1)\\
  &\vdots&\\
  (p^1_n, e^1_n) &==a& (p^2_n, e^2_n)
\end{eqnarray*}
and \fref{eq:alpha-con} holds if
\[
\sigma |- e^1_1 ==a e^1_n \land \ldots \land \sigma |- e^2_1 ==a e^2_n.
\]

\begin{definition}[Alpha equivalence of pattern-expression pairs, $==a$]\label{def:alpha-equivalence-patexp}
  Again we assume a mapping of bound variables. It is the case that
  \[
  \sigma |- (p_1, e_1) ==a (p_2, e_2)
  \]
  exactly when $p_1 ==_\pi p_2$ and
  \[
  \sigma ++ \pi |- e_1 ==a e_2
  \]
\end{definition}

If $e_1$ and $e_2$ are alpha equivalent expressions we write $e_1 ==a e_2$ which
is a shorthand for $[] |- e_1 ==a e_2$. Similarly for pattern-expression pairs.
\end{definition}

\begin{example}
\label{ex:alpha-equivalence1}

We have
\begin{eqnarray*}[c]
  \begin{eqnalign}[Tl]
\begin{lstlisting}
fun Nil . Nil
  | Cons (x, xs) . Cons(f x, g xs)
\end{lstlisting}
    \end{eqnalign}
  ==a
  \begin{eqnalign}[Tl]
\begin{lstlisting}
fun Nil . Nil
  | Cons (y, ys) . Cons(f y, g ys)
\end{lstlisting}
    \end{eqnalign}
  \end{eqnarray*}

but not (as the free variables are not the same)
\begin{eqnarray*}[c]
  \begin{eqnalign}[Tl]
\begin{lstlisting}
fun Nil . Nil
  | Cons (x, xs) . Cons(f x, g xs)
\end{lstlisting}
    \end{eqnalign}
  \not ==a
  \begin{eqnalign}[Tl]
\begin{lstlisting}
fun Nil . Nil
  | Cons (y, ys) . Cons(h y, j ys)
\end{lstlisting}
    \end{eqnalign}
  \end{eqnarray*}
\end{example}

\section{Semantic equivalence}
\label{sec:semantic-equivalence}
We write $e_1 \sim e_2$ if $e_1$ and $e_2$ are semantically equivalent. That is
if $e_2$ is substituted for $e_1$ (or vice versa) in any program $d$ to obtain
$d'$, then if $d$ evaluates to something in an environment $\sigma$ then $d'$
evaluates to that something in $\sigma$, and if $d$ diverges in $\sigma$ so does
$d'$.

\section{Orderings on patterns}
\label{sec:orderings-patterns}
We define a total ($<=$) relation on patterns, and a partial ($<='$) relation on
the quotient set of patterns by structural equivalence. Then we show that they
indeed are orderings.

We write $<$ and $<'$ for $<=$ and $<='$ strict (or irreflexive) counterparts
respectively.

As it turns out it is easier to define $<$ and $<='$ directly and then define
$<=$ and $<'$ in turn of those.

\begin{definition}[Strict total ordering, $<$]\label{def:pat-total-order-strict}
  Assume a total strict ordering $\lessdot$ on constructors and
  variables\footnote{For example let all constructors come before all variables
    and let variables and constructors be ordered lexicographically among
    themselves. Note that constructors and variables are not compared in the
    definition of $<$.}. We inductively define:
  \begin{eqnarray}
    v_1 &<& v_2 \quad \mrm{if}\ v_1 \lessdot v_2\label{eq:pat-total-order-strict-var}\\
    c\texttt{(}p_1\texttt{,} \ldots\texttt{,} p_n\texttt{)} &<& v\\
    c_1\texttt{(}p_1\texttt{,} \ldots\texttt{,} p_n\texttt{)} &<&
    c_2\texttt{(}p'_1\texttt{,} \ldots\texttt{,} p'_m\texttt{)}\label{eq:pat-total-order-strict-con}
  \end{eqnarray}
  Where \fref{eq:pat-total-order-strict-con} holds if
  \[
  c_1 \lessdot c_2 \lor (c_1 = c_2 \land ( p_1 < p'_1 \lor p_1 = p'_1 \land (\ldots p_n < p'_n \ldots )))
  \]
\end{definition}

\begin{definition}[Total ordering, $<=$]\label{def:pat-total-order-weak}
  We define the reflexive cousin:
  \begin{eqnarray*}
    p_1 <= p_2 \Longleftrightarrow p_1 < p_2 \lor p_1 = p_2
  \end{eqnarray*}
\end{definition}

\begin{definition}[Partial ordering, $<='$]\label{def:pat-partial-order-weak}
  We say that $p_2$ weakly generalises $p_1$ or $p_1$ is at least as specific as
  $p_2$ and we write $p_1 <=' p_2$. Inductively defined.
  \begin{eqnarray}
    p &<='& v \label{eq:pat-partial-order-weak-var}\\
    c_1\texttt{(}p_1\texttt{,} \ldots\texttt{,} p_n\texttt{)} &<='&
    c_2\texttt{(}p'_1\texttt{,} \ldots\texttt{,} p'_m\texttt{)}\label{eq:pat-partial-order-weak-con}
  \end{eqnarray}
  Where \fref{eq:pat-partial-order-weak-con} holds if
  \begin{eqnarray*}
    c_1 &=& c_2 \quad \land\\
    p_1 &<='& p'_1 \quad \land\\
    &\ldots&\\
    p_n &<='& p'_n
  \end{eqnarray*}
\end{definition}


\begin{definition}[Strict partial ordering, $<'$]\label{def:pat-partial-order-strict}
  We define the strict counterpart of $<='$ by
  \begin{eqnarray*}
      p_1 <' p_2 \Longleftrightarrow p_1 <=' p_2 \land p_1 \not == p_2
  \end{eqnarray*}
\end{definition}

\begin{lemma}[Total ordering]\label{lem:pat-total-orderings}
  The relation $<=$ is a total ordering, and $<$ is a strict total ordering on
  patterns.

  Proof is given in \fref{sec:proof-total-orderings}
\end{lemma}


\begin{lemma}[Partial ordering]\label{lem:pat-partial-orderings}
  The relation $<='$ is a partial ordering and $<'$ is a strict partial ordering
  on the equivalence classes of patterns modulo structural equivalence
  ($pat_{/_{==}}$).

  Proof is given in \fref{sec:proof-partial-orderings}
\end{lemma}

We write $p_1 > p_2$, $p_1 >= p_2$, $p_1 >' p_2$ and $p_1 >=' p_2$ to mean $p_2
< p_1$, $p_2 <= p_1$, $p_2 <' p_1$ and $p_2 <=' p_1$ respectively.

\begin{lemma}[]\label{lem:total-implies-partial}
  If two patterns $p_1$ and $p_2$ are ordered by the partial ordering then they
  are also ordered by the total one. That is
  \begin{eqnarray*}
    p_1 <' p_2 \Longrightarrow p_1 < p_2
  \end{eqnarray*}
\end{lemma}
\begin{proof}
  Straightforward using induction.
\end{proof}

\begin{definition}[Confusion, $||$]\label{def:pat-confusion}
  Let two patterns $p_1$ and $p_2$ be given. If it is the case that neither $p_1
  <=' p_2$ nor $p_1 >=' p_2$ we say that $p_1$ and $p_2$ are confused and we
  write $p_1 || p_2$.
\end{definition}

\begin{lemma}[Unique relation]\label{lem:unique-rel}
  Given two patterns $p_1$ and $p_2$ exactly one of the following hold
  \begin{eqnarray*}
    p_1 &==& p_2\\
    p_1 &<'& p_2\\
    p_1 &>'& p_2\\
    p_1 &||& p_2
  \end{eqnarray*}
\end{lemma}
\begin{proof}
  Immediately by inspection.
\end{proof}

\begin{lemma}[]\label{lem:more-specific-confused}
  If $p_1 <' p_2$ and $p_2 || p_3$, then $p_1 || p_3$.

  Proof is given in \fref{sec:proof-partial-orderings}
\end{lemma}

\begin{example}
  \begin{eqnarray*}[c]
    \begin{eqnalign}[Tl]
\begin{lstlisting}
Cons (y, z)
\end{lstlisting}
    \end{eqnalign}
    <
    \begin{eqnalign}[Tl]
\begin{lstlisting}
Cons (x, z)
\end{lstlisting}
    \end{eqnalign}
    <
    \begin{eqnalign}[Tl]
\begin{lstlisting}
Snoc (a, b)
\end{lstlisting}
    \end{eqnalign}
    <
    \begin{eqnalign}[Tl]
\begin{lstlisting}
a
\end{lstlisting}
    \end{eqnalign}
    <
    \begin{eqnalign}[Tl]
\begin{lstlisting}
b
\end{lstlisting}
    \end{eqnalign}
  \end{eqnarray*}
\end{example}

\begin{example}
  \begin{eqnarray*}[c]
    \begin{eqnalign}[Tl]
\begin{lstlisting}
Cons (y, Nil)
\end{lstlisting}
    \end{eqnalign}
    <'
    \begin{eqnalign}[Tl]
\begin{lstlisting}
Cons (x, z)
\end{lstlisting}
    \end{eqnalign}
    <'
    \begin{eqnalign}[Tl]
\begin{lstlisting}
b
\end{lstlisting}
    \end{eqnalign}
  \end{eqnarray*}
Note: Keep in mind that the exact names of the variables are unimportant.
\end{example}

\section{Eliminating unused patterns}
A function is simply a match. And a match is a list of pairs of patterns and
corresponding bodys.

The input to a function is tried against the patterns from top to bottom. An
unused pattern is a pattern that will never see a value which it matches.

This can happen for two reasons.
\begin{enumerate}
\item The pattern will never be tried against the input because the input
  matches an earlier pattern. \label{item:unused-reason-1}
\item The pattern is only tried against inputs it doesn't match. \label{item:unused-reason-2}
\end{enumerate}

\subsection{Cover}\label{sec:cover}
We define a cover to be a set of patterns such that for every input at least one
of the patterns will match that input. And we write $Cov(P)$ if $P$ is a cover.

Any pattern following patterns that taken together are a cover is unused because
of reason \ref{item:unused-reason-1}.

\subsection{Shadowed patterns}\label{sec:shadowed-patterns}
If a pattern is unused because of reason \ref{item:unused-reason-2} we say that
it is shadowed.
\begin{definition}[Shadowed]
  Let
  \[
  m = p_1\texttt{.}e_1 \texttt{|} \ldots \texttt{|} p_n\texttt{.}e_n
  \]
  If $p_j <=' p_i$ for some $1 \leq i < j \leq n$, then $p_j$ is shadowed (by
  $p_i$).
\end{definition}

We can now define the elimination of unused patterns.
\begin{definition}[Elimination, $->e$]
\label{def:shadowed-patterns-1}
  We define a reduction relation $->e$ that expresses the
  elimination of exactly one pattern from a match.

  Let
  \[
  m = p_1\texttt{.}e_1 \texttt{|} \ldots \texttt{|} p_n\texttt{.}e_n
  \]
  If there exist a $p_i$ such that $\{p_1, \ldots, p_{i-1}\}$ is a cover or
  $p_i$ is shadowed, then it is unused and can be eliminated. The resulting
  match is
  \[
  m' = p_1\texttt{.}e_1 \texttt{|} \ldots \texttt{|}
  p_{j-1}\texttt{.}e_{j-1} \texttt{|} p_{j+1}\texttt{.}e_{j+1} \texttt{|}
  \ldots \texttt{|} p_n\texttt{.}e_n,
  \]
  and we write $m ->e m'$.

\end{definition}

\begin{lemma}(Preservation)
  If an unused pattern is removed from a program, then the resulting program is
  semantically equivalent.

  That is
  \[
  m ->e m' ==> \fun m \sim \fun m'
  \]
\end{lemma}

\begin{proof}
  Trivial (as if).
\end{proof}

\begin{example}
  The first two patterns make a cover (assuming the only constructors are
  \ttt{Cons} and \ttt{Nil}) so the last pattern is eliminated.
  \begin{eqnarray*}[c]
    \begin{eqnalign}[Tl]
\begin{lstlisting}
  Cons(x, xs) . Cons (x, xs)
| Nil . Nil
| x . x
\end{lstlisting}
    \end{eqnalign}
    ->e
    \begin{eqnalign}[Tl]
\begin{lstlisting}
  Cons(x, xs) . Cons (x, xs)
| Nil . Nil
\end{lstlisting}
    \end{eqnalign}
  \end{eqnarray*}

  The second pattern in the example below is shadowed by the first pattern and is thus eliminated.
  \begin{eqnarray*}[c]
    \begin{eqnalign}[Tl]
\begin{lstlisting}
  Cons (x, y) . Cons (x, y)
| Cons (Cons (x, y), z) . Cons (Cons (x, y), z)
| Nil
\end{lstlisting}
    \end{eqnalign}
    ->e
    \begin{eqnalign}[Tl]
\begin{lstlisting}
  Cons (x, y) . Cons (x, y)
| Nil
\end{lstlisting}
    \end{eqnalign}
  \end{eqnarray*}

\end{example}

\section{Generalisiation}
Sometimes patterns get unnecessary complex. If for example a pattern (or one of
its subpatterns) is a constructor pattern whose subpatterns are all variables,
and those variables are only used as arguments to the same constructor (in the
same order) in the function body, then the constructor could simply be replaced
by a fresh variable in pattern and body. That is generalisation of the pattern.

Sometimes the generalisation of a pattern makes it equivalent to another pattern
in the match. And sometimes the two patterns corresponding bodys will merge
seamlessly, such that two patterns can be made to one.
\\[1em]
First we need some auxiliary definitions.

\subsection{Partially ordered form}
\begin{definition}\label{def:part-order-form}
  A match $m = p_1\texttt{.}e_1\texttt{|}\ldots\texttt{|}p_n\texttt{.}e_n$ is in
  partially ordered form if
  \[
  \forall i \in \{1, \ldots, n\} : p_j \not <=' p_i \quad \textnormal{where $j > i$}
  \]
  Note that every match $m$ can be transformed to an equivalent match $m'$ such
  that $m'$ is in partially ordered form, by repeated elimination of shadowed
  patterns (\fref{sec:shadowed-patterns}).
\end{definition}

\subsection{Generalisation of patterns}
We define generalisation of a single pair of a pattern and its body. We write
$(p, e) |> (p', e')$ to mean that the pattern $p$ with its body $e$ generalises
to the pattern $p'$ with the body $e'$.

\begin{definition}[Generalisation of single pattern-body pairs, $|>$]
\label{def:gener-patt}
Inductively defined:
\begin{eqnarray}
  (c \texttt{(} p_1 \texttt{,} \ldots \texttt{,} p_n \texttt{)} , e) &|>& (x , e[x
  / \kappa (c \texttt{(} p_1 \texttt{,} \ldots \texttt{,} p_n \texttt{)} )]) \label{eq:single-gen-1}\\
  (c \texttt{(} p_1 \texttt{,} \ldots \texttt{,} p_i \texttt{,} \ldots
  \texttt{,} p_n \texttt{)}, e) &|>&
  (c \texttt{(} p_1 \texttt{,} \ldots \texttt{,} p'_i \texttt{,} \ldots
  \texttt{,} p_n \texttt{)}, e') \label{eq:single-gen-2}
\end{eqnarray}
Where \fref{eq:single-gen-1} holds when $x$ is a fresh variable and
\[
FV_{pat}(c\texttt{(}p_1\texttt{,}\ldots\texttt{,}p_n\texttt{)}) \cap FV_{exp}(e[x/\kappa
(c\texttt{(}p_1\texttt{,}\ldots\texttt{,}p_n\texttt{)})]) = \emptyset
\]
and \fref{eq:single-gen-2} holds when \fref{eq:single-gen-1} does not and
\[
(p_i , e) |> (p'_i , e')
\]
\end{definition}

\begin{lemma}\label{lem:single-gen-imp-gen}
  If a pattern $p$ (and some body) is generalised to $p'$ (and some other
  body), then $p'$ strictly generalises $p$. In other words
  \[
  (p, e) |> (p', e') ==> p <' p'.
  \]
\end{lemma}
\begin{proof}
  Straightforward induction proof.
\end{proof}

\begin{example}
  This example shows a match with an unnecessary complex pattern that reverses
  the components of a pair.

  \begin{eqnarray*}[c]
    \begin{eqnalign}[Tl]
\begin{lstlisting}
Pair (Cons (x, xs), zs) . Pair (zs, Cons (x, xs))
\end{lstlisting}
    \end{eqnalign}
    |>
    \begin{eqnalign}[Tl]
\begin{lstlisting}
Pair (xs, zs) . Pair (zs, xs)
\end{lstlisting}
    \end{eqnalign}
  \end{eqnarray*}
\end{example}

\subsection{Generalisation of matches}
For the generalisation of a match $m$, we require $m$ to be in partially ordered
form.
\\[1em]
When generalising a pattern several things might happen. Assume
\begin{eqnarray*}[rqTcql]
  m = p_1 \texttt{.} e_1 \texttt{|} \ldots \texttt{|} p_i \texttt{.} e_i
  \texttt{|} \ldots \texttt{|} p_n \texttt{.} e_n & and & (p_i, e_i) |> (p'_i,
  e'_i).
\end{eqnarray*}
Now, perhaps $m$ can be generalised if we substitute $p'_i$ for $p_i$ and $e'_i$
for $e_i$. We would like the resulting match to be partially ordered too, so we
must be cautious. Since we know from \fref{lem:single-gen-imp-gen} that $p'_i >'
p_i$ the first part of the match $p_1 \texttt{.} e_1 \texttt{|} \ldots
\texttt{|} p'_i \texttt{.} e'_i$ must still be partially ordered. So we consider
the patterns $p_j$ for $j > i$.

Note that by \fref{lem:unique-rel} we know that either $p_i <' p_j$ or $p_i ||
p_j$.

Four scenarios arise
\begin{enumerate}
\item $p'_i$ and $p_j$ are equivalent. The pattern was generalised to one that
  already existed. Now the only hope is that $e'_i$ and $e_j$ merge. By this we
  mean $(p'_i, e'_i) ==a (p_j, e_j)$ (\Fref{def:alpha-equivalence-patexp}).

  If this is the case then either one of $p'_i \texttt{.} e'_i$ or $p_j
  \texttt{.} e_j$ shall be removed. \label{item:gen-scen-1}
\item $p'_i$ relates to $p_j$ in the same way that $p_i$ does. So $p_i <' p_j
  \Rightarrow p'_i <' p_j$ and $p_i || p_j \Rightarrow p'_i || p_j$. In this
  case nothing must be done. \label{item:gen-scen-2}
\item $p_i || p_j$ and $p'_i >' p_j$. So now $p'_i$ ``steals'' $p_j$s input. But
  because $p_i$ and $p_j$ were confused we know that $p_j$ will not steal any
  input originally intended for $p_i$. So we move $p_j$ and its body up, in the
  match so they come before $p'_i$. \label{item:gen-scen-3}
\item $p_i <' p_j$ and $p'_i >' p_j$. This means that $p'_i$ will match input
  intended for $p_j$ but we can not move $p_j$ above $p'_i$ for then it will
  steal input originally intended for $p_i$. So in this case $m$ can not be
  generalised. \label{item:gen-scen-4}
\end{enumerate}

\begin{definition}[Generalisation, $->g$]
\label{def:gener-match}
  We define a reduction relation $->g$ that expresses the
  generalisation of exactly one pattern from a match.

  Let
  \begin{eqnarray*}[rqTcql]
    m = p_1 \texttt{.} e_1 \texttt{|} \ldots \texttt{|} p_i \texttt{.} e_i
    \texttt{|} \ldots \texttt{|} p_n \texttt{.} e_n & and & (p_i, e_i) |> (p'_i,
    e'_i).
  \end{eqnarray*}
  and assume that a generalisation as described above can be done. Then the
  resulting match is
  \begin{eqnarray*}[rclqqqTl]
    m' &=& p_1 \texttt{.} e_1 \texttt{|} \ldots \texttt{|} p_{i-1} \texttt{.}
    e_{i-1} & (Untouched)\label{eq:gen-1}\\
    &\texttt{|}& p_{m_1} \texttt{.} e_{m_1} \texttt{|} \ldots \texttt{|} p_{m_k}
    \texttt{.} e_{m_k} & (Case \ref{item:gen-scen-3})\label{eq:gen-2}\\
    (&\texttt{|}& p'_i \texttt{.} e'_i \ ) & (Perhaps case \ref{item:gen-scen-1})\label{eq:gen-3}\\
    &\texttt{|}& p_{s_1} \texttt{.} e_{s_1} \texttt{|} \ldots \texttt{|} p_{s_l}
    \texttt{.} e_{s_l} & (Case \ref{item:gen-scen-2})\label{eq:gen-4}
  \end{eqnarray*}

  Where $m_1 < \ldots < m_k$ and $s_1 < \ldots < s_l$.

  The third line is put in parentheses because it should be deleted in the case
  of scenario \ref{item:gen-scen-1}.

  And we write $m ->g m'$.

  \begin{lemma}
    If $m ->g m'$ then $m'$ is in partially ordered form.
  \end{lemma}
  \begin{proof}
    Assume
    \begin{eqnarray*}[rqTcql]
      m = p_1 \texttt{.} e_1 \texttt{|} \ldots \texttt{|} p_i \texttt{.} e_i
      \texttt{|} \ldots \texttt{|} p_n \texttt{.} e_n & and & (p_i, e_i) |> (p'_i,
      e'_i).
    \end{eqnarray*}

    Since every $p_j <' p'_i$ for $j > i$ is moved in front of $p'_i$ the only
    thing that can break the partial ordering of $m'$ is if $p_j <' p_k$ for
    $i < j < k$ and $p_k$ moves, but $p_j$ does not. For this to happen $p_j$
    must fall into scenario \ref{item:gen-scen-2} and $p_k$ must fall into
    scenario \ref{item:gen-scen-3}.

    Either $p'_i <' p_j$ or $p'_i || p_j$. In the first case we have by
    transitivity that $p'_i <' p_k$, so $p_k$ can not fall into category
    \ref{item:gen-scen-3}, which is a contradiction.

    In the latter case we have by assumption that $p'_i || p_j$ and $p_k <'
    p'_i$. But then we get by lemma \fref{lem:more-specific-confused} that $p_k
    || p_j$, which is a contradiction.
  \end{proof}

\end{definition}

\section{Normal form}
We say that the function $\fun m$ is a normal form if there does not exist an
$m'$ such that $m ->e m'$ or $m ->g m'$.

\subsection{Reducing to normal form}
Consider a function $\fun m$. It can be converted to a normal form by repeatedly
eliminating and generalising patterns.

Note that $m$ should be in partially ordered form (\fref{def:part-order-form}) in
order to generalise its patterns. Luckily we can convert it to a partially
ordered form by repeated elimination.

\begin{lemma}
  If a function $f$ can be converted to the normal form $f'$, then $f$ and $f'$
  are semantically equivalent.
\end{lemma}

\begin{lemma}
  We expect the normal form to be unique (in some sense).
\end{lemma}

\begin{example}
Consider
\begin{lstlisting}
      f (A (A (a, b), c)) . A (c, A (a, b))
    | f (A (a, b))        . A (b, a)
    | f B                 . B
    | f x                 . x
    | f (A (a, A (b, c))  . A (A (c, b), a)
\end{lstlisting}
Assume that the only constructors are \texttt{A} and \texttt{B}.

As the match is not in partially ordered form (due to the last pattern being
more specific than the pattern \texttt{x}) our only choice is to
eliminate. Luckily we can, because the four (three actually) first patterns
forms a cover. With the last clause eliminated we have

\begin{lstlisting}
      f (A (A (a, b), c)) . A (c, A (a, b))
    | f (A (a, b))        . A (b, a)
    | f B                 . B
    | f x                 . x
\end{lstlisting}

From here we can choose to eliminate or generalise. We generalise. It is the
case that
\[
(\texttt{A (A (a, b), c)}, \texttt{A (c, A (a, b))}) |> (\texttt{A (x, y)},
\texttt{A (y, x)})
\]

We arrive at scenario \ref{item:gen-scen-1} since $\texttt{A (a, b)} ==
\texttt{A (x, y)}$. So we must have $(\texttt{A (a, b)}, \texttt{A (b, a)}) ==a
(\texttt{A (x, y)}, \texttt{A (y, x)})$, which is the case. So the first clause
can be deleted. We now have

\begin{lstlisting}
      f (A (a, b)) . A (b, a)
    | f B          . B
    | f x          . x
\end{lstlisting}

Since the two first patterns form a cover the last clause can be eliminated. We
end up with

\begin{lstlisting}
      f (A (a, b)) . A (b, a)
    | f B          . B
\end{lstlisting}

which cannot be generalised or eliminated further, so it is a normal form.
\end{example}


% \section{Weak unification}



%%% Local Variables: 
%%% mode: latex
%%% reftex-fref-is-default: t
%%% TeX-master: "../miniml"
%%% End: 


\chapter{Examples}

\newcommand{\patexp}[2]{#1 \texttt{.} #2}
\newcommand{\fargs}[2]{#1 \texttt{ | } #2}
\newcommand{\cargs}[2]{#1 \texttt{,} #2}
\newcommand{\con}[2]{#1 \texttt{(} #2 \texttt{)}}
\newcommand{\fcall}[2]{#1 \texttt{(} #2 \texttt{)}}
\newcommand{\pipe}{\phantom{\ttt{fu}}\ttt{|}}


% assume nil < ::

% Totally ordered (and thus partially due to lemma \fref{tot-imp-part}):
% fun sum nil = 0
%   | sum (x :: xs) = x + sum xs

% or

% fun sum nil = 0
%   | sum (x :: xs) = x + sum xs
%   | sum _ = raise Fail "Unused pattern"

% Partially but not totally ordered:
% fun sum (x :: xs) = x + sum xs
%   | sum nil = 0

\fixme{Fyld på med tekst}

\section{Auciliary definitions}

\begin{example}[Suppattern, $\subseteq$]
  \label{ex:suppattern1}
  \begin{eqnarray*}
    \ttt{x} &\subseteq& \ttt{A(x,y)} \\
    \ttt{y} &\subseteq& \ttt{A(x,y)}
\end{eqnarray*}
Whereas this is not a suppattern
\begin{eqnarray*}
  \ttt{z} \not \subseteq \ttt{A(x,y)}
\end{eqnarray*}
\end{example}


\begin{example}[Pattern equivalence, $==_{\pi}$]
  \label{ex:pattern-equiv1}
  \begin{eqnarray*}
    \ttt{A(x,y)} &==_{\pi}& \ttt{A(f,g)}\\
    \ttt{A(f,g)} &==_{\pi'}& \ttt{A(x,y)}
  \end{eqnarray*}
with 
  \begin{eqnarray*}
    \pi &=& [x\mapsto f, y \mapsto g] \\
    \pi' &=& [f \mapsto x, g \mapsto y]
\end{eqnarray*}

Whereas these patterns are not equivalent

  \begin{eqnarray*}
    \ttt{A(x,y)} &\not ==& \ttt{B(h,j)} \\
    \ttt{1} &\not ==& \ttt{2}
  \end{eqnarray*}
\end{example}

\begin{example}[Free variables, $\mrm{FV}$]
\label{ex:free-variables1}
\begin{eqnarray*}[c]
  \FV_{exp} \left( 
    \begin{eqnalign}[Tl]
\begin{lstlisting}
fun Nil . Nil
  | Cons (x, xs) . Cons(f x, g xs)
\end{lstlisting}
    \end{eqnalign}
  \right) = \{\ttt{f}, \ttt{g} \} \\
%
  \FV_{match} \left( 
    \begin{eqnalign}[Tl]
\begin{lstlisting}
Cons (Nil, Nil) . Cons(Nil, Nil)
\end{lstlisting}
    \end{eqnalign}
  \right) = \{\} \\
%
  \FV_{pat} \left( 
    \begin{eqnalign}[Tl]
\begin{lstlisting}
Cons (x, xs)
\end{lstlisting}
    \end{eqnalign}
  \right) = \{\ttt{x}, \ttt{xs}\} \\
\end{eqnarray*}
\end{example}


\begin{example}[Substitution]
\label{ex:substituation1}

\begin{eqnarray*}[c]
\left(
  \begin{eqnalign}[Tl]
\begin{lstlisting}
fun x . x
  | y . x
\end{lstlisting}
  \end{eqnalign}
\right) \left[ \ttt{z}/\ttt{x} \right] \quad = \quad
  \begin{eqnalign}[Tl]
\begin{lstlisting}
fun x . x
  | y . z
\end{lstlisting}
  \end{eqnalign} \\
%
\left(
  \begin{eqnalign}[Tl]
\begin{lstlisting}
fun x . Cons (1, x)
\end{lstlisting}
  \end{eqnalign}
\right) \left[ \ttt{z}/\ttt{Cons (1, x)} \right] \quad = \quad
  \begin{eqnalign}[Tl]
\begin{lstlisting}
fun x . z
\end{lstlisting}
  \end{eqnalign} \\
\end{eqnarray*}

whereas this is not

\begin{eqnarray*}
\left(
  \begin{eqnalign}[Tl]
\begin{lstlisting}
fun x . x
\end{lstlisting}
  \end{eqnalign}
\right) [ \ttt{z}/\ttt{x} ] \quad \neq \quad
  \begin{eqnalign}[Tl]
\begin{lstlisting}
fun z . z
\end{lstlisting}
  \end{eqnalign}
\end{eqnarray*}
\end{example}

\begin{example}[Alpha equivalence, $==a$]
\label{ex:alpha-equivalence1}

\begin{eqnarray*}[c]
  \begin{eqnalign}[Tl]
\begin{lstlisting}
fun Nil . Nil
  | Cons (x, xs) . Cons(f x, g xs)
\end{lstlisting}
    \end{eqnalign}
  ==a
  \begin{eqnalign}[Tl]
\begin{lstlisting}
fun Nil . Nil
  | Cons (y, ys) . Cons(f y, g ys)
\end{lstlisting}
    \end{eqnalign}
  \end{eqnarray*}

whereas this is not, as the free variables are not the same

\begin{eqnarray*}[c]
  \begin{eqnalign}[Tl]
\begin{lstlisting}
fun Nil . Nil
  | Cons (x, xs) . Cons(f x, g xs)
\end{lstlisting}
    \end{eqnalign}
  \not ==a
  \begin{eqnalign}[Tl]
\begin{lstlisting}
fun Nil . Nil
  | Cons (y, ys) . Cons(h y, j ys)
\end{lstlisting}
    \end{eqnalign}
  \end{eqnarray*}
\end{example}




\section{Ordering}

\paragraph{Note.} The below examples on orderings are not semantically
equivalent, but shows only how the ordering relation works.


\begin{example}[Strict total ordering, $<$]
\fixme{this is totally wrong but the idea is somewhat ok...}
  Given then following function $foo$
  \begin{eqnarray*}[rl]
    \texttt{rec } foo & \mapsto \fun (b\ttt{.}exp_1 \texttt{ | } a\ttt{.}exp_2 \texttt{ | }
    cons(y, z)\ttt{.}exp_3 \texttt{ | } cons(x, z)\ttt{.}exp_4) \\
\tabpause{then strict total ordering will result in}
    \texttt{rec } foo & \mapsto \fun (cons(x\texttt{,} z)\texttt{.}exp_4 \texttt{ |
    } cons(y\texttt{,} z)\texttt{.}exp_3 \texttt{ | } a\texttt{.}exp_2 \texttt{ | }
    b\texttt{.}exp_1)
\end{eqnarray*}
\end{example}


\begin{example}[Strict partial ordering, $<'$]
%   Given the following function $combine$
%   \begin{eqnarray*}[rll>{\ttt{.}}l]
%     \texttt{rec } combine \mapsto 
%     & \fun & \con{cons}{\cargs{x}{\con{cons}{\cargs{y}{xl}}}} & \fcall{cons}{\cargs{\con{pair}{\cargs{x}{y}}}{combine(xl)}} \\
%       & \pipe & \fcall{cons}{\cargs{xl}{nil}} & \mathrel{nil} \\
%       & \pipe & \mathrel{nil} & \mathrel{nil}\\
% \tabpause{then partial ordering will result in}
%     \texttt{rec } combine \mapsto 
%     & \fun & hej \\
%     & \pipe & ...
%   \end{eqnarray*}
%   \fixme{uheldigt, eftersom ingen af patternsne ordner partielt mellem hinanden
%     grundet aritet?}
\end{example}


\begin{example}[Elimination, $~>e$]

\end{example}


\begin{example}[Generalisation of single pattern-body pairs, $|>$]

\end{example}


\begin{example}[Reduction relation, $~>g$]

\end{example}

%%% Local Variables: 
%%% mode: latex
%%% TeX-master: "../miniml"
%%% End: 


\appendix

\chapter{Symbol table}

\begin{tabular}{| >{$}c<{$} | p{18em} | l|}
  \hline
  \textbf{Symbol} & \textbf{Description} & \textbf{Defined at} \\ \hline
  ++ & Union of two arbitrary mappings. & \Fref{sec:auxil-defin}  \\ \hline
  = & Syntactic equality. & \\ \hline
  ==_\pi & Equivalence of two patterns. & \Fref{def:equivalence-patterns} \\
  \hline
  ==a & Alpha equivalence of two expressions or pattern-expression
  pairs. & \Fref{def:alpha-equivalence} \\ \hline
  \sim & Semantic equivalence of two
  expressions. & \Fref{sec:semantic-equivalence} \\ \hline
  <, \ <=& Total ordering of patterns & \Fref{lem:pat-total-orderings} \\
  \hline
  <', \ <='& Partial ordering of patterns & \Fref{lem:pat-partial-orderings} \\
  \hline
  || & Confusion of two patterns (when they don't partially order) &
  \Fref{def:pat-confusion} \\ \hline
  ~>e & Elimination of one unused pattern from a match. &
  \Fref{def:shadowed-patterns-1} \\ \hline
  |> & Generalisation of a single pattern-expression pair. & \Fref{def:gener-patt} \\ \hline
  ~>g & Generalisation of one pattern in a match. & \Fref{def:gener-match} \\ \hline
\end{tabular}

%%% Local Variables: 
%%% mode: latex
%%% reftex-fref-is-default: t
%%% TeX-master: "../miniml"
%%% End: 



\chapter{Proofs}

\section{Orderings on patterns}

\subsection{Proof of the total ordering relation}
\label{sec:proof-total-orderings}


\begin{proof}[Proof of {\fref[plain]{lem:pat-total-orderings}}]
  We show that $<$ is irreflexive, transitive and total. Then it follows
  immediately by \fref{def:pat-total-order-weak} that $<=$ is a total ordering.
  \begin{description}
  \item[Irreflexivity.]
    Proof by contradiction. So assume $p < p$.

    If $p = v$ we get the contradiction immediately by irreflexivity of
    $\lessdot$.

    Therefore assume $p = c \texttt{(} p_1 \texttt{,} \ldots \texttt{,} p_n
    \texttt{)}$. By the irreflexivity of $\lessdot$ it must be the case that
    \begin{eqnarray*}
      p_1 < p_1 \land \ldots \land p_n < p_n
    \end{eqnarray*}
    which by induction we see can not be the case, and we again have a
    contradiction.

  \item[Transitivity.]
    Assume $p_1 < p_2 \land p_2 < p_3$. We show $p_1 < p_3$.

    If $p_3 = v$ then the result follows immediately. Therefore assume $p_3 = c_3
    \texttt{(}p^3_1 \texttt{,} \ldots \texttt{,} p^3_n\texttt{)}$.
    By the assumptions we have that $p_1 = c_1 \texttt{(}p^1_1 \texttt{,} \ldots
    \texttt{,} p^1_n\texttt{)}$, $p_2 = c_2 \texttt{(}p^2_1 \texttt{,} \ldots
    \texttt{,} p^2_n\texttt{)}$ and
    \begin{eqnarray}[c]
      c_1 < c_2 \lor c_1 = c_2 \land ( p^1_1 < p^2_1 \lor p^1_1 = p^2_1 \land
      (\ldots p^1_n < p^2_n \ldots )) \label{eq:trans-tot-proof-1}\\
      c_2 < c_3 \lor c_2 = c_3 \land ( p^2_1 < p^3_1 \lor p^2_1 = p^3_1 \land
      (\ldots p^2_n < p^3_n \ldots )) \label{eq:trans-tot-proof-2}
    \end{eqnarray}
    Combining \fref{eq:trans-tot-proof-1} and \fref{eq:trans-tot-proof-2} we get
    \begin{eqnarray*}[c]
      c_1 < c_2 \lor c_1 = c_2 \land c_2 < c_3 \lor c_2 = c_3 \land\\ ( p^1_1 < p^2_1
      \lor p^1_1 = p^2_1 \land p^2_1 < p^3_1 \lor p^2_1 = p^3_1 \land
      (\ldots p^1_n < p^2_n \land p^2_n < p^3_n \ldots ))
    \end{eqnarray*}
    And then using induction we get
    \begin{eqnarray*}
      c_1 < c_3 \lor c_1 = c_3 \land ( p^1_1 < p^3_1 \lor p^1_1 = p^3_1 \land
      (\ldots p^1_n < p^3_n \ldots ))
    \end{eqnarray*}
    and then by definition $p_1 < p_3$.

  \item[Totality.]
    Consider patterns $p_1 \neq p_2$. If either one is a variable we have $p_1
    < p_2$ or $p_2 < p_1$ directly by \fref{eq:pat-total-order-strict-con} or by
    the totality of $\lessdot$ and \fref{eq:pat-total-order-strict-var}.

    So assume $p_1 = c_1\texttt{(}p^1_1\texttt{,}\ldots\texttt{,}p^1_n\texttt{)}$
    and $p_2 = c_2\texttt{(}p^2_1\texttt{,}\ldots\texttt{,}p^2_n\texttt{)}$.

    If $c_1 \neq c_2$ we get $p_1 < p_2$ or $p_2 < p_1$ by the totality of
    $\lessdot$ and \fref{eq:pat-total-order-strict-con}.

    So assume $c_1 = c_2$. By induction we have $p^1_i < p^2_i \lor p^2_i <
    p^1_i \lor p^1_i = p^2_i$ for $i \in \{1, \ldots, n\}$. But it must be the
    case that $p^1_i \neq p^2_i$ for some $i$ for else $p_1 = p_2$ and then we
    have a contradiction. This implies that $p_1 < p_2$ (if $p^1_i < p^2_i$) or
    $p_2 < p_1$ (if $p^2_i < p^1_i$).

  \end{description}
\end{proof}

\subsection{Proof of the partial ordering relation}
\label{sec:proof-partial-orderings}

\begin{proof}[Proof of {\fref[plain]{lem:pat-partial-orderings}}]
  We show that $<='$ is reflexive, antisymmetric and transitive. Then we show
  that $<'$ is irreflexive and transitive.
  \begin{description}
  \item[Reflexivity.]
    Assume $p_1 == p_2$. We want to show $p_1 <=' p_2$.

    If either pattern is a variable then clearly so is the other. And then $p_1
    <=' p_2$ follows by \fref{eq:pat-partial-order-weak-var}.

    So assume $p_1 = c_1 \texttt{(} p^1_1 \texttt{,} \ldots \texttt{,} p^1_n
    \texttt{)}$ and $p_2 = c_2 \texttt{(} p^2_1 \texttt{,} \ldots \texttt{,}
    p^2_m \texttt{)}$. It must be the case that $c_1 = c_2$, $n = m$ and $p^1_i
    ==a p^2_i$ for $i \in \{1, \ldots, n\}$.

    It follows by induction that $p^1_1 <=' p^2_1 \land \ldots \land p^1_n <='
    p^2_n$ and then by \fref{eq:pat-partial-order-weak-con} that $p_1 <=' p_2$.

  \item[Antisymmetry.]
    Assume $p_1 <=' p_2$ and $p_2 <=' p_1$. We need to show $p_1 == p_2$.

    If either pattern is a variable then so is the other and structural
    equivalence follows directly.

    So assume that one of the patterns is a constructor pattern. By
    \fref{def:pat-partial-order-weak} it is easy to see that so must the other
    and the constructors must be the same. So $p_1 = c \texttt{(} p^1_1
    \texttt{,} \ldots \texttt{,} p^1_n \texttt{)}$ and $p_2 = c \texttt{(} p^2_1
    \texttt{,} \ldots \texttt{,} p^2_n \texttt{)}$, and furthermore
    \begin{eqnarray*}
      p^1_1 <=' p^2_1 &\land& p^2_1 <=' p^1_1 \quad \land\\
      &\vdots&\\
      p^1_n <=' p^2_n &\land& p^2_n <=' p^1_n
    \end{eqnarray*}
    By induction we have $p^1_1 == p^2_1 \land \ldots \land p^1_n == p^2_n$ and
    then $p_1 == p_2$.

  \item[Transitivity.]
    Assume $p_1 <=' p_2$ and $p_2 <=' p_3$. It must be shown that $p_1 <=' p_3$.

    If $p_3 = v$ the result follows immediately. So assume $p_3 = c_3 \texttt{(}
    p^3_1 \texttt{,} \ldots \texttt{,} p^3_n \texttt{)}$.

    Then clearly $p_1 = c_1 \texttt{(} p^1_1 \texttt{,} \ldots \texttt{,}
    p^1_n\texttt{)}$, $p_2 = c_2 \texttt{(}p^2_1 \texttt{,} \ldots \texttt{,}
    p^2_n\texttt{)}$ and by \fref{eq:pat-partial-order-weak-con} we have
    \begin{eqnarray}[c]
      c_1 = c_2 \land p^1_1 <=' p^2_1 \ldots p^1_n <=' p^2_n \label{eq:pat-partial-order-weak-trans-part-proof-1}\\
      c_2 = c_3 \land p^2_1 <=' p^3_1 \ldots p^2_n <=' p^3_n \label{eq:pat-partial-order-weak-trans-part-proof-2}
    \end{eqnarray}
    Combining \fref{eq:pat-partial-order-weak-trans-part-proof-1} and
    \fref{eq:pat-partial-order-weak-trans-part-proof-2} gives us
    \begin{eqnarray*}
      c_1 = c_2 \land c_2 = c_3 \land p^1 <=' p^2_1 \land p^2_1 <=' p^3_1 \ldots
      p^1_n <=' p^2_n \land p^2_n <=' p^3_n
    \end{eqnarray*}
    And then by induction we get
    \begin{eqnarray*}
      c_1 = c_3 \land p^1_1 <=' p^3_1 \ldots p^1_n <=' p^3_n
    \end{eqnarray*}
    which by \fref{eq:pat-partial-order-weak-con} gives us $p_1 <=' p_3$.

  \end{description}

  Now we show irreflexivity and transitivity of $<'$.
  \begin{description}
  \item[Irreflexivity.]
    Immediately by \fref{def:pat-partial-order-strict}.

  \item[Transitivity.]
    Immediately by the transitivity of $<='$.

  \end{description}
\end{proof}


%%% Local Variables: 
%%% mode: latex
%%% TeX-master: "../../report"
%%% End: 


\end{document}

%%% Local Variables: 
%%% mode: latex
%%% TeX-master: t
%%% reftex-fref-is-default: t
%%% End: 
