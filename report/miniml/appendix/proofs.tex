\section{Orderings on patterns}

\subsection{Proof of the total ordering relation}
\label{sec:proof-total-orderings}


\begin{proof}[Proof of {\fref[plain]{lem:pat-total-orderings}}]
  We show that $<$ is irreflexive, transitive and total. Then it follows
  immediately by \fref{def:pat-total-order-weak} that $<=$ is a total ordering.
  \begin{description}
  \item[Irreflexivity.]
    Proof by contradiction. So assume $p < p$.

    If $p = v$ we get the contradiction immediately by irreflexivity of
    $\lessdot$.

    Therefore assume $p = c \texttt{(} p_1 \texttt{,} \ldots \texttt{,} p_n
    \texttt{)}$. By the irreflexivity of $\lessdot$ it must be the case that
    \begin{eqnarray*}
      p_1 < p_1 \land \ldots \land p_n < p_n
    \end{eqnarray*}
    which by induction we see can not be the case, and we again have a
    contradiction.

  \item[Transitivity.]
    Assume $p_1 < p_2 \land p_2 < p_3$. We show $p_1 < p_3$.

    If $p_3 = v$ then the result follows immediately. Therefore assume $p_3 = c_3
    \texttt{(}p^3_1 \texttt{,} \ldots \texttt{,} p^3_n\texttt{)}$.
    By the assumptions we have that $p_1 = c_1 \texttt{(}p^1_1 \texttt{,} \ldots
    \texttt{,} p^1_n\texttt{)}$, $p_2 = c_2 \texttt{(}p^2_1 \texttt{,} \ldots
    \texttt{,} p^2_n\texttt{)}$ and
    \begin{eqnarray}[c]
      c_1 < c_2 \lor c_1 = c_2 \land ( p^1_1 < p^2_1 \lor p^1_1 = p^2_1 \land
      (\ldots p^1_n < p^2_n \ldots )) \label{eq:trans-tot-proof-1}\\
      c_2 < c_3 \lor c_2 = c_3 \land ( p^2_1 < p^3_1 \lor p^2_1 = p^3_1 \land
      (\ldots p^2_n < p^3_n \ldots )) \label{eq:trans-tot-proof-2}
    \end{eqnarray}
    Combining \fref{eq:trans-tot-proof-1} and \fref{eq:trans-tot-proof-2} we get
    \begin{eqnarray*}[c]
      c_1 < c_2 \lor c_1 = c_2 \land c_2 < c_3 \lor c_2 = c_3 \land\\ ( p^1_1 < p^2_1
      \lor p^1_1 = p^2_1 \land p^2_1 < p^3_1 \lor p^2_1 = p^3_1 \land
      (\ldots p^1_n < p^2_n \land p^2_n < p^3_n \ldots ))
    \end{eqnarray*}
    And then using induction we get
    \begin{eqnarray*}
      c_1 < c_3 \lor c_1 = c_3 \land ( p^1_1 < p^3_1 \lor p^1_1 = p^3_1 \land
      (\ldots p^1_n < p^3_n \ldots ))
    \end{eqnarray*}
    and then by definition $p_1 < p_3$.

  \item[Totality.]
    Consider patterns $p_1 \neq p_2$. If either one is a variable we have $p_1
    < p_2$ or $p_2 < p_1$ directly by \fref{eq:pat-total-order-strict-con} or by
    the totality of $\lessdot$ and \fref{eq:pat-total-order-strict-var}.

    So assume $p_1 = c_1\texttt{(}p^1_1\texttt{,}\ldots\texttt{,}p^1_n\texttt{)}$
    and $p_2 = c_2\texttt{(}p^2_1\texttt{,}\ldots\texttt{,}p^2_n\texttt{)}$.

    If $c_1 \neq c_2$ we get $p_1 < p_2$ or $p_2 < p_1$ by the totality of
    $\lessdot$ and \fref{eq:pat-total-order-strict-con}.

    So assume $c_1 = c_2$. By induction we have $p^1_i < p^2_i \lor p^2_i <
    p^1_i \lor p^1_i = p^2_i$ for $i \in \{1, \ldots, n\}$. But it must be the
    case that $p^1_i \neq p^2_i$ for some $i$ for else $p_1 = p_2$ and then we
    have a contradiction. This implies that $p_1 < p_2$ (if $p^1_i < p^2_i$) or
    $p_2 < p_1$ (if $p^2_i < p^1_i$).

  \end{description}
\end{proof}

\subsection{Proof of the partial ordering relation}
\label{sec:proof-partial-orderings}

\begin{proof}[Proof of {\fref[plain]{lem:pat-partial-orderings}}]
  We show that $<='$ is reflexive, antisymmetric and transitive. Then we show
  that $<'$ is irreflexive and transitive.
  \begin{description}
  \item[Reflexivity.]
    Assume $p_1 == p_2$. We want to show $p_1 <=' p_2$.

    If either pattern is a variable then clearly so is the other. And then $p_1
    <=' p_2$ follows by \fref{eq:pat-partial-order-weak-var}.

    So assume $p_1 = c_1 \texttt{(} p^1_1 \texttt{,} \ldots \texttt{,} p^1_n
    \texttt{)}$ and $p_2 = c_2 \texttt{(} p^2_1 \texttt{,} \ldots \texttt{,}
    p^2_m \texttt{)}$. It must be the case that $c_1 = c_2$, $n = m$ and $p^1_i
    ==a p^2_i$ for $i \in \{1, \ldots, n\}$.

    \fixme{Henvis til definition af struktural ækvivalens.}

    It follows by induction that $p^1_1 <=' p^2_1 \land \ldots \land p^1_n <='
    p^2_n$ and then by \fref{eq:pat-partial-order-weak-con} that $p_1 <=' p_2$.

  \item[Antisymmetry.]
    Assume $p_1 <=' p_2$ and $p_2 <=' p_1$. We need to show $p_1 == p_2$.

    If either pattern is a variable then so is the other and structural
    equivalence follows directly.

    So assume that one of the patterns is a constructor pattern. By
    \fref{def:pat-partial-order-weak} it is easy to see that so must the other
    and the constructors must be the same. So $p_1 = c \texttt{(} p^1_1
    \texttt{,} \ldots \texttt{,} p^1_n \texttt{)}$ and $p_2 = c \texttt{(} p^2_1
    \texttt{,} \ldots \texttt{,} p^2_n \texttt{)}$, and furthermore
    \begin{eqnarray*}
      p^1_1 <=' p^2_1 &\land& p^2_1 <=' p^1_1 \quad \land\\
      &\vdots&\\
      p^1_n <=' p^2_n &\land& p^2_n <=' p^1_n
    \end{eqnarray*}
    By induction we have $p^1_1 == p^2_1 \land \ldots \land p^1_n == p^2_n$ and
    then $p_1 == p_2$.

    \fixme{Henvis til definitionen af struktural ækvivalens.}

  \item[Transitivity.]
    Assume $p_1 <=' p_2$ and $p_2 <=' p_3$. It must be shown that $p_1 <=' p_3$.

    If $p_3 = v$ the result follows immediately. So assume $p_3 = c_3 \texttt{(}
    p^3_1 \texttt{,} \ldots \texttt{,} p^3_n \texttt{)}$.

    Then clearly $p_1 = c_1 \texttt{(} p^1_1 \texttt{,} \ldots \texttt{,}
    p^1_n\texttt{)}$, $p_2 = c_2 \texttt{(}p^2_1 \texttt{,} \ldots \texttt{,}
    p^2_n\texttt{)}$ and by \fref{eq:pat-partial-order-weak-con} we have
    \begin{eqnarray}[c]
      c_1 = c_2 \land p^1_1 <=' p^2_1 \ldots p^1_n <=' p^2_n \label{eq:pat-partial-order-weak-trans-part-proof-1}\\
      c_2 = c_3 \land p^2_1 <=' p^3_1 \ldots p^2_n <=' p^3_n \label{eq:pat-partial-order-weak-trans-part-proof-2}
    \end{eqnarray}
    Combining \fref{eq:pat-partial-order-weak-trans-part-proof-1} and
    \fref{eq:pat-partial-order-weak-trans-part-proof-2} gives us
    \begin{eqnarray*}
      c_1 = c_2 \land c_2 = c_3 \land p^1 <=' p^2_1 \land p^2_1 <=' p^3_1 \ldots
      p^1_n <=' p^2_n \land p^2_n <=' p^3_n
    \end{eqnarray*}
    And then by induction we get
    \begin{eqnarray*}
      c_1 = c_3 \land p^1_1 <=' p^3_1 \ldots p^1_n <=' p^3_n
    \end{eqnarray*}
    which by \fref{eq:pat-partial-order-weak-con} gives us $p_1 <=' p_3$.

  \end{description}

  Now we show irreflexivity and transitivity of $<'$.
  \begin{description}
  \item[Irreflexivity.]
    Immediately by \fref{def:pat-partial-order-strict}.

  \item[Transitivity.]
    Immediately by the transitivity of $<='$.

  \end{description}
\end{proof}


%%% Local Variables: 
%%% mode: latex
%%% TeX-master: "../miniml"
%%% End: 
