\section{Grammar}
\newcommand{\fun}{\ttt{fun}\ }
\newcommand{\rec}{\ttt{rec}\ }

\fixme{remember to introduce new relations with name.}
\begin{eqnarray*}[rqcql:Tl]
  var & = & \ttt{[a-z]+} & Identifiers\\
  con & = & \ttt{[A-Z][a-z]* | [0-9]+} & Constructors\\
  num & = & \mathbb{Z}    & Natural numbers\\
  match & ::= & \epsilon                            & Empty match\\
  & & pat\texttt{.}exp\ \texttt{|}\ match               & Pattern -> expression\\
  pat & ::= & var                                       & Variable\\
  & & con\texttt{(}pat_1\texttt{,} \ldots\texttt{,} pat_n\texttt{)} & Where con
  has arity $n$\\
  exp & ::= & var                                       & Variable\\
  & & exp_1 exp_2                                     & Application\\
  & & \fun match                                  & Abstraction\\
  & & con\texttt{(}exp_1\texttt{,} \ldots\texttt{,} exp_n\texttt{)} & Where con
  has arity $n$\\
  dec & ::= & var \mapsto exp                         & Plain value binding\\
  & & \rec var \mapsto \fun match         & Recursive value binding\\
  & & con \texttt{:} \overline{num}                           & Constructor of arity $num$\\
  & & \epsilon                             & Empty program\\
  & & dec_1 \texttt{;} dec_2
\end{eqnarray*} 

No pattern may contain a given variable more than once.

\paragraph{Note.} In constructor declarations the arity is overlined to enhance
the difference between numerals and their representation. In practice we write
the syntactic representation of a numeral as the numeral set in typewriter
font.

We use (with superfixes, subfixes and primes) $v$, $c$, $n$, $m$, $p$, $e$
and $d$ to range over $var$, $con$, $num$, $match$, $pat$, $exp$ and $dec$
respectively.

\section{A note about evaluation}
We expect programs to be run in an environment containing predefined functions
(that is variables bound to predefined functions) and constructors. Thus the
program
\begin{quote}
% \begin{verbatim}
\ttt{x $\mapsto$ plus (pair (1, 8))}
% \end{verbatim}
\end{quote}
might make perfect sense (if in particular \ttt{plus} is a variable bound to a
suitable function (perhaps addition), and \texttt{pair}, \texttt{1} and
\texttt{8} are constructors of arity 2, 0 and 0, respectively).
\section{Auxiliary definitions}
\label{sec:auxil-defin}

In the following we define what we mean by equivalence of patterns (with a
permutation of variables), free variables (for expressions, matches and
patterns), substitution (in expressions) and alpha equivalence (of expressions).

\paragraph{Note.}
\begin{enumerate}
\item 
If $f : A -> B$ and $g : A -> B$ are arbitrary mappings then
\begin{eqnarray*}[rlqTl]
  (g ++ f)(x) &= f(x) & if $x \in \Dom(f)$\\
  (g ++ f)(x) &= g(x) & otherwise
\end{eqnarray*}
and
\[
  \Dom (g ++ f) = \Dom (g) \cup \Dom (f).
\]

\item
We write $p \subseteq p'$ to mean that $p$ is a subpattern of $p'$. More
precisely this is the case if $p = p'$ or if $p' = c \ttt{(} p_1 \ttt{,} \ldots
\ttt{,} p_n \ttt{)}$ and $p \subseteq p_i$ for some $i \in \{1, \ldots, n\}$.

In particular we have\footnote{See \fref{sec:free-variables}
  for the definition of $\FV_{pat}$.} $x \subseteq p$ exactly when $x \in
\FV_{pat}(p)$. The relation is obviously reflexive.

\item
The syntactic category $pat$ is a proper subset of $exp$. Let $\kappa : pat ->
exp$ be the canonical mapping from $pat$ to $exp$. It is injective so it has a
left inverse $\kappa^{-1} : exp -> pat$. $\kappa^{-1}$ is clearly not total.
\end{enumerate}

\subsection{Equivalence of patterns}
\label{sec:equivalence-patterns}
We say that two patterns are equivalent if they can be transformed into
each other by a suitable renaming of the variables.

If $p_1$ and $p_2$ are equivalent we write $==_\pi$ where $\pi$ is a permutation
of variables, such that for each variable $x$ in $p_1$ its counterpart in $p_2$
is $\pi(x)$.

For example we have $A(x,y) ==_\pi A(z,x)$ where $\pi = [x \mapsto z, y \mapsto x]$.

\fixme{maybe: show that sigma is a permutation of variables}

\begin{definition}[Equivalence of patterns, $==_\pi$]
\label{def:equivalence-patterns}
  \begin{eqnarray}[rlqTl]
    v_1 &==_{\pi} v_2  & where $\pi = [v_1 \mapsto v_2]$ \label{eq:struct-eq-var} \\
    c\ttt{(}p^1_1 \ttt{,} \ldots \ttt{,} p^1_n \ttt{)} & ==_{\pi}
    c\ttt{(}p^2_1 \ttt{,} \ldots \ttt{,} p^2_n \ttt{)} & \label{eq:struct-eq-con}
  \end{eqnarray}
where \fref{eq:struct-eq-con} holds if
\begin{eqnarray*}[c]
  p^1_1 ==_{\pi_1} p^2_1 \\
  \vdots \\
  p^1_n ==_{\pi_n} p^2_n
\end{eqnarray*}
and $\pi = \pi_1 ++ \ldots ++ \pi_n$.

Note that the domains of each of the $\sigma$s are disjoint because no variable
can occur more than once in a pattern (by the definition of the syntax).

We write $==$ to mean $==_\pi$ (with a suitable non-fixed $\pi$) where $\pi$ has no
interest. Equivalence of patterns is defined by $==_\pi$, \emph{only} when $\pi$
is not fixed.
\end{definition}


\fixme{Show equivalence relation: reflexive, symmetric and transitive.}
\fixme{maybe: show $p_1 ==_\pi p_2$ iff $p_2 ==_{\pi^{-1}} p_1$.}


\subsection{Free variables}\label{sec:free-variables}

We denote the free variables of expressions, matches and patterns with the tree
functions $\FV_{exp}$, $\FV_{match}$ and $\FV_{pat}$, respectively.

\begin{definition}[Free variables of expressions, $\FV_{exp}$] \ \\
  Inductively defined:
  \begin{eqnarray}
    \FV_{exp} (v) &=& \{v\} \\
    \FV_{exp} (\fun m) &=& \FV_{match} (m) \\
    \FV_{exp} (e_1e_2) &=& \FV_{exp} (e_1) \cup \FV_{exp} (e_2) \\
    \FV_{exp} (c\ttt{(}e_1\ttt{,} \ldots \ttt{,} e_n \ttt{)}) &=& \FV_{exp}
    (e_1) \cup \ldots \cup \FV_{exp} (e_n)
  \end{eqnarray}
\end{definition}

\begin{definition}[Free variables of matches, $\FV_{match}$]\ \\ 
  Inductively defined:
  \begin{eqnarray}
    \FV_{match} (\epsilon) &=& \emptyset \\
    \FV_{match} (p\ttt{.}e\ \ttt{|}\ m) &=& \left( \FV_{exp}(e) \setminus
      \FV_{pat}(p) \right) \cup \FV_{match} (m)
  \end{eqnarray}
\end{definition}

\begin{definition}[Free variables of patterns, $\FV_{pat}$] \ \\
  Inductively defined:
  \begin{eqnarray}
    \FV_{pat} (v) &=& \{v\} \\
    \FV_{pat} (c\ttt{(}p_1\ttt{,} \ldots \ttt{,} p_n\ttt{)}) &=& \FV_{pat} (p_1)
    \cup \ldots \cup \FV_{pat} (p_n)
  \end{eqnarray}
\end{definition}

\subsection{Substitution}
We define substitution in expressions. An expression can be substituted for any
(sub)expression of an expressions, not just variables.

\begin{definition}[Substitution]
  If $e_1$, $e_2$ and $e_3$ are expressions we write $e_1[e_2/e_3]$ to be the
  result of substituting all occurrences of $e_3$ in $e_1$ with $e_2$.
  \begin{eqnarray}
    e_1[e_2/e_3] &=& e_2 \quad \mrm{if}\ e_1 = e_3 \label{eq:subst-sub}\\
    (e^1_1 e^2_1)[e_2/e_3] &=& e^1_1[e_2/e_3] e^2_1[e_2/e_3] \label{eq:subst-app}\\
    \fun p_1 \texttt{.} e_1 \texttt{|} m &=& \fun p_1 \texttt{.} e'_1
    \texttt{|} m' \label{eq:subst-lam}\\
    (c \texttt{(}e^1_1 \texttt{,} \ldots \texttt{,} e^1_n \texttt{)})[e_2/e_3]
    &=& c \texttt{(}e^1_1[e_2/e_3] \texttt{,} \ldots \texttt{,} e^1_n[e_2/e_3]
    \texttt{)} \label{eq:subst-con}
  \end{eqnarray}
Where in \fref{eq:subst-lam} we have
\begin{eqnarray*}[rlqTl]
  e'_1 &= e_1 & if $\kappa^{-1}(e_3) \subseteq p_1$\\
  e'_1 &= e_1[e_2/e_3]
\end{eqnarray*}
and
\[
(\fun m)[e_2/e_3] = \fun m'
\]

In \fref[plain]{eq:subst-app}, \fref[plain]{eq:subst-lam} and
\fref{eq:subst-con}  we require that \fref[plain]{eq:subst-sub} does not apply.

Note that in \fref[plain]{eq:subst-sub} we require $e_1$ and $e_3$ to be exactly
equal, not just alpha equivalent.
\end{definition}

\subsection{Alpha equivalence}
\label{sec:alpha-equivalence}

We define alpha equivalence for expressions and for pattern-expression pairs. We
use the symbol $==a$ for both relations.

\begin{definition}[Alpha equivalence of expressions, $==a$]
\label{def:alpha-equivalence}
  First we define alpha equivalence given a mapping of bound variables:
  \begin{eqnarray}
    \sigma |- v_1 &==a& v_2 \label{eq:alpha-var} \\
    \sigma |- e^1_1e^1_2 &==a& e^2_1e^2_1 \label{eq:alpha-exp} \\
    \sigma |- \fun p^1_1 \texttt{.} e^1_1 \texttt{|} \ldots \texttt{|} p^1_n
    \texttt{.} e^1_n &==a& \fun p^2_1 \texttt{.} e^2_1 \texttt{|} \ldots \texttt{|} p^2_n
    \texttt{.} e^2_n \label{eq:alpha-match} \\
    \sigma |- c\ttt{(}e^1_1 \ttt{,} \ldots \ttt{,} e^1_n \ttt{)} &==a& 
    c\ttt{(}e^2_1 \ttt{,} \ldots \ttt{,} e^2_n \ttt{)} \label{eq:alpha-con}
  \end{eqnarray}
where \fref{eq:alpha-var} holds
\begin{eqnarray*}[rlqTl]
\sigma (v_1) &= v_2 & if $v_1 \in \Dom(\sigma)$\\
v_1 &= v_2 & otherwise,
\end{eqnarray*}
\fref{eq:alpha-exp} holds if
\[
\sigma |- e^1_1 ==a e^2_1 \land \sigma |- e^1_2 ==a e^2_1,
\]
\fref{eq:alpha-match} holds if
\begin{eqnarray*}
  (p^1_1, e^1_1) &==a& (p^2_1, e^2_1)\\
  &\vdots&\\
  (p^1_n, e^1_n) &==a& (p^2_n, e^2_n)
\end{eqnarray*}
and \fref{eq:alpha-con} holds if
\[
\sigma |- e^1_1 ==a e^1_n \land \ldots \land \sigma |- e^2_1 ==a e^2_n.
\]

\begin{definition}[Alpha equivalence of pattern-expression pairs, $==a$]\label{def:alpha-equivalence-patexp}
  Again we assume a mapping of bound variables. It is the case that
  \[
  \sigma |- (p_1, e_1) ==a (p_2, e_2)
  \]
  exactly when $p_1 ==_\pi p_2$ and
  \[
  \sigma ++ \pi |- e_1 ==a e_2
  \]
\end{definition}

If $e_1$ and $e_2$ are alpha equivalent expressions we write $e_1 ==a e_2$ which
is a shorthand for $[] |- e_1 ==a e_2$. Similarly for pattern-expression pairs.
\end{definition}

\section{Semantic equivalence}
\label{sec:semantic-equivalence}
We write $e_1 \sim e_2$ if $e_1$ and $e_2$ are semantically equivalent. That is
if $e_2$ is substituted for $e_1$ (or vice versa) in any program $d$ to obtain
$d'$, then if $d$ evaluates to something in an environment $\sigma$ then $d'$
evaluates to that something in $\sigma$, and if $d$ diverges in $\sigma$ so does
$d'$.

\section{Orderings on patterns}
\label{sec:orderings-patterns}
We define a total ($<=$) relation on patterns, and a partial ($<='$) relation on
the quotient set of patterns by structural equivalence. Then we show that they
indeed are orderings.

We write $<$ and $<'$ for $<=$ and $<='$ strict (or irreflexive) counterparts
respectively.

As it turns out it is easier to define $<$ and $<='$ directly and then define
$<=$ and $<'$ in turn of those.

\begin{definition}[Strict total ordering, $<$]\label{def:pat-total-order-strict}
  Assume a total strict ordering $\lessdot$ on constructors and
  variables\footnote{For example let all constructors come before all variables
    and let variables and constructors be ordered lexicographically among
    themselves. Note that constructors and variables are not compared in the
    definition of $<$.}. We inductively define:
  \begin{eqnarray}
    v_1 &<& v_2 \quad \mrm{if}\ v_1 \lessdot v_2\label{eq:pat-total-order-strict-var}\\
    c\texttt{(}p_1\texttt{,} \ldots\texttt{,} p_n\texttt{)} &<& v\\
    c_1\texttt{(}p_1\texttt{,} \ldots\texttt{,} p_n\texttt{)} &<&
    c_2\texttt{(}p'_1\texttt{,} \ldots\texttt{,} p'_m\texttt{)}\label{eq:pat-total-order-strict-con}
  \end{eqnarray}
  Where \fref{eq:pat-total-order-strict-con} holds if
  \[
  c_1 \lessdot c_2 \lor (c_1 = c_2 \land ( p_1 < p'_1 \lor p_1 = p'_1 \land (\ldots p_n < p'_n \ldots )))
  \]
\end{definition}

\begin{definition}[Total ordering, $<=$]\label{def:pat-total-order-weak}
  We define the reflexive cousin:
  \begin{eqnarray*}
    p_1 <= p_2 \Longleftrightarrow p_1 < p_2 \lor p_1 = p_2
  \end{eqnarray*}
\end{definition}

\begin{definition}[Partial ordering, $<='$]\label{def:pat-partial-order-weak}
  We say that $p_2$ weakly generalises $p_1$ or $p_1$ is at least as specific as
  $p_2$ and we write $p_1 <=' p_2$. Inductively defined.
  \begin{eqnarray}
    p &<='& v \label{eq:pat-partial-order-weak-var}\\
    c_1\texttt{(}p_1\texttt{,} \ldots\texttt{,} p_n\texttt{)} &<='&
    c_2\texttt{(}p'_1\texttt{,} \ldots\texttt{,} p'_m\texttt{)}\label{eq:pat-partial-order-weak-con}
  \end{eqnarray}
  Where \fref{eq:pat-partial-order-weak-con} holds if
  \begin{eqnarray*}
    c_1 &=& c_2 \quad \land\\
    p_1 &<='& p'_1 \quad \land\\
    &\ldots&\\
    p_n &<='& p'_n
  \end{eqnarray*}
\end{definition}


\begin{definition}[Strict partial ordering, $<'$]\label{def:pat-partial-order-strict}
  We define the strict counterpart of $<='$ by
  \begin{eqnarray*}
      p_1 <' p_2 \Longleftrightarrow p_1 <=' p_2 \land p_1 \not == p_2
  \end{eqnarray*}
\end{definition}

\begin{lemma}[Total ordering]\label{lem:pat-total-orderings}
  The relation $<=$ is a total ordering, and $<$ is a strict total ordering on
  patterns.
  
  Proof is given in \fref{sec:proof-total-orderings}
\end{lemma}


\begin{lemma}[Partial ordering]\label{lem:pat-partial-orderings}
  The relation $<='$ is a partial ordering and $<'$ is a strict partial ordering
  on the equivalence classes of patterns modulo structural equivalence
  ($pat_{/_{==}}$).

  Proof is given in \fref{sec:proof-partial-orderings}
\end{lemma}

We write $p_1 > p_2$, $p_1 >= p_2$, $p_1 >' p_2$ and $p_1 >=' p_2$ to mean $p_2
< p_1$, $p_2 <= p_1$, $p_2 <' p_1$ and $p_2 <=' p_1$ respectively.

\begin{lemma}[]\label{lem:total-implies-partial}
  If two patterns $p_1$ and $p_2$ are ordered by the partial ordering then they
  are also ordered by the total one. That is
  \begin{eqnarray*}
    p_1 <' p_2 \Longrightarrow p_1 < p_2
  \end{eqnarray*}
\end{lemma}
\begin{proof}
  Straightforward using induction.
\end{proof}

\begin{definition}[Confusion, $||$]\label{def:pat-confusion}
  Let two patterns $p_1$ and $p_2$ be given. If it is the case that neither $p_1
  <=' p_2$ nor $p_1 >=' p_2$ we say that $p_1$ and $p_2$ are confused and we
  write $p_1 || p_2$.
\end{definition}

\begin{lemma}[Unique relation]\label{lem:unique-rel}
  Given two patterns $p_1$ and $p_2$ exactly one of the following hold
  \begin{eqnarray*}
    p_1 &==& p_2\\
    p_1 &<'& p_2\\
    p_1 &>'& p_2\\
    p_1 &||& p_2
  \end{eqnarray*}
\end{lemma}
\begin{proof}
  Immediately by inspection.
\end{proof}

\section{Eliminating unused patterns}
A function is simply a match. And a match is a list of pairs of patterns and
corresponding bodys.

The input to a function is tried against the patterns from top to bottom. An
unused pattern is a pattern that will never see a value which it matches.

This can happen for two reasons.
\begin{enumerate}
\item The pattern will never be tried against the input because the input
  matches an earlier pattern. \label{item:unused-reason-1}
\item The pattern is only tried against inputs it doesn't match. \label{item:unused-reason-2}
\end{enumerate}

\subsection{Cover}\label{sec:cover}
We define a cover to be a set of patterns such that at least one of them matches
any given value.

\begin{definition}[Cover]\label{def:cover}
  Let $P = \{p_1, \ldots, p_n \}$, and $C = \{c_1, \ldots, c_n\}$ be the constructors
  defined in the current environment. Then
  \[
  \Cov(P)
  \]
  if
  \[
  x \in \{p_1, \ldots, p_n \} \lor (\forall c \in C : \Cov(P_1) \land
  \ldots \Cov(P_m))
  \]
  where
  \begin{eqnarray*}
    P_1 &=& \{p'_1 \mid c \texttt{(} p'_1 \texttt{,} \ldots \texttt{,} p'_m \texttt{)} \in P\}\\
    &\vdots&\\
    P_m &=& \{p'_m \mid c \texttt{(} p'_1 \texttt{,} \ldots \texttt{,} p'_m \texttt{)} \in P\}
  \end{eqnarray*}
\end{definition}

Any pattern following patterns that taken together are a cover is unused because
of reason \ref{item:unused-reason-1}.

\subsection{Shadowed patterns}\label{sec:shadowed-patterns}
If a pattern is unused because of reason \ref{item:unused-reason-2} we say that
it is shadowed.
\begin{definition}[Shadowed]
  Let
  \[
  m = p_1\texttt{.}e_1 \texttt{|} \ldots \texttt{|} p_n\texttt{.}e_n
  \]
  If $p_j <=' p_i$ for some $1 \leq i < j \leq n$, then $p_j$ is shadowed (by
  $p_i$).
\end{definition}

We can now define the elimination of unused patterns.
\begin{definition}[Elimination, $~>e$]
\label{def:shadowed-patterns-1}
  We define a reduction relation $~>e$ that expresses the
  elimination of exactly one pattern from a match.

  Let
  \[
  m = p_1\texttt{.}e_1 \texttt{|} \ldots \texttt{|} p_n\texttt{.}e_n
  \]
  If there exist a $p_i$ such that $\{p_1, \ldots, p_{i-1}\}$ is a cover or
  $p_i$ is shadowed, then it is unused and can be eliminated. The resulting
  match is
  \[
  m' = p_1\texttt{.}e_1 \texttt{|} \ldots \texttt{|}
  p_{j-1}\texttt{.}e_{j-1} \texttt{|} p_{j+1}\texttt{.}e_{j+1} \texttt{|}
  \ldots \texttt{|} p_n\texttt{.}e_n,
  \]
  and we write $m ~>e m'$.

  \fixme{Decide upon a symbol for the reduction relation.}
\end{definition}

\begin{lemma}(Preservation)
  If an unused pattern is removed from a program, then the resulting program is
  semantically equivalent.

  That is
  \[
  m ~>e m' ==> \fun m \sim \fun m'
  \]
\end{lemma}

\begin{proof}
  Trivial (as if).
\end{proof}

\section{Generalisiation}
Sometimes patterns get unnecessary complex. If for example a pattern (or one of
its subpatterns) is a constructor pattern whose subpatterns are all variables,
and those variables are only used as arguments to the same constructor (in the
same order) in the function body, then the constructor could simply be replaced
by a fresh variable in pattern and body. That is generalisation of the pattern.

Sometimes the generalisation of a pattern makes it equivalent to another pattern
in the match. And sometimes the two patterns corresponding bodys will merge
seamlessly, such that two patterns can be made to one.
\\[1em]
First we need some auxiliary definitions.

\subsection{Partially ordered form}
\begin{definition}\label{def:part-order-form}
  A match $m = p_1\texttt{.}e_1\texttt{|}\ldots\texttt{|}p_n\texttt{.}e_n$ is in
  partially ordered form if
  \[
  \forall i \in \{1, \ldots, n\} : p_j \not <=' p_i \quad \textnormal{where $j > i$}
  \]
  Note that every match $m$ can be transformed to an equivalent match $m'$ such
  that $m'$ is in partially ordered form, by repeated elimination of shadowed
  patterns (\fref{sec:shadowed-patterns}).
\end{definition}

\subsection{Generalisation of patterns}
We define generalisation of a single pair of a pattern and its body. We write
$(p, e) |> (p', e')$ to mean that the pattern $p$ with its body $e$ generalises
to the pattern $p'$ with the body $e'$.

\begin{definition}[Generalisation of single pattern-body pairs, $|>$]
\label{def:gener-patt}
Inductively defined:
\begin{eqnarray}
  (c \texttt{(} p_1 \texttt{,} \ldots \texttt{,} p_n \texttt{)} , e) &|>& (x , e[x
  / \kappa (c \texttt{(} p_1 \texttt{,} \ldots \texttt{,} p_n \texttt{)} )]) \label{eq:single-gen-1}\\
  (c \texttt{(} p_1 \texttt{,} \ldots \texttt{,} p_i \texttt{,} \ldots
  \texttt{,} p_n \texttt{)}, e) &|>&
  (c \texttt{(} p_1 \texttt{,} \ldots \texttt{,} p'_i \texttt{,} \ldots
  \texttt{,} p_n \texttt{)}, e') \label{eq:single-gen-2}
\end{eqnarray}
Where \fref{eq:single-gen-1} holds when $x$ is a fresh variable and
\[
FV_{pat}(c\texttt{(}p_1\texttt{,}\ldots\texttt{,}p_n\texttt{)}) \cap FV_{exp}(e[x/\kappa
(c\texttt{(}p_1\texttt{,}\ldots\texttt{,}p_n\texttt{)})]) = \emptyset
\]
and \fref{eq:single-gen-2} holds when \fref{eq:single-gen-1} does not and
\[
(p_i , e) |> (p'_i , e')
\]
\end{definition}

\begin{lemma}\label{lem:single-gen-imp-gen}
  If a pattern $p$ (and some body) is generalised to $p'$ (and some other
  body), then $p'$ strictly generalises $p$. In other words
  \[
  (p, e) |> (p', e') ==> p <' p'.
  \]
\end{lemma}
\begin{proof}
  Straightforward induction proof.
\end{proof}

\subsection{Generalisation of matches}
For the generalisation of a match $m$, we require $m$ to be in partially ordered
form.
\\[1em]
When generalising a pattern several things might happen. Assume
\begin{eqnarray*}[rqTcql]
  m = p_1 \texttt{.} e_1 \texttt{|} \ldots \texttt{|} p_i \texttt{.} e_i
  \texttt{|} \ldots \texttt{|} p_n \texttt{.} e_n & and & (p_i, e_i) |> (p'_i,
  e'_i).
\end{eqnarray*}
Now, perhaps $m$ can be generalised if we substitute $p'_i$ for $p_i$ and $e'_i$
for $e_i$. We would like the resulting match to be partially ordered too, so we
must be cautious. Since we know from \fref{lem:single-gen-imp-gen} that $p'_i >'
p_i$ the first part of the match $p_1 \texttt{.} e_1 \texttt{|} \ldots
\texttt{|} p'_i \texttt{.} e'_i$ must still be partially ordered. So we consider
the patterns $p_j$ for $j > i$.

% Note that by \fref{lem:unique-rel} we know that either $p_i <' p_j$ or $p_i ||
% p_j$.

Four scenarios arise
\begin{enumerate}
\item $p'_i$ and $p_j$ are equivalent. The pattern was generalised to one that
  already existed. Now the only hope is that $e'_i$ and $e_j$ merge. By this we
  mean $(p'_i, e'_i) ==a (p_j, e_j)$ (\fref{def:alpha-equivalence-patexp}).

  If this is the case then either one of $p'_i \texttt{.} e'_i$ or $p_j
  \texttt{.} e_j$ shall be removed. \label{item:gen-scen-1}
\item $p'_i$ relates to $p_j$ in the same way that $p_i$ does. So if $p_i <'
  p_j$, then $p'_i <' p_j$ and if $p_i || p_j$ then $p'_i || p_j$. In this case
  nothing must be done. \label{item:gen-scen-2}
\item $p_i || p_j$ and $p'_i >' p_j$. So now $p'_i$ ``steals'' $p_j$s input. But
  because $p_i$ and $p_j$ were confused we know that $p_j$ will not steal any
  input originally intended for $p_i$. So we move $p_j$ and its body up, in the
  match so they come before $p'_i$. \label{item:gen-scen-3}
\item $p_i <' p_j$ and $p'_i >' p_j$. This means that $p'_i$ will match input
  intended for $p_j$ but we can not move $p_j$ above $p'_i$ for then it will
  steal input originally intended for $p_i$. So in this case $m$ can not be
  generalised. \label{item:gen-scen-4}
\end{enumerate}

\begin{definition}[Generalisation, $~>g$]
\label{def:gener-match}
  We define a reduction relation $~>g$ that expresses the
  generalisation of exactly one pattern from a match.

  Let
  \begin{eqnarray*}[rqTcql]
    m = p_1 \texttt{.} e_1 \texttt{|} \ldots \texttt{|} p_i \texttt{.} e_i
    \texttt{|} \ldots \texttt{|} p_n \texttt{.} e_n & and & (p_i, e_i) |> (p'_i,
    e'_i).
  \end{eqnarray*}
  and assume that a generalisation as described above can be done. Then the
  resulting match is
  \begin{eqnarray}
    m' &=& p_1 \texttt{.} e_1 \texttt{|} \ldots \texttt{|} p_{i-1} \texttt{.}
    e_{i-1} \label{eq:gen-1}\\
    &\texttt{|}& p_{m_1} \texttt{.} e_{m_1} \texttt{|} \ldots \texttt{|} p_{m_k}
    \texttt{.} e_{m_k} \label{eq:gen-2}\\
    (&\texttt{|}& p'_i \texttt{.} e'_i \ ) \label{eq:gen-3}\\
    &\texttt{|}& p_{s_1} \texttt{.} e_{s_1} \texttt{|} \ldots \texttt{|} p_{s_l}
    \texttt{.} e_{s_l} \label{eq:gen-4}
  \end{eqnarray}

  Where (\ref{eq:gen-1}) is the part up until $p_i$, (\ref{eq:gen-2}) are the
  patterns that get moved in scenario \ref{item:gen-scen-3} and (\ref{eq:gen-4})
  are the ones that stay (scenario \ref{item:gen-scen-2}). Line (\ref{eq:gen-3})
  is put in parentheses because is should be deleted in the case of scenario
  \ref{item:gen-scen-1}. And of course $m_1 < \ldots < m_k$.

  And we write $m ~>g m'$.
  
\end{definition}

\section{Normal form}
We say that the function $\fun m$ is a normal form if there does not exist an
$m'$ such that $m ~>e m'$ or $m ~>g m'$.

\subsection{Reducing to normal form}
Consider a function $\fun m$. It can be converted to a normal form by repeatedly
eliminating and generalising patterns.

Note that $m$ should be in partially ordered form (\fref{def:part-order-form}) in
order to generalise its patterns. Luckily we can convert it to a partially
ordered form by repeated elimination.

\begin{lemma}
  If a function $f$ can be converted to the normal form $f'$, then $f$ and $f'$
  are semantically equivalent.
\end{lemma}

\begin{lemma}
  We expect the normal form to be unique (in some sense).
\end{lemma}


% \section{Weak unification}



%%% Local Variables: 
%%% mode: latex
%%% reftex-fref-is-default: t
%%% TeX-master: "../miniml"
%%% End: 
