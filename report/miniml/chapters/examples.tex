
\newcommand{\patexp}[2]{#1 \texttt{.} #2}
\newcommand{\fargs}[2]{#1 \texttt{ | } #2}
\newcommand{\cargs}[2]{#1 \texttt{,} #2}
\newcommand{\con}[2]{#1 \texttt{(} #2 \texttt{)}}
\newcommand{\fcall}[2]{#1 \texttt{(} #2 \texttt{)}}
\newcommand{\pipe}{\phantom{\ttt{fu}}\ttt{|}}


% assume nil < ::

% Totally ordered (and thus partially due to lemma \fref{tot-imp-part}):
% fun sum nil = 0
%   | sum (x :: xs) = x + sum xs

% or

% fun sum nil = 0
%   | sum (x :: xs) = x + sum xs
%   | sum _ = raise Fail "Unused pattern"

% Partially but not totally ordered:
% fun sum (x :: xs) = x + sum xs
%   | sum nil = 0

\fixme{Fyld på med tekst}

\section{Auxiliary definitions}

\begin{example}[Suppattern, $\subseteq$]
  \label{ex:suppattern1}
  \begin{eqnarray*}
    \ttt{x} &\subseteq& \ttt{A(x,y)} \\
    \ttt{y} &\subseteq& \ttt{A(x,y)}
\end{eqnarray*}
Whereas this is not a suppattern
\begin{eqnarray*}
  \ttt{z} \not \subseteq \ttt{A(x,y)}
\end{eqnarray*}
\end{example}

\begin{example}[Pattern equivalence, $==_{\pi}$]
  \label{ex:pattern-equiv1}
  \begin{eqnarray*}
    \ttt{A(x,y)} &==_{\pi}& \ttt{A(f,g)}\\
    \ttt{A(f,g)} &==_{\pi'}& \ttt{A(x,y)}
  \end{eqnarray*}
with
  \begin{eqnarray*}
    \pi &=& [x\mapsto f, y \mapsto g] \\
    \pi' &=& [f \mapsto x, g \mapsto y]
\end{eqnarray*}

Whereas these patterns are not equivalent

  \begin{eqnarray*}
    \ttt{A(x,y)} &\not ==& \ttt{B(h,j)} \\
    \ttt{1} &\not ==& \ttt{2}
  \end{eqnarray*}
\end{example}

\begin{example}[Free variables, $\mrm{FV}$]
\label{ex:free-variables1}
\begin{eqnarray*}[c]
  \FV_{exp} \left(
    \begin{eqnalign}[Tl]
\begin{lstlisting}
fun Nil . Nil
  | Cons (x, xs) . Cons(f x, g xs)
\end{lstlisting}
    \end{eqnalign}
  \right) = \{\ttt{f}, \ttt{g} \} \\
%
  \FV_{match} \left( 
    \begin{eqnalign}[Tl]
\begin{lstlisting}
Cons (Nil, Nil) . Cons(Nil, Nil)
\end{lstlisting}
    \end{eqnalign}
  \right) = \{\} \\
%
  \FV_{pat} \left( 
    \begin{eqnalign}[Tl]
\begin{lstlisting}
Cons (x, xs)
\end{lstlisting}
    \end{eqnalign}
  \right) = \{\ttt{x}, \ttt{xs}\} \\
\end{eqnarray*}
\end{example}


\begin{example}[Substitution]
\label{ex:substituation1}

\begin{eqnarray*}[c]
\left(
  \begin{eqnalign}[Tl]
\begin{lstlisting}
fun x . x
  | y . x
\end{lstlisting}
  \end{eqnalign}
\right) \left[ \ttt{z}/\ttt{x} \right] \quad = \quad
  \begin{eqnalign}[Tl]
\begin{lstlisting}
fun x . x
  | y . z
\end{lstlisting}
  \end{eqnalign} \\
%
\left(
  \begin{eqnalign}[Tl]
\begin{lstlisting}
fun x . Cons (1, x)
\end{lstlisting}
  \end{eqnalign}
\right) \left[ \ttt{z}/\ttt{Cons (1, x)} \right] \quad = \quad
  \begin{eqnalign}[Tl]
\begin{lstlisting}
fun x . z
\end{lstlisting}
  \end{eqnalign} \\
\end{eqnarray*}

whereas this is not

\begin{eqnarray*}
\left(
  \begin{eqnalign}[Tl]
\begin{lstlisting}
fun x . x
\end{lstlisting}
  \end{eqnalign}
\right) [ \ttt{z}/\ttt{x} ] \quad \neq \quad
  \begin{eqnalign}[Tl]
\begin{lstlisting}
fun z . z
\end{lstlisting}
  \end{eqnalign}
\end{eqnarray*}
\end{example}

\begin{example}[Alpha equivalence, $==a$]
\label{ex:alpha-equivalence1}

\begin{eqnarray*}[c]
  \begin{eqnalign}[Tl]
\begin{lstlisting}
fun Nil . Nil
  | Cons (x, xs) . Cons(f x, g xs)
\end{lstlisting}
    \end{eqnalign}
  ==a
  \begin{eqnalign}[Tl]
\begin{lstlisting}
fun Nil . Nil
  | Cons (y, ys) . Cons(f y, g ys)
\end{lstlisting}
    \end{eqnalign}
  \end{eqnarray*}


whereas the example below is not, as the free variables are not the same
\begin{eqnarray*}[c]
  \begin{eqnalign}[Tl]
\begin{lstlisting}
fun Nil . Nil
  | Cons (x, xs) . Cons(f x, g xs)
\end{lstlisting}
    \end{eqnalign}
  \not ==a
  \begin{eqnalign}[Tl]
\begin{lstlisting}
fun Nil . Nil
  | Cons (y, ys) . Cons(h y, j ys)
\end{lstlisting}
    \end{eqnalign}
  \end{eqnarray*}
\end{example}


\section{Ordering}
\paragraph{Note.} In the following the ordering $\lessdot$ on variables and
constructors is normal lexicographic ordering.

\begin{example}[Strict total ordering on patterns, $<$]
  \begin{eqnarray*}[c]
    \begin{eqnalign}[Tl]
\begin{lstlisting}
Cons (y, z)
\end{lstlisting}
    \end{eqnalign}
    <
    \begin{eqnalign}[Tl]
\begin{lstlisting}
Cons (x, z)
\end{lstlisting}
    \end{eqnalign}
    <
    \begin{eqnalign}[Tl]
\begin{lstlisting}
Snoc (a, b)
\end{lstlisting}
    \end{eqnalign}
    <
    \begin{eqnalign}[Tl]
\begin{lstlisting}
a
\end{lstlisting}
    \end{eqnalign}
    <
    \begin{eqnalign}[Tl]
\begin{lstlisting}
b
\end{lstlisting}
    \end{eqnalign}
  \end{eqnarray*}
\end{example}

\begin{example}[Strict partial ordering on patterns, $<'$]
  \begin{eqnarray*}[c]
    \begin{eqnalign}[Tl]
\begin{lstlisting}
Cons (y, Nil)
\end{lstlisting}
    \end{eqnalign}
    <'
    \begin{eqnalign}[Tl]
\begin{lstlisting}
Cons (x, z)
\end{lstlisting}
    \end{eqnalign}
    <'
    \begin{eqnalign}[Tl]
\begin{lstlisting}
b
\end{lstlisting}
    \end{eqnalign}
  \end{eqnarray*}
  Keeping in mind that the exact names of the variables are unimportant.
\end{example}

\begin{example}[Elimination, $->e$]
  The first two patterns make a cover so the last pattern is eliminated.
  \begin{eqnarray*}[c]
    \begin{eqnalign}[Tl]
\begin{lstlisting}
  Cons(x, xs) . Cons (x, xs)
| Nil . Nil
| x . x
\end{lstlisting}
    \end{eqnalign}
    ->e
    \begin{eqnalign}[Tl]
\begin{lstlisting}
  Cons(x, xs) . Cons (x, xs)
| Nil . Nil
\end{lstlisting}
    \end{eqnalign}
  \end{eqnarray*}

  The second pattern in the example below is shadowed by the first pattern and is thus eliminated.
  \begin{eqnarray*}[c]
    \begin{eqnalign}[Tl]
\begin{lstlisting}
  Cons (x, y) . Cons (x, y)
| Cons (Cons (x, y), z) . Cons (Cons (x, y), z)
| Nil
\end{lstlisting}
    \end{eqnalign}
    ->e
    \begin{eqnalign}[Tl]
\begin{lstlisting}
  Cons (x, y) . Cons (x, y)
| Nil
\end{lstlisting}
    \end{eqnalign}
  \end{eqnarray*}

\end{example}

\begin{example}[Generalisation of single pattern-body pairs, $|>$]
  This example shows a match with an unnecessary complex pattern that reverses
  the components of a pair.

  \begin{eqnarray*}[c]
    \begin{eqnalign}[Tl]
\begin{lstlisting}
Pair (Cons (x, xs), zs) . Pair (zs, Cons (x, xs))
\end{lstlisting}
    \end{eqnalign}
    |>
    \begin{eqnalign}[Tl]
\begin{lstlisting}
Pair (xs, zs) . Pair (zs, xs)
\end{lstlisting}
    \end{eqnalign}
  \end{eqnarray*}
\end{example}


\begin{example}[Reduction relation, $->g$]
Consider
\begin{lstlisting}
      addodd Cons (Cons (Pair (a, b), Pair (c, d)), xs) .
               Cons (Cons (Pair (a, b), + (Pair (c, d))), addodd xs)
    | addodd Cons (Pair (a, b), Nil) . Cons (Pair (a, b), Nil)
    | addodd Nil . Nil
    | addodd x . x
\end{lstlisting}
Assume that the constructor environment is $C = \{\mathtt{Cons}, \mathtt{Nil},
\mathtt{Pair}\}$, and that there exists a function \texttt{+}.

As the match is not in partially ordered form (due to the pattern \texttt{x})
our only choice is to eliminate. Luckily we can, because the three first
patterns forms a cover, so we can eliminate the fourth. We get
\begin{lstlisting}
      addodd Cons (Cons (Pair (a, b), Pair (c, d)), xs) .
               Cons (Cons (Pair (a, b), + (Pair (c, d))), addodd xs)
    | addodd Cons (Pair (a, b), Nil) . Cons (Pair (a, b), Nil)
    | addodd Nil . Nil
\end{lstlisting}
which is in partially ordered form. The second clause can be generalised to
\texttt{x . x}. This means that the clause \texttt{Nil . Nil} should be moved
up. We get
\begin{lstlisting}
      addodd Cons (Cons (Pair (a, b), Pair (c, d)), xs) .
               Cons (Cons (Pair (a, b), + (Pair (c, d))), addodd xs)
    | addodd Nil . Nil
    | addodd x . x
\end{lstlisting}
Next we can generalise \texttt{Nil . Nil} to \texttt{x . x} which coincide with
the third clause, so we remove it.
\begin{lstlisting}
      addodd Cons (Cons (Pair (a, b), Pair (c, d)), xs) .
               Cons (Cons (Pair (a, b), + (Pair (c, d))), addodd xs)
    | addodd x . x
\end{lstlisting}
The first clause can be generalised to \texttt{
  Cons (Cons (x, y), xs) . Cons (Cons (x, + y), addodd xs)
} (by generalising the two \texttt{Pair} patterns), so we get
\begin{lstlisting}
      addodd Cons (Cons (x, y), xs) .
               Cons (Cons (x, + y), addodd xs)
    | addodd x . x
\end{lstlisting}
which cannot be generalised or eliminated upon. So it is a normal form.

\end{example}

%%% Local Variables: 
%%% mode: latex
%%% TeX-master: "../miniml"
%%% End: 
