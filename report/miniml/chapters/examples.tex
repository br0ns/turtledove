
\newcommand{\patexp}[2]{#1 \texttt{.} #2}
\newcommand{\fargs}[2]{#1 \texttt{ | } #2}
\newcommand{\cargs}[2]{#1 \texttt{,} #2}
\newcommand{\con}[2]{#1 \texttt{(} #2 \texttt{)}}
\newcommand{\fcall}[2]{#1 \texttt{(} #2 \texttt{)}}
\newcommand{\pipe}{\phantom{\ttt{fu}}\ttt{|}}


% assume nil < ::

% Totally ordered (and thus partially due to lemma \fref{tot-imp-part}):
% fun sum nil = 0
%   | sum (x :: xs) = x + sum xs

% or

% fun sum nil = 0
%   | sum (x :: xs) = x + sum xs
%   | sum _ = raise Fail "Unused pattern"

% Partially but not totally ordered:
% fun sum (x :: xs) = x + sum xs
%   | sum nil = 0

\fixme{Fyld på med tekst}

\section{Auciliary definitions}

\begin{example}[Suppattern, $\subseteq$]
  \label{ex:suppattern1}
  \begin{eqnarray*}
    \ttt{x} &\subseteq& \ttt{A(x,y)} \\
    \ttt{y} &\subseteq& \ttt{A(x,y)}
\end{eqnarray*}
Whereas this is not a suppattern
\begin{eqnarray*}
  \ttt{z} \not \subseteq \ttt{A(x,y)}
\end{eqnarray*}
\end{example}


\begin{example}[Pattern equivalence, $==_{\pi}$]
  \label{ex:pattern-equiv1}
  \begin{eqnarray*}
    \ttt{A(x,y)} &==_{\pi}& \ttt{A(f,g)}\\
    \ttt{A(f,g)} &==_{\pi'}& \ttt{A(x,y)}
  \end{eqnarray*}
with 
  \begin{eqnarray*}
    \pi &=& [x\mapsto f, y \mapsto g] \\
    \pi' &=& [f \mapsto x, g \mapsto y]
\end{eqnarray*}

Whereas these patterns are not equivalent

  \begin{eqnarray*}
    \ttt{A(x,y)} &\not ==& \ttt{B(h,j)} \\
    \ttt{1} &\not ==& \ttt{2}
  \end{eqnarray*}
\end{example}

\begin{example}[Free variables, $\mrm{FV}$]
\label{ex:free-variables1}
\begin{eqnarray*}[c]
  \FV_{exp} \left( 
    \begin{eqnalign}[Tl]
\begin{lstlisting}
fun Nil . Nil
  | Cons (x, xs) . Cons(f x, g xs)
\end{lstlisting}
    \end{eqnalign}
  \right) = \{\ttt{f}, \ttt{g} \} \\
%
  \FV_{match} \left( 
    \begin{eqnalign}[Tl]
\begin{lstlisting}
Cons (Nil, Nil) . Cons(Nil, Nil)
\end{lstlisting}
    \end{eqnalign}
  \right) = \{\} \\
%
  \FV_{pat} \left( 
    \begin{eqnalign}[Tl]
\begin{lstlisting}
Cons (x, xs)
\end{lstlisting}
    \end{eqnalign}
  \right) = \{\ttt{x}, \ttt{xs}\} \\
\end{eqnarray*}
\end{example}


\begin{example}[Substitution]
\label{ex:substituation1}

\begin{eqnarray*}[c]
\left(
  \begin{eqnalign}[Tl]
\begin{lstlisting}
fun x . x
  | y . x
\end{lstlisting}
  \end{eqnalign}
\right) \left[ \ttt{z}/\ttt{x} \right] \quad = \quad
  \begin{eqnalign}[Tl]
\begin{lstlisting}
fun x . x
  | y . z
\end{lstlisting}
  \end{eqnalign} \\
%
\left(
  \begin{eqnalign}[Tl]
\begin{lstlisting}
fun x . Cons (1, x)
\end{lstlisting}
  \end{eqnalign}
\right) \left[ \ttt{z}/\ttt{Cons (1, x)} \right] \quad = \quad
  \begin{eqnalign}[Tl]
\begin{lstlisting}
fun x . z
\end{lstlisting}
  \end{eqnalign} \\
\end{eqnarray*}

whereas this is not

\begin{eqnarray*}
\left(
  \begin{eqnalign}[Tl]
\begin{lstlisting}
fun x . x
\end{lstlisting}
  \end{eqnalign}
\right) [ \ttt{z}/\ttt{x} ] \quad \neq \quad
  \begin{eqnalign}[Tl]
\begin{lstlisting}
fun z . z
\end{lstlisting}
  \end{eqnalign}
\end{eqnarray*}
\end{example}

\begin{example}[Alpha equivalence, $==a$]
\label{ex:alpha-equivalence1}

\begin{eqnarray*}[c]
  \begin{eqnalign}[Tl]
\begin{lstlisting}
fun Nil . Nil
  | Cons (x, xs) . Cons(f x, g xs)
\end{lstlisting}
    \end{eqnalign}
  ==a
  \begin{eqnalign}[Tl]
\begin{lstlisting}
fun Nil . Nil
  | Cons (y, ys) . Cons(f y, g ys)
\end{lstlisting}
    \end{eqnalign}
  \end{eqnarray*}

whereas this is not, as the free variables are not the same

\begin{eqnarray*}[c]
  \begin{eqnalign}[Tl]
\begin{lstlisting}
fun Nil . Nil
  | Cons (x, xs) . Cons(f x, g xs)
\end{lstlisting}
    \end{eqnalign}
  \not ==a
  \begin{eqnalign}[Tl]
\begin{lstlisting}
fun Nil . Nil
  | Cons (y, ys) . Cons(h y, j ys)
\end{lstlisting}
    \end{eqnalign}
  \end{eqnarray*}
\end{example}




\section{Ordering}

\paragraph{Note.} The below examples on orderings are not semantically
equivalent, but shows only how the ordering relation works.


\begin{example}[Strict total ordering, $<$]
\fixme{this is totally wrong but the idea is somewhat ok...}
  Given then following function $foo$
  \begin{eqnarray*}[rl]
    \texttt{rec } foo & \mapsto \fun (b\ttt{.}exp_1 \texttt{ | } a\ttt{.}exp_2 \texttt{ | }
    cons(y, z)\ttt{.}exp_3 \texttt{ | } cons(x, z)\ttt{.}exp_4) \\
\tabpause{then strict total ordering will result in}
    \texttt{rec } foo & \mapsto \fun (cons(x\texttt{,} z)\texttt{.}exp_4 \texttt{ |
    } cons(y\texttt{,} z)\texttt{.}exp_3 \texttt{ | } a\texttt{.}exp_2 \texttt{ | }
    b\texttt{.}exp_1)
\end{eqnarray*}
\end{example}


\begin{example}[Strict partial ordering, $<'$]
%   Given the following function $combine$
%   \begin{eqnarray*}[rll>{\ttt{.}}l]
%     \texttt{rec } combine \mapsto 
%     & \fun & \con{cons}{\cargs{x}{\con{cons}{\cargs{y}{xl}}}} & \fcall{cons}{\cargs{\con{pair}{\cargs{x}{y}}}{combine(xl)}} \\
%       & \pipe & \fcall{cons}{\cargs{xl}{nil}} & \mathrel{nil} \\
%       & \pipe & \mathrel{nil} & \mathrel{nil}\\
% \tabpause{then partial ordering will result in}
%     \texttt{rec } combine \mapsto 
%     & \fun & hej \\
%     & \pipe & ...
%   \end{eqnarray*}
%   \fixme{uheldigt, eftersom ingen af patternsne ordner partielt mellem hinanden
%     grundet aritet?}
\end{example}


\begin{example}[Elimination, $~>e$]

\end{example}


\begin{example}[Generalisation of single pattern-body pairs, $|>$]

\end{example}


\begin{example}[Reduction relation, $~>g$]

\end{example}

%%% Local Variables: 
%%% mode: latex
%%% TeX-master: "../miniml"
%%% End: 
