% \documentclass[a4paper, oneside, final]{memoir} 
\documentclass[a4paper, oneside, draft]{memoir} 
\usepackage[T1]{fontenc}
\usepackage[utf8]{inputenc}
\usepackage[british]{babel}

% bedre orddeling Gør at der som minimum skal blive to tegn på linien ved
% orddeling og minimum flyttes to tegn ned på næste linie. Desværre er værdien
% anvendt af babel »12«, hvilket kan give orddelingen »h-vor«.
\renewcommand{\britishhyphenmins}{22} 

% Fix of fancyref to work with memoir. Makes references look
% nice. Redefines memoir \fref and \Fref to \refer and \Refer.
% \usepackage{refer}             %
% As we dont really have any use for \fref and \Fref we just undefine what
% memoir defined them as, so fancyref can define what it wants.
\let\fref\undefined
\let\Fref\undefined
\usepackage{fancyref} % Better reference. 

\usepackage{pdflscape} % Gør landscape-environmentet tilgængeligt
\usepackage{fixme}     % Indsæt "fixme" noter i drafts.
\usepackage{hyperref}  % Indsæter links (interne og eksterne) i PDF

\usepackage[format=hang]{caption,subfig}
\usepackage{graphicx}
\usepackage{stmaryrd}
\usepackage{amssymb}
\usepackage{listings}
\usepackage{ulem} % \sout - strike-through
\usepackage{tikz}

\usepackage{mdwtab}
\usepackage{mathenv}
\usepackage{amsthm}
\usepackage{semantic} % for the \mathlig function


\renewcommand{\ttdefault}{txtt} % Bedre typewriter font
% \usepackage[sc]{mathpazo}     % Palatino font
% \renewcommand{\rmdefault}{ugm} % Garamond
% \usepackage[garamond]{mathdesign}

% \overfullrule=5pt
% \setsecnumdepth{part}
\setcounter{secnumdepth}{1} % Sæt overskriftsnummereringsdybde. Disable = -1.
\chapterstyle{hangnum} % changes style of chapters, to look nice.

\theoremstyle{definition}
\newtheorem{judgement}{Judgement}
\newtheorem{definition}{Definition}
\newtheorem{lemma}{Lemma}
\newtheorem{theorem}{Theorem}
\newtheorem{corollary}{Corollary}
\newtheorem{example}{Example}


\mathlig{||}{\parallel}
\mathlig{<'}{\prec}
\mathlig{>'}{\succ}
\mathlig{<='}{\preccurlyeq}
\mathlig{>='}{\succcurlyeq}
\mathlig{<=}{\leqslant}
\mathlig{>=}{\geqslant}
\mathlig{<>}{\neq}
\mathlig{|=}{\sqsubset}
\mathlig{=|}{\sqsupset}
\mathlig{==}{\equiv}
\mathlig{==a}{=_{\alpha}}
\mathlig{<|}{\lhd}
\mathlig{|>}{\rhd}
\mathlig{++}{\mathrel{\mbox{+\!\!\!+}}}
\mathlig{~>e}{\stackrel{elim}{\leadsto}}
\mathlig{~>g}{\stackrel{gen}{\leadsto}}

\newcommand{\patexp}[2]{#1 \texttt{.} #2}
\newcommand{\fargs}[2]{#1 \texttt{ | } #2}
\newcommand{\cargs}[2]{#1 \texttt{,} #2}
\newcommand{\con}[2]{#1 \texttt{(} #2 \texttt{)}}
\newcommand{\fcall}[2]{#1 \texttt{(} #2 \texttt{)}}

\begin{document}
\chapter{Examples}

% assume nil < ::

% Totally ordered (and thus partially due to lemma \fref{tot-imp-part}):
% fun sum nil = 0
%   | sum (x :: xs) = x + sum xs

% or

% fun sum nil = 0
%   | sum (x :: xs) = x + sum xs
%   | sum _ = raise Fail "Unused pattern"

% Partially but not totally ordered:
% fun sum (x :: xs) = x + sum xs
%   | sum nil = 0

\section{Ordering}

\paragraph{Note.} The below examples on orderings are not semantically
equivalent, but shows only how the ordering relation works.}


\begin{example}[Strict total ordering, $<$]
  Given then following function $foo$
  \begin{eqnarray*}[rl]
    \texttt{rec } foo & \mapsto \lambda (b.exp_1 \texttt{ | } a.exp_2 \texttt{ | }
    con(y, z).exp_3 \texttt{ | } con(x, z).exp_4) \\
\tabpause{then strict total ordering will result in}
    \texttt{rec } foo & \mapsto \lambda (con(x\texttt{,} z)\texttt{.}exp_4 \texttt{ |
    } con(y\texttt{,} z)\texttt{.}exp_3 \texttt{ | } a\texttt{.}exp_2 \texttt{ | }
    b\texttt{.}exp_1)
\end{eqnarray*}
\end{example}


\begin{example}[Strict partial ordering, $<'$]
  Given the following function $combine$
  \begin{eqnarray*}[rll]
    \texttt{rec } combine \mapsto 
    & \lambda ( & \patexp{\con{::}{\cargs{x}{\con{::}{\cargs{y}{xl}}}}}{\fcall{::}{\cargs{\con{pair}{\cargs{x}{y}}}{combine(xl)}}} \\
      & \texttt{|} & \patexp{\fcall{::}{\cargs{xl}{nil}}}{nil} \\
      & \texttt{|} & \patexp{nil}{nil} )\\
\tabpause{then partial ordering will result in}
    \texttt{rec } combine \mapsto 
    & \lambda ( & hej \\
    & \texttt{|} & ... )
  \end{eqnarray*}
  \fixme{uheldigt, eftersom ingen af patternsne ordner partielt mellem hinanden
    grundet aritet?}
\end{example}


\begin{example}[Elimination, $~>e$]

\end{example}


\begin{example}[Generalisation of single pattern-body pairs, $|>$]

\end{example}


\begin{example}[Reduction relation, $~>g$]

\end{example}

\end{document}



%%% Local Variables: 
%%% mode: latex
%%% TeX-master: t
%%% reftex-fref-is-default: t
%%% End: 
